\subsubsection{Suma directa externa}

\begin{df}
Tomando el producto cartesiano de \(t\) módulos sobre el mismo anillo
y tomando la suma usual de tuplas y definiendo el siguiente producto:
\[
  a(m_1,\ldots, m_t)=(am_1,\ldots,am_t)
\]

Es un módulo que se llama suma directa externa de \(M_1,\ldots, M_t\)
con \(M^t\) si son todos iguales.

  Se denota \(M_1\oplus\cdots\oplus M_t\).
\end{df}


Ejercicio: Sea \(\subscriptbefore{A}{M}\), \(N_1,\ldots,
N_t\in\mathcal{L}(\subscriptbefore{A}{M})\). Se pide demostrar que existe
un homomorfismo \(f:N_1\oplus\cdots\oplus N_t\longrightarrow
N_1{+}\cdots{+}N_t\)
sobreyectivo de \(A\)-módulos tal que entre la suma directa
externa y la suma interna, tal que \(f\) es un isomorfismo si y solo si
la suma interna es directa. Podría ser interesante usar coordenadas.
\begin{df}[Base de un módulo libre]
  Consideramos \(A^n=A\oplus\cdots\oplus A\), donde la suma se repite
  \(n\) veces. Tenemos que \(\{e_i = (0,\ldots, 0,1,0, \ldots, 0) : i \in \{1, \ldots, n\}\}
  \) forman un sistema de
  generadores de \(A^n\). Por tanto \(a=\sum_i a_i e_i\in A^n\)
  es una expresión única.
\end{df}

Dicha base puede no existir.

\begin{prop}
  Dado un módulo cualquiera \(\subscriptbefore{A}{M}\) y \(m_1, m_n\in M\),
  existe un único homomorfismo de módulos \(f:A^n\longrightarrow M\)
  tal que \(f(e_i)=m_i\).
\end{prop}

\begin{cor}
  Si \(M\) es finitamente generado con generadores \(\{m_i\}\),
  entonces \(M\cong A^n/L\) para \(L\) un cierto submódulo.
\end{cor}

\begin{proof}
  Unicidad: si existe una tal aplicación \(f\), entonces para
  cualquier \(a\in A^n\),
  \[
    f(a)=\sum_i a_i f(e_i)=\sum_i a_i m_i
  \]

  Veamos la existencia,
  Definiendo \(f(a)=\sum_i a_i m_i\) obtenemos un homomorfismo de módulos
  que cumple lo exigido en el enunciado.

  Si \(M=Am_1+\cdots+Am_n\) tenemos que \(L=\ker f\) cumple lo que se
  pide por el teorema de isomorfía para módulos.
\end{proof}