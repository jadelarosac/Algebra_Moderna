\subsection{El caso de la circunferencia unidad}

¿Y si \(G\) no es finito? En general es intratable. Lo más fácil es estudiar
grupos compactos, habitualmente \(p\)-ádicos o grupos de Lie.

Veamos un ejemplo de un grupo de Lie compacto:
\[
  \Sphere^1=\{z\in\C:|z|=1\}
\]

Tenemos \(\C \Sphere^1\). Tomamos \(\C \Sphere^1\)-módulos de dimensión
compleja finita que provengan de representaciones continuas de \(\Sphere^1\).
Estas son los homomorfismos continuos de grupos
\(\rho:\Sphere^1\longrightarrow GL(V)\) con dimensión finita.

Dada \(\rho\) quiero definir un producto directo \(\langle\cdot|\cdot
\rangle_\Sphere\) en \(V\) tal que:
\[
  \langle\rho(z)(v)|\rho(z)(w)\rangle_\Sphere=\langle v|w\rangle_\Sphere
\]
para todo \(v,w\in V\). En notación de módulos:
\[
  \rho(z)(v)=zv
\]

En efecto, tómese un producto interno cualquiera \(\langle\cdot|\cdot\rangle\)
y definimos:
\[
  \langle v|w\rangle_{\Sphere^1}:=\int_{\Sphere^1}\langle
  zv|zw\rangle dz
\]
que es integrable por ser continua sobre un compacto. Se puede ver sin mucha
dificultad que es un producto interno.

Veamos que es unitario.
Sea \(z'\), tenemos:

\[
  \langle z'v|z'w\rangle_{\Sphere^1}=\int_{\Sphere^1}\langle
  zz'v|zz'w\rangle dz
  =
  \int_{\Sphere^1}\langle uv|uw\rangle du=
  \langle v|w\rangle_{\Sphere^1}
\]
con un cambio de variable adecuado (es una isometría), la medida
de \(\Sphere^1\) es invariante por la acción de \(\Sphere^1\).

Dicha acción es unitaria.

