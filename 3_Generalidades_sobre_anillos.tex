
\begin{obs}
  Para anillos conmutativos denotamos
  \[
    \langle a\rangle=\{ba:b\in A\}
  \]
  el ideal generado por \(a\).
\end{obs}

Vamos a hacer un ejemplo, aplicando el teorema anterior.

\subsubsection{Interpolación}         

Tomamos \(A=K[x]\), un anillo de polinomios con coeficientes en un
cuerpo \(K\).

Sea \(A_i = K\) con \(i\in\{1,\ldots,t\}\).
Tomamos \(\alpha_i\in K\) para cada \(i\) y definimos
\(\xi_i:K[x]\longrightarrow K\), \(\xi(g)=g(\alpha_i)\),
para cada \(g\in K[x]\) y es un homomorfismo de anillos.

\(\Im \xi_i=K\) y \(\xi:K[x]\longrightarrow K\times
\cdots \times K=K^t\).

\(\ker \xi_i=\langle x-\alpha_i\rangle\) que es ideal de un anillo de
polinomios, luego principal. Está generado por el polinomio de grado menor,
como las constantes no pueden anular a \(\xi_i\), tiene que estar generado
por ese, que es de grado uno.

\[
  I=\bigcap_{i=1}^t\langle x-\alpha_i\rangle =\langle p(x)\rangle
\]
donde \(p(x)=\mcm\{x-\alpha_i: i\in\{1,\ldots, t\}\}\).


El teorema chino del resto nos asegura que \(\tilde{\xi}:
K[x]/\langle p(x)\rangle\longrightarrow K^t\) es un isomorfismo si y solo si
\(\mcd\{x-\alpha_i, x-\alpha_j\}\) para todo \(j\neq i\), es decir,
si \(\alpha_i\neq \alpha_j\).

Lo que estamos viendo es que para cualquier tupla
\((y_1,\ldots, y_t)\in K^t\), existe un \(g\in K[x]\) tal que
\(g(\alpha_i)=y_i\), si y solo si \(\alpha_i\neq\alpha_j\). En tal
caso, \(p(x)=\prod_{i=1}^t(x-\alpha_i)\).

Existe un único representante \(g\in K[x]\) tal que \(g(\alpha_i)=y_i\)
de grado menor que \(t\), siempre que
\(p(x)=\prod_{i=1}^t(x-\alpha_i)\).

\(\alpha_1,\ldots,\alpha_t\in K\) distintos dos a dos
\[
  \tilde{\xi}:K[x]/\langle p(x)\rangle\longrightarrow K^t
\]
es un isomorfismo de anillos.

\(K[x]/\langle p(x)\rangle\) es un espacio vectorial cociente.

\(\tilde{\xi}\) es también un isomorfismo entre espacios vectoriales.

\[
  \tilde{\xi}(\alpha(g+p))=\tilde{\xi}(\alpha g+p)=\tilde{\xi}((\alpha
  + p)(g +p))=\]\[\tilde{\xi}(\alpha + p)\tilde{\xi}(g+p)=
  (\alpha,\ldots, \alpha)(g(\alpha_1),\ldots g(\alpha_t))=
  \alpha\tilde{\xi}(g+p)
\]

Sea \(\{1+p, x+p, x^2+p,\ldots, x^{t-1} +p\}\)
\(K\)-base de \(K[x]/\langle p(x)\rangle\).
Notamos:
\[ 1 = 1+p\]
\[ x = x+p\]

Sea \(\{e_1,\ldots, e_n\}\) es la base canónica de \(K^t\).
Nuestro objetivo es calcular sus preimágenes por \(\xi\), en concreto
un polinomio de grado menor que \(t\).

\[
  g_i(x)=\prod_{j\neq i}(x-\alpha_j)
\]

\[
  L_i(x)=\frac{g_i(x)}{g(\alpha_i)}=\prod_{j\neq i}\frac{x-x_j}{x_i-x_j}
\]

que vale 0 en \(\alpha_j\) para cualquier \(j\)
salvo en \(\alpha_i\) que vale 1.

Tenemos que
\[
  g(x)=\sum_{i=1}^t y_i L_i(x)
\]
satisface que \(g(\alpha_i)=y_i\).


Finalmente vamos a ver que la matriz de \(\tilde{\xi}\) en las bases
consideradas es:
\[
  \begin{pmatrix}
    1&\cdots& 1\\
    \alpha_1 &\cdots&\alpha_t\\
    \cdots&\cdots&\cdots\\
    \alpha_1^t&\cdots&\alpha_t^t
  \end{pmatrix}
\]
