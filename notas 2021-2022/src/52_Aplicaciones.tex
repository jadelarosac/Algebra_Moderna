\begin{cor}
  \[
    |G| = d_1^2+\cdots+d_t^2
  \]
\end{cor}
\begin{proof}
  \(\mu(G)=C(\Sigma_1)\dot{+}\cdots\dot{+}C(\Sigma_t)\) con lo que:
  \[|G|=\sum_{i=1}^t \dim_\C C(\Sigma_i)=\sum_{i=1}^t d_i^2\]
\end{proof}

\begin{prop}
  Sea \(G\) abeliano finito, \(\Sigma\) un \(\C G\)-módulo simple.
  Entonces \(\dim_\C \Sigma = 1\).
\end{prop}
\begin{proof}
  \(\Sigma\) tiene dimensión compleja finita, \(x\in G\),
  \(f_x:\Sigma\longrightarrow\Sigma\), \(f_x(v)=xv\)
  con \(y\in G\):
  \[
    f_x(yv)=xyv=yxv=yf_x(v)
  \]
  entonces \(f_x\in\End_{\C G}(\Sigma)=
  \{\lambda\id_\Sigma:\lambda\in\C\}\). Así que \(f_x=\lambda_x\id_\Sigma\)
  para cierto \(\lambda_x\in\C\).

  Sea \(v\in\Sigma\setminus\{0\}\), \(w\in\Sigma\),
  \(w=(\sum_{x\in G} \alpha_x x)v\) porque todo módulo simple está generado
  por cualquiera de sus elementos.
  Pero entonces: \[
    w=\sum_{x\in G} \alpha_x f_x(v)=\sum_{x\in G} \alpha_x\lambda_x v
  \]
  luego \(\dim_\C \Sigma = 1\).

\end{proof}

\begin{cor}
  Si \(G\) es abeliano, \(n=|G|\), entonces \(|\Sigma_{\C G}|=n\).
\end{cor}
\begin{proof}
  \(\Sigma_{\C G} =\{\Sigma_1,\ldots,\Sigma_t\}\), por el teorema
  de Webber-Artin,
  \[
    \C G \cong \End_{\C G}(\Sigma_1)\times\cdots\times
    \End_{\C G}(\Sigma_t)
    \cong \C\times\overset{(t)}{\cdots}\times\C
  \]
  con lo que \(n =t\).

\end{proof}

Ejemplo: \(G=\Z_n=\{0,1,\ldots, n-1\}\).
Tenemos que ver que \(\Omega_{\C G}=\{\Sigma_0,\ldots,\Sigma_{n-1}\}\), con
\(\dim_\C \Sigma_j = 1\) para todo \(j\in\{1,\ldots,n-1\}\).

Sea \(u_j\) una base de \(\Sigma_j\) (\(\Sigma_j =\C u_j\)).
Sea \(\omega =e^{2\pi i/n}\in\C\), ponemos \(k u_j :=\omega^{kj} u_j\) para
\(k\in\Z_n\).

Es claro que \((k+k')u_j=\omega^{kj+k'j}u_j=(k\circ k')u_j\) y el 0 va a la
identidad. \(\Sigma_j\) es un \(\C \Z_n\)-módulos simples (tiene
dimensión 1). Basta ver que ningún par de ellos son isomorfos entre sí.

Supongamos \(f:\Sigma_j\longrightarrow \Sigma_k\) \(\C G\)-lineal y no
nulo. Veamos que \(k=j\). \(\exists\alpha\in\C\setminus\{0\}\)
tal que \(f(u_j)=\alpha u_k\).
\[
  \omega^j \alpha u_k = \omega^j f(u_j)=f(\omega^j u_j)=f(1u_j)=1\alpha u_k
  =\alpha\omega_k u_k
\]
Luego \(\omega^j=\omega^k\) y por tanto \(j=k\).

Cada \(C(\Sigma_j)\) tiene como base \(\{t^{\Sigma_j}\}\), donde
\(t^{\Sigma_j}(k)=k u_j=\omega^{kj}\). Son una base ortonormal
de \(\mu(\Z_n)\) respecto del producto interno:
\[
  \langle\varphi|\psi\rangle=\frac{1}{n}\sum_{k\in\Z_n}\varphi(k)
  \overline{\psi(k)}
\]

Si \(\varphi\in\mu(\Z_n)\), \(\varphi=\sum_{j=0}^{n-1}\langle
\varphi|t^{\Sigma_j}\rangle t^{\Sigma_j}\). Es decir, obtenemos
la transformada de Fourier Discreta.





