\begin{obs}
  \(\Cont^\infty(\R)\) dotado de estructura de \(\R[x]\)-módulo
  a través del endomorfismo lineal \(T=\frac{d}{dt}\) es un ejemplo
  ilustrativo en el siguiente sentido.

  Tomemos \(\sin\), \(x\sin t=T(\sin t)=\cos t\)
  \(x^2\sin t= -\sin t\) con lo que
  \[
    (x^2+1)\sin t=0
  \]
  es decir, en un \(A\)-módulo \(M\) puede pasar que \(a m=0\)
  \(a\neq 0\), \(m\neq 0\).
\end{obs}

Ejemplo en el \(\Z\)-módulo \(\Z_4\) tenemos que
\(2\cdot \bar{2}=\bar{0}\).
