\begin{obs}
  \(R\), \(S\) anillos. Si es \(\{e_1,\ldots, e_t\}\) CCIO centrales
  indescomponibles y \(phi:R\longrightarrow S\) es un isomorfismo de anillos.
  Entonces \(\{\phi(e_1),\ldots,\phi(e_t)\}\) es el CCIO centrales
  indescomponibles de \(S\). Además se tiene
  \[
    R=Re_1\dot{+}\cdots\dot{+}Re_t
  \]
  entonces
  \[
    S=S\phi(e_1)\dot{+}\cdots\dot{+}S\phi(e_t)
  \]
  donde \(Re_i\cong S\phi(e_i)\) como anillos.
\end{obs}

Imaginemos que sabemos que \(R\) es semisimple y que disponemos de un
isomorfismo de anillos \(R\cong R_1\times\cdots\times R_s\) con \(R_i\)
indescomponibles.

Por otra parte, \(\Omega_R=\{\Sigma_1,\ldots,\Sigma_t\}\), tengo un isomorfismo
de anillos \(R\cong
\End_{D_1}(\Sigma_1)\times\cdots\times\End_{D_t}(\Sigma_t)\), donde
\(D_i=\End_R(\Sigma_i)\).

Se deduce de la observación:
\[
\End_{D_1}(\Sigma_1)\times\cdots\times\End_{D_t}(\Sigma_t)
\cong
R_1\times\cdots\times R_s
\]
sin más que componer isomorfismos. En primer lugar, \(s=t\) y tras
reordenación \(\End_{D_i}(\Sigma_i)\cong R_i\).
Conocemos los \(e_i\) CCIO centrales
del segundo factor.

\begin{teo}[Teorema de Artin-Wedderburn]
  Si \(R\cong\End_{D_1}(\Sigma_1)\times\cdots\End_{D_t}(\Sigma_t)
  \cong\End_{E_1}(T_1)\times\cdots\End_{E_s}(T_s)\)
  donde \(D_i\), \(E_i\) son todos anillos de división y \(\Sigma_i\),
  \(T_i\) de dimensión finita como espacios vectoriales.

  Entonces \(s=t\) y tras reordenación \(D_i\cong E_i\) y
  \(\dim_{D_i}(\Sigma_i)=\dim_{T_i}(T_i)\).
\end{teo}

La demostración consiste en juntar observaciones, comentarios
y teoremas anteriores.

\begin{df}[Anillo opuesto]
Sea \(R\) un anillo. Mantengo su estructura de grupo aditivo, pero cambiamos
el producto. El nuevo producto va a ser el producto opuesto dado por
\(r*s:=sr\).

  A \(R\) con este nuevo producto lo vamos a llamar \(R^{op}\), el anillo
  opuesto.
\end{df}

Ejemplo: \[
  R=
  \begin{pmatrix}
    \Z&\Q\\
    0&\Q
  \end{pmatrix}
  \le \mathcal{M}_2(\Q)
\]

Tenemos que \(\subscriptbefore{R}{R}\) no es noetheriano, pero
\(\subscriptbefore{R^{op}}{R^{op}}\) sí que lo es.
Es decir, es noetheriano a derecha pero no a izquierda.

\begin{df}[Anillo noetheriano a derecha]
  Un anillo es noetheriano a derecha si el anillo opuesto es noetheriano
  a izquierda.
\end{df}
\begin{df}[Módulo a derechas]
  Definimos \(M\) módulo a derechas como \(M_R:=\subscriptbefore{R^{op}}{M}\).
\end{df}

\begin{df}[Ideal bilátero]
  Un ideal bilátero es un ideal a izquierda que es ideal a derecha también.
\end{df}

\begin{df}[Dual de un módulo]
  Sea \(\subscriptbefore{R}{M}\) un módulo. Tomamos
  \[
    \superscriptbefore{*}{M}:=\{f:M\longrightarrow R: f
    \textrm{ es homomorfismo de \(R\)-módulos}\}
  \]
  que es un grupo aditivo y un módulo a derechas, es decir,
  \(R^{op}\)-módulo, por la acción:
  \[
    (r\varphi)(m):=\varphi(m)r
  \]
  con \(r\in R\), \(m\in M\) y \(\varphi\in \superscriptbefore{*}{M}\).
  Es decir, \(\superscriptbefore{*}{M}_R\).
\end{df}

\begin{lema}
  \(\theta:{(\End_R(M))}^{op}
  \longrightarrow\End_{R^{op}}(\superscrpitbefore{*}{M})\)
  tenemos que \(\theta(f)(\varphi):=\varphi\circ f\) es un homomorfismo
  de anillos.
  Nota: el producto en \({(\End_R(M))}^{op}\) es \(f*g=g\circ f\).
\end{lema}

Si \(M=\Z_n\) como \(Z\) módulos, \(\superscriptbefore{*}{M}=\{0\}\) así que
nos olvidamos de cualquier idea de reflexividad o isomorfismo.

\begin{df}[Módulos reflexivos]
  Un módulo en el que la anterior \(\theta\) es un isomorfismo.
\end{df}

\begin{teo}
  \(R\) es semisimple si y solo si \(R^{op}\) es semisimple.
\end{teo}
\begin{proof}
  Hay que fabricar alguna herramienta adicional.

\end{proof}


