\section{Descomposición primaria sobre un DIP}

\begin{df}[Módulo acotado sobre un DIP]
  Sea \(A\) un dominio de ideales principales, \(\subscriptbefore{A}{M}\)
  un módulo. Recordemos que \(\Ann_A(M)=\langle\mu\rangle\) para cierto \(\mu\in A\).

  Si \(\mu\neq 0\), \(M\) se dice \textbf{acotado}.
\end{df}

Supongamos que \(\subscriptbefore{A}{M}\) es acotado y \(\mu\not\in
\mathcal{U}(A)\) (unidad de \(A\)). Supongamos \(\mu\) lo fuese. Sea \(m \in M\). Entonces
\(m = 1 \cdots m = \mu \mu^{-1}m = 0\). Esto implicaría que \(M = \{0\}\).

Así \(\mu= p_1^{e_1}\cdots p_t^{e_t}\), con \(p_i\in A\) irreducible y \(e_i \in \N\backslash
\{0\}\) para todo \(i = 1, \ldots, t\), porque todo DIP es un dominio de factorización
única (DFU).


\begin{df}[Componentes primarias]
  Sean \(A\) un DIP, \({}_AM\) un módulo y \(M_1, \ldots, M_t\) submódulos de \({}_AM\).
  Si \(M = M_1 \dotplus \ldots \dotplus M_t\), esta expresión es la \textbf{descomposición
    primaria} de \(M\).
\end{df}

\begin{prop}[Descomposición primaria del módulo]
  Sean \(A\) un DIP, \({}_AM\) un módulo donde \(\Ann_A(M) = \langle\mu\rangle\), con
  \(\mu = p_1^{e_1}\cdots p_t^{e_t}\) y \(M_i = \{q_im \ : \ m \in M\}\), donde
  \(q_i = \frac{\mu}{p_i^{e_i}} \in A\) y \(p_1^{e_1}\cdots p_t^{e_t}\), con \(p_i\in A\)
  irreducible y \(e_i \in \N\backslash\{0\}\) para todo \(i = 1, \ldots, t\).
  Entonces
  \(M = M_1 \dotplus \ldots \dotplus M_t\) es la descomposición primaria de \(M\). A los
  \(M_i\) se le llama \textbf{componente \(p_i\)-primaria de \(M\)}.
\end{prop}
\begin{proof}
  Veamos que \(M_i\in\mathcal{L}(\subscriptbefore{A}{M})=\{\textrm{submódulos
    de } \subscriptbefore{A}{M}\}\).

  Queremos que \(M=M_1\dot{+}\cdots \dot{+}M_t\), con \(t>1\) para
  evitar trivialidades. En ese caso, \(\mcd\{q_1,\ldots,q_t\} = 1\),
  donde se ha usado que estamos en un DFU.

  Por la identidad de Bezout, que es válida porque estamos en un DIP,
  tenemos que \(1=\sum_{i=1}^t q_i a_i\). Dado cualquier \(m \in M\),
  \(m = 1 \cdot m = \sum_i q_i a_i m\), luego \(M = M_1+ \cdots + M_t\).

  Vamos a ver que la suma es directa.
  Sean \(m_1, \ldots, m_t \in M\) tales que \(0 = m_1 + \ldots + m_t\). Observamos que
  \(q_im_j = 0\) si \(i \neq j\), ya que \(m_j = q_jm\), con \(m \in M\) y
  \(q_iq_j \in \langle \mu \rangle = \Ann_A(M)\). Entonces obtenemos que
  \[
    0 = m_1 + \ldots + m_t = q_j(m_1 + \ldots + m_t) = q_jm_j
  \]
  para todo \(j = 1, \ldots, t\). Ahora bien, como cualquier \(m_j \in M\), entonces
  \[
    m_j = \sum_{i=1}^{t}a_iq_im_j = a_jq_jm_j = a_j0 = 0
  \]
  
  Por tanto, por la proposición \ref{prop:suma-interna}, \(M=M_1\dotplus \cdots \dotplus
  M_t\).

  De hecho, \(M_i = \{m \in M \ | \ m = a_iq_im\} = \{m \in M \ | \ p_i^{e_i}m = 0\}\). La
  inclusión hacia la izquierda se prueba con Bezout.
  Además, \(\langle \mu \rangle = \Ann_A(M) = \bigcap_{i=1}^{t}\Ann_A(M_i)
  \supseteq \bigcap_{i=1}^{t} \langle p_i^{e_i}\rangle = \langle \mu \rangle\),
  luego \(\Ann_A(M_i) = \langle p_i^{e_i} \rangle\) para todo \(i=1, \ldots, t\).
\end{proof}

\begin{ejemplo}
  Sean \( M = \Z_{60}\) y \(A = \Z\). Entonces \(\Ann_{\Z}(\Z_{60}) = 60\Z\),
  luego \(\nu = 2^2 \cdot 3 \cdot 5\) y \(Z_{60} = \mu_2 \dotplus
  \mu_3 \dotplus \mu_5\).
  \begin{itemize}
  \item \(\mu_2, \ q_2 = 15 \implies \mu_2 = \{15m \ | \ m \in \Z_{60}\} =
    \{0, 15, 30, 45\}\).
  \item \(\mu_3, \ q_3 = 20 \implies \mu_3 = \{20m \ | \ m \in \Z_{60}\} = \{0, 20, 40\}\).
  \item \(\mu_5, \ q_5 = 12 \implies \mu_5 = \{12m \ | \ m \in \Z_{60}\} = \{0, 12, 24, 36,
    48\}\).
  \end{itemize}
\end{ejemplo}

\begin{ejemplo}
  Sea \(M = \Z_2 \times \Z_2\). Su anulador es \(\Ann_{\Z}(M) = 2\Z\). \(M\) es 2-primario
  pero \(M = (\Z_2 \times \{0\}) \dotplus (\{0\} \times \Z_2)\).
\end{ejemplo}

\begin{ejercicio}
  Obtener la descomposición primaria de \(\Z_{8000}\).
\end{ejercicio}

\begin{ejemplo}
  Sea \(T\) endomorfismo \(K\)-lineal de un espacio vectorial \(V\) de dimensión finita.

  Un \(W\) es un submódulo de \(V\) es un subespacio vectorial tal que
  \(T(W)\subseteq W\), es decir, \(W\) es \(T\) invariante.

  Por el ejercicio \ref{ejer:anuladorV}, \(\Ann_{K[x]}(V)\neq\{0\}\),
  tomo \(\mu(x)\in K[x]\), el polinomio
  mínimo de \(T\). Es decir, \(\Ann_{K[x]}(V)=\langle\mu(x)\rangle\). Como \(K[x]\) es
  un DIP, podemos expresar el polinomio \(\mu\) como producto de potencias de polinomios
  irreducibles:

  \[
    \mu=p_1^{e_1}\cdots p_t^{e_t}
  \]

  Entonces la descomposición primaria de \(V\) es \(V=V_1\dot{+}
  \cdots\dot{+}V_t\), con
  \[
    V_i=\{v\in V:p_i(x)v=0\}
  \]

  Estudemos el siguiente caso particular: \(\dim(V)<\infty\) y que
  \(\mu(x)=(x-\alpha_1) \cdots (x-\alpha_t)\) con \(\alpha_i\neq\alpha_j\).
  \[
    V_i=\{v\in V:(x-\alpha_i)v=\{v\in V:T(v)=\alpha_i v\}
  \]
  es decir, el subespacio propio asociado al valor propio \(\alpha_i\).

  Si el polinomio factoriza como producto de polinomios de grado 1 distintos,
  \(T\) es diagonalizable.
  Veremos en el futuro que el polinomio mínimo divide siempre al polinomio
  característico mediante el Teorema de Cayley-Hamilton.
\end{ejemplo}

Después de ver varios ejemplos de espacios vectoriales y sus anuladores, es natural
pensar en la siguiente pregunta: ¿cómo se calcula el polinomio mínimo de un
endomorfismo lineal entre espacios vectoriales? Precisamente, el Teorema de
Cayley-Hamilton nos ayudará a responder a dicha pregunta.

\begin{ejercicio}
  Sea \(V\) un espacio vectorial real euclídeo con producto
  escalar \(\langle -, - \rangle\). Sea \(T: V \longrightarrow V\) una isometría.
  Se pide demostrar que si \(W\) es un subespacio \(T\) invariante de \(V\),
  entonces su ortogonal \(W^\perp\) es también \(T\)-invariante.
  Entonces \(V=W\dot{+}W^\perp\). Se usa inducción. Como consecuencia, usando
  el teorema fundamental del álgebra, deducir que \(V\) admite una base
  ortonormal con respecto de la cual la matriz de \(T\) es diagonal por
  bloques, con bloques de dimensión 1 o 2. ¿Qué aspecto tienen dichos
  bloques? Hay que ver que uno de los dos subespacios invariantes tienen
  dimensión 1 o 2.
\end{ejercicio}
