\section{Descomposición primaria sobre un DIP}

\begin{df}[Módulo acotado sobre un DIP]
  Sea \(A\) un dominio de ideales principales, \(\subscriptbefore{A}{M}\)
  un módulo. Recordemos que \(\Ann_A(M)=\langle\mu\rangle\) para cierto \(\mu\in A\).

  Si \(\mu\neq 0\), \(M\) se dice \textbf{acotado}.
\end{df}

Supongamos que \(\subscriptbefore{A}{M}\) es acotado y \(\mu\not\in
\mathcal{U}(A)\) (unidad de \(A\)). Supongamos \(\mu\) lo fuese. Sea \(m \in M\). Entonces
\(m = 1 \cdots m = \mu \mu^{-1}m = 0\). Esto implicaría que \(M = \{0\}\).

Así \(\mu= p_1^{e_1}\cdots p_t^{e_t}\), con \(p_i\in A\) irreducible y \(e_i \in \N\backslash
\{0\}\) para todo \(i = 1, \ldots, t\), porque todo DIP es un dominio de factorización
única (DFU).


\begin{df}[Componentes primarias]
  Sean \(A\) un DIP, \({}_AM\) un módulo y \(M_1, \ldots, M_t\) submódulos de \({}_AM\).
  Si \(M = M_1 \dotplus \ldots \dotplus M_t\), esta expresión es la \textbf{descomposición
    primaria} de \(M\).
\end{df}

\begin{prop}[Descomposición primaria del módulo]
  Sean \(A\) un DIP, \({}_AM\) un módulo donde \(\Ann_A(M) = \langle\mu\rangle\), con
  \(\mu = p_1^{e_1}\cdots p_t^{e_t}\) y \(M_i = \{q_im \ : \ m \in M\}\), donde
  \(q_i = \frac{\mu}{p_i^{e_i}} \in A\) y \(p_1^{e_1}\cdots p_t^{e_t}\), con \(p_i\in A\)
  irreducible y \(e_i \in \N\backslash\{0\}\) para todo \(i = 1, \ldots, t\).
  Entonces
  \(M = M_1 \dotplus \ldots \dotplus M_t\) es la descomposición primaria de \(M\). A los
  \(M_i\) se le llama \textbf{componente \(p_i\)-primaria de \(M\)}.
\end{prop}
\begin{proof}
  Veamos que \(M_i\in\mathcal{L}(\subscriptbefore{A}{M})=\{\textrm{submódulos
    de } \subscriptbefore{A}{M}\}\).

  Queremos que \(M=M_1\dot{+}\cdots \dot{+}M_t\), con \(t>1\) para
  evitar trivialidades. En ese caso, \(\mcd\{q_1,\ldots,q_t\}=1\),
  donde se ha usado que estamos en un DFU.

  Por la identidad de Bezout (válida porque estamos en un DIP),
  tenemos que \(1=\sum_{i=1}^t q_i a_i\), para ciertos \(q_i\in A\).
  Para en \(m\in M\), \(m = 1 \cdot m = \sum_i q_i a_i m\), luego
  \(M=M_1+\cdots+ M_t\).

  Vamos a ver que la suma es directa.
  Se cumple que \(q_i q_j\in\langle\mu\rangle\) si \(i\neq j\). Eso significa que si
  \(m\in M_i\), entonces \(q_j m= 0\) si \(i\neq j\).
  En consecuencia, podemos escribir \(M_i\) como \(M_i=\{m\in N: m=q_i a_i m\}\).

  Si \(0=\sum_{i=1}^t\) con \(m_i\in M_i\), entonces
  \[
    0=q_j a_j 0=m_j
  \]
  y, por tanto, \(M=M_1\dot{+}\cdots\dot{+} M_t\).

  Además, \(\langle \mu \rangle = \Ann_A(M) = \bigcap_{i=1}^{t}\Ann_A(M_i)\),
  luego \(\Ann_A(M_i) = \langle p_i^{e_i} \rangle\) para todo \(i=1, \ldots, t\).
\end{proof}

\begin{prop}
  \[
    M_i=\{m\in M: p_i^{e_i} m=0\}
  \]

  Así, \(\langle\mu\rangle=\Ann_A(M)=\bigcap_{i=1}^t\Ann_A(M_i)
  \supseteq\bigcap_{i=1}^t\langle p_i^{e_i}\rangle=\langle\mu\rangle\)
\end{prop}

Ejercicio: Obtener la descomposición primaria usando \(\dot{+}\) de
\(\Z_{8000}\).
