%%%%%%%%%%%%%%%%%%%%%%%%%%%%%%%%%%%%%%%%%%%%%%%%%%%%%%%%%%%%%%%%%%%%%%%%%%%%%%%%
%
% Plantilla para libro de texto de matemáticas.
%
% Esta plantilla ha sido desarrollada desde cero para el proyecto apuntesDGIIM
% de LibreIM.
%
%	Copyright (C) 2019 LibreIM
%
% This program is free software: you can redistribute it and/or modify
% it under the terms of the GNU Affero General Public License as
% published by the Free Software Foundation, either version 3 of the
% License, or (at your option) any later version.
% This program is distributed in the hope that it will be useful,
% but WITHOUT ANY WARRANTY; without even the implied warranty of
% MERCHANTABILITY or FITNESS FOR A PARTICULAR PURPOSE.  See the
% GNU Affero General Public License for more details.
% You should have received a copy of the GNU Affero General Public License
% along with this program.  If not, see <https://www.gnu.org/licenses/>.
%
%%%%%%%%%%%%%%%%%%%%%%%%%%%%%%%%%%%%%%%%%%%%%%%%%%%%%%%%%%%%%%%%%%%%%%%%%%%%%%%%

% ------------------------------------------------------------------------------
% CONFIGURACIÓN BÁSICA DEL DOCUMENTO
% ------------------------------------------------------------------------------

\documentclass[fontsize=12pt]{scrartcl}
\usepackage[spanish, es-tabla, es-lcroman, es-noquoting, es-minimal]{babel}
\usepackage{graphicx}

\usepackage{templatetools}

% ------------------------------------------------------------------------------
% CONFIGURACIÓN DE ASIGNATURA
% ------------------------------------------------------------------------------

\usepackage{config}

% ------------------------------------------------------------------------------
% TIPOGRAFÍA
% ------------------------------------------------------------------------------

% Tipografía personalizada
\usepackage[bitstream-charter]{mathdesign}
\usepackage[scale=0.88, type1]{FiraSans}
\usepackage[scale=0.88, type1]{FiraMono}
\usepackage[T1]{fontenc}

% Microtype
\usepackage[activate={true, nocompatibility}, final, tracking=true,factor=1100, stretch=10, shrink=10]{microtype}
\SetTracking{encoding={*}, shape=sc}{0}

% ------------------------------------------------------------------------------
% DISEÑO DE PÁGINA
% ------------------------------------------------------------------------------

% Márgenes
\usepackage[bottom=3.125cm, top=2.5cm, left=2.5cm, right=4.5cm, marginparwidth=70pt]{geometry}

% Párrafos
\linespread{1.1}
\setlength{\parindent}{12pt}
\setlength{\parskip}{0pt}

% Listas
\setlist{noitemsep}

% Cabeceras
\usepackage[automark]{scrlayer-scrpage}
\clearpairofpagestyles
\chead{\leftmark}
\cfoot*{\pagemark}

% Enlaces
\usepackage{hyperref}
\hypersetup{
    colorlinks=true,
    linkcolor=500,
}

% No mostrar cabecera en páginas de sección
\usepackage{xpatch}
\xpretocmd{\section}{\vspace*{1cm}\thispagestyle{plain}}{}{}

% Páginas de color para portada
\usepackage{pagecolor}
\usepackage{afterpage}

% ------------------------------------------------------------------------------
% ENTORNOS PERSONALIZADOS
% ------------------------------------------------------------------------------

\usepackage[theorems, skins, breakable]{tcolorbox}

\tcolorboxenvironment{teo}{
    blanker,
    breakable,
    left=12pt,
    before skip=12pt,
    after skip=12pt,
    borderline west={2pt}{0pt}{500},
    before upper={\parindent 12pt},
}

\tcolorboxenvironment{prop}{
    blanker,
    breakable,
    left=12pt,
    before skip=12pt,
    after skip=12pt,
    borderline west={2pt}{0pt}{300},
    before upper={\parindent 12pt},
}

\tcolorboxenvironment{cor}{
    blanker,
    breakable,
    left=12pt,
    before skip=12pt,
    after skip=12pt,
    borderline west={2pt}{0pt}{300},
    before upper={\parindent 12pt},
}

\tcolorboxenvironment{df}{
    skin=enhancedmiddle jigsaw,
    frame hidden,
    colback=50,
    breakable = true,
    break at = -6pt,
    top = 4pt,       % Estos márgenes están un poco a ojo
    bottom = 4pt,
    left= 8pt,
    right = 8pt,
    before skip=8pt, % Normalmente dejamos 12pt, pero
    after skip=8pt,  % aquí tenemos espacio adicional por el fondo
    no borderline,
    borderline west={2pt}{0pt}{500},
    borderline east={2pt}{0pt}{50},
    before upper={\parindent 12pt},
}

\tcolorboxenvironment{ejer}{
    skin=enhancedmiddle jigsaw,
    frame hidden,
    colback=50,
    breakable = true,
    break at = -6pt,
    top = 4pt,       % Estos márgenes están un poco a ojo
    bottom = 4pt,
    left= 8pt,
    right = 8pt,
    before skip=8pt, % Normalmente dejamos 12pt, pero
    after skip=8pt,  % aquí tenemos espacio adicional por el fondo
    no borderline,
    borderline west={2pt}{0pt}{500},
    borderline east={2pt}{0pt}{50},
    before upper={\parindent 12pt},
}

% ------------------------------------------------------------------------------
% LISTINGS
% ------------------------------------------------------------------------------

\IfPackageLoaded{listings}{
    \renewcommand{\lstlistingname}{Listado}
    
    \lstset{
        frame=lines,
        rulecolor=\color{black},
        framerule=1pt,
        numbers=left,
        belowcaptionskip=1\baselineskip,
        basicstyle=\ttfamily\color{black},
        keywordstyle=\bfseries\color{700},
        commentstyle=\color{300},
        stringstyle=\color{500},
        numberstyle=\color{black},
        breaklines=true,
        showstringspaces=false,
        tabsize=2,
    }
}

\begin{document}
    
    % ---------------------------------------------------------------------------
    % PORTADA EXTERIOR
    % ---------------------------------------------------------------------------
    
    \newpagecolor{500}
    \begin{titlepage}
        \noindent
        \setlength\fboxsep{0cm}
        \parbox[t]{\textwidth}{
            \raggedright
            \fontsize{50pt}{50pt}\selectfont\sffamily\color{white}{
                \textbf{\asignatura}
            }
        }
        
        \vfill
        
        \noindent
        \parbox[b]{\textwidth}{
            \raggedright
            \sffamily\large\color{white}
            {\Large \autor }\\[4pt]
            \grado\\
            \universidad\\[4pt]
            \texttt{\enlaceweb}
        }
        
    \end{titlepage}
    \restorepagecolor
    
    % ---------------------------------------------------------------------------
    % PORTADA INTERIOR
    % ---------------------------------------------------------------------------
    
    \begin{titlepage}
        
        \noindent
        \parbox[t]{\textwidth}{
            \raggedright
            \fontsize{50pt}{50pt}\selectfont\sffamily\color{500}{
                \textbf{\asignatura}
            }
        }
        
        \vfill
        
        \noindent
        \parbox[b]{\textwidth}{
            \raggedright
            \sffamily\large
            {\Large \autor}\\[4pt]
            \grado\\
            \universidad\\[4pt]
            \texttt{\enlaceweb}
        }
        
    \end{titlepage}
    
    % ---------------------------------------------------------------------------
    % ÍNDICE
    % ---------------------------------------------------------------------------
    
    \thispagestyle{empty}
    {\hypersetup{linkcolor=black}\tableofcontents}
    \newpage
    
    % ---------------------------------------------------------------------------
    % CONTENIDO
    % ---------------------------------------------------------------------------

    \section{Introducción}
\subsection{Generalidades sobre anillos}
\begin{df}[Anillo]
  Sea \(A\) un conjunto en el que existen dos operaciones
  \(+,\cdot:A\times A\longrightarrow A\) tales que:
  \begin{enumerate}
    \item \((A, +,0)\) es un grupo aditivo (conmutativo):
      \begin{itemize}
        \item \((a+b)+c=a+(b+c)\) para todos \(a,b,c\in A\).
        \item \(a+b=b+a\) para todos \(a,b\in A\).
        \item \(a+0=a\) para todo \(a\in A\).
        \item Para todo \(a\in A\) existe un \(-a\in A\)
          tal que \(-a+a=0\).
      \end{itemize}
    \item \((A, \cdot, 1)\) es un monoide:
      \begin{itemize}
        \item \((ab)c=a(bc)\) para todos \(a,b,c\in A\).
        \item \(a\cdot 1=1\cdot a=a\) para todo \(a\in A\).
      \end{itemize}
    \item Se cumplen las siguientes propiedades distributivas:
      \begin{itemize}
        \item \((a+b)c=ac+bc\) para todos \(a,b,c\in A\).
        \item \(a(b+c)=ab+ac\) para todos \(a,b,c\in A\).
      \end{itemize}
     \end{enumerate}
  \end{df}

  \begin{df}[Ideales]
    Sea \(A\) un anillo. \(I\subset A\) se dice ideal si cumple las
    siguientes propiedades:
    \begin{itemize}
      \item \(I\) es un subgrupo aditivo de \(A\) (es decir,
        \(I\) es un conjunto no vacío que cumple
        \(a + b\in I\) para todo \(a, b\in I\)).
      \item \(ax, xa\in I\) para todo \(a\in I\) y \(x \in A\).
    \end{itemize}
  \end{df}

  \begin{obs}
      La primera condición para que un subconjunto de A sea un ideal suyo
      es equivalente a que \(b - a \in I\) para todo \(a, b \in I\). 
      Recordemos que, dado cualquier anillo \(A\), \(0a = (0+0)a = 0a + 0a
      \implies 0a = 0, \forall a \in A\).
      
      Esta propiedad, aunque evidentemente intuitiva, no viene explícita
      en la definición de anillo. Ahora bien, comprobemos que la propiedad
      de ideal anteriormente descrita se cumple.
      
      Sean \(a, b \in I\). \(b - a = b + (-1)a \in I\), pues \((-1) \in A\)
      y \(a \in I \implies (-1)a \in I)\) por la segunda propiedad de un
      ideal. 
  \end{obs}

  \begin{teo}[Teorema de Isomorfía]
    Sea \(f:A\longrightarrow B\) un homomorfismo de anillos. Entonces:
    \begin{itemize}
      \item \(\ker f\) es un ideal de \(A\),
      \item \(\Im f\) es un subanillo de \(B\),
      \item Si \(I\subset \ker f\) es un ideal de \(A\), entonces
        existe un único homomorfismo de anillos tal que
        \(\tilde{f}:A/I\longrightarrow B\) tal que \(\tilde{f}(a+I)=f(a)\).
      \item El homomorfismo anterior es inyectivo si y solo si
        \(I=\ker f\).
      \item El homomorfismo anterior es sobreyectivo si y solo si lo era
        \(f\).
    \end{itemize}
  \end{teo}


    \begin{df}[Homomorfismo de anillos]
  \(A,B\) anillos. Se dice que \(f:A\longrightarrow B\) se dice un
  (homo)morfismo de anillos si para todos \(a,a'\in A\) se tiene:
  \begin{enumerate}
    \item \(f(a+a')=f(a)+f(a')\)
    \item \(f(aa')=f(a)f(a')\)
    \item \(f(1)=1\)
  \end{enumerate}
\end{df}

\begin{df}[Producto de ideales]
  Sean \(I, J\) ideales. Definimos su producto por:
  \[
    IJ=\{\sum_i x_i y_i: x_i\in I, y_i\in J\}\subseteq I\cap J
  \]
\end{df}


\noindent Recordemos que tanto la suma como el producto de dos ideales de un anillo es un ideal del mismo anillo.

\begin{df}[Ideales coprimos]
  Dos ideales \(I, J\subset A\) se dirán primos entre sí
  o coprimos si \(I+J=A\). Equivalentemente, existen \(x\in I\), \(y\in J\) tales que
  \(1=x+y\).
\end{df}

\noindent La motivación de la definición anterior reside en la identidad
de Bezout, que estamos generalizando.

\begin{lema}
  Sean \(I, J, K\) ideales de \(A\),
  \(I+J=I+K=A\) si, y solo si, \(I+(J\cap K)=I+J\cap K=A\).

  Es decir, son coprimos entre sí si, y solo si, uno es coprimo
  con la intersección de los otros dos.
\end{lema}
\begin{proof}
  \[1=x+y=x'+z\]
  con \(x,x'\in I\), \(y\in J\), \(z\in K\).
  \[1=x+y=x+y1=x+y(x'+z)=x+yx'+yz\]
  \(x+yx'\in I\), y \(yz\in J\cap K\).

  Para el recíproco, \(A\supseteq I+J\supseteq I+J\cap K=A\),
  luego \(A=I+J\).
\end{proof}

\begin{lema}
  Sean \(I_1,\ldots, I_t\) ideales de \(A\).
  \(I_1\cap I_i=A\) si y solo si
  \(I_1+\bigcap_{i=2}^t I_i=A\).
\end{lema}
\begin{proof}
  Para \(t=2\) es trivial.

  Supongamos cierto \(I_1\cap I_i=A\implies
  I_1+\bigcap_{i=2}^t I_i=A\) para \(t\), veamos para \(t+1\).

  Llamo \(I=I\), \(J=\bigcap_{i=2}^t I_i\), \(K=I_{t+1}\).
  Por hipótesis de inducción \(I+J=A\) y \(I+K=A\) por ser coprimos
  (hipótesis del lema). Por el lema anterior tenemos:
  \[
    I+J\cap K=I_1+I_{t+1}\cap \bigcap_{i=2}^{t}
    I_i =I_1 + \bigcap_{i=2}^{t+1} I_i
  \]

  La otra implicación es muy sencilla.
\end{proof}

  Hipótesis de trabajo para el teorema chino del resto:
  \begin{enumerate}
    \item \(A\) un anillo.
    \item \(A_1,\ldots,A_t\) anillos, con \(t \ge 2\).
    \item \(f_i:A\longrightarrow A_i\) un homomorfismo de anillos
      para cada \(i\in\{1,\ldots,t\}\).
    \item \(\Im f_i\subseteq A_i\) es un subanillo.
    \item A \(\Im f_1\times\cdots\times\Im f_t\) se le llama el anillo
      producto.
    \item Definimos \(f:A\longrightarrow\Im f_1\times\cdots\times\Im f_t\),
      \(f(x)=(f_1(x),\ldots,f_t(x))\) para cada \(x\in A\).
    \item Tenemos que \(f\) es un homomorfismo de anillos, cuyo núcleo es la
      intersección de todos los núcleos.
      En efecto, dado \(x\in A\), \(x\in\ker f\) si, y solo si, \(f_i(x)=0\) para todo \(i\),
      es decir, \(x\in\bigcap_{i=0}^t \ker f_i\). Llamaremos \(I=\ker f\).
    \item  Además, por el primer teorema de isomorfía, existe un homomorfismo
    \(\tilde{f}:A/I \longrightarrow\Im f_1\times\cdots\times\Im f_t\),
      con \(x+I\mapsto f(x)\). Por construcción, \(\tilde{f}\) es inyectiva, 
      pero no sobreyectiva. El Teorema Chino del Resto se basa en demostrar
      que \(\tilde{f}\) es sobreyectiva bajo ciertas condiciones.
    \item Cada \(\ker f_i\) es coprimo con cualquier \(\ker f_j\)
      para \(j\neq i\).
    \item Llamamos \(I_i=\ker f_i\).
  \end{enumerate}

\begin{teo}[Teorema Chino del Resto]
  \(\tilde{f}\) es isomorfismo si y solo si
  \(I_i+I_j=A\) para todo \(i\neq j\).
\end{teo}
\begin{proof}
  Probemos primero la implicación a la derecha. Vamos a suponer \(\tilde{f}\) sobreyectiva, es decir, que \(f\) lo es, 
  pues \(\tilde{f}(a + \ker{f}) = f(a)\).
  
  Veamos que todos los \(I_i\) son coprimos entre sí.

  Dado \(i\), tomamos \(x\in A\) tal que \(f_i(x)=1\) y
  \(f_j(x)=0\) para todo \(j\neq i\).

  Observemos que \(1-x\in I_i\), ya que \(f_i(1-x) = f_i(1) - f_i(x) = 
  1 - 1 = 0\), y que \(x\in\bigcap_{j\neq i} I_j\).
  \[
    1=1-x+x \in I_i+\bigcap_{j\neq i} I_j
  \]

  Por tanto, \(I_i+\bigcap I_j =A\) y entonces por el lema anterior
  \(I_i+I_j=A\).
  
  Veamos el recíproco.
  Suponemos que \(I_i+I_j=A\) para cualquier \(i\neq j\). Por el lema
  anterior, \(\forall i \in \{1, \cdots, t\} \ I_i + \bigcap_{i \neq j} I_j = A\).

  Tomamos \((f(b_1),\ldots,f(b_t))\in I_1\times\cdots\times I_t\).

  Para cada \(i\), tomamos \(a_i\in I_i\) y \(p_i\in\bigcap I_j\) tales que
  \(1=a_i+p_i\) y sea \(x=\sum_{i=1}^t b_i p_i\). Entonces

  \[
    f_j(x)=\sum_{k=1}^t f_j(b_k) f_j(p_k)=f_j(b_j)f_j(p_j)=f_j(b_j(1-a_j))
    =f_j(b_j)-f_j(b_j)f_j(a_j)=f_j(b_j)
  \]
  porque \(f_j(p_k)=0\) si \(k\neq j\) y \(a_j\in\ker f_j\).

\end{proof}

    
\begin{obs}
  Para anillos conmutativos denotamos
  \[
    \langle a\rangle=\{ba:b\in A\}
  \]
  el ideal generado por \(a\).
\end{obs}

Vamos a hacer un ejemplo, aplicando el teorema anterior.

\subsubsection{Interpolación}         

Tomamos \(A=K[x]\), un anillo de polinomios con coeficientes en un
cuerpo \(K\).

Sea \(A_i = K\) con \(i\in\{1,\ldots,t\}\).
Tomamos \(\alpha_i\in K\) para cada \(i\) y definimos
\(\xi_i:K[x]\longrightarrow K\), \(\xi_i(g)=g(\alpha_i)\),
para cada \(g\in K[x]\) y es un homomorfismo de anillos.

\(\Im \xi_i=K\) y \(\xi:K[x]\longrightarrow K\times
\cdots \times K=K^t\).

\(\ker \xi_i=\langle x-\alpha_i\rangle\) que es ideal de un anillo de
polinomios, luego principal. Está generado por el polinomio de grado menor,
como las constantes no pueden anular a \(\xi_i\), tiene que estar generado
por ese, que es de grado uno.

\[
  I=\bigcap_{i=1}^t\langle x-\alpha_i\rangle =\langle p(x)\rangle
\]
donde \(p(x)=\mcm\{x-\alpha_i: i\in\{1,\ldots, t\}\}\).


El teorema chino del resto nos asegura que \(\tilde{\xi}:
K[x]/\langle p(x)\rangle\longrightarrow K^t\) es un isomorfismo si y solo si
\(\mcd\{x-\alpha_i, x-\alpha_j\} = 1\) para todo \(j\neq i\), es decir,
si \(\alpha_i\neq \alpha_j\).

Lo que estamos viendo es que para cualquier tupla
\((y_1,\ldots, y_t)\in K^t\), existe un \(g\in K[x]\) tal que
\(g(\alpha_i)=y_i\), si y solo si \(\alpha_i\neq\alpha_j\). En tal
caso, \(p(x)=\prod_{i=1}^t(x-\alpha_i)\).

Existe un único representante \(g\in K[x]\) tal que \(g(\alpha_i)=y_i\)
de grado menor que \(t\), siempre que
\(p(x)=\prod_{i=1}^t(x-\alpha_i)\).

\(\alpha_1,\ldots,\alpha_t\in K\) distintos dos a dos
\[
  \tilde{\xi}:K[x]/\langle p(x)\rangle\longrightarrow K^t
\]
es un isomorfismo de anillos.

\(K[x]/\langle p(x)\rangle\) es un espacio vectorial cociente.

\(\tilde{\xi}\) es también un isomorfismo entre espacios vectoriales.

\[
  \tilde{\xi}(\alpha(g+p))=\tilde{\xi}(\alpha g+p)=\tilde{\xi}((\alpha
  + p)(g +p))=\]\[\tilde{\xi}(\alpha + p)\tilde{\xi}(g+p)=
  (\alpha,\ldots, \alpha)(g(\alpha_1),\ldots g(\alpha_t))=
  \alpha\tilde{\xi}(g+p)
\]

Sea \(\{1+p, x+p, x^2+p,\ldots, x^{t-1} +p\}\)
\(K\)-base de \(K[x]/\langle p(x)\rangle\).
Notamos:
\[ 1 = 1+p\]
\[ x = x+p\]

Sea \(\{e_1,\ldots, e_n\}\) es la base canónica de \(K^t\).
Nuestro objetivo es calcular sus preimágenes por \(\xi\), en concreto
un polinomio de grado menor que \(t\).

\[
  g_i(x)=\prod_{j\neq i}(x-\alpha_j)
\]

\[
  L_i(x)=\frac{g_i(x)}{g(\alpha_i)}=\prod_{j\neq i}\frac{x-x_j}{x_i-x_j}
\]

que vale 0 en \(\alpha_j\) para cualquier \(j\)
salvo en \(\alpha_i\) que vale 1.

Tenemos que
\[
  g(x)=\sum_{i=1}^t y_i L_i(x)
\]
satisface que \(g(\alpha_i)=y_i\).


Finalmente vamos a ver que la matriz de \(\tilde{\xi}\) en las bases
consideradas es:
\[
  \begin{pmatrix}
    1&\cdots& 1\\
    \alpha_1 &\cdots&\alpha_t\\
    \cdots&\cdots&\cdots\\
    \alpha_1^t&\cdots&\alpha_t^t
  \end{pmatrix}
\]

    \subsubsection{Transformada discreta de Fourier}

Ahora vamos a reindexar. En lugar de usar \(1,\ldots, t\)
vamos a tomar los índices \(0,\ldots, n-1\).

Vamos a suponer que el cuerpo \(K\)
contiene una raíz primitiva de 1, o sea,
existe un \(\omega\in K\) tal que \(\omega^n = 1\) y
\(1, \omega, \omega^2,\ldots, \omega^{n-1}\) son distintos.

Seguro que \(\car K \not\| n\) ya que \(1, \omega, \omega^2,\ldots,
\omega^{n-1}\) son las raíces de \(x^n-1\) y son distintas.

Vamos a interpolar las raíces de la unidad.

Tomo \(\alpha_j =\omega^j\), \(j\in\{0,\ldots, n-1\}\) y
\[M=A_\omega=
\begin{pmatrix}
  1&1&\cdots1\\
  \omega^0&\omega^1&\cdots\omega^{n-1}\\
  {(\omega^0)}^2&{(\omega^1)}^2&\cdots{(\omega^{n-1})}^2\\
  \cdots&\cdots&\cdots
\end{pmatrix}= (\omega^{ij})
\]


Tenemos que \(x^n-1=(x-1)(x^{n-1}+\cdots+x+1)\)
y evaluando en \(\omega^{j}\) obtenemos
\[
  \omega^{(n-1)j}+\cdots+\omega^j+1=0
\]

Entonces \(\sum_{k=0}^{n-1}\omega^{ik}=0\) para \(0<i<n\).

\[
  \begin{pmatrix}
  	\omega^{0i}&
    \omega^{i}&
    \omega^{2i}&
    \cdots&
    \omega^{(n-1)i}
  \end{pmatrix}
  \begin{pmatrix}
  	\omega^{-0j}\\
    \omega^{-j}\\
    \omega^{-2j}\\
    \cdots\\
    \omega^{-(n-1)j}
  \end{pmatrix}
  =\sum_{k=0}^{n-1}\omega^{k(i-j)}=0
\]

Tenemos entonces que \(A_\omega A_{\omega^{-1}}^T=nI\),
es decir, \(A^{-1}_\omega=\frac{1}{n}A_{\omega^{-1}}^T\).

\(\tilde{\xi}:K[x]/\langle x^n-1\rangle\longrightarrow K^n\),
con \(\tilde{\xi}^{-1}(y)\) es el polinomio interpolador.

Tenemos unos datos \((y_0,\ldots, y_{n-1})\in K^n\). El polinomio
interpolador de esos datos en los nodos \(1,\omega,\ldots,\omega^{n-1}\)
viene dado por
\[
  \hat{y} =\sum_{j=0}^{n-1}\hat{y_j}x^j
\]
donde \(\hat{y}=y\frac{1}{n}A^T_{\omega^{-1}}\).

Explicitamente, se calcula que los coeficientes quedan:
\[
  \hat{y_j}=\frac{1}{n}\sum_{k=0}^{n-1}y_k\omega^{-jk}
\]

Tomamos \(K=\C\). \(\omega = e^{i2\pi/n}\):
\[
  \hat{y_j}=\frac{1}{n}\sum_{k=0}^{n-1}y_k\omega^{-i2\pi jk/n}
\]
que es la transformada de Fourier de \(y\).

¿Qué interpretación le damos? Supongamos una función periódica de periodo
\(2\pi\), \(f:[0,2\pi]\longrightarrow\C\) con \(f(0)=f(2\pi)\).
Dividimos el intervalo en \(n\) partes iguales, una muestra:
\(y_j=f(\frac{2\pi j}{n})\) con \(j=0,\ldots,n-1\).

Tomamos \(g:[0,2\pi]\longrightarrow\C\) con
\(g(t)=\sum_{j=0}^{n-1}\hat{y_j}e^{ijt}\).

Tenemos entonces que \(g(\frac{2\pi l}{n})=
\sum_{l=0}^{n-1}\hat{y_j}e^{i2\pi lj/n} = y_l=f(\frac{2\pi j}{n})\)

A los \(\hat{y}\) también se le llama el espectro de \(y\).

    \section{Introducción al concepto de módulo}

\begin{df}
  Sean \(M\), \(N\) grupos aditivos:
  \[
    \Ad(M,N)=\{f:M\longrightarrow N| f\textrm{ homomorfismo de grupos}\}
  \]
\end{df}

El conjunto anterior es no vacío porque \(0\in\Ad(M,N)\).
\(\Ad(M,N)\) es un grupo aditivo con la suma:
\[
  (f+g)(m):=f(m)+g(m)\hspace{1cm} \forall m\in M
\]


\begin{df}[Anillo de endomorfismo de \(M\)]
  Definimos directamente \(\End(M):= \Ad(M,M)\).
\end{df}

\begin{prop}
  \((\End(M),+,0,\circ,\id) \) es un anillo.
\end{prop}
\begin{proof}
  Se comprueba que es cerrado para composición. Es obvio que la
  composición es asociativa y tiene como elemento neutro la identidad.

  Finalmente se ve que se cumplen las propiedades distributivas, que
  se siguen de que son homomorfismos.
\end{proof}


\begin{obs}
  Consideramos el grupo \(\{0\}\), es el anillo \(\{0\}\) (anillo
  cero o trivial).

  Si \(M\neq \{0\}\), entonces \(\End(M)\) no es trivial.
\end{obs}

\begin{df}[Módulo]
  Sea \(M\) un grupo aditivo y \(A\) un anillo. Una estructura de
  \(A\)-módulo sobre \(M\) es un homomorfismo de anillos
  \(\rho: A\longrightarrow\End(M)\).
\end{df}

Ejemplo: los números enteros. \(M\) grupo aditivo, \(A=\Z\).
Existe un único \(\chi:\Z\longrightarrow\End(M)\) determinado
por \(\chi(1)=\id_M\), es decir, una única estructura de
\(\Z\)-módulo sobre \(M\) (y su núcleo te da la
característica del anillo).

Ejemplo: cuerpos. Sea \(K\) un cuerpo.
Si \(V\) es un \(K\)-espacio vectorial, definimos \(\rho:K\longrightarrow
\End(V)\), tomamos \(\rho(\alpha):V\longrightarrow V\)
cumpliendo \(\rho(\alpha)(v)=\alpha v\). Trivialmente se cumple que
\(\rho\) es un homomorfismo por la estructura de espacio vectorial de \(V\).
Con lo cual tenemos una estructura de \(K\)-módulo sobre \(V\).
Se puede demostrar el recíproco trivialmente.

\begin{obs}
  Sean  \(X, Y, Z\) conjuntos. \(\Map(X,Y)\) es el conjunto de
  aplicaciones de \(X\) en \(Y\).

  Entonces:
  \[
    \psi:\Map(X\times Y, Z)\longrightarrow\Map(X,\Map(Y,Z))
  \]
  es una biyección dada por \(\psi(f)(x)(y):=f(x,y)\) y
  \(\psi^{-1}(g)(x,y):=g(x)(y)\).
\end{obs}

Ejercicio: comprobar que \(\psi^{-1}\) es realmente la inversa de
\(\psi\).

\begin{obs}
  Sean \(M, N, L\) grupos aditivos.
  \[
    \psi:\Biad(M\times N, L)\longrightarrow\Ad(M,\Ad(N,L))
  \]
  donde \(b\in\Biad(M\times N, L)\) si \(b\) es biaditiva:
  \[
    b(m+m',n)=b(m,n)+b(m',n)
  \]\[
    b(m,n+n')=b(m,n)+b(m,n')
  \]
\end{obs}

Ejercicio, demostrar que la aplicación \(\psi\) es una biyección.

\begin{teo}[Caracterización de módulos]
  Sea \(A\) anillo, \(M\) un grupo aditivo. Sea \(\Ring(A,\End(M))\),
  llamamos \(A\)-módulo a la imagen por \(\psi\) de ese conjunto.
\end{teo}

    \subsection{\(K[x]\)-módulos con \(K\) cuerpo}

Sea \(K\) un cuerpo y se considera el \(K[x]\)-módulo \(M\), es decir,
\(M\) es un grupo aditivo
y \(\rho: K[x] \longrightarrow \End(M)\) es un homomorfismo de anillos.
El cuerpo \(K\) se puede ver como subanillo de \(K[x]\), aplicando la
restricción de escalares aplicada a la aplicación inclusión y, por tanto,
\(M\) es un \(K\)-espacio vectorial.

Veamos \(\rho(x) \in \End(M)\) que es un endomorfismo de espacios vectoriales.
\[
  \rho(x)(km)=x\cdot (km)=x\cdot(k\cdot m)=(xk)\cdot m
  =kx\cdot m=k(xm)=k\rho(x)(m)
\]

\noindent Así que \(\rho(x)\) es \(K\)-lineal.

Si \(p=\sum_i p_i x^i\in K[x]\), tenemos que
\[
  pm = \rho(p)(m)= \rho(\sum_ip_ix^i)(m) = \sum_i p_i {\rho(x)}^i(m)
\]

\begin{prop}
  Sea \(K\) un cuerpo y \(V\) un grupo aditivo. Entonces existe una correspondencia
  biyectiva entre:

  \begin{enumerate}
  \item \(V\) tiene estructura de \(K[x]\)-módulo.
  \item \(V\) tiene estructura de \(K\)-espacio vectorial junto con un endomorfismo
    de espacios vectoriales \(T: V \longrightarrow V\).
  \end{enumerate}
\end{prop}
\begin{proof}
  Veamos primero que 2 implica 1. Se define la aplicación \(\rho: K[x] \longrightarrow
  End(V)\) definida por \(\rho(f(x))(v) = f(T)(v)\) para todo \(f(x) \in K[x]\) y
  \(v \in V\). Recordemos que la expresión de un polinomio en \(K[x]\) puede ser
  \(f = f_0 + \cdots + f_nx^n\), luego \(f(T) = id_V + f_1T \cdots + f_nT^n\). A
  continuación,
  se comprueba que \(\rho\) es un homomorfismo de anillos y acabaría la demostración.
\end{proof}

\begin{ejemplo}
  El espacio de las funciones infinitamente diferenciables sobre \(\R\),
  \(\Cont^\infty(\R)\), es un \(\R\)-espacio vectorial y la aplicación \(T = \frac{d}{dt}\)
  es una aplicación lineal. Por la proposición anterior, \((\Cont^\infty(\R), \frac{d}{dt}\)
  es un \(\R[x]\)-módulo. Tomemos \(\sin \in \Cont^\infty(\R)\). Entonces
  \[x\sin t=T(\sin t)=\cos t \implies x^2\sin t= -\sin t \implies (x^2+1)\sin t=0,\]
  \noindent es decir, en un \(A\)-módulo \(M\) puede pasar que \(a m=0\) aunque \(a\neq 0\)
  y \(m\neq 0\), a diferencia de lo que ocurre en un espacio vectorial.
\end{ejemplo}

\begin{ejemplo}
  En el \(\Z\)-módulo \(\Z_4\) tenemos que \(2\cdot \bar{2}=\bar{0}\).
\end{ejemplo}

\begin{obs}
  En la notación de operación externa \(\cdot\), \(x \cdot v = T(v)\) se
  convierte a la notación de \(K[x]\)-módulo.
\end{obs}

\begin{nt}
  Cuando se esté en presenca de un \(A\)-módulo \(M\), se escribirá simplmente que
  \({}_AM\) es un módulo.
\end{nt}


    \subsection{\(K[x]\)-módulos con \(K\) cuerpo}

Tenemos \(K[x]\)-módulo \(M\). O sea, \(M\) es un grupo aditivo
y \(\rho: K[x]\longrightarrow\End(M)\) es un homomorfismo de anillos.

\(K\) se puede ver como subanillo de \(K[x]\), aplicando la
restricción de escalares aplicada a la aplicación inclusión,
\(M\) es un \(K\)-espacio vectorial.

Veamos que ocurre con la indeterminada. \(\rho(x)\in\End(M)\).

Veamos que es un endomorfismo de espacios vectoriales:
\[
  \rho(x)(km)=x\cdot (km)=x\cdot(k\cdot m)=(xk)\cdot m
  =kx\cdot m=k(xm)=k\rho(x)(m)
\]

Así que \(\rho(x)\) es \(K\)-lineal.

Si \(p=\sum_i p_i x^i\in K[x]\), tenemos que
\[
  pm=\rho(p)(m)=\sum_i p_i {\rho(x)}^i(m)
\]

\begin{prop}
  Si tengo un \(K\)-espacio vectorial \(V\) y una aplicación
  lineal \(T:V\longrightarrow V\), podemos definir para \(p\in K[x]\)
  y \(v\in V\) el operador
  \[
    pv:=p(T)(v)=\sum_i p_i T^i(v)
  \]
  resulta que \(V\) es un \(K[x]\)-módulo.
\end{prop}

Ejemplo, \(\Cont^\infty(\R)\) con \(T=\frac{d}{dt}\)
es un \(\R[x]\)-módulo.

    \begin{obs}
  \(\Cont^\infty(\R)\) dotado de estructura de \(\R[x]\)-módulo
  a través del endomorfismo lineal \(T=\frac{d}{dt}\) es un ejemplo
  ilustrativo en el siguiente sentido.

  Tomemos \(\sin\), \(x\sin t=T(\sin t)=\cos t\)
  \(x^2\sin t= -\sin t\) con lo que
  \[
    (x^2+1)\sin t=0
  \]
  es decir, en un \(A\)-módulo \(M\) puede pasar que \(a m=0\)
  \(a\neq 0\), \(m\neq 0\).
\end{obs}

Ejemplo en el \(\Z\)-módulo \(\Z_4\) tenemos que
\(2\cdot \bar{2}=\bar{0}\).

    \subsection{Módulos abstractos}

\begin{df}
  Sea \({}_AM\) un módulo cuyo homomorfismo es \(\rho\). Se dice que \({}_AM\) es
  \textbf{fiel} si \(\rho\) es inyectivo.
\end{df}
\begin{df}
  Se define el \textbf{anulador de M} como
  \[\Ann_A(M) = {a \in A \ /\ am = 0\ \text{para todo}\ m \in M}.\]
\end{df}

\begin{obs}
  Sean \(A\) un anillo, \(\subscriptbefore{A}{M}\) un \(A\)-módulo y
  un homomorfismo de anillos \(\rho:A\longrightarrow\End(M)\) y \(a \in A\).
  \(\rho(a) = 0\) siempre que \(a \in \Ann_A(M)\), luego \(\Ann_A(M) = \ker(\rho)\).
  Por eso, se puede afirmar que \(\Ann_A(M)\) es un ideal de \(A\) y se tiene que
  \(M\) es fiel si, solo si, \(\Ann_A(M) = {0}\).

  Aplicando el primer teorema de isomorfía, tenemos:
  \[
    A/\ker\rho = A/\Ann_A(M) \simeq \Im\rho \leqslant \End(M)
  \]
  \noindent y entonces \(M\) es un \(A/\ker\rho\)-módulo.
  De hecho, \((a+\ker\rho)m=\rho(m)\).
  \(M\) siempre es fiel sobre \(A/\Ann_A(M)\), aunque \(\rho\) no sea inyectiva.
\end{obs}

\begin{ejercicio}
  Si tenemos una aplicación lineal entre espacios vectoriales
  de dimensión finita, entonces el anulador está generado por un único
  polinomio, el polinomio mínimo de \(T\). Se denota como \(\Ann_{K[x]}(V) =
  < \mu(x) >\).
\end{ejercicio}

\begin{ejercicio}
  Sea \(K\) un cuerpo, \(V\) un \(K\)-espacio vectorial de dimensión finita y
  \(T:V \longrightarrow V\) una aplicación linea. Considerando que \(V\) tiene estructura
  de \(K[x]\)-módulo, probar que el anulador es no vacío.
\end{ejercicio}

\subsection{Submódulos}
\begin{df}
  Dado un módulo \({}_AM\), un \textbf{submódulo} de \({}_AM\)
  es un subgrupo aditivo \(N\subseteq M\)
  tal que \(am\in N\) para cualquier \(a\in A\) y \(m\in N\).
\end{df}

\begin{ejemplo}
  Los submódulos del módulo regular \(A\) se llaman \textbf{ideales por la izquierda de
    \(A\)}. Todo ideal es un ideal a izquierda. Si \(A\) es conmutativo, los ideales
  a izquierda coinciden con los ideales.

  Tomando \(A=\mathcal{M}_2(K)\) con \(K\) un cuerpo.
  \[
    \mathcal{M}_2(K)=\left\{
      \begin{pmatrix}
        a&b\\
        c&d
      \end{pmatrix}:
      a,b,c,d\in K
    \right\}
  \]

  Tenemos que el conjunto:
  \[
    \left\{
      \begin{pmatrix}
        0&b\\
        0&d
      \end{pmatrix}:
      b,d\in K
    \right\}
  \]
  es un ideal a izquierda de \(A\).
\end{ejemplo}

\begin{ejemplo}
  \(T:V\longrightarrow V\), \(K\)-lineal.
  ¿Qué es un \(K[x]\)-submódulo de \(V_{K[x]}\)?
  Sea \(W\) un tal submódulo.
  \(W\) es un subespacio vectorial y además \(T(w)=xw\in W\),
  es decir, un subespacio \(T\)-invariante (un ejemplo
  de subespacio \(T\) invariante es un subespacio propio).
  El recíproco es también cierto.
\end{ejemplo}

\begin{df}[Submódulo cíclico]
  Dado \(\subscriptbefore{A}{M}\), y un \(m\in M\). Es claro que
  \(Am=\{am:a\in A\}\) es un submódulo de \(\subscriptbefore{A}{M}\) que se llama
  \textbf{submódulo cíclico generado por \(m\)}.
\end{df}

\begin{ejemplo}
  \(\R[x]\sin t=\R\sin t+\R\cos t\).
\end{ejemplo}

\begin{df}[Submódulo finitamente generado]
  Dados \(m_1,\ldots, m_n\in M\), el conjunto
  \[
    Am_1+\cdots+Am_n=\{a_1 m_1+\cdots+a_n m_n: a_i\in A\}
  \]
  es un submódulo de \(\subscriptbefore{A}{M}\) llamado el \textbf{submódulo generado por
  \(m_1,\ldots, m_n\)}.
  Si \(M=Am_1+\cdots+Am_n\), diremos que \(M\) es \textbf{finitamente generado}
  con generadores
  \(m_1,\ldots, m_n\).
\end{df}

\subsubsection{Suma directa interna}

\begin{df}[Módulo suma]
  Dados \(N_1,\ldots, N_n\) submódulos de \(\subscriptbefore{A}{M}\),
  \[
    N_1+\cdots+ N_n=\{m_1+\cdots+m_n: m_i\in N_i\}
  \]
  es un submódulo de \(M\), que se llama \textbf{suma de \(N_1, \cdots, N_n\)}.
\end{df}

\begin{nt}
  Se puede expresar \(N_1+\cdots+ N_n\) como \(\sum_{i=1}^n N_i\).
\end{nt}

\begin{prop}
  Sean \(N_1,\ldots, N_t\) submódulos de \(A\). Son equivalentes:
  \begin{enumerate}
    \item Para cada \(i = 1, \cdots t\), \(N_i\cap\sum_{j\neq i}N_j =\{0\}\).
    \item Si \(0=n_1+\cdots+n_t\), \(n_i\in N_i\) entonces \(n_i=0\)
      para todo \(i = 1, \cdots t\).
    \item Cada \(n\in N_1+\cdots+N_t\) admite una representación
      única como \(n=n_1+\cdots+n_t\), con \(n_i\in N_i\).
  \end{enumerate}
\end{prop}
\begin{proof}
  Veamos que 1 implica 2. Tenemos que \(0=n_1+\cdots+n_t\),
  si despejamos, \(n_i=-\sum_{j\neq i} n_j\in N_i\cap\left(
  \sum_{j\neq i} N_j\right)=\{0\}\).

  Veamos que 2 implica 3. Si \(n=\sum n_i=\sum n_i'\),
  entonces \(0=\sum(n_i-n_i')\) lo que implica que
  \(n_i=n_i'\).

  Finalmente, tomando \(n\in N_i\cap\left(
  \sum_{j\neq i} N_j\right)\), es decir,
  \(n=\sum_{j\neq i}n_j\) con lo que
  \(0=n-\sum_{j\neq i} n_j\) y como las descomposiciones
  son únicas, \(n=0\).
\end{proof}

\begin{df}[Suma interna]
  Si \(M=N_1+\cdots+ N_t\) tales que satisfacen una de las condiciones
  equivalentes anteriores, diremos que \(M\) es la suma directa interna
  y usaremos la notación \(M=N_1\dot{+}\cdots\dot{+} N_t\).
\end{df}

\begin{df}
  Si \(\{N_1,\ldots, N_t\}\) verifican las condiciones equivalentes
  anteriores y \(N_i\neq \{0\}\), se dice que el conjunto
  \(\{N_1,\ldots, N_t\}\) es una
  familia independiente.
\end{df}

\begin{ejemplo}
  \(\Z_6\) es un \(\Z\) módulo.
\[
  \Z_6=\{0,1,2,3,4,5,6\}
\]
Tomamos
\[
  N_1=\{0,3\}
\]
y
\[
  N_2=\{0,2,4\}
\]

Tenemos que \(N_1, N_2\) es una familia independiente. Además es obvio que:
\[
  N_1\dot{+}N_2=\Z_6
\]
ya que tienen como intersección \(\{0\}\) y su suma es el total.
\end{ejemplo}
    
\begin{proof}
  Veamos que 1 implica 2. Tenemos que \(0=n_1+\cdots+n_t\),
  si despejamos, \(n_i=-\sum_{j\neq i} n_j\in N_i\cap\left(
  \sum_{j\neq i} N_j\right)=\{0\}\).

  Veamos que 2 implica 3. Si \(n=\sum n_i=\sum n_i'\),
  entonces \(0=\sum(n_i-n_i')\) lo que implica que
  \(n_i=n_i'\).

  Finalmente, tomando \(n\in N_i\cap\left(
  \sum_{j\neq i} N_j\right)\), es decir,
  \(n=\sum_{j\neq i}n_j\) con lo que
  \(0=n-\sum_{j\neq i} n_j\) y como las descomposiciones
  son únicas, \(n=0\).
\end{proof}

\begin{df}[Suma interna]
  Si \(M=N_1+\cdots+ N_t\) tales que satisfacen una de las condiciones
  equivalentes anteriores, diremos que \(M\) es la suma directa interna
  y usaremos la notación \(M=N_1\dot{+}\cdots\dot{+} N_t\).
\end{df}

\begin{df}
  Si \(\{N_1,\ldots, N_t\}\) verifican las condiciones equivalentes
  anteriores y \(N_i\neq \{0\}\), se dice que el conjunto
  \(\{N_1,\ldots, N_t\}\) es una
  familia independiente.
\end{df}

Ejemplo: \(\Z_6\) es un \(\Z\) módulo.
\[
  \Z_6=\{0,1,2,3,4,5,6\}
\]
Tomamos
\[
  N_1=\{0,3\}
\]
y
\[
  N_2=\{0,2,4\}
\]

Tenemos que \(N_1, N_2\) es una familia independiente. Además es obvio que:
\[
  N_1\dot{+}N_2=\Z_6
\]
ya que tienen como intersección \(\{0\}\) y su suma es el total.


    \subsubsection{Módulos acotados sobre un DIP}

\begin{df}[Módulo acotado sobre un DIP]
  Sea \(A\) un dominio de ideales principales, \(\subscriptbefore{A}{M}\)
  un módulo, \(\Ann_A(M)=\langle\mu\rangle\) para cierto \(\mu\in A\).

  Si \(\mu\neq 0\), diré que \(M\) es acotado.
\end{df}

Supongamos que \(\subscriptbefore{A}{M}\) es acotado y \(\mu\not\in
\mathcal{U}(A)\), ya que si \(\mu\in\mathcal(A)\) entonces \(M=\{0\}\).

Si \(\mu=p_1^{e_1}\cdots p_t^{e_t}\), posible porque todo DIP es un
dominio de factorización única (DFU), con \(p_i\in A\) irreducible
y \(e_i>0\).

\begin{prop}[Descomposición primaria del módulo]
  Tomamos \(q_i=\frac{\mu}{p_i^{e_i}}\in A\).

  Llamamos \(M_i=\{q_i m:m \in M\}\subseteq M\). Veamos
  que \(M_i\in\mathcal{L}(\subscriptbefore{A}{M})=\{\textrm{submódulos
  de } \subscriptbefore{A}{M}\}\).

  Queremos que \(M=M_1\dot{+}\cdots \dot{+}M_t\), con \(t>1\) para
  evitar trivialidades. En ese caso, \(\mcd\{q_1,\ldots,q_t\}=1\),
  donde se ha usado que estamos en un DFU.\@

  Por la identidad de Bezout (válida porque estamos en un DIP),
  tenemos que \(1=\sum_{i=1}^t q_i a_i\), para ciertos \(q_i\in A\).
  Para en \(m\in M\), \(M=1\cdot m=\sum_i q_i a_i m\), luego
  \(M=M_1+\cdots+ M_t\).

  Vamos a ver que la suma es directa.
  \(q_i q_j\in\langle\mu\rangle\) si \(i\neq j\). Eso significa que si
  \(m\in M_i\) y entonces \(q_j m= 0\) si \(i\neq j\).
  Por tanto \(M_i=\{m\in N: m=q_i a_i m\}\).

  Si \(0=\sum_{i=1}^t\) con \(m_i\in M_i\), entonces
  \[
    0=q_j a_j 0=m_j
  \]
  y por tanto \(M=M_1\dot{+}\cdots\dot{+} M_t\).
\end{prop}

\begin{df}[Componentes primarias]
  Tenemos que los \(M_i\) se llaman componentes primarias.
\end{df}
\begin{prop}
  \[
    M_i=\{m\in M: p_i^{e_i} m=0\}
  \]

  Así, \(\langle\mu\rangle=\Ann_A(M)=\bigcap_{i=1}^t\Ann_A(M_i)
  \supseteq\bigcap_{i=1}^t\langle p_i^{e_i}\rangle=\langle\mu\rangle\)
\end{prop}

Ejercicio: Obtener la descomposición primaria usando \(\dot{+}\) de
\(\Z_{8000}\).

    Ejemplo: \(T\) endomorfismo \(K\)-lineal. \(V=\subscriptbefore{K[x]}{V}\).

Un \(W\) es un submódulo de \(V\) es un subespacio vectorial tal que
\(T(W)\subseteq W\), es decir, \(W\) es \(T\) invariante.

Si \(\Ann_{K[x]}(V)\neq\{0\}\), tomo \(\mu(x)\in K[x]\), el polinomio
mínimo de \(T\). Es decir, \(\Ann_{K[x]}(V)=\langle\mu(x)\rangle\).

\[
  \mu=p_1^{e_1}\cdots p_t^{e_t}
\]

Entonces la descomposición primaria de \(V\) es \(V=V_1\dot{+}
\cdots\dot{+}V_t\) con
\[
  V_i=\{v\in V:p_i(x)v=0\}
\]

Caso particular: \(\dim(V)<\infty\) y que \(\mu(x)=(x-\alpha_1)\cdots
(x-\alpha_t)\) con \(\alpha_i\neq\alpha_j\).
\[
  V_i=\{v\in V:(x-\alpha_i)v=\{v\in V:T(v)=\alpha_i v\}
\]
es decir, el subespacio propio asociado al valor propio \(\alpha_i\).

Si el polinomio factoriza como producto de polinomios de grado 1 distintos,
\(T\) es diagonalizable.
Veremos en el futuro que el polinomio mínimo divide siempre al polinomio
característico.

¿Cómo se calcula el polinomio mínimo de un endomorfismo lineal?

Ejercicio: Sea \(V\) un espacio vectorial real euclídeo (con producto
escalar). Sea \(T:V\longrightarrow V\) una isometría.
Se pide demostrar que si \(W\) es un subespacio \(T\) invariante de \(V\),
entonces su ortogonal \(W^\perp\) es también \(T\) invariante.
Entonces \(V=W\dot{+}W^\perp\). Se usa inducción. Como consecuencia, usando
el teorema fundamental del álgebra, deducir que \(V\) admite una base
ortonormal con respecto de la cual la matriz de \(T\) es diagonal por
bloques, con bloques de dimensión 1 o 2. ¿Qué aspecto tienen dichos
bloques? Hay que ver que uno de los dos subespacios invariantes tienen
dimensión 1 o 2.


    \subsection{Homomorfismos de módulos}

\begin{prop}[Módulo cociente o factor]
  Sea \(\subscriptbefore{A}{M}\) y \(L\in\mathcal{L}(M)\).
  Consideramos \(M/L\) grupo aditivo y se define la acción:
  \[
    a(m+L):=am+L
  \]
  Entonces \(M/L\) es un módulo.
\end{prop}

\begin{df}[Homomorfismo de módulos]
  Se dice que
  \(f:\subscriptbefore{A}{M}\longrightarrow \subscriptbefore{A}{N}\)
  es un homomorfismo de módulos si respeta sumas y productos.
\end{df}

\begin{df}[Proyección canónica]
  Es la aplicación \(\pi:M\longrightarrow M/L\) dada por
  \(\pi(m)=m+L\). Es un homomorfismo de módulos.
\end{df}

\begin{teo}[Teorema de isomorfía para módulos]
  \(f:M\longrightarrow N\) un homorfismo de \(A\)-módulos. Entonces
  el núcleo \(\ker f\in\mathcal{L}(\subscriptbefore{A}{M})\) y
  \(\Im f\in\mathcal{L}(N)\). Para cada \(L\in\mathcal{L}
  (\subscriptbefore{A}{M})\) tal que \(L\subseteq \ker f\) existe
  un único homomorfismo de módulos \(\tilde{f}:M/L\longrightarrow N\)
  tal que \(\tilde{f}\circ\pi=f\). Finalmente, \(\tilde{f}\) es
  inyectiva si y solo si \(L=\ker f\), en cuyo caso, \(\tilde{f}\)
  da un isomorfismo de \(A\)-módulos
  \(M/\ker f\cong \Im f\).
\end{teo}

Ejemplo \(\subscriptbefore{A}{M}\), definimos \(f:A\longrightarrow M\)
dada por:
\[
  f(a)=am\hspace{1cm} \forall a \in A
\]
es un homomorfismo de \(A\)-módulos.

Tenemos \(\Im f = Am\) y
\(\ann(a)=\ker f=\{a\in A: am=0\}\) es un ideal izquierda y se tiene
\[
  A/\ann_A(m)\cong Am
\]

\[
  a+\ann_A(m)\mapsto am
\]

Ejemplo: \(S=\Map(\N,K)\), el conjunto de las sucesiones (que forman
un \(K\)-espacio vectorial). Tomamos \(T:S\longrightarrow S\)
tal que \(T(s)(n)=s(n+1)\). Es lineal. Entonces
\(\subscriptbefore{K[x]}{S}\), donde \(xs=T(s)\).

Para cualquier \(f\in K[x]\), es decir \(f=\sum_i f_i x^i\), se tiene:
\[
  (fs)(n)=\sum_i f_i s(n+i)
\]

Imaginémosnos que \(s\) verifica que \(\ann_{K[x]}(s)\neq\langle0\rangle\).
Podemos tomar entonces un polinomio tal que \(fs = 0\) y que sea mónico.
Tenemos entonces que \(s(n+m)=-\sum_{i=0}^{m-1} f_i s(n+i)\)
para todo \(n\in\N\). Es decir, la sucesión es linealmente recursiva.

Caso particular, \(s(0)=s(1)=1\), tenemos que
\[
  s(n+2)=s(n)+s(n+1)
\]

\[
  x^2-x-1\in\ann_{\Q[x]}(s)
\]

Volviendo al caso general, tenemos que
\[
  K[x]/\ann_{K[x]}(s)\cong K[x]s
\]

Tenemos que \(\dim_{K}(K[x]s)<\infty\) si y solo si
\(\ann_{K[x]}(s)\neq\langle0\rangle\) si y solo si
\(s\) es una sucesión linealmente recursiva.

El generador \(p(x)\) de \(\ann_{K[x]}(s)\) se le llama el polinomio
mínimo de \(s\). El grado de dicho polinomio, coincide con
\(\dim_{k}(K[x]s)\) y se le llama complejidad lineal de \(s\).

\(s,t\) dos sucesiones linealmente recursivas.
\(K[x](s+t)\subseteq K[x]s+K[x]t\), luego la primera tiene dimensión finita.
Luego \(s+t\) es una sucesión linealmente recursiva, de complejidad menor
o igual a la suma de las complejidades lineales.
Puede argumentarse lo mismo para combinaciones lineales.

Las sucesiones linealmente recursivas forman un subespacio vectorial
del espacio de sucesiones. De hecho forman un submódulo. Sea
\(S^l\) el conjunto de las sucesiones linealmente recursivas, forma
un \(S^l\) es un \(K[x]\)-submódulo de \(S\), ya que es ivariante por la
acción de \(x\) (es \(T\)-invariante).

Otro ejemplo: \(T\) endomorfismo de \(\Cont^\infty(\R)\) tal que
\(T(\varphi)=\varphi'\). Tenemos que
\(\subscriptbefore{R[x]}{\Cont^\infty(\R)}\). Dada \(\varphi\),
tenemos que
\[
  \ann_{\R[x]}(\varphi)=\{f\in\R[x]: f(x)\varphi=0\}
  =\{f=\sum_i f_i\frac{d^i}{dt^i}: f\varphi=0\}
\]
\(\ann(\varphi)\neq\langle 0\rangle\) si \(\varphi\) satisface una ecuación
diferencial lineal homogénea con coeficientes constantes. Bla bla.

\(\R[x]/\ann_{\R[x]}(\varphi)\cong \R[x]\varphi\), donde \(\varphi\)
satisface bla bla.

Tenemos que \(\varphi''-\varphi'-\varphi=0\), cuya solución
\(\varphi(t)=e^{\phi t}\), donde \(\phi\) es la razón aúrea.

    \input{14_Homomorfismos_de_modulos.tex}
    \subsection{Suma directa externa}

\begin{df}
Tomando el producto cartesiano de \(t\) módulos sobre el mismo anillo
y tomando la suma usual de tuplas y definiendo el siguiente producto:
\[
  a(m_1,\ldots, m_t)=(am_1,\ldots,am_t)
\]

Es un módulo que se llama suma directa externa de \(M_1,\ldots, M_t\)
con \(M^t\) si son todos iguales.

  Se denota \(M_1\oplus\cdots\oplus M_t\).
\end{df}


Ejercicio: Sea \(\subscriptbefore{A}{M}\), \(N_1,\ldots,
N_t\in\mathcal{L}(\subscriptbefore{A}{M})\). Se pide demostrar que existe
un homomorfismo \(f:N_1\oplus\cdots\oplus N_t\longrightarrow
N_1{+}\cdots{+}N_t\)
sobreyectivo de \(A\)-módulos tal que entre la suma directa
externa y la suma interna, tal que \(f\) es un isomorfismo si y solo si
la suma interna es directa. Podría ser interesante usar coordenadas.

    \input{16_Homomorfismos_de_modulos.tex}
    \section{Módulos noetherianos}

\begin{df}[Sucesiones exactas]
  Una sucesión de homomorfismos de módulos \(f_i:M_i\longrightarrow  M_{i+1}\) se dice
  exacta en \(M_{i+1}\) si \(\ker f_{i+1}=\Im f_i\).
\end{df}

\begin{ejemplo}
  Dada una sucesión \(\{0\}\longrightarrow L 
  \overset{\alpha}{\longrightarrow} M 
  \overset{\beta}{\longrightarrow} N\longrightarrow \{0\}\)
  es exacta en \(L\) si y solo si \(\ker \alpha=\{0\}\), es decir,
  \(\alpha\) es inyectiva, en \(N\) si y solo si \(\Im \beta = N\),
  es decir, \(\beta\) sobreyectiva y en \(M\) si y solo si
  \(\ker\beta=\Im\alpha\).
   
  A \(\alpha\) se les llama monomorfismos de módulos y a
  \(\beta\) epimorfismos de módulos.

  A esta sucesión se le llama sucesión exacta corta.

  Caso particular: Por ejemplo, si \(f:M\longrightarrow N\) es
  un homorfismo de módulos, obtenemos:
  \[
    0\longrightarrow\ker f\overset{\iota}{\longrightarrow}
    M \overset{f}{\longrightarrow}\Im f
    \longrightarrow 0
  \]
\end{ejemplo}

\begin{prop}
  Sea \(0\longrightarrow L\overset{\psi}{\longrightarrow} M
  \overset{\varphi}{\longrightarrow}
  N\longrightarrow 0\) una sucesión exacta de \(A\)-módulos. Entonces:
  \begin{enumerate}
    \item Si \(M\) es finitamente generado, lo es también \(N\).
    \item Si \(L\) y \(N\) son finitamente generados, lo es también \(M\).
  \end{enumerate}
\end{prop}

\begin{proof}
  Veamos primero la primera afirmación. Sea \(\{m_i\}\) generadores de \(M\).
  Es claro que \(\{\varphi(m_i)\}\) generan \(N\).

  Para la segunda, \(\{n_i\}\) generadores de \(N\), y tomamos
  \(\{m_i\}\subseteq M\) tales que \(\varphi(m_i)= n_i\).

  Tomamos \(\{e_i\}\) generadores de \(L\). Tomamos \(m\in M\).
  \[
    \varphi(m)=\sum_{i=1}^s r_i n_i = \sum_{i=1} r_i\varphi(m_i)
    =\varphi\left(\sum r_i m_i\right)
  \]
  con lo que \(m-\varphi(\sum r_i m_i)\in\ker\varphi=\Im\psi\).
  Luego existen \(b_1,\ldots, b_t\) tales que
  \[
    m-\varphi\left(\sum r_i m_i\right)=
    \psi\left(\sum_j b_j e_j\right)
  \]
  y finalmente:
  \[
    m=\varphi\left(\sum r_i m_i\right)+\sum r_j \varphi(e_j)
  \]
  con lo que \(\{m_i\}\cup\{\psi(e_j)\}\).
\end{proof}

Ejemplo de que no se puede mejorar la proposición anterior:
Sea \(I\) un conjunto infinito, \(K\) un cuerpo.
\[
  K^I=\{{(\alpha_i)}_i\in I:\alpha_i\in K\}
\]
\(K^I\) es un anillo finitamente generado
por \((\ldots,1,1,1,\ldots)\). Definimos:
\[
  K^{(I)}=\{{(\alpha_i)}_i\in I:\alpha_i\in K \textrm{ y } \alpha_i=0
  \textrm{ salvo un número finito de } i\in I\}
\]

Tenemos que \(K^{(I)}\) es un ideal de \(K^I\), y por tanto ideal a
izquierda, pero no es finitamente generado como ideal a izquierda.

Es decir, \(M\) finitamente generado no implica que un submódulo suyo
sea finitamente generado.

\begin{df}[Módulos Noetherianos]
  Un módulo finitamente generado \(M\) se dice Noetheriano si todo
  submódulo de \(M\) es finitamente generado.
\end{df}

El ejemplo anterior no era un módulo Noetheriano.

\begin{prop}
  Equivalen:
  \begin{enumerate}
    \item \(M\) es noetheriano.
    \item Cualquier cadena ascendente \(L_1\subseteq L_2\subseteq\ldots
      \subseteq L_n\subseteq\ldots\) se estabiliza, es decir,
      a partir de un cierto \(m\) las inclusiones se vuelven igualdades.
    \item Cada subconjunto no vacío de \(\mathcal{L}(M)\) tiene un elemento
      maximal con respecto de la inclusión.
  \end{enumerate}
\end{prop}

    \input{18_Algebra_Homologica.tex}
    \section{Módulos artinianos}

\begin{prop}[Módulo artiniano]
  Dado un módulo \(\subscriptbefore{A}{M}\), son equivalentes:
  \begin{enumerate}
    \item Cada cadena descendente
      \(L_1\supseteq L_2\supseteq\ldots\supseteq L_n\supseteq\ldots\)
      de submódulos de \(M\)
      se estabiliza; esto es, a partir de cierto \(m \in \N\)
      se tiene \(L_n = L_m\) para todo \(n \ge m\).
    \item Cada subconjunto no vacío de \(\mathcal{L}(M)\) tiene un elemento
      minimal.
  \end{enumerate}
  A un tal módulo lo llamaremos artiniano.
\end{prop}
\begin{proof}
  Se adapta la demostración de los módulos noetherianos. (Propsición \ref{prop:noether}).
\end{proof}

\begin{ejercicio}
  Sea \(A\) un dominio de integridad conmutativo. Si el
  módulo regular es artiniano, entonces \(A\) es un cuerpo.
\end{ejercicio}

En particular \(\Z\) no es artiniano, aunque por ser un DIP, sí que
es noetheriano.

\begin{ejercicio}
  \(K\) un cuerpo de característica 0. Tomo \(K[x]\) anillo
  de polinomios. Veo \(K[x]\) como \(K\)-espacio vectorial.
  Tomamos \(T\) la aplicación lineal \(T(f):=f'\), donde \(f'\) es el
  polinomio derivado. Esto nos da una estructura de \(K[x]\)-módulo
  sobre \(K[x]\) que no es la del módulo regular. Se pide demostrar
  que ese módulo es artiniano y no finitamente generado.
\end{ejercicio}

En consecuencia, la estructura que hemos definido no es la misma
que la del módulo regular.

\begin{prop}
  Sea la sucesión exacta corta de módulos
  \[
    0\longrightarrow L\longrightarrow M\longrightarrow N\longrightarrow 0
  \]

  Entonces \(M\) es artiniano si, y solo si, \(L\) y \(N\) son artinianos.
\end{prop}
\begin{proof}
  Se adapta la demostración para módulos noetherianos (Proposición \ref{prop:suc_noether}).
\end{proof}

\begin{ejercicio}
  Sea \(p\) un número primo. Definimos:
  \[
    C_{p^\infty}=\{z\in\C:z^{p^n}=1\textrm{ para algún } n\ge1\}
  \]
  Se pide comprobar que es un subgrupo \(\Sphere=\{z\in\C:\md{z}=1\}\) y, mediante que
  \(C_{p^\infty}\) es conmutativo y visto como \(\Z\)-módulo, es artiniano, pero no es
  finitamente generado.
\end{ejercicio}


    \section{Módulos de longitud finita}

\begin{df}[Serie de composición]
  Sea \(M\) un módulo. Una \textbf{serie de composición de \(M\)}
  es una cadena de submódulos
  \[
    M=M_n\supsetneq M_{n-1}\supsetneq\ldots\supsetneq M_1
    \supsetneq M_0=\{0\}
  \]
  tal que si \(M_i\supseteq N\supseteq M_{i-1}\), con \(N\) submódulo de \(M\),
  entonces \(N=M_i\) o \(N=M_{i-1}\). Es decir, cada submódulo es maximal
  en el anterior.

  A \(n\) se le llama la \textbf{longitud de la serie} y se denota por \(\ell(M)\).
\end{df}

\begin{ejemplo}
  Sea una serie de composición de \(\Z_{12}\). Tiene como subgrupos
  a \(\Z_m\) con \(m\) divisor de 12.
  \[
    M_3=\Z_{12}
  \]
  Tiene como subgrupo maximal (argumentando por Lagrange):
  \[
    M_2=\langle 2 \rangle
  \]
  que a su vez tiene como subgrupo maximal
  \[
    M_1=\langle 4\rangle
  \]
  y ya solo tiene
  \[
    M_0=\{0\}
  \]
\end{ejemplo}

\begin{ejemplo}
  Sea \(T: \R^3 \longrightarrow \R^3\) un giro de ángulo \(\frac{\pi}{3}\) respecto al eje
  vertical. Sabemos que \(\R^3, T\) es un \(\R[x]\)-módulo. Los submódulos serán el vacío,
  el eje de giro, \(E\), el plano vectorial, \(\Pi\), perpendicular a \(E\) y \(\R^3\).
  Entonces
  \[\{0\} \subset E \subset E + \Pi = \R^3\]
  \[\{0\} \subset \Pi \subset \Pi + E = \R^3\]
  son series de composición. 
\end{ejemplo}

\begin{df}[Módulo simple]
  \(M\) se dice \textbf{simple} si \(M\supset\{0\}\) (estrictamente) es una serie de
  composición.
  Es decir, si no tiene submódulos propios y no es el módulo 0.
\end{df}

\begin{prop}
  La condición de que cada submódulo sea maximal en el anterior es equivalente
  a que los factores \(M_i/M_{i-1}\) sean simples.
\end{prop}

\begin{teo}
  Toda serie de composición del mismo módulo tiene la misma longitud
  y los mismos factores, salvo isomorfismos y reordenaciones.
\end{teo}

\(\Z_{12}\) tiene como factores \(\Z_2\), \(\Z_2\) y \(\Z_3\).

\begin{prop}
  Un módulo no nulo admite una serie de composición si, y solo si, es
  noetheriano y artiniano.
\end{prop}
\begin{proof}
  Sea \(M_i\) una serie de composición. Inducción sobre \(n\).
  Si \(n=1\), tenemos que \(M\) es simple y en particular noetheriano
  y artiniano.

  Si \(n>1\), entonces \(M_{n-1}\) admite una serie de composición de
  longitud \(n-1\), luego es noetheriano y artiniano. Tomamos
  la sucesión exacta corta \[0\longrightarrow M_{n-1}
  \longrightarrow M_n\longrightarrow M_n/M_{n-1}\longrightarrow 0\]
  El primer elemento es noetheriano y artiniano, el último es simple (luego
  noetheriano y artiniano), con lo que \(M_n\) es noetheriano y artiniano.

  Para el recíproco, como \(M\) es artiniano, contiene un submódulo
  simple \(M_1\). Entonces hay un \(M_2\supsetneq M_1\) donde
  \(M_2/M_1\) es simple. Reiterando el proceso, tenemos
  \(0\subsetneq M_1\subsetneq M_2\subsetneq \ldots\) y como es
  noetheriano, habrá un \(M_n\) que termine la cadena.
\end{proof}

\begin{cor}
  Dada una sucesión exacta corta, \(0\longrightarrow L
  \longrightarrow M\longrightarrow N\longrightarrow 0\),
  \(L\) y \(N\) admiten una serie de composición si, y solo si,
  \(M\) admite serie de composición.
\end{cor}

\begin{cor}
  Los \(A\)-módulos \(M_1\) y \(M_2\) admiten series de composición si, y solo si,
  \(M_1\oplus M_2\) admite una serie de composición.
\end{cor}


\begin{teo}[Jordan-Hölder]
  Sea \({}_AM\) un módulo y supongamos que \(M\) admite series de composición:
  \[
    \{0\}=M_0\subsetneq M_1\subsetneq M_2\subsetneq\ldots\subsetneq M_n=M
  \]
  \[
    \{0\}=N_0\subsetneq N_1\subsetneq N_2\subsetneq\ldots\subsetneq N_m=M
  \]
  Entonces \(n=m\) y existe una permutación \(\sigma\) tal que
  \[
    M_i/M_{i-1}\cong N_{\sigma(i)}/N_{\sigma(i)-1}
  \]
\end{teo}
\begin{proof}
  Si \(n=1\), entonces \(M\) es simple y \(m=1\) y el el único factor posible
  es el \(M/\{0\}=M\).

  Si \(n>1\), como \(M\) no es simple, \(m>1\).

  Vamos a observar un caso particular. Supongamos que \(N_{m-1}
  =M_{n-1}\). Por hipótesis de inducción aplicado a \(N_{m-1}\),
  tenemos que  \(n-1=m-1\), luego \(n=m\) y se da el enunciado
  (tomando la permutación \(\sigma\) para los \(n-1\) primeros elementos
  y extendiendola a una permutación de \(n\) elementos \(\sigma'\) tal
  que \(\sigma'(n):=n\), \(\sigma'(k):=\sigma(k)\)).

  Vamos ahora al caso general. Como hemos visto
  en el caso particular anterior, podemos suponer
  \(M_{n-1}\neq N_{m-1}\), por lo que
  \(M_{n-1}+M_{m-1}=M\) (ya que \(M_{n-1}\subsetneq M_{n-1}+
  N_{m-1}\subseteq M\) y \(M_{n-1}\) es maximal).

  Tomamos \(N_{m-1}\cap M_{n-1}\) que admite una serie de composición:
  \[
    \{0\}=L_0\subsetneq L_1\subsetneq\ldots\subsetneq L_k
    =N_{m-1}\cap M_{n-1}
  \]
  y tenemos que, por el teorema de isomorfía:
  \[
    N_m/N_{m-1}=
    M/N_{m-1}=
    (M_{n-1}+N_{m-1})/N_{m-1}\cong M_{n-1}/(M_{n-1}\cap N_{m-1})
  \]
  que al ser un factor es simple.

  Aplicando la inducción, \(n-1=k+1\) y existe una permutación
  \(\tau\) de \(n-1\) elementos tal que
  \[
    L_i/L_{i-1}\cong M_{\tau(i)}/M_{\tau(i)-1}
  \]
  donde \(i=1,\ldots, n-2\)
  y
  \[
    M_{n-1}/L_{n-2}=M_{n-1}/(M_{n-1}\cap N_{m-1})\cong
    M_{\tau(n-1)}/M_{\tau(n-1)-1}
  \]

  Tenemos que, por el teorema de isomorfía:
  \[
    M_n/M_{n-1}=
    M/M_{n-1}=
    (N_{m-1}+M_{n-1})/M_{n-1}\cong N_{m-1}/(N_{m-1}\cap M_{n-1})
  \]
  que al ser un factor es simple.

  Aplicando la inducción, \(m-1=k+1\) y existe una permutación
  \(\rho\) de \(m-1\) elementos tal que
  \[
    L_i/L_{i-1}\cong N_{\rho(i)}/N_{\rho(i)-1}
  \]
  donde \(i=1,\ldots, n-2\)
  y
  \[
    N_{n-1}/L_{n-2}=N_{n-1}/(M_{n-1}\cap N_{m-1})\cong
    N_{\rho(n-1)}/N_{\rho(n-1)-1}
  \]

  Tenemos ya que \(n=k+2=m\), y si definimos \(\sigma\) la permutación
  de \(n\) elementos:
  \[
    \sigma(i)=\left\{
      \begin{matrix}
        \rho\circ\tau^{-1}(i),
        &i\in\{1,\ldots n-1\}, &\tau^{-1}(i)\in\{1,\ldots, n-2\}\\
        n,
        &i\in\{1,\ldots n-1\}, &\tau^{-1}(i)=n-1\\
        \rho(n-1),
        &i=n &
      \end{matrix}
      \right.
  \]
\end{proof}

\begin{df}[Módulo de longitud finita]
  Un módulo se dice de longitud finita si tiene una serie de composición
  finita o es \(\{0\}\). La longitud es la de cualquiera
  de sus series de composición, o cero si \(M=\{0\}\).
\end{df}

\begin{ejercicio} sea \(M\) un módulo de longitud finita. Se pide demostrar
que si \(0\longrightarrow L\longrightarrow M\longrightarrow N\longrightarrow
0\) es una sucesión exacta corta, entonces:
\[
  \ell(M)=\ell(N)+\ell(M)
\]
Si \(U,V\in\mathcal{L}(M)\), entonces:
\[
  \ell(U+V)=\ell(U)+\ell(V)-\ell(U\cap V)
\]
\end{ejercicio}
Este último resultado se aplica a que si \(\{0\} \longrightarrow N \subset
M \longrightarrow \frac{M}{N} \longrightarrow \{0\}\) implica que
\(\ell(M) = \ell(N) + \ell(\frac{M}{N})\). También se obtiene que
\(\ell(\frac{M}{N}) < \ell(M)\) si \(N \neq \{0\}\). 

\begin{ejercicio}
  Si \(V\) es un \(K\)-espacio vectorial, \(\ell(V)=\dim(V)\).
\end{ejercicio}

\begin{ejemplo}
  \(\ell(\Z_{12})=3\), ya que calculamos antes una serie de
  composición.
\end{ejemplo}

\begin{ejemplo}
  \(\ell(\Z_p)=1\) si \(p\) es primo.
\end{ejemplo}

\begin{ejercicio}
  \(\ell(\Z_n)\) es la suma de los exponentes de su descomposición
  en primos.
\end{ejercicio}

\begin{ejemplo}
  Si \(n=\prod {p_i}^{e_i}\) entonces\(\Z_n\cong
  \Z_{{p_t}^{e_t}}\oplus\cdots \oplus\Z_{{p_t}^{e_t}}\)
\end{ejemplo}
    \begin{cor}
  Dada una sucesión exacta corta, \(0\longrightarrow L
  \longrightarrow M\longrightarrow N\longrightarrow 0\),
  \(L\) y \(N\) admite serie de composición si y solo si
  \(M\) admite serie de composición.
\end{cor}

\begin{cor}
  \(M_1\), \(M_2\) admiten series de composición si y solo si
  \(M_1\oplus M_2\) admite serie de composición.
\end{cor}


\begin{teo}[Jordan-Hölder]
  Supongan que \(M\) admite series de composición:
  \[
    \{0\}=M_0\subsetneq M_1\subsetneq M_2\subsetneq\ldots\subsetneq M_n=M
  \]
  \[
    \{0\}=N_0\subsetneq N_1\subsetneq N_2\subsetneq\ldots\subsetneq N_m=M
  \]
  Entonces \(n=m\) y existe una permutación \(\sigma\) tal que
  \[
    M_i/M_{i-1}\cong N_{\sigma(i)}/N_{\sigma(i)-1}
  \]
\end{teo}
\begin{proof}
  Si \(n=1\), entonces \(M\) es simple y \(m=1\) y el el único factor posible
  es el \(M/\{0\}=M\).

  Si \(n>1\), como \(M\) no es simple, \(m>1\).

  Vamos a observar un caso particular. Supongamos que \(N_{m-1}
  =M_{n-1}\). Por hipótesis de inducción aplicado a \(N_{m-1}\),
  tenemos que  \(n-1=m-1\), luego \(n=m\) y se da el enunciado
  (tomando la permutación \(\sigma\) para los \(n-1\) primeros elementos
  y extendiendola a una permutación de \(n\) elementos \(\sigma'\) tal
  que \(\sigma'(n):=n\), \(\sigma'(k):=\sigma(k)\)).

  Vamos ahora al caso general. Como hemos visto
  en el caso particular anterior, podemos suponer
  \(M_{n-1}\neq N_{m-1}\), por lo que
  \(M_{n-1}+M_{m-1}=M\) (ya que \(M_{n-1}\subsetneq M_{n-1}+
  N_{m-1}\subseteq M\) y \(M_{n-1}\) es maximal).

  Tomamos \(N_{m-1}\cap M_{n-1}\) que admite una serie de composición:
  \[
    \{0\}=L_0\subsetneq L_1\subsetneq\ldots\subsetneq L_k
    =N_{m-1}\cap M_{n-1}
  \]
  y tenemos que, por el teorema de isomorfía:
  \[
    N_m/N_{m-1}=
    M/N_{m-1}=
    (M_{n-1}+N_{m-1})/N_{m-1}\cong M_{n-1}/(M_{n-1}\cap N_{m-1})
  \]
  que al ser un factor es simple.

  Aplicando la inducción, \(n-1=k+1\) y existe una permutación
  \(\tau\) de \(n-1\) elementos tal que
  \[
    L_i/L_{i-1}\cong M_{\tau(i)}/M_{\tau(i)-1}
  \]
  donde \(i=1,\ldots, n-2\)
  y
  \[
    M_{n-1}/L_{n-2}=M_{n-1}/(M_{n-1}\cap N_{m-1})\cong
    M_{\tau(n-1)}/M_{\tau(n-1)-1}
  \]

  Tenemos que, por el teorema de isomorfía:
  \[
    M_n/M_{n-1}=
    M/M_{n-1}=
    (N_{m-1}+M_{n-1})/M_{n-1}\cong N_{m-1}/(N_{m-1}\cap M_{n-1})
  \]
  que al ser un factor es simple.

  Aplicando la inducción, \(m-1=k+1\) y existe una permutación
  \(\rho\) de \(m-1\) elementos tal que
  \[
    L_i/L_{i-1}\cong N_{\rho(i)}/N_{\rho(i)-1}
  \]
  donde \(i=1,\ldots, n-2\)
  y
  \[
    N_{n-1}/L_{n-2}=N_{n-1}/(M_{n-1}\cap N_{m-1})\cong
    N_{\rho(n-1)}/N_{\rho(n-1)-1}
  \]

  Tenemos ya que \(n=k+2=m\), y si definimos \(\sigma\) la permutación
  de \(n\) elementos:
  \[
    \sigma(i)=\left\{
      \begin{matrix}
        \rho\circ\tau^{-1}(i),
        &i\in\{1,\ldots n-1\}, &\tau^{-1}(i)\in\{1,\ldots, n-2\}\\
        n,
        &i\in\{1,\ldots n-1\}, &\tau^{-1}(i)=n-1\\
        \rho(n-1),
        &i=n &
      \end{matrix}
      \right.
  \]
\end{proof}

\begin{df}[Módulo de longitud finita]
  Un módulo se dice de longitud finita si tiene una serie de composición
  finita o es \(\{0\}\). La longitud \(\ell(M)\) es la de cualquiera
  de sus series de composición, o cero si \(M=\{0\}\).
\end{df}

Ejercicio: sea \(M\) un módulo de longitud finita. Se pide demostrar
que si \(0\longrightarrow L\longrightarrow M\longrightarrow N\longrightarrow
0\) es una sucesión exacta corta, entonces:
\[
  \ell(M)=\ell(N)+\ell(M)
\]
Si \(U,V\in\mathcal{L}(M)\), entonces:
\[
  \ell(U+V)=\ell(U)+\ell(V)-\ell(U\cap V)
\]

Ejemplo: si \(V\) es un \(K\)-espacio vectorial, \(\ell(V)=\dim(V)\).

Ejemplo: \(\ell(\Z_{12})=3\), ya que calculamos antes una serie de
composición.

Otro ejemplo: \(\ell(\Z_p)=1\) si \(p\) es primo.

Ejercicio: \(\ell(\Z_n)\) es la suma de los exponentes de su descomposición
en primos.

Ejemplo: si \(n=\prod {p_i}^{e_i}\) entonces\(\Z_n\cong
\Z_{{p_t}^{e_t}}\oplus\cdots
\oplus\Z_{{p_t}^{e_t}}\)

    Sea \(\subscriptbefore{A}{M}\) un módulo, \(\mathcal{L}(M)\)
es el conjunto de todos los submódulos de \(M\).
Dado \(\Gamma\subseteq\mathcal{L}(L)\) no vacío,
tenemos \(\bigcap_{N\in\Gamma}
N\in\mathcal{L}(M)\) (no tiene por qué ocurrir que estén en \(\Gamma\),
\(\bigcap_{n\ge 1} n\Z=\{0\}\notin m\Z\) para ningún \(m\ge 1\)).


\begin{df}[Zócalo]
  \textbf{El zócalo de \(M\)} es el menor submódulo de \(M\) que contiene a todos
  los submódulos simples de \(M\).

  Si \(M\) no tiene ningún submodulo simple, definimos el zócalo como
  \(\{0\}\).

  En ambos casos usaremos la notación \(\Soc(M)\).
\end{df}

\begin{ejemplo}
  Si \(V\) es un \(K\)-espacio vectorial, \(\Soc(V)=V\).
\end{ejemplo}

\begin{ejemplo}
  \(\Soc(\Z)=\{0\}\), puesto que cada \(n\Z\) contiene
  un \(2n\Z\), luego no es simple.
\end{ejemplo}

De hecho, si \(A\) es un dominio de
integridad que no es un cuerpo, \(\Soc(A)=\{0\}\). Tienes que sus submódulos
son ideales. Para \(x\in I\), el ideal generado por \(x^2\) está dentro de
\(I\), luego \(I\) no es simple.

\begin{prop}
  Sea \(M\) un \(A\)-módulo de longitud finita. Entonces existen submódulos
  \(S_i\) simples de \(M\)
  tales que
  \[
    \Soc(M)=S_1\dot{+}\cdots\dot{+} S_n.
  \]
  Además, si existen otros submódulos \(T_i\) simples tales que
  \(
    \Soc(M)=T_1\dot{+}\cdots\dot{+} T_m
  \), entonces \(n=m\) y tras reordenación, \(S_i\cong T_i\).
\end{prop}

\begin{proof}
  Si \(\Gamma\) es el conjunto de todos los submódulos de la forma
  \(
    S_1\dot{+}\cdots\dot{+} S_n
  \)

  Si \(M\neq\{0\}\), entonces \(\Gamma\neq\emptyset\), ya que \(M\)
  contiene algún submódulo simple.

  Como \(M\) es noetheriano, existe un
  \(
    S_1\dot{+}\cdots\dot{+} S_n
  \) maximal.

  \(
    S_1\dot{+}\cdots\dot{+} S_n\subseteq\Soc(M)
  \). Sea \(S\in\mathcal{L}(M)\) simple.
  \[
    S\cap(
    S_1\dot{+}\cdots\dot{+} S_n
    )
  \]
  puesto que \(S\) es simple y la intersección es submódulo, se tiene
  que dicha intersección o es \(\{0\}\) o es \(S\).

  Consideramos
  \[
    S\cap(
    S_1\dot{+}\cdots\dot{+} S_n
    ) = \{0\}
  \]
  luego
  \[
    S\dot{+}S_1\dot{+}\cdots\dot{+} S_n\in\Gamma
  \]
  con lo que no sería maximal.

  Luego se tiene:
  \[
    S\subseteq
    S_1\dot{+}\cdots\dot{+} S_n\in\Gamma
  \]
  luego, como \(S\) era un modulo simple arbitrario, tenemos que
  \(\Soc(M)=
    S_1\dot{+}\cdots\dot{+} S_n
  \).

  Resulta que
  \[
    \{0\}\subsetneq S_1\subsetneq S_1\dot{+} S_2\subsetneq\ldots
    \subsetneq
    S_1\dot{+}\cdots\dot{+} S_n=\Soc(M)
  \]
  es una serie de composición, ya que:
  \[
    (S_1\dot{+}\cdots\dot{+} S_i)/
    (S_1\dot{+}\cdots\dot{+} S_{i-1})\cong
    S_i
  \]
  Aplicando Jordan-Hölder se obtiene el resultado.
\end{proof}

\begin{df}[Módulo semisimple]
  Sea \(M\) un \(A\)-módulo de longitud finita. Decimos que \(M\) es \textbf{semisimple}
  si es \(\Soc(M)=M\).
\end{df}

\begin{ejercicio}
  Sea \(A\) un DIP que no sea un cuerpo,
  \(I\) ideal de \(A\). Se pide demostrar
  que \(A/I\) es de longitud finita si, y solo si, \(I\neq\langle 0\rangle\).
\end{ejercicio}

¿Se puede deducir cuál es la longitud de \(A/I\) de un generador de \(I\)?

\subsection{Módulos de longitud finita sobre un DIP}

Sea de ahora en adelante \(A\) un dominio de ideales principales que no
sea un cuerpo. El objetivo es describir la estructura de los \(A\)-módulos
de longitud finita. Recordemos que \(A\) es un anillo conmutativo cualquiera.

\begin{prop}
  \(\subscriptbefore{A}{M}\) es un módulo de longitud finita si, y solo si,
  \(\subscriptbefore{A}{M}\) es finitamente generado y acotado.
\end{prop}
\begin{proof}
  Supongamos que \(M\) es distinto del 0, porque si no es trivial.

  \(M\) es de longitud finita, por tanto noetheriano, lo que implica que también
  es finitamente generado: \(M=Am_1+\cdots+Am_n\), con \(m_i\in M\) para cada
  \(i \in \{1, \ldots, n\).
  \[
    \langle\mu\rangle=\Ann_A(M)=\bigcap_{i=1}^n\ann_A(m_i)
  \]
  porque el anillo \(A\) es conmutativo, donde ademas cada anulador
  de cada elemento es un ideal (a izquierdas en un conmutativo, luego ideal).

  Sea \(\langle f_i\rangle=\ann_A(m_i)\), entonces
  \[
    \langle\mu\rangle=\bigcap_{i=1}^n\langle f_i\rangle
  \]
  donde \(\mu=\mcm\{f_i:1\le i\le n\}\).

  Veamos que \(f_i\neq 0\) para cada \(i\).
  \[
    M\subseteq Am_i\cong A/\langle f_i\rangle
  \]
  luego \(\ell(Am_i)<\infty\), como \(A\) no es un cuerpo y por
  tanto \(M\) no es artiniano, entonces
  \(\langle f_i\rangle\neq0\).

  Luego \(\langle\mu\rangle\neq 0\) y por tanto \(M\) es acotado.

  Veamos el recíproco: \(M\) acotado y finitamente generado.
  \[
    M=Am_1+\cdots+Am_n
  \]

  Vemos que cada \(Am_i\) es de longitud finita (\(\mu\neq 0\) por ser
  acotado, luego cada \(\langle f_i\rangle\neq 0\)).
  Tenmos que \(Am_i\cong A/\langle f_i\rangle\) es de longitud finita.

  Existe un epimorfismo entre \(Am_1\oplus\cdots\oplus Am_n\) (que es
  de longitud finita) y \(Am_1\oplus\cdots\oplus Am_n\), con lo que
  el segundo tiene longitud finita.
\end{proof}

\(\ell_A(M)<\infty\), entonces es acotado, o sea
\(\langle\mu\rangle=\Ann_A(M)=\langle 0\rangle\). Entonces
\[
  M=M_1\dot{+}\cdots\dot{+} M_t
\]
donde \(M_i\) es la componente \(p_i\) primaria que viene de
\(\mu={p_1}^{e_1}\cdots{p_t}^{e_t}\)
(\(M_i=\{m\in M:m\cdot{p_i}^{e_i}=0\}\)).
Además \(M_i\) es finitamente generado.
¿Se puede descomponer como suma directa de submódulos indescomponibles?


\[
  M=M_1\dot{+}\cdots\dot{+} M_t
\]
donde
\[
  M_i=\{q_i m: m\in M\}=\{m\in M: p_i^{e_i} m=0\}=\{m\in M:
  a_i q_i m = m\}
\]
con \(q_i=\frac{\mu}{p_i^{e_i}}\) y \(\sum_i a_i q_i=1\)
y \(\langle \mu\rangle=\Ann_A(M)\). Se tiene que
\(\Ann_A(M_i)=\langle p_i^{e_i}\rangle\).
    \begin{df}[Módulo p-primario]
  \(\subscriptbefore{A}{M}\) se dice \textbf{\(p\)-primario} si
  \(\Ann_A(M)=\langle p^e\rangle\), \(p\) un irreducible con \(e \ge 1\).
\end{df}

\begin{obs}
  Sea \(\subscriptbefore{A}{M}\) un módulo \(p\)-primario con \(\ell(M) < \infty\).
  \[
    \Ann_A(M)=\langle p^t\rangle
  \]

  Si \(0\neq m\in M\), el ideal principal \(\ann_A(m)\supseteq \Ann_A(M)=\langle p^t
  \rangle\) y tenemos que \(\ann_A(m)=\langle p^r\rangle \),
  con \(r\le t\).

  Si \(M=Am_1+\cdots+Am_m\), entonces \(\langle p^t\rangle
  =\ann_A(m_1)\cap\ldots\cap\ann_A(m_m)\). Luego
  \(\langle p^t\rangle =\ann_A(m_i)\) para algún \(i\).
\end{obs}

\begin{cor}
  Existe un \(x \in M\) tal que \(\Ann_A(M)=\ann_A(x)\).
\end{cor}

\begin{lema}
  \(\ell(M)<\infty\), \(M\) \(p\)-primario. Para \(0\neq m\in M\),
  entonces:
  \[
    Am\textrm{ es simple} \iff \ann_A(m)=\langle p\rangle
  \]
  y como consecuencia
  \[
    \Soc(M)=\{m\in M: pm=0\}
  \]
\end{lema}
\begin{proof}
  Dado \(m\), tenemos \(Am\cong A/\ann_a(m)\). Si \(Am\) es simple,
  entonces \(\ann_A(m)\) es ideal maximal (generado por irreducible o ideal
  primo) y
  \(\ann_A(m)\supseteq\Ann_A(M)=\langle p^t\rangle\).
  Entonces \(\ann_A(m)=\langle p\rangle\).

  Recíprocamente, si \(\ann_A(m)=\langle p\rangle\) entonces
  \(Am\cong A/\langle p\rangle\) es simple.

  \(\Soc(M)=S_1\dot{+}\cdots\dot{+}S_n\) con \(S_i\) simple.
  Sea \(m\) en el zócalo, \(\ann_A(m)
  \supseteq\Ann_A(S_1\dot{+}\cdots\dot{+}S_n)=
  \bigcap_{k=1}^{n} \Ann_A(S_k)\). Tomamos \(s_i\) tal que
  \(\Ann_A(S_i)=\ann_A(s_i)\), tenemos que \(S_i=As_i\), luego
  \(As_i\cong A/\ann_A(s_i)\) y es simple, luego
  \(\ann_A(s_i)=\langle p\rangle\),
  tenemos que \(\ann_A(m)\supseteq\langle p\rangle\) y finalmente
  \(pm=0\).

  Tomamos ahora \(m\in M\) tal que \(pm=0\). \(\langle p\rangle
  \subseteq\ann_A(m)\) pero es maximal, luego se da la igualdad.
  \[
    Am\cong A/\ann_A(m)= A/\langle p\rangle
  \]
  luego es simple, y \(Am\subseteq\Soc(M)\) y en particular
  \(m\in\Soc(M)\).

\end{proof}

\begin{prop}
  Suponemos que tenemos \(M\) un módulo \(p\)-primario y de longitud finita. Sea
  \(x\in M\) tal que \(\Ann_A(M)=\ann_A(x)\). Entonces \(Ax\) es un
  sumando directo interno de \(M\).
\end{prop}
\begin{proof}
  Por inducción sobre la longitud \(\ell(M)<\infty\).

  Si la longitud es 1, \(M\) es simple y entonces \(M=Ax\).

  Si \(\ell(M)>1\) y \(Ax=M\), no hay nada que demostrar.

  Veamos que pasa si \(Ax\neq M\). Veamos que existe un \(y \in M\)
  tal que \(y\neq Ax\) y \(\ann_A(y)=\langle p\rangle\).
  \(\ell(M/Ax) < \ell(M) < \infty\), luego debe contener algún
  submódulo simple \(S \subseteq M/Ax\).
  Tomamos \(s\in S\) tal que \(S=As\).
  \[
    \langle p^t\rangle =\Ann_A(M)\subseteq\Ann_A(M/Ax)\subseteq\Ann_A(S)
    =\ann_A(s)
  \]
  El \(\ann_A(s)\) es un ideal maximal generado por un irreducible, y como
  \(\langle p^t\rangle \subset \ann_A(s)\), \(\ann_A(s)=\langle p\rangle\).

  Tomamos \(z\in M\) tal que \(s = z + Ax\), es decir, \(pz \in Ax\) ya que
  \(0 + Ax = ps = pz + Ax\). Equivalentemente,
  existe un cierto \(a \in A\) tal que \(pz = ax\). Afirmamos que \(p|a\) (no es obvio
  porque es un módulo).

  Supongamos que no es así. Por Bezout, \(1=ua+vp\) para \(u,v\in A\)
  adecuados. En dicho caso, \(x=uax+vpx=upz+vpx=p(uz+vx)\).
  \[
    \ann_A(uz+vx)=\langle p^{t'}\rangle
  \]
  para \(t'\le t\). Se deduce que \(p^{t'-1} x=0\). \(p^{t-1}x=0\), y
  entonces como el anulador de \(x\) es el de \(M\) y está generado
  por \(p^t\), no puede anularlo \(p^{t'-1}\) ya que
  \(t'-1\le t-1<t\).

  Una cuenta alternativa podrá haber sido que \(p^{t-1}ax=p^t z=0\), entonces \(p^{t-1} a
  \in\ann_A(x)=\langle p^t\rangle\), y tenemos que \(a=pa'\).

  Hemos obtenido un elemento \(s=z+Ax\in M/Ax\) tal que \(pz=ax\) y hemos
  visto que \(p|a\). Así tenemos que \(pz=pa'x\) y entonces
  \(p(z-a'x)=0\). Sea \(y = z - a'x \neq 0\) y \(py=0\), con lo que
  \(\ann_A(y)=\langle p\rangle\).

  Tenemos que \(Ay\) es simple y \(y\notin Ax\) asi que \(Ay\cap Ax=
  \{0\}\).
  \[
    Ax\cong Ax/(Ay\cap Ax)\cong (Ax+Ay)/Ay\cong A(x+Ay)\subseteq M/Ay
  \]
  La segunda isomorfía se verifica gracias al segundo o tercer Teorema de Isomorfía,
  propuesto como ejercicio. 

  Empleando la expresión anterior, 
  \[
    \langle p^t\rangle=\ann_A(x)=\ann_A(A(x+Ay))\supseteq
    \Ann_A(M/Ay)\supseteq\Ann_A(M)=\langle p^t\rangle
  \]
  con lo cual todas las inclusiones son igualdades.

  Tenemos que \(\Ann_A(M/Ay)=\langle p^t\rangle=\ann_A(x+Ay)\), que
  están en las mismas condiciones de la hipótesis pero
  con \(\ell(M/Ay)<\ell(M)\). Aplicando la hipótesis de inducción,
  tenemos que \(M/Ay=(Ax+Ay)/Ay \dot{+} N/Ay\) para cierto
  \(N\in\mathcal{L}(M)\) tal que \(N\supseteq Ay\). De aquí se deduce
  que \(M=Ax+Ay+N=Ax+N\).

  Veamos ahora que es suma directa. Tomamos \(Ax\cap N\subseteq(Ax+Ay)\cap N=Ay\).
  Entonces \(Ax\cap N = Ax\cap N\cap Ay= Ax\cap Ay=\{0\}\).
\end{proof}

    \begin{teo}
  Sea \(\subscriptbefore{A}{M}\) \(p\)-primario de longitud finita. Existen
  \(x_1,\ldots, x_n\in M\setminus\{0\}\) tales
  que \(M=Ax_1\dot{+}\cdots\dot{+} Ax_n\) y
  \[
    \Ann_A(M)=\ann_A(x_1)\supseteq\ann_A(x_2)\supseteq\ldots
    \supseteq\ann_A(x_n)
  \]
  Además, si \(y_1,\ldots, y_n\in M\) no nulos son tales que
  \(
    M=Ay_1\dot{+}\ldots\dot{+} Ay_n
  \)
  y
  \(
    \Ann_A(M)=\ann_A(y_1)\supseteq\ann_A(y_2)\supseteq\ldots
    \supseteq\ann_A(y_m)
  \), entonces \(n=m\) y \(\ann_A(x_i)=\ann_A(y_i)\).
\end{teo}
\begin{proof}
  Tomo \(x_1\in M\) tal que \(\Ann_A(M)=\ann_A(x)\), por la proposición,
  \(M=Ax_1\dot{+} N\) para cierto submódulo \(N\) de \(M\).
  Es claro que \(\Ann_A(N)\supseteq\Ann_A(M)=\langle p^t\rangle\),
  con lo que \(\Ann_A(N)=\langle p^{t'}\rangle\) con \(t'\le t\) y
  \(\ell(N)<\ell(M)\).

  Por inducción sobre \(\ell(M)\), tenemos \(x_1,x_2,\ldots, x_n\in N\)
  y \(N=Ax_2\dot{+}\cdots\dot{+} Ax_n\).
  De esto se deduce
  \[
    M=Ax_1\dot{+}\cdots\dot{+} Ax_n
  \]
  y \(\ann_A(x_1)=\Ann_A(M)\subseteq\ann_A(x_2)\subseteq\ldots\subseteq
  \ann_A(x_n)\).

  Veamos la unicidad. Hacemos inducción sobre \(\ell(M)\).

  Si \(\ell(M)=1\), tenemso que es simple y \(M=Ax=Ay\) y \(n=1=m\).

  Si \(\ell(M)>1\), tenemos que \(M\) no es simple. Consideramos
  \(M/pM\) donde \(pM:=\{pm:m\in M\}\) que es un submódulo por ser
  \(A\) conmutativo. \(\Ann_A(pM)=\langle p\rangle\).
  \[
    \Soc(M/pM)=M/pM
  \]
  luego \(M/pM\) es semisimple.

  Tengo un homomorfismo de módulos \(M\longrightarrow
  Ax_1/Apx_1\oplus\cdots
  Ax_n/Apx_1
  \) tal que \(\sum A-ix_i\mapsto (a_1 x_1+Apx_1,\ldots,
  a_n x+Apx_n)\).

  Se puede demotrar que dicha aplicación es sobreyectivo y su núcleo es
  \(pM\).
  \[
    M/pM\cong
    Ax_1/Apx_1\oplus\cdots
    Ax_n/Apx_1
  \]
  \(n = \ell(M/pM)\). Argumentando de forma análoga para \(y\);
  obtenemos \(n= \ell(M/pM)=m\).

  Si \(pM=\{0\}\), tenemos que todos los anuladores son iguales:
  \(\ann_A(x_i)=\langle p\rangle=\ann_A(y_i)\).

  Supongamos que \(pM\neq\{0\}\).
  \[
    pM=Apx_1\dot{+}\cdots\dot{+} Apx_r
  \]
  para cierto \(r\le n\).

  Así, \(\ann_A(x_i)=\langle p\rangle\) si solo si \(i>r\).
  y también \(\ann_A(y_i)=\langle p\rangle\) si solo si \(i>r\).
  Para cualquier \(i\le r\), tenemos que \(\ann_A(px_i)=\langle
  p^{t_i-1}\rangle\) si \(\ann_A(x_i)=\langle p^{t_i}\rangle\).

  \[
   \ann_A(px_1)\supseteq\ann_A(px_2)\supseteq\ldots
    \supseteq\ann_A(px_r)
  \]
  \[
   \ann_A(py_1)\supseteq\ann_A(py_2)\supseteq\ldots
    \supseteq\ann_A(py_s)
  \]
  donde \(\ann_A(y_i)=\langle p^{s_i}\rangle\) si y solo si \(i>s\).
  Pero \(\ell(pM)<\ell(M)\), por inducción \(s=r\) y que \(s_i-1=
  r_i-1\) y como sabemos que si \(i>r=s\) tenemos que
  \(\ann_A(x_i)=\ann(y_i)=\langle p\rangle\).
\end{proof}

    \begin{obs}
  Si \(A=\Z\), \(M\) grupo abeliano, \(x\in M\),
  \(\ann_\Z(x)=n\Z\), \(n\) recibe el nombre de el orden.
\end{obs}

\begin{obs}
  Si \(A=K[x]\), \(T:V\longrightarrow V\), \(n=\dim_K V<\infty\),
  \(v\in V\), \(\ann_{K[x]}(v)=\langle f(x)\rangle\).
  Tenemos que \(f\) tiene grado \(n\). \(\{v, Tv,\ldots, T^{n-1}v\}\)
  es una base de \(V\).
\end{obs}

\begin{ejemplo}
  \(\mathcal{U}(\Z_8)=\{1,3,5,7\}\).
  Viendo los ordenes de los elementos:
  \[\mathcal{U}(\Z_8)=\langle 3\rangle\dot{+}\langle 5\rangle\]
  donde \(\langle \cdot\rangle\) es la generación como subgrupo.
\end{ejemplo}

\begin{ejemplo}
  Suponemos un espacio vectorial \(V\) de dimensión 3 y un
  endomorfismo \(T\) cuyo polinomio mínimo es de la forma
  \({(x-\lambda)}^2\) con \(\lambda\in K\).
  Sabemos que existen dos vectores \(v_1,v_2 \in V\) tales que
  \[
    V=K[x]v_1\dot{+} K[x]v_2
  \]
  con \(\ann_{K[x]} v = \langle {(x-\lambda)}^2\rangle
  \subsetneq\langle {x-\lambda}\rangle = \ann_{K[x]} v_2\).
\end{ejemplo}

\begin{cor}
  Si \(\subscriptbefore{A}{M}\) es un módulo \(p\)-primario, entonces existen
  \(C_1, \ldots, C_n\) módulos cíclicos tales que
  \[
    M\cong C_1\oplus\cdots\oplus C_n.
  \]

  Si existen otros \(D_1, \ldots, D_m\) módulos cíclicos tales que \(M\cong D_1\oplus
  \cdots\oplus D_m\), entonces
  \(n=m\) y tras reordenación, \(D_i\cong C_i\) para todo \(i \in \{1, \ldots, n\).
\end{cor}
\begin{proof}
  De \(M\cong C_1\oplus\cdots\oplus C_n\), se puede exigir que
  \(x_1,\ldots,x_n\in M\) tales que
  \[
    M=Ax_1\dot{+}\cdots\dot{+}Ax_n
  \]
  con \(\ann_A(x_1)\subseteq\ann_A(x_2)\subseteq\ldots\subseteq\ann_A(x_n)\)

  Con \(D_1\oplus\cdots\oplus D_m\) hago lo mismo.
  \[
    M=Ay_1\dot{+}\cdots\dot{+}Ay_n
  \]
  ordenados bajo el mismo criterio.

  El enunciado se sigue de aplicar el teorema anterior. De
  \(\ann(x_i)=\ann(y_i)\) se deduce
  \[
    C_i\cong Ax_i\cong A/\ann(x_i)=A/\ann(y_i)\cong Ay_i\cong D_i
  \]
\end{proof}

\begin{ejercicio}
  Decimos que un módulo \(M\) es indescomponible si \(M\cong
  L\oplus N\) implica que \(L=\{0\}\) (o \(N=\{0\}\)).
  Razonar que en el corolario cada uno de los \(C_i\) es indescomponible.
\end{ejercicio}

\begin{ejemplo}
  \(M\) grupo abeliano de longitud finita y \(p\)-primario.
  Aplicando el corolario, \(M\cong C_1\oplus\cdots\oplus C_n\) con
  \(C_i\) cíclico y de longitud finita \(p\)-primarios.
  Tenemos que \(M\cong \Z_{p^{m_1}}\oplus\cdots\oplus\Z_{p^{m_n}}\),
  \(M\) es finito de cardinal \(p^{m_1+\cdots+m_n}\).
\end{ejemplo}

\begin{teo}[Estructura de módulos sobre un DIP]
  Sea \(A\) un dominio de integridad primaria, \(M\) un grupo aditivo y el módulo
  \(\subscriptbefore{A}{M}\neq\{0\}\) de longitud finita. Entonces existen
  irreducibles distintos \(p_1,\ldots,p_r\in A\) y 
  \(n_1,\ldots, n_r \in \N\) tales que \(e_{i1}\ge\ldots\ge e_{in_i}\) con
  \(i\in\{1,\ldots,r\}\) que determinan \(M\):
  \[
    M=\dotplus_{i=1}^r\left(\dotplus_{j=1}^{n_i} Ax_{ij}\right),
  \]
  donde los \(x_{ij}\in M\) verifican:
  \[
    \ann_A(x_{ij})=\langle p_i^{e_{ij}}\rangle
  \]
  con \(i\in\{1,\ldots,r\}\) y \(j\in\{1,\ldots, n_i\}\).
  A esta expresión
  se le llama la \textbf{descomposición cíclica-primaria de \(M\)} (la
  primaria sería la primera suma y luego cada factor primario se
  descompone en factores cíclicos).
  A los \(x_{ij}\) se les llama
  \textbf{divisores elementales de \(M\)} y determinan \(M\) salvo isomorfismos.
\end{teo}

\begin{proof}
  Sea \(\mu \in A\) tal que \(\langle \mu \rangle = \Ann_A(M)\). Tomamos
  \(\mu = p_1^{e_1} \cdots p_r^{e_r}\), con \(p_1, \ldots, p_r \in A\) irreducibles.
  Entonces \(M = M_1 \dotplus \cdots \dotplus M_r\) (descomposición primaria) para
  \(M_i\) un submódulo \(p_i\)-primario. Para cada \(i \in \{1, \ldots, r\}\),
  \(M_i = \dotplus^{m_i}_{j=1} Ax_{ij}\)(descomposición cíclica de un primario),
  con \(\Ann_A(M_i) = \ann_A(x_{i1}) \subset \cdots \subset \ann_A(x_{in_i})\)
  para ciertos \(x_{ij} \in M_i\) y \(j \in \{1, \ldots, n_i\}\).

  De hecho, \(\Ann_A(M_i) = \langle p_i^{e_i} \rangle = \ann_A(x_{i1}) \subset \ldots
  \subset \ann_A(x_{in_i}) = \langle p_i^{e_{n_i}} \rangle\).
  Así, se toma la sucesión \(\{e_{ij} \ : \ j \in \{1, \ldots, n_i\}\}\) que viene dada
  por \(\ann_A(x_{ij}) = \langle p_i^{e_{ij}} \rangle \), con \(j = 1, \ldots, n_i\).
  
  Supongamos otra descomposición:
  \[
    M=N_1\dotplus \cdots \dotplus N_t
  \]
  con \(N_i\) \(s_i\)-primario para \(s_1,\ldots,s_t\in A\) irreducibles.
  Entonces
  \[
    \left\langle \mu\right\rangle =\Ann_A(M)=\bigcap_{i=1}^t \Ann_A(N_i)
    =\bigcap_{i=1}^t\left\langle s_i^{t_i}\right\rangle
    = \left\langle\mcm\{s_i^{t_i}\}\right\rangle=\left\langle\prod s_i^{t_i}\right\rangle
  \]
  y \(\mu\) es asociado con \(s_1^{t_1}\cdots s_t^{t_t}\).
  Tras reordenación, por ser \(A\) un DFU, \(t=r\) y \(s_i=p_i\).

  \(N_i\subseteq\{m\in M:p_i^{e_i} m=0\}=M_i\), entonces
  \(N_i=M_i\), argumentando sobre las longitudes.
\end{proof}

    \begin{proof}
  Supongamos otra descomposición:
  \[
    M=N_1\dot{+} N_t
  \]
  con \(N_i\) \(s_i\)-primario para \(s_1,\ldots,s_t\in A\) irreducibles.
  Entonces
  \[
    \left\langle \mu\right\rangle =\Ann_A(M)=\bigcap_{i=1}^t \Ann_A(N_i)
    =\bigcap_{i=1}^t\left\langle s_i^{t_i}\right\rangle
    = \left\langle\mcm\{s_i^{t_i}\}\right\rangle=\left\langle\prod s_i^{t_i}\right\rangle
  \]
  y \(\mu\) es asociado con \(s_1^{t_1}\cdots s_t^{t_t}\).
  Tras reordenación, por ser \(A\) un DFU, \(t=r\) y \(s_i=p_i\).

  \(N_i\subseteq\{m\in M:p_i^{e_i} m=0\}=M_i\), entonces
  \(N_i=M_i\), argumentando sobre las longitudes.

\end{proof}

\begin{obs}
  Sea \(M\) un grupo abeliano de longitud finita, \(A=\Z\).
  Los grupos abelianos son de longitud finita si y solo si son
  finitos.
\end{obs}
\begin{proof}
  \(\mu=p_1^{e_1}\cdots p_r^{e_r}\)
  \[
    M=\dot{+}_{i=1}^r\dot{+}_{j=1}^{n_i}\Z x_{ij}\cong
    \oplus_{i=1}^r\oplus_{j=1}^{n_i}\Z_{p_i^{e_{ij}}}
  \]
  con \(x_{ij}\). Luego es finito de cardinal:
  \[
    m=\prod_{i=1}^r\prod_{j=1}^{n_i} p_i^{e_{ij}}
    =p_1^{f_1}\cdots p_r^{f_r}
  \]
  donde \(f_i=\sum_{j=1}^{n_i} e_{ij}\).

  \(\mu| m\).

\end{proof}

Ejemplo: si \(m=12\), \(p_1=2\) y \(p_2=3\). Entonces \(M\cong \Z_4\oplus
\Z_3\cong\Z_{12}\) o \(M\cong \Z_2\oplus
\Z_2\oplus\Z_3\cong\Z_2\Z_{6}\).

Ejemplo: \(A=K[x]\) y \(V\) un \(K[x]\)-módulo de longitud finita.
\(V\) es dimensión finita:
\[
  V=\dot{+}_{i=1}^r\dot{+}_{j=1}^{n_i} K[x]x_{ij}
\]
luego es suma directa de espacios de dimensión finita.

\[
  V_{ij}=K[x]x_{ij}\subseteq V
\]
donde \(T(V_{ij})\subseteq V_{ij}\). Tenemos que
\[
  \minpol(\left.T\right|_{V_{ij}})=p_i^{e_{ij}}
\]
existen \(x_{ij}\) tales que \(\{x_{ij},Tx_{ij},\ldots,
T^{\dim V -1}x_{ij}\}\) base de \(V_{ij}\).

Caso particular: \(\dim V=n\), \(\minpol(T)={(x-\lambda)}^n\).
Existe un \(v\in V\) tal que
\[
  \{v, (T-\lambda)v,\ldots,{(T-\lambda)}^{n-1} v\}
\]

Aplicamos \(T{(T-\lambda)}^i v=(T-\lambda+\lambda){(T-\lambda)}^i v=
{(T-\lambda)}^{i+1}v+\lambda{(T-\lambda)}^i v\).
La matriz asociada es:
\[
  M_B(T)=
  \begin{pmatrix}
    \lambda&1&0&0&\cdots&0\\
    0&\lambda&1&0&\cdots&0\\
    0&0&\lambda&1&\cdots&0\\
    \vdots&\vdots&\vdots&\vdots&\cdots&\vdots\\
    0&0&0&0&\cdots&1\\
    0&0&0&0&\cdots&\lambda\\
  \end{pmatrix}
\]
A matrices de este tipo las llamaremos bloque de Jordan.

Si le aplicamos al caso general en el que
\(\mu={(x-\lambda_1)}^{e_1}\cdots{(x-\lambda_r)}^{e_r}\). Tomamos en
cada \(V_{ij} =K[x]x_{ij}\) la base \(\{x_{ij},\ldots,
{(T-\lambda)}^{e_{ij}-1} x_{ij}\}\) y obtenemos uniendo ordenadamente las
bases una base de \(V\), llámase \(B\), tal que por bloques se expresa:
\[
  M_B(T)=
  \begin{pmatrix}
    J_{e_{ij}}(\lambda_i)&0&0&0&\cdots&0\\
    0&J_{e_{ij}}(\lambda_i)&0&0&\cdots&0\\
    0&0&J_{e_{ij}}(\lambda_i)&0&\cdots&0\\
    \vdots&\vdots&\vdots&\vdots&\cdots&\vdots\\
    0&0&0&0&\cdots&0\\
    0&0&0&0&\cdots&J_{e_{ij}}(\lambda_i)\\
  \end{pmatrix}
\]

    \begin{ejemplo}
  Sea \(V={\Cont^\infty(\R)}^n=\bigoplus_{i=1}^n\Cont^\infty(\R)\),
  \(B\in\mathcal{M}_n(\R)\), \(y=(y_1,\ldots,y_n)\in V\).
  Tenemos la ecuación diferencial \(y'=yB\).

  Sea \(M=\{y\in{\Cont^\infty(\R)}^n: y'=yB\}\) es un subespacio vectorial
  de \(V\). Entonces \(V\) es un \(\R[x]\)-módulo. Sabemos que \(M\)
  es un submódulo (\(xy=y'=yB\in M\)). Por análisis, sabemos que
  la dimensión es finita. Entonces \(M\) tiene una descomposición cíclica
  primaria.

  Si \(x\in\R^n\), tomamos \(y=xe^{tB}\) y \(y'=xe^{tB}B=yB\)
  donde \(e^S=\sum_{m\ge 0}\frac{1}{m!}S^m\).

  Tomamos la forma canónica de Jordan \(J\) de \(B\). Existe una matriz
  \(P\in\mathcal{GL}_n(\C)\) tal que \(PBP^{-1}=J\) con lo que:
  \[
    e^{tB}=P^{-1}e^{tJ}P
  \]

  Se puede calcular \(e^{tJ}\).

  Caso particular: Sea \(n=2\). Sea \(\mu\) el polinomio mínimo de \(B\)
  sobre \(\C\). Tenemos tres casos.

  La primera posibilidad es que \(\mu=(x-\lambda_1)(x-\lambda_2)\)
  o \(\mu=x-\lambda\). En este segundo caso tomamos \(\lambda_1
  =\lambda_2=\lambda\) y en cualquiera de las dos posibilidades
  podemos escribir:
  \[
    J=\begin{pmatrix}
        \lambda_1& 0\\
        0&         \lambda_2
      \end{pmatrix}
  \]
  y por tanto
  \[
      e^{tJ}=\begin{pmatrix}
               e^{t\lambda_1}& 0\\
               0&         e^{t\lambda_2}
             \end{pmatrix}
           \]

           La otra posibilidad es que \(\mu={(x-\lambda)}^2\) con \(\lambda\in\R\).
           entonces:
           \[
             J=\begin{pmatrix}
                 \lambda& 1\\
                 0&         \lambda
               \end{pmatrix}
             \]
             y por tanto
             \[
               tJ=\begin{pmatrix}
                    {t\lambda_1}& t\\
                    0&         {t\lambda_2}
                  \end{pmatrix}=
                  \begin{pmatrix}
                    {t\lambda_1}& 0\\
                    0&         {t\lambda_2}
                  \end{pmatrix}
                  +
                  \begin{pmatrix}
                    0& t\\
                    0& 0
                  \end{pmatrix}
                  =
                  tA+tC
                \]
                que son dos matrices que conmutan, luego:
                \[
                  e^{tJ}=
                  e^{tA+tC}=
                  e^{tA}e^{tC}
                  =
                  \begin{pmatrix}
                    e^{t\lambda_1}& te^{t\lambda_2}\\
                    0&         e^{t\lambda_2}
                  \end{pmatrix}
                \]

                Por último puede suceder que \(\mu=(x-z)(x-\bar{z})\) y tenemos
                \[
                  J=\begin{pmatrix}
                      z & 0\\
                      0&  \bar{z}
                    \end{pmatrix}
                  \]
                  y por tanto
                  \[
                    e^{tJ}=\begin{pmatrix}
                             e^{tz}& 0\\
                             0&         e^{t\bar{z}}
                           \end{pmatrix}
                         \]

                         Alternativamente \(\mu=x^2+bx+c\), tenemos que
                         \(\alpha=\sqrt{\frac{c-b^2}{4}}\) y \(\beta=-\frac{b}{2}\).
                         Tenemos que \(T:\R^2 \longrightarrow\R^2\) tal que \(T(v)=vB\).
                         Tomamos \(v\in\R^2\setminus\{0\}\) y tomamos la base:
                         \(\mathcal{B}=\{-\beta v,{(T-\alpha)}v\}\).
                         Vamos a calcular la matriz de \(T\)
                         respecto de esta nueva base:
                         \[
                           T(-\beta v)=-\beta{(T-\alpha)}v-\alpha\beta v
                         \]
                         \[
                           T((T-\alpha v))=\cdots=\alpha{(T-v)}v-\beta^2 v
                         \]

                         Entonces
                         \[
                           C=M_T(\mathcal{B})=
                           \begin{pmatrix}
                             \alpha&-\beta\\
                             \beta&\alpha\\
                           \end{pmatrix}=
                           \begin{pmatrix}
                             \alpha&0\\
                             0&\alpha\\
                           \end{pmatrix}+
                           \begin{pmatrix}
                             0&-\beta\\
                             \beta&0\\
                           \end{pmatrix}
                           =A+B
                         \]
                         que conmutan. Además
                         existe \(Q\in\mathcal{GL}_2(\R)\) tal que \(C=Q^{-1}BQ\). Tenemos que:
                         \[
                           e^{tC}=e^{tA+tB}=e^{tA}e^{tB}=
                           \begin{pmatrix}
                             e^{t\alpha}&0\\
                             0&e^{t\alpha}\\
                           \end{pmatrix}
                           \begin{pmatrix}
                             \cos(\beta t)& -\sin(\beta t)\\
                             \sin(\beta t)&\cos(\beta t) \\
                           \end{pmatrix}=
                           \begin{pmatrix}
                             e^{t\alpha}\cos(\beta t)& -e^{t\alpha}\sin(\beta t)\\
                             e^{t\alpha}\sin(\beta t)&e^{t\alpha}\cos(\beta t) \\
                           \end{pmatrix}
                         \]
\end{ejemplo}

\begin{ejercicio}
  Tomamos la sucesión \(c_k=\cos(k\nu)\) con \(\nu\in\R\) fijo.
  \[
    c_k=\frac{e^{ik\nu}+e^{-ik\nu}}{2}
  \]
  usando este hecho, demostrar que \(\cos((k+2)\nu)=
  2\cos((k+1)\nu)\cos\nu-\cos k\nu\) para \(k\ge0\). Se pide buscar
  el polinomio mínimo de la sucesión en \(\C[x]\).
\end{ejercicio}
    \section{Teoría de módulos}

Sea \(R\) un anillo, \(\subscriptbefore{R}{M}\) un módulo. Sea la
familia no vacía de submódulos
\(\Gamma\subseteq\mathcal{L}(M)\) entonces
\(\bigcap_{N\in\Gamma} N\in\mathcal{L}(M)\).

\begin{df}[Submódulo generado por un conjunto \(X\)]
  Si \(X\) es un subconjunto de \(M\), el menor submódulo de \(M\) que contiene
  a \(X\) se llama submódulo generado por \(X\). Lo denotaremos por \(RX\).
\end{df}

\begin{lema}
  \[
    RX=\left\{\sum_{x\in F}v_x x:F\subseteq X \textrm{ finito}, v_x\in
    R\right\}
  \]
\end{lema}
\begin{proof}
  \(X\subseteq RX\) por ser el menor submódulo que contiene a \(X\).

  \[
    C=\left\{\sum_{x\in F}v_x x:F\subseteq X \textrm{ finito}, v_x\in
    R\right\}
  \]

  Entonces \(C\subseteq RX\).
  Tenemos que, como \(C\) es un submódulo, se tiene que dar la igualdad.

\end{proof}

Si \(X=\{x_1,\ldots, x_n\}\), tenemos que \(RX=Rx_1+\cdots Rx_n\).

\begin{df}[Módulo producto]
  Tomamos \(I\neq\emptyset\) un conjunto de índices, tal que
  \(i\in I\), tomamos un módulo \(M_i\).
  \[
    \prod_{i\in I} M_i=\{{(m_i)}_{i\in I}:m_i\in M_i\}
  \]

  Son tuplas, pero no ordenadas.
\end{df}

\begin{prop}
  El producto de módulos es un módulo, con la suma término a término y el
  producto por escalares también término a término.
\end{prop}

\begin{df}[Proyecciones e inclusiones canónicas]
  Vamos a tomar \(M_i\)y \( \prod_{i\in I} M_i\). Definimos
  la inclusión canónica \(\iota_i\)  mediante la
  aplicación que asigna \(m_i\mapsto {(a_j)}_{j\in I}\) dado por
  \(a_j=\delta_i^j m_i\).
  Del mismo modo, definimos la proyección canónica \(\pi_i\) como
  la aplicación que asigna \({(a_j)}_{j\in I}\mapsto a_i\).

  Evidentemente \(\pi_i\circ\iota_i=\id\).
\end{df}

\begin{df}[Suma directa externa]
  \[
    \bigoplus_{i\in I} M_i:=\{{(m_i)}_{i\in I}: \textrm{ tiene soporte
    finito}\}
  \]
\end{df}


En el caso de \(I\) finito \(\bigoplus_{i\in I} M_i=\prod_{i\in I} M_i\),
y en el caso general \(\bigoplus_{i\in I} M_i\subseteq\prod_{i\in I} M_i\)

\begin{df}[Suma de módulos]
  Definimos \(\sum_{i\in I} M_i\) como el menor submódulo que contiene
  a cualquier \(M_i\) o equivalentemente:
  \[
    \sum_{i\in I} M_i=\left\{\sum_{i\in F} m_i: F\subseteq I \textrm{ finito}\right\}
  \]
\end{df}

\begin{prop}[Relación entre sumas]
  Tomamos \(\theta:\bigoplus M_i\longrightarrow \sum M_i\) tal que
  \(\theta({(m_i)}_{i\in I})=\sum_{i\in I} m_i\) es un homomorfismo
  sobreyectivo de \(R\)-módulos.

  Para \(\{N_i:i\in I\}\subseteq\mathcal{L}(M)\), son equivalentes:
  \begin{enumerate}
    \item Para todo \(j\in I\), \(N_j\cap\sum_{i\in I\setminus\{j\}} N_i
      =\{0\}\).
    \item Para todo \(F\subseteq I\) finito, y para todo \(j\in F\),
      \(N_j\cap\sum_{i\in F\setminus\{j\}} N_i
      =\{0\}\).
    \item Si \(0=\sum_{i\in I} m_i\) con \(m_i\in M_i\) para todo
      \(i\in I\), entonces \(m_i=0\) para todo \(i\in I\).
    \item \(\theta\) es inyectivo y por tanto un isomorfismo.
    \item Para cada par \(J_1,J_2\subseteq I\) con
      \(J_1\cap J_2=\emptyset\), se tiene que
      \(\left(\sum_{i\in J_1} N_i\right)\cap
      \left(\sum_{i\in J_2} N_i\right)=\{0\}\)
  \end{enumerate}
\end{prop}

\begin{df}
  En caso de satisfacerse cualquiera de las condiciones anteriores
  equivalentes, diremos que la suma \(\sum_{i\in I} N_i\) es una suma
  directa interna, que notaremos por \(\dot{+}_{i\in I} N_i\).
\end{df}

\begin{cor}
  Si la familia \(\{N_i:i\in I\}\subseteq\mathcal{L}(M)\) verifican
  las condiciones y \(N\in\mathcal{L}(M)\) tal que
  \(N\cap\dot{+}_{i\in I} N_i=\{0\}\), entonces \(\{N_i:i\in I\}\cup\{N\}\).
\end{cor}

\begin{df}[Independencia]
  Si la familia \(\{N_i:i\in I\}\) donde cada módulo es distinto de 0 y
  satisface alguna de las condiciones anteriores equivalente, entonces
  diremos que dicha familia es independiente.
\end{df}

Caso particular: El módulo regular \(M_i=R\), llamamos:
\[
  R^{(I)}=\bigoplus_{i\in I} M_i=\{
    {(r_i)}_{i\in I}\in R^I:\textrm{ con soporte finito}\}
\]


    \begin{df}
\(A\) es un DIP, \(\subscriptbefore{A}{M}\) módulo.
\[
  t(M)=\{m\in M:\ann_A(m)\neq\langle 0\rangle\}
\]
es un submódulo de \(M\), que se llama submódulo de torsión de \(M\).
\end{df}


Ejemplo: sea \(A\) un DIP,
sea \(\subscriptbefore{A}{M}\) un módulo y consideramos
su submódulo de torsión.

Supongamos que \(t(M)\neq\{0\}\).
Definimos \(P\) como el conjunto de representantes de las clases de
equivalencia, bajo la relación ser asociados, de los irreducibles de \(A\).

Sea \(p\in P\), tomamos \(M_p=\{m\in M: p^e m=0\textrm{ para algún }
e\ge 1\}\). Tenemos que \(M_p\subseteq t(M)\), \(M_p\) es un submódulo.
Entonces:
\[
  t(M)=\dot{+}_{p\in P} M_p
\]
Demostremos esto.

Tomemos un \(m\in t(M)\), \(Am\) es un módulo de longitud finita.
\[
  Am=N_1\dot{+}\cdots\dot{+}N_r
\] donde \(N_i\) es una componente \(p_i\)-primaria.

En particular, \(m=m_1+\cdots+ m_r\) de manera que \(m_i\in N_i\subseteq
M_{p_i}\).

Luego \(M=\sum_{p\in P} M_p\). La unicidad es sencilla de deducir:
cada \(m\) estará en una componente primaria.

Caso particular. Tomamos \(M=\Cont^\infty(\R)\), \(M\) es un
\(\R[x]\)-módulo si \(xf=f'\). Entonces \(t(M)\) es el conjunto
de las funciones que satisfacen una EDO con coeficientes constantes.

\(P=\{\textrm{ Polinomios mónicos o bien lineales o bien
cuadráticos irreducibles}\}\). Es decir, cualquier función que se puede
definir mediante una EDO lineal con coeficientes constantes se puede
escribir como suma de funciones que resuelven
\({(\alpha\frac{\textrm{d}^2}{\textrm{d}x^2}+
\beta\frac{\textrm{d}}{\textrm{d}x}+\gamma)}^e f=0\) con \(e\in\N\).

Como hemos visto en ese caso particular, \(M_p\) no tiene por qué
tener longitud finita.

Consideremos \(I\) un conjunto infinito y \(R^{(I)}\) tal y como lo hemos
definido antes.
\begin{lema}
  Si \(M\) es un \(R\) módulo, existe una sucesión exacta de la forma
  \[
    0\longrightarrow L\longrightarrow R^{(I)}\longrightarrow M
    \longrightarrow 0
  \]
  para \(I\) adecuado.
\end{lema}
\begin{proof}
  Tomo \(\{m_i:i\in I\}\) tal que \(M=\sum_{i\in I}Rm_i\).
  Definimos \(\varphi:R^{(I)}\longrightarrow M\) dada por
  \(\varphi({(r_i)}_{i\in I})=\sum_{i\in I}r_i m_i\).

  \(L=\ker\varphi\overset{\iota}{\longrightarrow} M\).
\end{proof}

\begin{lema}[Existencia de bases]
  Para \(\{m_i:i\in I\}\subseteq M\), son equivalentes:
  \begin{enumerate}
    \item \(\sum_{i\in I} r_i m_i=0\) implica que \(r_i=0\) para todo índice.
    \item El homomorfismo \(\varphi:R^{(I)}\longrightarrow M\) con
      \(\varphi({(r_i)}_{i\in I})=\sum_i r_i m_i\) es inyectiva.
  \end{enumerate}
  Si se satisface 1, diremos que el conjunto \(\{m_i:i\in I\}\) es linealmente
  independiente. Si además estos elementos son además un conjunto de
  generadores, diremos que forman una base.
\end{lema}

La demostración es trivial.

\begin{obs}
  \(M\) tiene una base si y solo si \(M\cong R^I\) para algún \(I\).
\end{obs}

\begin{df}[Módulo libre]
  Un módulo se llama libre si admite una base.
\end{df}

\begin{obs}
  Advertencia: hay muchos módulos que no son libres.
\end{obs}

Ejemplos de módulos no libres:
\begin{enumerate}
  \item Ningún grupo abeliano finito es libre como \(\Z\) módulo.
  \item \(t(M)\), \(\subscriptbefore{A}{M}\) con \(A\) un DIP, nunca es libre.
    En otras palabras \(A^{(I)}\) no es nunca un módulo de torsión (por
    ser un dominio de integridad).
\end{enumerate}

    \subsection{Presentaciones de módulos}

\begin{prop}[Módulo presentado]
  Sea \(M\) un módulo. Existe una sucesión exacta
  \[
    \cdots\overset{f_{-2}}{\longrightarrow} F_{-1}
    \overset{f_{-1}}{\longrightarrow} F_{0}
    \overset{f_{0}}{\longrightarrow} M
    \overset{}{\longrightarrow} 0
  \]
  donde \(F_{-n}\) es libre para todo \(n\in\N\). Esa sucesión se llama
  resolución libre de \(M\).
\end{prop}
\begin{proof}
  Tomo un conjunto de generadores de \(M\), y tomo un homomorfismo de módulos
  sobreyectivo \(F_0\overset{p_0}{\longrightarrow} M\).
  \[
    F_{-1}
    \overset{p_{-1}}{\longrightarrow} K_0
    \overset{\iota}{\longrightarrow} F_0
    \overset{p_{0}}{\longrightarrow} M
    \overset{}{\longrightarrow} 0
  \]
  y reiteramos el proceso.

  Exactitud vista en \(F_{-1}\) ya que otro caso sería análogo.
  \(\ker f_{-1}=:K_{-1}=\Im p_{-2}=\Im  f_{-2}\).

  La resolución puede pero no tiene por qué ser finita.

\end{proof}

\begin{df}[Módulo finitamente presentado]
  \(M\) se dice finitamente presentado si existe un presentación finita que no
  es sino una sucesión exacta de la forma
  \[
    F_{-1}\overset{f_{-1}}{\longrightarrow} F_0
    \overset{f_{0}}{\longrightarrow}M \longrightarrow 0
  \]
\end{df}

Ejercicio: dar una presentación finita del \(\Z\)-módulo \(\Z_2\oplus\Z_4\).

\begin{prop}
  Un anillo \(R\) es noetheriano a izquierda si y solo si todo módulo
  finitamente generado es finitamente presentado.
\end{prop}
\begin{proof}
  Veamos solo una implicación: que si \(\subscriptbefore{R}{R}\) es
  noetheriano entonces que submódulo finitamente generado es finitamente
  presentado.

  Como \(M\) es finitamente generado, \(K_0\) es finitamente
  generado \(F_{-1}\overset{p_{-1}}{\longrightarrow}
  K_0\overset{\iota}{\longrightarrow}
  F_0\overset{f_0}{\longrightarrow} M\longrightarrow 0\).

\end{proof}

Tenemos que \(M\cong F_0/\Im f_{-1}\). Tomemos \(E_s\), \(F_t\) módulos
libres con bases finitas de cardinales \(s\) y \(t\) respectivamente.
Diremos que \(E_s\) tiene rango \(s\) (a pesar de que no es una invariante
del módulo, problema de la base de número invariante o INB, incluso se
puede dar \(R\cong R\oplus R\)). Llamamos \(e=\{e_1,\ldots,e_s\}\) base
de \(E_s\), y \(f=\{f_1,\ldots,f_t\}\) base de \(F_t\). Sea
\(\psi: E_s\longrightarrow F_t\), definido por \(\psi(e_i)=
\sum_{j=1}^t a_{ij}f_j\). Definimos la matriz \(A_\psi={(a_{ij})}_{
  1\le i\le s, 1\le j\le t}\in\mathcal{M}_{s\times t}(R)\).

Dado \(u=\sum_{i=1}^s x_i e_i\), \(x_i\in R\). Entonces
\[
  \psi(u)=\sum_{j=1}^t y_j f_j
\]
Resulta que si \(u_e=x=(x_1,\ldots,x_s)\) y \(y=(y_1,\ldots,y_t)\),
tenemos que \(y=xA_\psi\) y por tanto
\({\psi(u)}_f=u_e A_\psi\).

Tenemos que \((\cdot)A_\psi\circ{(\cdot)}_e={(\cdot)}_f\circ\psi\).


Sean \(E_s\overset{\psi}{\longrightarrow} F_t\overset{\varphi}{\longrightarrow}
G_r\), entonces \(A_{\varphi\circ\psi}=A_\varphi A_\psi\).

Ejemplo: Sea \(T:V\longrightarrow V\) un endomorfismo de espacios vectoriales,
y \(V\) de dimensión finita. \(\subscriptbefore{K[x]}{V}\) es un módulo
finitamente presentado. Buscamos una presentación finita.




    Ejemplo: \(T:V\longrightarrow V\) aplicación \(K\)-lineal,
\(n=\dim_K(V)<\infty\). Queremos una presentación
libre finita de \(\subscriptbefore{K[x]}{V}\). Tomo una
\(K\)-base (base como espacio vectorial)
\(\{v_1,\ldots, v_n\}\) de \(V\).

Tenemos que
\[
  T(v_i)=\sum_{j=1}^n b_{ij} v_i
\]
donde \((b_{ij})\in\mathcal{M}_n(K)\) es la matriz asociada a \(T\). Tomo
\(F_n\) un \(K[x]\)-módulo libre con base \(\{f_1,\ldots, f_n\}\)
y \(\phi: F_n\longrightarrow V\) tal que \(\phi(f_i)=v_i\)
para todo \(i\in\{1,\ldots,n\}\). \(\phi\) es un homomorfismo de
\(K[x]\)-módulos sobreyectivo.

Tenemos que \(F_n\overset{\phi}{\longrightarrow} V\longrightarrow 0\).
Tomamos \(Xf_i-\sum_{j=1}^n b_{ij}f_i\in\ker\phi\).

Afirmamos que \(\{Xf_i-\sum_{j=1}^n b_{ij}f_i:i\in\{1,\ldots,n\}\}\)
es un conjunto de generadores de \(\ker\phi\).

Tomemos \(x\in F_n\), tenemos que \(x=\sum_{i=1}^n p_i(x)f_i\). Supongamos
que \(x\neq 0\), definimos el peso como \(w(x):=\sum_{i=1}^n \gr(p_i)\ge 0\).

Observemos que \(w(x)=0\) es solo posible si \(p_i\in K\) para todo \(i\).
Si \(w(x)=0\), entonces \(x=\sum_{i=1}^n p_i f_i\).
Entonces aplicando \(\phi\) tenemos
\(0=\sum_{i=1}^n p_i v_i\), y por tanto \(x=0\) lo que es una contradicción.

Así que si \(x\in\ker\phi\setminus\{0\}\), \(w(x)\ge 1\).

Vamos a aplicar inducción sobre \(w(x)\). \(w(x)=1\). Entonces existe un
único índice \(k\in\{1,\ldots,n\}\) tal que \(p_k\) no es constante
y además \(p_k=aX+b\) con \(a,b\in K\).
\begin{eqnarray*}
  x&=&\sum_{i\neq k} p_i f_i + (aX+b)f_k\\
  &=&\sum_{i\neq k} p_i f_i + a(Xf_k-\sum_j b_{kj}f_j)+a\sum_j b_{kj}f_j
  +bf_k\\
\end{eqnarray*}
Luego
\[
  \sum_{i\neq k}p_i f_i+a\sum_j b_{kj}f_j+bf_k\in\ker\phi
\]
donde como son todos constantes, se tiene
\[
  \sum_{i\neq k}p_i f_i+a\sum_j b_{kj}f_j+bf_k=0
\]
y por tanto \(x=a(Xf_k-\sum_j b_{kj}f_j)\).

Supongamos \(w(x)>1\). Existe algún \(k\in\{1,\ldots,n\}\) para el que
\(\gr(p_k)\ge 1\). Así, \(p_k=q(X)X+b\), con \(\gr(q)=\gr(p_k)-1\) y
\(b\in K\).

\begin{eqnarray*}
  x&=&\sum_{i\neq k} p_i f_i +q(X)(Xf_k-\sum_j b_{kj}f_j)
  +q(X)\sum_j b_{kj}f_j+bf_k\\
\end{eqnarray*}
Tenemos que \(y=\sum_{i\neq k} p_i f_i+q(X)\sum_j b_{kj}f_j+bf_k\in\ker\phi\)
y \(w(y)\le w(x)-1<w(x)\). Por inducción, sabemos que \(y=\sum_i q_i (Xf_i
-\sum_j b_{ij}f_j)\),
y tenemos:
\[
  x= q(X)(Xf_k-\sum_j b_{kj}f_j)+\sum_i q_i (Xf_i
-\sum_j b_{ij}f_j)
\]
con lo que se demuestra el enunciado, sacando factor común lo que haga
falta y redondeando.

Definimos
\[
  F_n\overset{\psi}{\longrightarrow} F_n\overset{\phi}{\longrightarrow}
  V\longrightarrow 0
\]
que es una representación libre finita, donde
\[
  \psi(f_i)=Xf_i-\sum_j b_{ij}f_j
\]

Con lo que la matriz nos queda:
\[
  A_\psi=
  \begin{pmatrix}
    X-b_{11}  & -b_{12}&\cdots&-b_{1n}\\
    -b_{21}  & X-b_{22}&\cdots&-b_{2n}\\
    \cdots&\cdots&\cdots&\cdots\\
    -b_{n1}  & -b_{n2}&\cdots&X-b_{nn}\\
  \end{pmatrix}\in\mathcal{M}_n(K[x])
\]
o si se quiere, \(A_\psi=XI-A_T\) con \(A_T=(b_{ij})\).

\begin{lema}
  Sea \(F\) un \(R\)-módulo libre y \(\varphi:M\longrightarrow N\) un
  epimorfismo de \(R\)-módulos. Para cada homomorfismo de \(R\)-módulos
  \(\alpha:F\longrightarrow N\) existe un homomorfismo de \(R\)-módulos
  \(\beta: F\longrightarrow M\) tal que \(\varphi\circ\beta=\alpha\).
  Es decir, \(\alpha\) se levanta como homomorfismo a \(M\).
\end{lema}
\begin{proof}
  Tomo en \(F\) una base \(\{e_i:i\in I\}\). Como \(\varphi\) es sobreyectivo
  para cada \(\alpha(e_i)\) existe un \(m_i\in M\) tal que \(\varphi(m_i)
  =\alpha(e_i)\).
  Ahora tenemos \(\beta\) dado por \(\beta(e_i)=m_i\).

\end{proof}

Sean \(\subscriptbefore{R}{M}\) y \(\subscriptbefore{R}{N}\) finitamente
presentados y \(h:M\longrightarrow N\) homomorfismo de \(R\)-módulos.
\[
  E_s\overset{\psi}{\longrightarrow}
  F_t\overset{\phi}{\longrightarrow} M\longrightarrow 0
\]
\[
  E_{s'}\overset{\psi'}{\longrightarrow}
  F_{t'}\overset{\phi'}{\longrightarrow} N\longrightarrow 0
\]
Por el lema anterior, existe un \(q\) tal que \(\phi'\circ q=h\circ\phi\).
Observemos que \(\Im q\circ\psi\subseteq \ker\phi'=\Im\psi'\).
Aplicando el lema sobre la imagen de \(\psi'\), existe un
\(p\) tal que \(\psi'\circ p=q\circ\psi\).

    Donde \(p:E_s\longrightarrow E_{s'}\)
y \(q:F_t\longrightarrow F_{t'}\).

Supongamos ahora que tenemos que existen \(p\) y \(q\)
tales que \(q\psi=\psi' p\). Vamos
a construir un \(h\) homomorfismo. Tomamos \(u\in F_t\)
tal que \(\phi(u)=m\). Queremos definir \(h(m)\) como
\(\phi'(q(u))\in N\). Hay que demostrar que está bien definida.

Tomamos \(v\in F_t\), tal que \(\phi(v)=m\). Tenemos que:
\[
  \phi'(q(v)-q(u))=\phi'(q(u-v))
\]
tomando un \(x\in E_s\) tal que \(v-u=\psi(x)\), ya que
\(0=\phi(v-u)\in\ker\phi=\Im\psi\).
\[
  \phi'(q(v)-q(u))=\phi'(q(u-v))=\phi'(q(\psi(x)))=
  \phi'(\psi'(p(x)))=0
\]
y entonces \(h\) no depende del representante elegido.
Es fácil ver que \(h\) es un homomorfismo de módulos y que
\(\phi\circ h= q\circ\phi'\).

Fijadas bases en \(E_s, F_t, E_{s'}, F_{t'}\), definir \(h\) se reduce a
dar dos matrices \(A_q\) y \(A_p\) tales que
\[
  A_\psi A_q=A_p A_{\psi'}
\]
entonces \(A_\psi\), \(A_{\psi'}\) representan a los módulos \(M\) y \(N\)
y \(A_q\), \(A_p\) representan al homormorfismo \(h\).

Concretamente, si \(f=\{f_1,\ldots, f_t\}\) es una base de \(F_t\)
y \(f'=\{f_1',\ldots, f_t'\}\) de \(F_{t'}\) y
\(A_q=(q_{ij})\) y tomamos \(m_i=\phi(f_i)\) y \(n_j=\phi'(f_j')\),
tenemos:
\begin{enumerate}
  \item \(\{m_1,\ldots, m_t\}\) genera \(M\).
  \item \(\{n_1,\ldots, n_{t'}\}\) genera \(N\).
  \item \(h(m_i)=\sum_{j=1}^{t'} q_{ij}n_j\).
\end{enumerate}

\begin{prop}[Teorema de Cayley-Hamilton]
  Sea \(T:V\longrightarrow V\) un homomorfismo \(K\)-lineal, con
  la dimensión de \(V\) finita. Sea \(d\in K[x]\) el polinomio característico
  de \(T\). Entonces el polinomio mínimo de \(T\) divide a \(d(x)\).
  En particular, \(d(T)=0\).
\end{prop}
\begin{proof}
  Tomamos la presentación finita de \(\subscriptbefore{K[x]}{V}\) que
  vimos anteriormente:
  \[
    F_n\overset{\psi}{\longrightarrow} F_n
    \overset{\phi}{\longrightarrow} V\longrightarrow 0
  \]
  Tomamos \(A_\psi\) y \(P\) su matriz adjunta (o de cofactores), o sea,
  la que hace que se cumpla la ecuación \(PA_\psi=d(x)I\).

  Sea \(\delta:F_n\longrightarrow F_n\) el homomorfismo que fijada
  bases \(f\) de \(F_n\) tiene como matriz \(d(x)I\), o sea,
  \(\delta(f_i)=d(x)f_i\). Consideramos la proyección canónica
  \(\pi:F_n\longrightarrow F_n/\Im\delta\) y nos queda:
  \[
    F_n\overset{\delta}{\longrightarrow} F_n
    \overset{\pi}{\longrightarrow} F_n/\Im\delta\longrightarrow 0
  \]

  Tomando \(p\) la aplcación tal que \(A_p=P\) y \(q=\id\), de aquí
  obtenemos que \(\psi_p=\id\circ\delta\), con lo que se induce \(h\),
  un homomorfismo de módulos sobreyectivo (\(h\circ\pi=\phi\)).

  \[
    \Ann_{K[x]}(V)\supseteq\Ann_{K[x]}(F_n/\Im\delta)=\langle\delta(x)\rangle
  \]
  donde la última igualdad viene de que \(F_n/\Im\delta\cong
  \bigoplus_{i=1}^n K[x]f_i/K[x]d(x)f_i\cong
  \bigoplus_{i=1}^n K[x]/\langle d(x)\rangle\).

  Por tanto, el polinomio mínimo de \(T\) (que es el anulador de \(V\)),
  divide a \(d(x)\). Como al evaluar en \(T\) el polinomio mínimo se anula,
  tenemos que el polinomio característico se anula también.

\end{proof}




    \begin{df}[Matrix quasidiagonal]
  Sea \(A=(a_{ij})\in\mathcal{M}_{s\times t}(R)\). Diremos que
  \(A\) es quasidiagonal si \(a_{ij}=0\) para todo \(i\neq j\).
  Usaremos \(d_i=a_{ii}\) para \(i=1,\ldots,m\) con
  \(m=\min\{s,t\}\). La notación
  \[
    A=\diag_{s\times t}(d_1,\ldots, d_m)
  \]
\end{df}

Ejemplos:
\[
  \diag_{3\times 2}(1,3)
  \begin{pmatrix}
    1&0\\
    0&3\\
    0&0
  \end{pmatrix}
\]
\[
  \diag_{2\times 3}(1,3)
  \begin{pmatrix}
    1&0&0\\
    0&3&0\\
  \end{pmatrix}
\]

\begin{nt}
  Denotaremos \(GL_n(R)\) al grupo de unidades de \(\mathcal{M}_n(R)\):
  es decir, las matrices \(Q\) tales que existe
  otra matriz \(Q^{-1}\) que cumpla \(QQ^{-1}=Q^{-1}Q=I_n\).
\end{nt}

\begin{prop}
  Sea la presentación finita:
  \[
    E_s\overset{\psi}{\longrightarrow} F_t
    \overset{\phi}{\longrightarrow} M\longrightarrow 0
  \]
  de \(\subscriptbefore{R}{M}\). Supongamos que existen \(P\in GL_s(R)\),
  \(Q\in GL_t(R)\) y \\\(D=\diag_{s\times t}(d_1,\ldots, d_m)\)
  tales que \(PA_\psi = DQ\). Si \(\{m_1,\ldots, m_t\}\) es el conjunto
  de generadores de \(M\) y tales que \(m_i=\phi(f_i)\) con
  \[
    x_i=\sum_{j=1}^t q_{ij}m_j
  \]
  entonces \(M=\dot{+}_{i=1}^t Rx_i\) y \(\ann_R(x_i)=Rd_i\) si
  \(i\le m\) y \(\ann_R(x_i)=\{0\}\) si \(i>m\) si se da el caso.
\end{prop}

\begin{proof}
  Tomemos otra presentación:
  \[
    E_s\overset{\psi_1}{\longrightarrow} F_t
    \overset{\phi_1}{\longrightarrow} M\longrightarrow 0
  \]

  Tomemos \(\id:M\longrightarrow M\) y dos homomorfismos
  \(p:E_s\longrightarrow E_s\) y
  \(q:F_t\longrightarrow F_t\), tales que \(A_p=P\),
  \(A_q=Q\) y \(A_\psi =D\) y que conmuten todas las aplicaciones.

  Para que conmuten, definimos \(\phi_1=\phi\circ q\),
  con lo que \(\phi_1(f_i)=\phi(q(f_i))=\sum_{j=1}^t q_{ij} m_j=x_i\).

  La condición de matrices \(PA_\psi=DQ\) garantiza que
  \(\psi\circ p=q\circ A_\phi\).

  Hay que comprobar que la sucesión que nos hemos inventado
  es exacta en \(F_t\). Para demostrarlo, usamos que \(P\) y \(Q\) son
  inversibles: \(p,q\) son isomorfismos y podemos recuperar la exactitud
  de la sucesión del enunciado.

  \[
    M=Rx_1+\cdots+Rx_t
  \]
  porque \(x_i=\phi_1(f_i)\) y \(\phi_1\) es sobreyectiva. Para ver que es
  directa, tomamos el \(0=r_1 x_1+\cdots+ r_t x_t\). Hay que ver
  que cada \(r_i x_i=0\).
  \[
    0=\phi_1(r_1 f_1+\cdots+r_t f_t)\implies
    r_1 f_1+\cdots+r_t f_t\in\ker\phi_1=\Im\psi_1
  \]

  Por otro lado, \(\Im\psi_1=R\psi_1(e_1)+\cdots+ R\psi_1(e_s)\).
  Ahora bien, \(A_{\psi_1}=D\), con lo que
  \(\Im\psi_1=Rd_1 f_1+\cdots +Rd_m f_m\). Tenemos que esos módulos
  son independientes y la suma es directa:
  \(\Im\psi_1=Rd_1 f_1\dot{+}\cdots\dot{+} Rd_m f_m\).
  Entonces \(r_i\in Rd_i\) para \(i\le m\),
  y si \(t>m\), entonces \(r_i=0\) para \(i>m\).

  Así, cada \(r_i f_i = s_i d_i f_i\).
  Tomamos \(r_1 x_i\phi_1(r_i f_i) \), tenemos que
  \(r_i f_i\in\Im\psi_1=\ker\phi_1\), luego \(r_i x_i=0\).
  Luego:
  \[
    M=Rx_1\dot{+}\cdots\dot{+}Rx_t
  \]

  Se deduce también que \(\ann_R(x_i)\supseteq Rd_i\).

  \begin{eqnarray*}
    M&\cong& F_t/\Im\psi_1=
    (Rf_1\dot{+}\cdots\dot{+}Rf_t)/
    (Rd_1f_1\dot{+}\cdots\dot{+}Rd_m f_m)\\
    &\cong&
    Rf_1/Rd_1 f_1\oplus\cdots\oplus Rf_m/Rd_m f_m
    \oplus R/\{0\}\oplus\overset{(t-m)}{\cdots}\oplus R/\{0\}\\
    &\cong&
    Rf_1/Rd_1\oplus\cdots\oplus R/Rd_m
    \oplus R\oplus\overset{(t-m)}{\cdots}\oplus R
  \end{eqnarray*}

\end{proof}

Caso particular: \(R=\Z\). Aquí siempre podemos calcular P y Q.
Si \(M\) es un grupo abeliano finitamente generado como \(\Z\)-módulo,
entonces existen \(d_1,\ldots, d_m\in\N\) tales que
\[
  M\cong \Z_{d_1}\oplus\cdots\oplus\Z_{d_m}\oplus\Z^{t-m}
\]
si \(t>m\) y en otro caso:
\[
  M\cong \Z_{d_1}\oplus\cdots\oplus\Z_{d_m}
\]
Es la suma de una parte de torsión y una libre de torsión.


    ¿Es posible encontrar \(P,Q\) cuadradas inversibles tales que
\(PA_\psi Q^{-1}\) sea una matriz quasidiagonal?

\begin{df}[Matrices y operaciones elementales]
  \(E_{ij}\in\mathcal{M}_n(R)\) definida por su única entrada no nula es
  la \((i,j)\)-ésima, que vale 1.
  Se verifica:
  \[
    E_{ij}=\left\{
      \begin{matrix}
        E_{ie},&\textrm{si } j=k\\
        0,&\textrm{si } j\neq k\\
      \end{matrix}
      \right.
  \]

  Para cualquier matriz \(B\) de entradas \(b_{ij}\), se puede escribir:
  \[
    B=\sum_{i,j} b_{ij}E_{ij}=\sum_{i,j} E_{ij}b_{ij}
  \]

  Sea \(A\) una matriz rectangular de tamaño adecuado, \(r\in R,
  u\in\mathcal{U}(R)\).
  \[
    {(E_{ij}A)}_{rs}=\left\{
      \begin{matrix}
        a_{is},&\textrm{si } r=j\\
        0,&\textrm{si } r\neq j\\
      \end{matrix}
      \right.
  \]
  La matriz \(I+rE_{ij}\) es inversible para \(i\neq j\). (multiplicando
  por \(I-rE_{ij}\) sale).

  La matriz \(I+E_{ij}+E_{ji}-E_{ii}-E_{jj}\) es inversible para \(i\neq j\),
  pues al cuadrado es la identidad.

  La matrix \(I+(u-1)E_{ii}\) es inversible, se prueba multiplicando
  por \(I+(u^{-1}-1)E_{ii}\).

  A las siguientes matrices las llamamos matrices elementales:
  \begin{enumerate}
    \item \(I+rE_{ij}\) (multiplicar por un escalar una fila o columna
      y sumársela a otra).
    \item \(I+E_{ij}+E_{ji}-E_{ii}-E_{jj}\) (intercambiar sus filas o
      columnas).
    \item \(I+(u-1)E_{ii}\) (multiplicar una fila o columna por una unidad).
  \end{enumerate}
  es un grupo.
\end{df}

Ejemplo: Sea \(M\) un grupo aditivo generado por \(\{m_1,m_2,m_3\) sujeto
a las relaciones:
\begin{enumerate}
  \item \(2m_1+m_2-m_3=0\)
  \item \(4m_1+m_2-3m_3=0\)
\end{enumerate}

Tomamos \(\Z\)-módulos libres \(F_3\) con bases \(\{f_1,f_2,f_3\}\) y
\(E_2\) con bases \(\{e_1,e_2\}\).
\[
  F_3\overset{\psi}{\longrightarrow}
  F_3\overset{\phi}{\longrightarrow}
  M\longrightarrow 0
\]

Definimos \(\phi(f_i)=m_i\) y
\[
  A_\psi =
  \begin{pmatrix}
    2&1&-1\\
    4&1&-3\\
  \end{pmatrix}
\]

Solo apuntamos las operaciones por columnas porque solo nos interesa la
matriz \(Q\). Para tener una sencilla, vamos a hacer el máximo número
de matrices por filas.
\[
  \begin{pmatrix}
    2&1&-1\\
    4&1&-3\\
  \end{pmatrix}\sim
  \begin{pmatrix}
    2&1&-1\\
    0&-1&-1\\
  \end{pmatrix}\sim
\]
Ahora comenzamos a hacer operaciones por columnas, anotándolas:
\[
  \begin{pmatrix}
    2&0&-2\\
    0&-1&-1\\
  \end{pmatrix}\underset{c_3+c_2}{\sim}
  \begin{pmatrix}
    2&0&-2\\
    0&-1&0\\
  \end{pmatrix}\underset{c_3-c_1}{\sim}
  \begin{pmatrix}
    2&0&0\\
    0&-1&0\\
  \end{pmatrix}
\]
Tenemos que
\[
  D=
  \begin{pmatrix}
    2&0&0\\
    0&-1&0\\
  \end{pmatrix}
\]
y que
\[
  \begin{pmatrix}
    1&0&0\\
    0&1&0\\
    0&0&1\\
  \end{pmatrix}\underset{c_3-c_1}{\sim}
  \begin{pmatrix}
    1&0&-1\\
    0&1&0\\
    0&0&1\\
  \end{pmatrix}\underset{c_3-c_2}{\sim}
  \begin{pmatrix}
    1&0&-1\\
    0&1&1\\
    0&0&1\\
  \end{pmatrix}
  = Q
\]

\[
  x_1=\sum_{j=1}^t q_{ij}m_j=m_1-m3
\]
\[
  x_2=\sum_{j=1}^t q_{ij}m_j=m_2+m3
\]
\[
  x_3=\sum_{j=1}^t q_{ij}m_j=m3
\]

\[M=\Z x_1\dot{+}\Z x_2\dot{+}\Z x_3\dot{+}\]
\[
  \ann_\Z(x_1)=2\Z
\]
\[
  \ann_\Z(x_2)=-1\Z
\]
\[
  \ann_\Z(x_3)=\langle 0\rangle
\]

Con lo que
\[M=\Z x_1\dot{+}\Z x_3\dot{+}\cong\Z_2\oplus\Z\]

    Ejemplo: Sea \(K\) un cuerpo, \(T:V\longrightarrow V\), con
\(\dim_K V=3\), \(\{v_1,v_2,v_3\}\) es una base de \(V\).

Sea
\[
  B=
  \begin{pmatrix}
    \phantom{-}1  & -1  & -1  \\
    -1  & \phantom{-}1  & -1  \\
    -1  & -1 &  \phantom{-}1  \\
  \end{pmatrix}
\]
la matriz de \(T\) en dicha base. Obtengamos la descomposición cíclica
primaria de \(\subscriptbefore{K[x]}{V}\).

Tenemos para
\[
  A=A_\psi=
  \begin{pmatrix}
    x{-}1  & 1  & -1  \\
    1  & x{+}1  & -1  \\
    1  & -1 &  x{-}1  \\
  \end{pmatrix}\in\mathcal{M}(K[x])
\]

Busquemos \(P, Q\) y \(D\).
\(v_i=\varphi(f_i)\). Al final obtendremos \(PAQ^{-1}=D\).

Partimos de \(A\) y hacemos operaciones por filas:
\[
  A=
  \begin{pmatrix}
    x{-}1  & 1  & -1  \\
    1  & x{+}1  & -1  \\
    1  & -1 &  x{-}1  \\
  \end{pmatrix}\sim
\]
(colocamos el polinomio de menor grado como pivote)
\[
  \begin{pmatrix}
    1  & -1 &  x{-}1  \\
    1  & x{+}1  & -1  \\
    x{-}1  & 1  & -1  \\
  \end{pmatrix}\sim
  \begin{pmatrix}
    1   & -1   &  x-1  \\
    0   & x+2  & -x  \\
    0   & x    & -x^2-2x-2  \\
  \end{pmatrix}\sim
\]
(Suponiendo que el cuerpo tiene característica
  distinta de 2)
\[
  \begin{pmatrix}
    1   & -1   &  x-1  \\
    0   & 2    & x^2-3x+2  \\
    0   & x+2  & -x  \\
  \end{pmatrix}\sim
  \begin{pmatrix}
    1   & -1   &  x-1  \\
    0   & 2    & x^2-3x+2  \\
    0   & x+2  & -x  \\
  \end{pmatrix}\sim
\]
\[
  \begin{pmatrix}
    1   & -1   &  x-1  \\
    0   & 2    & x^2-3x+2  \\
    0   & 0    & -\frac{1}{2}x^3+\frac{1}{2}x^2+x-2  \\
  \end{pmatrix}\sim
  \begin{pmatrix}
    1   & -1   &  x-1  \\
    0   & 2    & x^2-3x+2  \\
    0   & 0    & x^3-x^2-2x+4  \\
  \end{pmatrix}
\]
Empezamos con las operaciones por columnas
\[
  \begin{pmatrix}
    1   & -1   &  x-1  \\
    0   & 2    & x^2-3x+2  \\
    0   & 0    & x^3-x^2-2x+4  \\
  \end{pmatrix}
  \overset{c_2+c_1}{\sim}
  \begin{pmatrix}
    1   & 0   &  x-1  \\
    0   & 2    & x^2-3x+2  \\
    0   & 0    & x^3-x^2-2x+4  \\
  \end{pmatrix}
  \overset{c_3-(x-1)c_1}{\sim}
\]
\[
  \begin{pmatrix}
    1   & 0   &  0  \\
    0   & 2    & x^2-3x+2  \\
    0   & 0    & x^3-x^2-2x+4  \\
  \end{pmatrix}
  \overset{c_3-(\frac{1}{2})(x^2-3x+2)c_2}{\sim}
  \begin{pmatrix}
    1   & 0   &  0  \\
    0   & 2    & 0\\
    0   & 0    & x^3-x^2-2x+4  \\
  \end{pmatrix} = D
\]

Calculamos ahora \(Q\):
\[
  Q=
  \begin{pmatrix}
    1 & 1 & x\\
    0 & 1 & \frac{1}{2}(x^2-3x+2)\\
    0 & 0 & 1
  \end{pmatrix}
\]

Quiero encontrar \(x_1, x_2, x_3\) tales que
\(\subscriptbefore = K[x]x_1\dot{+}K[x]x_2\dot{+}K[x]x_3\).

Tenemos que \(\ann_{K[x]}(x_1)=K[x]\), \(\ann_{K[x]}(x_2)=2K[x]=K[x]\) y
\(\ann_{K[x]}(x_3)=\langle x^3-x^2-2x+4 \rangle\).
Con esto, \(x_1=x_2=0\) y por tanto
\[
  \subscriptbefore{K[x]}{V}=K[x]v_3
\]
donde la última igualdad es porque \(q_{33}=1\).

Es cíclica primaria si \( x^3-x^2-2x+4 \) es una potencia del irreducible.
Vamos a estudiar según quien sea el cuerpo \(K\), al menos en un par de
casos.

Caso particular \(K=\Q\): Probando con \(\pm 1, \pm 2, \pm 4\) vemos que
no tiene raíces en \(\Q\). Por tanto, como el grado es 3,
\(\mu= x^3-x^2-2x+4\) es el polinomio mínimo y es irreducible.
\[
  \subscriptbefore{\Q[x]}{V}=\Q[x]v_3
\]
es la descomposición cíclica primaria. Además, \(\subscriptbefore{\Q[x]}{V}\)
es simple.

Caso particular \(K=\R\): Por análisis, al ser grado impar, existe al menos
una raíz real. \(\mu'(x)=3x^2-2x-2\), que tiene como raíces
\(\frac{1\pm\sqrt{7}}{3}\) y tenemos una parábola con coeficiente líder
positivo. Luego hay 1 raíces o 3 si los máximos y mínimos son positivos
o negativos: \(\mu(\frac{1+\sqrt{7}}{3})>0\) luego \(\mu\) tiene una única
raíz en \(\R\).

Tenemos que, si \(\alpha\) es la raíz real y \(z\in\C\setminus\R\),
\(\mu=(x-\alpha)(x^2-2\Re(z)x+|z|^2)\) en \(\R[x]\).


    La descomposición cíclica primaria se consigue mediante el siguiente
procedimiento.
Sea \(u_1=(x-\alpha)v_3=(T-\alpha)v_3\).
\(\ann_{\R[x]} u_1=\langle x^2-2\Re(z)x+|z|^2\rangle\) y
sea \(u_2=(x^2-2\Re(z)x+|z|^2)v_3=(T^2-2\Re(z)T+|z|^2)v_3\).
\(\ann_{\R[x]} u_1=\langle(x-\alpha) \rangle\).

La descomposición cíclica primaria queda:
\[
  \subscriptbefore{\R[x]}{V}=\R[x]u_1\dot{+}\R[x]u_2
\]
Tomamos la base de \(V\) dada por \(\{u_1, T(u_1), u_2\}\).
La matriz de \(T\) con respecto de esa base por filas es:
\[
  \begin{pmatrix}
    0 & 1 & 0\\
    -|z|^2 & 2\Re(z) & 0\\
    0 & 0 & \alpha\\
  \end{pmatrix}
\]
donde hemos usado que \(T^2(u_1)=2\Re(z)T(u_1)-|z|^2 u_1\)
y que \(T(u_2)=\alpha u_2\). Como vemos es diagonal por bloques.

Caso particular, \(K=\C\). Al ser algebraicamente cerrado,
\(\mu=(x-\alpha)(x-z)(x-\bar{z})\) donde \(x\in\R\) y
\(z\in\C\setminus\R\).

\[
  \subscriptbefore{\C[x]}{V}=\C[x]u_1\dot{+}\C[x]u_2
\]
Pero podemos dividir \(\R[x]u_1\) aún más. Llamamos \(x_1=(x-z)u_1\),
\(\ann_{\C[x]}(x_1)=\langle x-\bar{z}\rangle\) y
\(\ann_{\C[x]}(x_2)=\langle x-{z}\rangle\), con lo que queda
\[
  \subscriptbefore{\C[x]}{V}=\C[x]x_1\dot{+}\C[x]x_2\dot{+}\C[x]u_2
\]

En la base \(\{x_1,x_2,u_2\}\) la matriz de \(T\) es:
\[
  \begin{pmatrix}
    \bar{z} & 0 & 0\\
    0 & z & 0\\
    0 & 0 & \alpha\\
  \end{pmatrix}
\]

Caso particular, \(K\) con característica 2:
\[
  B=
  \begin{pmatrix}
    1 & 1 & 1\\
    1 & 1 & 1\\
    1 & 1 & 1\\
  \end{pmatrix}
\]

Tenemos que
\[
  X-B=
  \begin{pmatrix}
    x+1 & 1 & 1\\
    1 & x+1 & 1\\
    1 & 1 & x+1\\
  \end{pmatrix}\sim
  \begin{pmatrix}
    1 & 1 & x+1\\
    1 & x+1 & 1\\
    x+1 & 1 & 1\\
  \end{pmatrix}\sim
\]\[
  \begin{pmatrix}
    1 & 1 & x+1\\
    0 & x & x\\
    0 & x & x^2\\
  \end{pmatrix}\sim
  \begin{pmatrix}
    1 & 1 & x+1\\
    0 & x & x\\
    0 & 0 & x^2+x\\
  \end{pmatrix}
\]

Y ahora por columnas
\[
  \begin{pmatrix}
    1 & 1 & x+1\\
    0 & x & x\\
    0 & 0 & x^2+x\\
  \end{pmatrix}
  \overset{c_2+c_1}\sim
  \begin{pmatrix}
    1 & 0 & x+1\\
    0 & x & x\\
    0 & 0 & x^2+x\\
  \end{pmatrix}
  \overset{c_3+(x+1)c_1}\sim
\]\[
  \begin{pmatrix}
    1 & 0 & 0\\
    0 & x & x\\
    0 & 0 & x^2+x\\
  \end{pmatrix}
  \overset{c_3+c_2}\sim
  \begin{pmatrix}
    1 & 0 & 0\\
    0 & x & 0\\
    0 & 0 & x^2+x\\
  \end{pmatrix}
  =D
\]

\[
  Q =
  \begin{pmatrix}
    1&1&x+1\\
    0&1&1\\
    0&0&1\\
  \end{pmatrix}
\]

Tenemos que:
\[
  \subscriptbefore{K[x]}{V}=K[x]x_2\dot{+}K[x]x_3
\]
donde \(\ann_{K[x]}(x_2)=\langle x\rangle\) y
\(\ann_{K[x]}(x_3)=\langle x^2+x\rangle\), con lo que
\(x_2=v_2+v_3\) y \(x_3=v_3\) (como el anulador de \(x_1\) es \(K[x]\),
\(x_1=0\) y no nos interesa).

Como \(x^2+x=x(x+1)\), tomamos \(y_1=(x+1)x_3=(T+1)x_3\)
y \(\ann_{K[x]}(y_1)=\langle x\rangle\).
Como \(x^2+x=x(x+1)\), tomamos \(y_2=x x_3=Tx_3\)
y \(\ann_{K[x]}(y_2)=\langle x+1\rangle\).
La descomposición cíclica primaria queda:
\[
  \subscriptbefore{K[x]}{V}=K[x]x_2\dot{+}K[x]y_1\dot{+}K[x]y_2
\]

Y la matriz de \(T\) en la base \(\{x_2, y_1,y_2\}\) es:
\[
  \begin{pmatrix}
    0&0&0\\
    0&0&0\\
    0&0&1\\
  \end{pmatrix}
\]

\begin{teo}
  Si \(A\) es un dominio euclídeo con función euclídea \(\nu\) y \(B\)
  es una matriz con coeficientes en \(A\), existen \(P,Q\) inversibles
  de tamaño adecuado y \(D\) quasidiagonal tal que:
  \[
    PB=DQ
  \]
\end{teo}
\begin{proof}
  Necesitamos demostrar que \(PBQ^{-1}=D\).
  Suponemos que \(B\neq 0\). Vamos a demostrar que mediante operaciones
  elementales sobre filas y columnas, podemos reducir \(B\) a una
  del tipo
  \[
    \begin{pmatrix}
      b &O\\
      O &B'\\
    \end{pmatrix}
  \]

  Llamemos \(\nu(B)=\min\{\nu(b_{ij}):b_{ij}\neq 0\}\).
  Intercambiando filas y columnas en \(B\) podemos conseguir
  \(\nu(B)=\nu(b_{11})\).

  Caso \(a\): Si \(b_{11}|b_{i1}\) y \(b_{11}|b_{1j}\) para todos \(i,j\),
  entonces reduzco haciendo ceros en las filas y columnas.

  Caso \(b\): Si \(b_{11}|b_{i1}\) o \(b_{11}|b_{1j}\) para algún \(i\)
  o \(j\) (supongamos \(i\)),
  entonces
  \[ b_{i1}=qb_{11}+r\]
  y tenemos que \(\nu(r)<\nu(b_{11})\). Basta restar a la fila
  \(i\) la primera multiplicada por \(q\) e intercambiarlas.

  Hacemos finalmente inducción sobre \(\nu(B)\).

\end{proof}


    \subsection{Módulos semisimples}

\begin{prop}
  Sea \(M\) un módulo.
  Sea \(\{M_i:i\in I\}\) una familia no vacía de submódulos simples
  (no cero y que sus únicos submódulos son el 0 y el total).

  Ponemos \(M'=\sum_{i\in I} M_i\), es decir, el menor submódulo que los
  contiene a todos. Tomamos  \(N\subsetneq M'\). Entonces existe un
  \(J\subseteq I\) tal que \(\{M_i:i\in J\}\) es independiente,
  \(N\cap\left(\dot{+}_{i\in J} M_i\right)=\{0\}\) y
  \(M'=N\dot{+}\left(\dot{+}_{j\in J} M_i\right)\).
\end{prop}
\begin{proof}
  La demostración pasa por utilizar el lema de Zorn.
  Sea \(\Gamma\) el conjunto de los subconjuntos \(J\) de \(I\) tales que
  \(\{M_i:i\in J\}\) es independiente y
  \(N\cap\left(\dot{+}_{i\in J} M_i\right)=\{0\}\).

  Veamos que \(\Gamma\neq\emptyset\). Si \(N=\{0\}\), tomamos \(i\in I\)
  y tenemos que \(\{i\}\in\Gamma\}\), que cumple trivialmente ambas
  propiedades. Si \(N\neq\{0\}\), pero \(N\cap M_i=\{0\}\), tomamos
  de nuevo \(\{i\}\in\Gamma\) que es otro caso trivial.

  Supongamos que \(N\neq\{0\}\) y \(N\cap M_i=\{0\}\) para todo \(i\in I\).
  Como cada \(M_i\) es simple, \(N\cap M_i=M_i\) para todo \(i\in I\),
  con lo que \(N=M'\), caso que hemos excluído.

  El orden que definimos en \(\Gamma\) es la inclusión. Tenemos que ver
  que cualquier cadena (subconjunto totalmente ordenado) tiene un elemento
  maximal. Sea \(\chi\subseteq\Gamma\). Definimos \(J=\bigcup_{C\in\chi} C\).
  Lo que tenemos que demostrar es que \(J\in \Gamma\).

  Veamos que \(\{M_i:i\in J\}\) es independiente. Por una proposición anterior,
  basta ver que cualquier \(\{M_i:i\in F\}\) es independiente para cualquier
  \(F\subseteq J\) finito. Por ser \(\chi\) una cadena, existe un \(C\in\chi\)
  tal que \(F\subseteq C\). Pero \(C\in\Gamma\), luego \(\{M_i:i\in C\}\)
  es idependiente, y en particular, \(\{M_i:i \in F\}\) es independiente.

  Tomamos \(m\in N\cap\left(\dot{+}_{j\in J} M_j\right)\). Entonces existe
  un \(F\subseteq J\) finito tal que \(m\in N\cap\left(\dot{+}_{j\in F}
  M_j\right)\), entonces existe un \(C\in\chi\) tal que \(F\subseteq C\)
  y por consiguiente \(m\in N\cap\left(\dot{+}_{j\in C} M_j\right)=\{0\}\).

  Por tanto \(\Gamma\) es inductivo y el lema de Zorn nos asegura que
  existe un \(J\in\Gamma\) maximal.

  Solo basta ver que \(M'=N\dot{+}\left(\dot{+}_{j\in J}M_j\right)\) y
  basta ver que es la suma (ya sabemos que es directa). Para \(i\notin J\),
  \(M_i\cap\left(N+\left(\dot{+}_{j\in J} M_j\right)\right)\neq \{0\}\).
  De lo contrario, \(J\cup\{i\}\in\Gamma\) y \(J\) no sería maximal.
  Como \(M_i\) es simple, \(M_i\subseteq\left(N+\left(\dot{+}_{j\in J}
  M_j\right)\right)\). Al final tenemos que
  \(M_i\subseteq\left(N+\left(\dot{+}_{j\in J}
  M_j\right)\right)\) para todo \(i\in I\), con lo que
  \(M'=N\dot{+}\left(\dot{+}_{j\in J}M_j\right)\).

\end{proof}

\begin{df}[Anillo de división]
  Un anillo \(D\) se dice que es un anillo de división si
  para todo \(d\in D\setminus\{0\}\) existe un \(d^{-1}\) tal que
  \[
    dd^{-1}=d^{-1}d=1
  \]
  Si \(D\) es además conmutativo, es entonces un cuerpo.
\end{df}

\begin{cor}
  Sea \(D\) un anillo de división y \(\subscriptbefore{D}{V}\) un
  \(D\)-espacio vectorial no nulo. Si \(\{v_i:i\in I\}\) es un conjunto
  de generadores no nulos de \(V\), existe \(J\subseteq I\) tal que
  \(\{v_j:j\in J\}\) es una base de \(\subscriptbefore{D}{V}\).
\end{cor}
\begin{proof}
  Tomo la familia \(\{Dv_i:i\in I\}\). Cada \(Dv_i\cong D/\ann_D (v_i)
  \cong \subscriptbefore{D}{D}\) ya que el anulador de cualquier elemento
  en un anillo de división es cero.

  \(\subscriptbefore{D}{D}\) es un módulo simple.
  \[
    V=\sum_{i\in I} Dv_i
  \]
  Tomando \(N=\{0\}\) en la proposición, existe un \(J\in I\) tal que
  \[
    V=\dot{+}_{j\in J} Dv_j
  \]
  o equivalentemente \(\{v_j: j\in J\}\) es base de \(V\).

\end{proof}

\begin{obs}
  En la proposición anterior se ve que \(V\cong D^{(J)}\).
\end{obs}

\begin{df}[Homomorfismo escindido]
  Dado homomorfismos de módulos \(N\overset{g}{\longrightarrow} M
  \overset{f}{\longrightarrow} N\) tales que \(f\circ g=\id_N\), diremos
  que \(f\) es un epimorfismo escindido (o roto o partido)
  y que \(g\) es un monomorfismo escindido (o roto o partido).
\end{df}

\begin{teo}
  Las siguientes condiciones son equivalentes para un módulo \(M\):
  \begin{enumerate}
    \item Todo submódulo de \(M\) es un sumando directo.
    \item Todo monomorfismo \(L\longrightarrow M\) es escindido.
    \item Todo epimorfismo \(M\longrightarrow M \) es escindido.
    \item \(\Soc(M)=M\).
    \item \(M\) es suma de una familia de submódulos simples.
    \item \(M\) es suma directa interna de una familia de submódulos simples.
  \end{enumerate}
\end{teo}



    \begin{teo}
  Las siguientes condiciones son equivalentes para un módulo \(M\):
  \begin{enumerate}
    \item Todo submódulo de \(M\) es un sumando directo.
    \item Todo monomorfismo \(L\longrightarrow M\) es escindido.
    \item Todo epimorfismo  \(M\longrightarrow N\) es escindido.
    \item \(\Soc(M)=M\).
    \item \(M\) es suma de una familia de submódulos simples.
    \item \(M\) es suma directa interna de una familia de submódulos simples.
  \end{enumerate}
  En cualquiera de los casos diremos que \(M\) es semisimple.
\end{teo}

\begin{proof}
  Como todas las afirmaciones son triviales ciertas si \(M=\{0\}\),
  suponemos que \(M\neq \{0\}\).

  Vamos a ver que la primera afirmación implica la tercera.
  Sea \(\phi: M\longrightarrow N\). Tomemos \(L=\ker\phi\).
  Por hipótesis \(M=L\dot{+}X\) para cierto \(X\in\mathcal{L}(M)\).
  Tenemos que, por los teoremas de isomorfía:
  \[
    N\cong M/L=(L\dot{+}X)/L\cong X/(L\cap X)\cong X/\{0\}\cong X
  \]

  Tenemos que para cada \(x\in X\) se va identificando con \(x+\{0\}\),
  y \(x+L\), que se identifica con \(\phi(x)\) a través de los isomorfismos
  anteriores.

  Es decir, la aplicación anterior es \(\phi|_X:X\longrightarrow N\).

  Definimos \(\varphi:N\longrightarrow M\) como \(\varphi:=\iota\circ
  {(\phi)}^{-1}\), que cumple que \(\phi\circ\varphi=\id_N\).

  Veamos que la tercera afirmación implica la segunda. Sea
  \(\varphi:L\longrightarrow M\) un monomorfismo. Consideramos la sucesión
  exacta corta dada por \(0\longrightarrow L\overset{\varphi}{\longrightarrow}
  M\overset{\kappa}{\longrightarrow} C\longrightarrow 0\) donde
  \(C=M/\Im\varphi\) y \(\kappa\) es la proyección canónica.

  Existe un \(g:C\longrightarrow M\) tal que \(\kappa\circ g=\id_C\).
  Defino \(h=\id_M-g\circ\kappa:M\longrightarrow M\).
  \[
    \kappa\circ h=\kappa -\kappa\circ g\circ\kappa =\kappa-\kappa =0
  \]
  con lo que \(\Im h\subseteq\ker\kappa\).

  Tenemos \(f:M\longrightarrow L\) tal que \(\varphi\circ f=h\) (es
  decir, \(h\) pero visto en \(L\)). Se dejan como ejercicio los detalles.
  \[\varphi\circ(f\circ\varphi)=h\circ\varphi=
  \varphi-g\circ\kappa\circ\varphi=\varphi\]
  donde el segundo sumando se anula por exactitud.

  Por la inyectividad de \(\varphi\), tenemos que podemos cancelar
  a izquierda y por tanto \(f\circ\varphi=\id_L\).

  Veamos que la segunda afirmación implica la primera, con lo que tendremos
  ya que las tres primeras son equivalentes.

  Tomamos \(X\in\mathcal{L}(M)\), tenemos que \(\iota:X\longrightarrow M\)
  es un monomorfismo. Por hipótesis, existe un \(p:M\longrightarrow X\) tal
  que \(p|_X=\id_X\). Entonces se tiene que:
  \[
    M=X\dot{+}\ker p
  \]
  que es un ejercicio sencillo.

  Vamos a ver que de la cuarta afirmación se deduce la quinta.
  La cuarta afirmación dice que \(M=\sum N_i\) donde \(N_i\) son los
  submódulos simples de \(M\).

  Veamos ahora que de la quinta se sigue la sexta. Por una proposición
  anterior (la 23)
  tomando \(N=0\), tenemos que es cierta.

  Trivialmente, la última afirmación implica la primera, tomando \(N\)
  cualquiera en la proposición 23.

  Basta ver ahora que la primera afirmación implica la cuarta.
  Por hipótesis \(M=\Soc(M)\dot{+}X\) para cierto \(X\). Veamos que
  \(X=\{0\}\). Si no fuera así, tomamos \(m\in X\setminus\{0\}\).
  El lema previo nos asegura que hay un epimorfismo
  \(p:Rm\longrightarrow S\) para \(S\) simple.

  De nuevo, \(Rm\) es un sumando directo de \(M\), existe un epimorfismo
  \(\pi:M\longrightarrow Rm\). Hacemos la composición \(p\circ\pi:
  M\longrightarrow S\).

  Como la hipótesis primera equivale a la tercera, existe
  \(\iota:S\longrightarrow M\) (por una vez no es inclusión) tal que
  \(p\circ\pi\circ\iota=\id_S\).
  \[
    S\cong\Im(\pi\circ\iota)\subseteq Rm\subseteq X
  \]
  donde hemos usado el primer teorema de isomorfía a una aplicación
  inyectiva.\@
  luego \(X\) contiene a una copia de un simple y no es simple. Así que
  \(X=0\) y \(M=\Soc(M)\).

\end{proof}

\begin{cor}
  Si M es finitamente generado y no nulo, existe un \(N\le M\) tal que
  \(M/N\) es simple.
\end{cor}

    \begin{cor}
  Todo cociente de y todo submódulo de un módulo semisimple es semisimple.
\end{cor}
\begin{proof}
  Sea \(M\) semisimple y tomamos \(N\) un submódulo. Veamos que \(M/N\) es
  también semisimple. \(M\) es semisimple, luego es suma de módulos simples:
  \[
    M=\sum_{i\in I} S_i
  \]

  Consideremos \(p:M\longrightarrow M/N\) la proyección canónica (\(m\mapsto
  m+N\)). Tenemos que \(M/N=\sum_{i\in I} p(S_i)\). Para cada \(i\in S_i\)
  puede que \(p(S_i)=0\) (que sobran de la suma) o \(p(S_i)\neq 0\).

  \(p(S_i)\) es simple porque \(p:S_i\longrightarrow p(S_i)\) es un
  isomorfismo (inyectiva y definida sobre su imagen).

  Veamos que pasa con los submódulos. \(M=N\dot{+}X\) para algún \(X\).
  Esto implica que \(m=n+x\overset{\pi}{\mapsto} n\) es un epimorfismo de
  módulos entre \(M\) y \(N\).
  Entonces \(N\cong N/\ker\pi\), luego es semisimple.
\end{proof}

\begin{cor}
  \(M\) es semisimple finitamente generado si y solo si \(M=S_1\dot{+}\cdots
  \dot{+} S_n\) para \(S_i\) simple.
\end{cor}
\begin{proof}
  La implicación hacia la izquierda es una aplicación directa del teorema.

  Por otro lado, \(M=\dot{+}_{i\in I} S_i\), por ser simple.
  Sean \(m_1,\ldots, m_t\) generadores de \(M\). Entonces existe \(F\subseteq
  I\) finito tal que
  \[m_j\in \dot{+}_{i\in F} S_i\] para todo \(j\). Entonces
  \[ M\subseteq \dot{+}_{i\in F}\subseteq M\]
  con lo que \(M\) es una suma finita.
\end{proof}

\subsubsection{Anillos semisimples}
\begin{df}[Anillos semisimples]
  Un anillo \(R\) es semisimple si todo \(R\)-módulo es semisimple.
\end{df}

\begin{obs}
  Todo anillo de división es semisimple. ¿Hay más?
\end{obs}

\begin{teo}
  \(R\) es semisimple si y solo si \(\subscriptbefore{R}{R}\) es semisimple.
  Es decir, todos los módulos sobre \(R\) son semisimples si y solo si lo
  es el regular.
\end{teo}
\begin{proof}
  Sea \(\subscriptbefore{R}{M}\) un módulo.
  Está claro que \(Rm\cong R/\ann_R(m)\), que es un cociente de un semisimple,
  luego semisimple para cualquier \(m\). Tenemos que para ciertos \(m\in M\):
  \[
    M=\sum_{m\in M} Rm
  \]
  con lo que \(M\) es suma de semisimples, luego semisimple.

\end{proof}

\begin{df}[Anillo de endomorfismos]
  Sea \(M\) un \(R\)-módulo, definimos:
  \[
    \End_R(M)=\{f:M\longrightarrow M:
    f\textrm{ homomorfismo de módulos sobre }R\}
  \]
  es un subanillo de \(\End(M)\).

  Llamemos \(S=\End_R(M)\), tenemos que \(M\) es un \(S\)-módulo puesto que
  \(S\subseteq \End(M)\). \(\End_R(M)\) es el anillo de endomorfismos de
  \(M\).

  ¿Cuál es la acción en \(M\)? La inclusión: \(f\in S\), tenemos
  \(fm=f(m)\).
\end{df}
\begin{df}[Biendomorfismos]
  ¿Quién es \(\End_S(M)\)?
  Obviamente, \(\End_S(M)\subseteq\End(M)\) subanillo.

  Dado \(g\in\End(M)\), \(g\in \End_S(M)\) si y solo si \(g(fm)=fg(m)\) para
  todo \(f\in S=\End_R(M)\). Pero \(g(f(m))=g(fm)=fg(m)=f(g(m))\) con
  \(m\in M\). Pero esto es lo mismo que decir que \(g\circ f=f\circ g\).

  \[
    \End_S(M)=\{g\in\End(M):g\circ f=f\circ g \quad\forall f\in\End_R(M)\}
  \]

  Llamaremos \(T=\End_S(M)\).
\end{df}

\begin{lema}
  \(R\overset{\lambda}{\longrightarrow} \End_S(M)\) dado por
  \(\lambda(r):M\longrightarrow M\) y \(\lambda(r)(m)=rm\).
  Dicho \(\lambda\) es un homomorfismo de anillos.
\end{lema}
\begin{proof}
  Basta con ver que \(\Im\lambda\subseteq\End_S(M)\). O sea que
  \(\lambda(r)\circ f=f\circ\lambda(r)\) para todo \(r\in R\).
  En efecto:
  \[
    (\lambda(r)\circ f)(m)=rf(m)=f(rm)=(f\circ\lambda(r))(m)
  \]
  para todo \(f\in S\) y todo \(m \in M\).
\end{proof}
\begin{obs}
  El conjunto de ``triendomorfismos'' coincide con el de endomorfismos.
\end{obs}
\begin{prop}
  Los \(R\)-sumandos directos de \(M\) son los mismos que los \(T\)-sumandos
  directos de \(M\).

  Como consecuencia, si \(\subscriptbefore{R}{M}\) es semisimple, entonces
  \(\subscriptbefore{T}{M}\).
\end{prop}
\begin{proof}
  Si \(N\) es un \(T\)-sumando directo de \(M\) tenemos que
  \(M=N\dot{+}X\) para cierto \(X\in\mathcal{L}(\subscriptbefore{T}{M})\).
  Entonces \(N\dot{+}X=M\) como \(R\)-módulos.

  Recíprocamente, \(\subscriptbefore{R}{M}=X\dot{+}Y\) con \(X,Y\mathcal{L}
  (\subscriptbefore{R}{M})\). Basta ver que \(X\) es un \(T\)-módulo.

  Tomo \(p:M\longrightarrow M\), tal que \(p(x+y)=x\) y \(p\in S\).
\end{proof}


    
\begin{cor}
  Si \(\subscriptbefore{R}{M}\) es semisimple,
  \(\ell(\subscriptbefore{R}{M})<\infty\), entonces \(\subscriptbefore{T}{M}\)
  y \(\ell(\subscriptbefore{R}{M})=\ell(\subscriptbefore{T}{M})\).
\end{cor}

Tenemos que, dado \(\subscriptbefore{R}{M}\),
\(\subscriptbefore{R}{M}^n=M\oplus\overset{(n)}{\cdots}\oplus M\).
Sea \(S'=\End_R(M^n)\).

Sea \(\iota_i\) la aplicación dada por \(m\mapsto(0,\ldots,0,m,0,\ldots0)\) y
\(\pi_j\) la que aplica \(m_i=(m_1,\ldots, m_j,\ldots, m_n)\mapsto m_j\).

Tenemos que \(\id_{M^n}=\sum_{i=1}^n\iota_i\circ\pi_i\in S'\).
Dado \(f\in\End_S(M)\), definimos \(\bar{f}=\sum_{i=1}^n\iota_i\circ
f\circ\pi_i\in\End(M^n)\), en concreto
\[
  \bar{f}(m_1,\ldots, m_n)=(f(m_1),\ldots,f(m_n))
\]

A partir de ahora prescindimos del símbolo \(\circ\) para indicar composición.

Tomando \(g\in S'\), tenemos que
\[
  g\bar{f}=\sum_{i,j=1}^n \iota_i\pi_i g \iota_j f\pi_j
  =\sum_{i,j=1}^n \iota_i f\pi_i g\iota_j\pi_j=\bar{f}g
\]
con lo que \(f\in \End_{S'}(M^n)\).

\begin{teo}[de densidad de Jacobson]
  Sea \(M\) un \(R\)-módulo semisimple. Sean \(m_1,\ldots, m_n\in M\)
  y \(S=\End_R(M)\). Para cada \(f\in\End_S(M)\) existe un \(r\in R\)
  tal que \(f(m_i)=rm_i\) para todo \(i\in\{1,\ldots,n\}\).
\end{teo}
\begin{proof}
  Sea \(m=(m_1,\ldots, m_n)\in M^n\). Sé que \(M^n\) es \(R\)-semisimple.
  \(
    Rm
  \) es un \(R\)-sumando directo de \(M^n\). Entonces \(Rm\) es un
  \(\End_{S'}(M^n)\)-submódulo de \(M^n\).

  Como \(\bar{f}\in\End_{S'}(M^n)\), entonces
  \((f(m_1),\ldots,f(m_n))=\bar{f}(m)=\bar{f} m\in Rm\), con lo que
  existe un \(r\in R\) tal que \(f(m_i)=rm_i\).

\end{proof}

\begin{lema}[de Schur]
  Sean \(\subscriptbefore{R}{M}\), \(\subscriptbefore{R}{N}\) y
  \(f:M\longrightarrow N\) es homomorfismo de \(R\)-módulos,
  entonces \(f\) o es 0 o es un isomorfismo.

  Así, \(\End_R(M)\) es una anillo de división.
\end{lema}

\begin{prop}
  Sea \(R\) tal que \(\subscriptbefore{R}{R}\) es artiniano y
  \(\subscriptbefore{R}{M}\) un módulo simple. Si \(\subscriptbefore{R}{M}\)
  es fiel (\(\Ann_R(M)=\{0\}\)),
  entonces \(\lambda:R\longrightarrow\End_D(M)\) es un isomorfismo, donde
  \(D=\End_R(M)\).
  Además, \(\dim_D M<\infty\).
\end{prop}


    \begin{prop}
  Sea \(R\) tal que \(\subscriptbefore{R}{R}\) es artiniano y
  \(\subscriptbefore{R}{M}\) un módulo simple. Si \(\subscriptbefore{R}{M}\)
  es fiel (\(\Ann_R(M)=\{0\}\)),
  entonces \(\lambda:R\longrightarrow\End_D(M)\) es un isomorfismo, donde
  \(D=\End_R(M)\).
  Además, \(\dim_D M<\infty\).
\end{prop}
\begin{proof}
  Supongamos que \(\subscriptbefore{D}{M}\) no fuera de dimensión finita.
  Entonces \(M\) admite una base \(B\) infinita. Tomamos \(\{x_i:i\in N\}
  \subseteq B\) linealmente independiente.

  Dado \(i\in \N\), tomamos \(f_i:M\longrightarrow M\) la aplicación
  \(D\)-lineal que vale 0 sobre todo elemento de \(B\) y sobre
  \(f_i(x_i)=x_i\).

  Cada \(f_i\in\End_D(M)\). El teorema de densidad nos permite asegurar
  que existe \(r_i\in R\) tal que \(f_i(x_j)=r_i x_j\) para
  \(j=0,\ldots, i\).

  \(r_i\in\ann_R(x_0)\cap\ldots\cap\ann_R(x_{i-1})\), pero el
  \(r_i\in\ann_R(x_0)\cap\ldots\cap\ann_R(x_{i-1})\cap\ann_R(x_i)\).
  Tenemos que \[\ann_R(x_0)\cap\ldots\cap\ann_R(x_{i-1})
  \supsetneq\ann_R(x_0)\cap\ldots\cap\ann_R(x_{i-1})\cap\ann_R(x_i)\]
  con \(i\ge 1\) y tenemos una cadena descendente y por tanto
  \(\subscriptbefore{R}{R}\) no es artiniano.

  Tomo \(\{m_1,\ldots,m_n\}\) una base de \(M\). Dado \(f\in\End_D(M)\),
  el teorema de densidad asegura que existe \(r\in R\), entonces
  \(f(m_i)=rm_i\) para todo \(i\in\{1,\ldots,n\}\). Basta tomar
  \(\lambda(r)=f\) y tenemos que es sobreyectivo. Como además \(M\)
  es fiel, \(\lambda\) es un isomorfismo.

\end{proof}

\begin{df}[Idempotentes]
  Un elemento \(e\in R\) se dice idempotente si \(e^2=e\).

  Un conjunto \(e_1,\ldots,e_n\in R\) de idempotentes se dice un conjunto
  completo de idempotentes ortogonales (CCIO) si:
  \[
    e_i e_j=0
  \] siempre que \(i\neq j\) y además:
  \[
    e_1+\cdots+e_n = 1
  \]
\end{df}
\begin{prop}
  Si \(\{e_1,\ldots,e_n\}\) es CCIO, entonces
  \(R=Re_1\dot{+}\cdots\dot{+}Re_n\).
\end{prop}
\begin{proof}
  Sea \(r\in R\), tenemos que \(r=r1=re_1+\cdots+ren\).

  Por otro lado, si \(0=r_1 e_1+\cdots+r_n e_n\) para \(r_i\in R\),
  entonces multiplicando la identidad por \(e_i\) nos queda
  \(0=r_i e_i^2=r_i e_i\) para cada \(i\in\{1,\ldots,n\}\).

\end{proof}

\begin{df}[Anillo simple]
  \(R\) es simple si y solo si los únicos ideales de \(R\) son \(\{0\}\)
  y \(R\).
\end{df}

\begin{teo}
  Son equivalentes, para un anillo no trivial:
  \begin{enumerate}
    \item \(\subscriptbefore{R}{R}\) semisimple y todos los \(R\)-módulos
      simples son isomorfos entre sí.
    \item \(R\) es isomorfo como anillo a \(\End_D(M)\) con \(D\) de división
      y \(\subscriptbefore{D}{M}\) es de dimensión finita.
    \item \(\subscriptbefore{R}{R}\) artiniano y existe un \(R\)-módulo simple
      y fiel.
    \item \(\subscriptbefore{R}{R}\) es artiniano y simple.
  \end{enumerate}

  Además, para la segunda afirmación se da necesariamente que
  \(D\cong\End_R(\Sigma)\) para cualquier \(\subscriptbefore{R}{\Sigma}\)
  el único sumódulo simple salvo isomorfismo dado en la primera afirmación.
  Por último, necesariamente, \(\dim_D(M)=\ell(\subscriptbefore{R}{R})\).
\end{teo}
\begin{proof}
  Veamos que la primera afirmación implica la cuarta.
  Sabemos que \(\subscriptbefore{R}{R}\) tiene longitud finita.
  Sea \(I\) un ideal de \(R\) propio (\(R\neq \{0\}\)).
  \(R/I\) es semisimple como \(R\)-módulo. Como es finitamente generado,
  es suma directa finita de simples. Todos esos submódulos son isomorfos
  entre sí.
  \[
    R/I\cong \Sigma^n
  \]
  donde \(\subscriptbefore{R}{\Sigma}\) es simple.

  Tenemos que \(I=\Ann_R(R/I)\) (aquí es donde hace falta que sea ideal y no
  solo ideal por la izquierda).
  \[
    I=\Ann_R(R/I)=\Ann_R(\Sigma^n)=\Ann_R(\Sigma)
  \]

  Por otro lado \(R\cong\Sigma^m\), con  \(m=\ell(\subscriptbefore{R}{R})\).
  \[
    I=\Ann_R(\Sigma)=\Ann_R(\Sigma^m)=\Ann_R(R)=\{0\}
  \]

  Demostremos ahora que la cuarta afirmación implica la tercera.
  Tomamos \(\subscriptbefore{R}{\Sigma}\) simple (existe tomando el primero
  de la serie de descomposición, por ser artiniano).
  \(R\neq\Ann_R(\Sigma)\) por se simple, luego \(\Ann_R(\Sigma)=\{0\}\).
  Luego \(\subscriptbefore{R}{\Sigma}\) fiel.

  La segunda afirmación se deduce de la tercera por la proposición anterior.

  Veamos finalmente que la cuarta afirmación implica la primera. Tomamos
  \(S=\End_D(M)\). Si \(m, m'\in M\) con \(m\neq 0\), existe entonces un
  \(f\in S\) tal que \(f(m)=m'\).

  Así, \(Sm=M\) con lo que \(\subscriptbefore{S}{M}\) es simple.
  Sea \(\{m_1,\ldots, m_n\}\) \(D\)-base de \(M\). Para \(i\in\{1,\ldots,n\}\)
  defino \(e_i\in S\) tal que
  \[
    e_i(m_j)=
    \left\{
    \begin{matrix}
      0&\textrm{si } j\neq i\\
      m_i&\textrm{si } j = i
    \end{matrix}
    \right.
  \]

  \(\{e_1,\ldots,e_n\}\) es CCIO de \(S\), entonces \(S=Se_1\dot{+}
  \cdots\dot{+}Se_n\).
  Veamos que \(Se_i\) es simple. Basta con demostrar que si \(f\in S\)
  tal que \(fe_i\neq 0\) entonces \(Sfe_i=Se_i\).
  \[
    fe_i=f(e_i)=\sum_{j=1}^n a_j m_j
  \]
  con \(a_j\in D\). Tomamos un índice \(k\) tal que \(a_k\neq 0\) (posible
  porque \(fe_i\neq 0\)). Tenemos que \(s(m_k)=a_k^{-1} m_i\) y
  \(s(m_j)=0\) si \(j\neq k\).

  Tenemos entonces
  \[
    sfe_i(m_i)=s(\sum_j a_j m_j)=a_k^{-1}a_k m_i=m_i
  \]
  con lo que \(sfe_i=e_i\) y por tanto \(Se_i=Sfe_i\) con lo que
  \(\subscriptbefore{S}{S}\) es semisimple.

  Veamos que cualquier módulo simple es isomorfo a \(\subscriptbefore{S}{S}\).
  Para ver que cada \(Se_i\) es isomorfo a \(\subscriptbefore{S}{M}\), por
  el lema de Schur, basta encontrar un homomorfismo no nulo
  \(Se_i\longrightarrow M\). Sea \(F:Se_i\longrightarrow M\) dado por
  \(F(f):=f(m_i)=f m_i\). Es fácil ver que \(F\) es un \(S\)-módulo.
  \(F(e_i)=e_i(m_i)=m_i\neq 0\), con lo que \(F\neq 0\) y por el lema
  de Schur es un isomorfismo.

  Si \(\subscriptbefore{S}{\Sigma}\) es simple, luego existe un epimorfismo
  \(p:S\longrightarrow\Sigma\) de \(S\)-módulos (descomponer por anuladores
  de cualquiera de sus elementos). Como \(p\neq 0\), existe un \(i\) tal
  que \(p|_{Se_i}\neq 0\) y por tanto es un isomorfismo.

  Sea \(\phi:S\longrightarrow R\) un isomorfismo de anillos.
  \(\{\phi(e_1),\longrightarrow,\phi(e_n)\}\) es claramente CCIO de R.\@
  En particular, \(R=\dot{+}_{i=1}^n R\phi(e_i)\).
  Cada \(R\phi(e_i)\) es simple como \(R\)-módulo.
  \(Se_i\cong \subscriptbefore{S}{M}\), y \(\subscriptbefore{R}{M}\) por
  restricción de escalares. Comprobando que
  \(\mathcal{L}(\subscriptbefore{S}{M})=\mathcal{L}(\subscriptbefore{R}{M})\),
  deducimos que \(\subscriptbefore{R}{M}\) es simple.

  \(R\phi(e_i)\), veamos que
  \(\mathcal{L}{R\phi(e_i)}\cong\mathcal{L}{R\phi(e_i)}\)
  dados por \(I\mapsto\phi(I)\) y \(J\mapsto\phi^{-1}(M)\), luego son dos
  conjuntos ordenados por la inclusión isomorfos. Luego como uno solo tiene
  dos elementos, en el otro también.

  \(R\phi(e_i)\) es simple y por tanto \(\subscriptbefore{R}{R}\) es
  semisimple. Además:
  \[
    \dim_D(M)=n=\ell(\subscriptbefore{S}{S})=\ell(\subscriptbefore{R}{R})
  \]

  Sea \(\Sigma\) un \(R\)-módulo simple. Mediante restricción de escalares
  es un \(S\)-módulo simple, luego \(\subscriptbefore{S}{\Sigma}\cong
  \subscriptbefore{S}{M}\) con lo que \(\subscriptbefore{R}{\Sigma}\cong
  \subscriptbefore{R}{M}\).

  \[
    \lambda:D\longrightarrow\End_S(M)=\End_R(M)
  \]
  es, por densidad, un isomorfismo.

\end{proof}



    
\begin{lema}
  Sea \(R\) un anillo. Existe un conjunto \(\Omega_R\) (y es un conjunto
  y no una clase) de \(R\)-módulos simples no isomorfos entre sí tal que
  cualquier \(R\)-módulo simple es isomorfo a uno de los \(\Omega_R\).
\end{lema}
\begin{proof}
  Sea \(\subscriptbefore{R}{\Sigma}\) simple. Tomo
  \(0\neq s\in\Sigma\) entonces \(\subscriptbefore{R}{\Sigma}\cong
  R/\ann_R(s)\). Tomo \(\Omega_R\) un conjunto de representantes de los
  \(R\)-módulos \(R/I\) para \(I\) ideal izquierda maximales bajo
  la relación de equivalencia \(I\sim J\) si y solo si
  \(R/I\cong R/D\).

\end{proof}

\begin{prop}
  \(\subscriptbefore{R}{M}\) un módulo. Para \(\Sigma\in\Omega_R\), defino
  \(\Soc_\Sigma(M)\) como la suma de todos los submódulos simples de \(M\)
  isomorfos a \(\Sigma\). Entonces:
  \[
    \Soc(M)=\dot{+}_{\Sigma\in\Omega_R}\Soc_\Sigma(M)
  \]
\end{prop}
\begin{proof}
  \[
    \Soc(M)=\sum_{\Sigma\in\Omega_R}\Soc_\Sigma(M)
  \]
  por la definición de \(\Omega_R\). Muchos de ellos serán cero.

  Llamamos \(N=\Soc_{\Sigma'}(M)\cap\sum_{\Sigma\neq\Sigma'}\Soc_\Sigma(M)\).
  Tomamos \(m\in N\setminus\{0\}\), suponiendo que \(N\neq \{0\}\). Tenemos
  que \(Rm\) es semisimple y es finitamente generado y por tanto de longitud
  finita. Entonces contiene un \(S\) \(R\)-submódulo simple de \(Rm\).

  Tenemos que \(S\subseteq\Soc_{\Sigma'}(M)\). Existe entonces
  \(g:\Soc_{\Sigma'}(M)\longrightarrow S\) epimorfismo (porque escinde).
  Existe \(S'\in\Soc_{\Sigma'}(M)\) tal que \(S'\cong \Sigma'\) tal que
  \(g|_{S'}\neq0\) entonces por Schur \(S'\cong S\). Análogamente,
  se demuestra que \(S''\cong\Sigma\neq\Sigma'\) tal que \(S''\cong S\).
  Resulta que \(\Sigma'\cong S\cong\Sigma\) y están relacionados, lo que
  contradice la definición de \(\Omega_R\).

\end{proof}

    \begin{obs}
  Sea \(f\in\End_R(M)\). Entonces:
  \[
    f(\Soc_\Sigma(M))=f\left(\sum_{S\cong\Sigma, S\in\mathcal{L}(M)} S\right)
    =\sum_{S\cong\Sigma, S\in\mathcal{L}(M)}f(S)\subseteq\Soc_\Sigma(M)
  \]

  Tomando \(M=R\) y \(f=\rho_r\) para \(\rho_r:R\longrightarrow R\) definida
  por \(\rho_r(r')=r'r\), entonces \(\rho_r(\Soc_\Sigma(R))\subseteq
  \Soc_\Sigma(R)\) y tenemos que \(\Soc_\Sigma(R)\) es un ideal de \(R\).
\end{obs}

\begin{obs}
  \(\Omega_\Z\) es biyectivo con
  \(\{\Z_p:p\textrm{ es primo}\}\), luego es un conjunto
  infinito.
\end{obs}

\begin{teo}[Estructura de anillos semisimples]
  Sea \(R\) un anillo semisimple. Entonces \(\Omega_R\) es finito.
  Si ponemos \(\Omega_R=\{\Sigma_1,\ldots,\Sigma_t\}\) y
  \(D_i=\End_R(\Sigma_i)\), entonces
  \[
    R\cong\End_{D_i}(\Sigma_1)\times \cdots\times\End_{D_t}(\Sigma_t)
  \]
  y \(\dim_{D_i}(\Sigma_i)\) es finita.
\end{teo}
\begin{proof}
  Sé que \(\subscriptbefore{R}{R}=S_1\dot{+}\cdots\dot{+} S_n\) donde
  \(\subscriptbefore{R}{S_i}\) es simple. Así, si
  \(\subscriptbefore{R}{\Sigma}\) es simple, entonces
  \(\R\overset{p}{\longrightarrow}\Sigma\) epimorfismo, donde
  \(p|_{S_i}\) es un isomorfismo para algún \(i\) y por Schur,
  \(S_i\cong\Sigma\).
  Así que \(\Omega_R\) es finito.

  Tenemos que \(\Soc_{\Sigma_i}(R)\Soc_{\Sigma_j}(R)\subseteq
  \Soc_{\Sigma_i}(R)\cap\Soc_{\Sigma_j}(R)=\{0\}\). Eso implica
  que
  \(\Soc_{\Sigma_j}(R)\subseteq\Ann_R(\Soc_{\Sigma_j}(R))=\Ann_R(\Sigma_i)\).

  Llamo a \(I_i=\sum_{j\neq i}\Soc_{\Sigma_j}(R)\). Tenemos que
  \(I_i+I_j=R\) si \(i\neq j\). De la inclusión anterior se deduce
  \(\Ann_R(\Sigma_i)+\Ann_R(\Sigma_j)=R\) si \(i\neq j\).
  Se cumple que:
  \[
    R\longrightarrow R/\Ann_R(\Sigma_1)\times\cdots
    \times R/\Ann_R(\Sigma_t)
  \]
  tal que
  \[
    r\mapsto (r+\Ann_R(\Sigma_1),\ldots,r+\Ann_R(\Sigma_t))
  \]
  es un homomorfismo de anillos cuyo núcleo es
  \(\bigcap_{i=1}^t \Ann_R(\Sigma_i)=\bigcap_{i=1}^n \Ann_R(S_i)=\{0\}\).
  donde simplemente puede haber algún \(\Ann_R(S_i)\) repetido.

  Cada \(R/\Ann_R(\Sigma_i)\) es artiniano (de longitud finita por ser
  cociente de uno de longitud finita). \(\Sigma_i\) es un
  \(R/\Ann:R(\Sigma_i)\)-módulo simple fiel. Nuestro teorema nos garantiza
  que \(R/\Ann_R(\Sigma_i)\cong\End_{D_i}(\Sigma_i)\) para
  \(D_i=\End_{R/\Ann_R(\Sigma_i)}(\Sigma_i)\cong\End_R(\Sigma_i)\) y
  \(\dim_{D_i}(\Sigma_i)\) es finita.

\end{proof}

Ejemplo: \(R, S\) dos anillos. Sea \(T=R\times S\). Vamos a definir
\(e=(1,0)\), \(\mathcal{L}(\subscriptbefore{T}{Te})\longrightarrow
\mathcal{L}(\subscriptbefore{R}{R})\) dada por \(I\mapsto\pi(I)\) donde
\(\pi\) es la proyección en la primera componente, es una biyección que
preserva la inclusión.

Como consecuencia \(\subscriptbefore{T}{Te}\) es semisimple si y solo si
\(\subscriptbefore{R}{R}\).

Así, \(T\) es semisimple si y solo si \(R\) y \(S\) si y solo si \(T\)
son semisimples, ya que:
\[
  \subscriptbefore{T}{T}=Te\dot{+}T(1-e)
\]

Ejercicio: Sean \(D, E\) dos anillos de división, \(\subscriptbefore{D}{M}\),
\(\subscriptbefore{E}{N}\) espacios vectoriales. Se pide demostrar que
\[
  \End_D(M)\cong\End_E(N)\;\iff\;
  \left\{
  \begin{matrix}
    D\cong E\\
    \dim_D(M)=\dim_E(N)
  \end{matrix}
  \right.
\]

    \subsection{Descomposición de anillos en ideales indescomponibles}
\begin{df}[El centro de un anillo]
  Sea \(R\) un anillo. El conjunto
  \[
    Z(R)=\{r\in R:rs=sr\quad\forall s\in R\}
  \]
  es un subanillo conmutativo de \(R\) que se llama centro de \(R\).
\end{df}

\begin{df}[Idempotente central]
  Si \(e\in Z(R)\) verifica \(e^2=e\) diremos que es un idempotente central
  de \(R\).
\end{df}

Si \(e\) es un idempotente central, \(Re\) es ideal y de \(R\) y además
es un anillo con la suma y el producto heredados de \(R\) cuyo 1 es \(e\).

Ejemplo: dados \(R_1\) y \(R_2\) anillos, \(R=R_1\times R_2\), \(e=(1,0)\)
que es idempotente central, entonces \(Re=R_1\times\{0\}\) es un anillo
isomorfo a \(R_1\).

\begin{obs}
  Si \(e\) es idempotente central, \(1-e\) es idempotente central.
\end{obs}

Tenemos que \(\{e, 1-e\}\) CCIO centrales. De hecho:
\[
  R=Re\dot{+}R(1-e)\cong Re\times R(1-e)
\]

Al revés, si \(R=I\dot{+}J\) con \(I,J\) son ideales, \(1=e+(1-e)\),
con \(e\in I\), \(1-e\in J\) con ambos centrales y \(I=Re\) y \(J=R(1-e)\).

Contraejemplo: \(R=\mathcal{M}_{2\times 2}(K)\) con \(K\) un cuerpo.
\[
  e=
  \begin{pmatrix}
    1&0\\
    0&0
  \end{pmatrix}
\]
es idempotente no central.
\[
  Re=
  \begin{pmatrix}
    K&0\\
    K&0
  \end{pmatrix}
\]
pero
\[
  eR=
  \begin{pmatrix}
    K&K\\
    0&0
  \end{pmatrix}
\]

\begin{df}[Ideal indescomponibles]
  Si \(I=I_1\dot{+}I_2\) con \(I_i\) ideales implica que al menos uno
  de ellos es \(\{0\}\), entonces diremos que es indescomponible.
\end{df}

\begin{df}[Anillos indescomponibles]
  Diremos que \(R\) anillo es indescomponible si lo es como ideal.
\end{df}

\begin{df}[Idempotentes indescomponibles]
  Sea \(e\) un idempotente central de \(R\). \(e\) se dice
  indescomponible si \(e=e'+e''\), \(e'\) y \(e''\) idempotentes centrales
  ortogonales (\(e'e''=0\)), uno de ellos es cero.
\end{df}

\begin{obs}
  Hay una equivalencia entre los ideales indescomponibles y los idempotentes
  indescomponibles.
\end{obs}

Ejercicio: \(R\) es indescomponible si y solo si no es isomorfo a ningún
anillo de la forma \(R_1\times R_2\) con \(R_1\), \(R_2\) anillos no
triviales.

\begin{obs}
  Ningún dominio de integridad puede expresarse como producto de dos anillos.
  Luego es indescomponible.
\end{obs}

\begin{prop}
  Si un anillo tiene un CCIO centrales indescomponibles, entonces es único.
  Además, si \(\ell(\subscriptbefore{R}{R})<\infty\), entonces \(R\) admite
  un CCIO central indescomponibles.
\end{prop}
\begin{proof}
  Supongamos que haya dos tales conjuntos \(\{e_1,\ldots, e_n\}\) y
  \(\{f_1,\ldots, f_m\}\). Basta ver que uno está incluido en el otro.

  \(e_i f_i\) es idempotente central. Tenemos que
  \[
    e_i=e_i f_j +e_i(1-f_j)
  \]
  es una descomposición de un idempotente indescomponible, así que o bien
  \(e_i f_j\) o bien \(e_i (1-f_j)\) es cero. Si ocurriera que
  \(e_i f_j\neq 0\),
  \(e_i=e_i f_j\). Análogamente \(f_j=e_i f_j\), aplicando el razonamiento
  a \(f_j\). Entonces \(e_i=f_j\).

  Dado \(e_i\), \(0\neq e_i=e_i 1=e_i(f_1+\cdots+f_m)\) y al menos hay algún
  \(j\) tal que \(e_i f_j\neq 0\) con lo que \(e_i = f_j\).

  Para la segunda parte vamos a aplicar inducción sobre la longitud.
  Si \(R\) es indescomponible no hay nada que demostrar. Si no, es porque
  \(R=Re+R(1-e)\) para \(e\notin\{0,1\}\) indepontente central.
  Tenemos que \(\ell(\subscriptbefore{Re}{Re})<\ell(R)\), y podemos aplicar
  la hipótesis de inducción.

\end{proof}





    \begin{obs}
  \(R\), \(S\) anillos. Si es \(\{e_1,\ldots, e_t\}\) CCIO centrales
  indescomponibles y \(phi:R\longrightarrow S\) es un isomorfismo de anillos.
  Entonces \(\{\phi(e_1),\ldots,\phi(e_t)\}\) es el CCIO centrales
  indescomponibles de \(S\). Además se tiene
  \[
    R=Re_1\dot{+}\cdots\dot{+}Re_t
  \]
  entonces
  \[
    S=S\phi(e_1)\dot{+}\cdots\dot{+}S\phi(e_t)
  \]
  donde \(Re_i\cong S\phi(e_i)\) como anillos.
\end{obs}

Imaginemos que sabemos que \(R\) es semisimple y que disponemos de un
isomorfismo de anillos \(R\cong R_1\times\cdots\times R_s\) con \(R_i\)
indescomponibles.

Por otra parte, \(\Omega_R=\{\Sigma_1,\ldots,\Sigma_t\}\), tengo un isomorfismo
de anillos \(R\cong
\End_{D_1}(\Sigma_1)\times\cdots\times\End_{D_t}(\Sigma_t)\), donde
\(D_i=\End_R(\Sigma_i)\).

Se deduce de la observación:
\[
\End_{D_1}(\Sigma_1)\times\cdots\times\End_{D_t}(\Sigma_t)
\cong
R_1\times\cdots\times R_s
\]
sin más que componer isomorfismos. En primer lugar, \(s=t\) y tras
reordenación \(\End_{D_i}(\Sigma_i)\cong R_i\).
Conocemos los \(e_i\) CCIO centrales
del segundo factor.

\begin{teo}[Teorema de Artin-Wedderburn]
  Si \(R\cong\End_{D_1}(\Sigma_1)\times\cdots\End_{D_t}(\Sigma_t)
  \cong\End_{E_1}(T_1)\times\cdots\End_{E_s}(T_s)\)
  donde \(D_i\), \(E_i\) son todos anillos de división y \(\Sigma_i\),
  \(T_i\) de dimensión finita como espacios vectoriales.

  Entonces \(s=t\) y tras reordenación \(D_i\cong E_i\) y
  \(\dim_{D_i}(\Sigma_i)=\dim_{T_i}(T_i)\).
\end{teo}

La demostración consiste en juntar observaciones, comentarios
y teoremas anteriores.


    \subsection{Módulos a derecha}
\begin{df}[Anillo opuesto]
Sea \(R\) un anillo. Mantengo su estructura de grupo aditivo, pero cambiamos
el producto. El nuevo producto va a ser el producto opuesto dado por
\(r*s:=sr\).

  A \(R\) con este nuevo producto lo vamos a llamar \(R^{op}\), el anillo
  opuesto.
\end{df}

Ejemplo: \[
  R=
  \begin{pmatrix}
    \Z&\Q\\
    0&\Q
  \end{pmatrix}
  \le \mathcal{M}_2(\Q)
\]

Tenemos que \(\subscriptbefore{R}{R}\) no es noetheriano, pero
\(\subscriptbefore{R^{op}}{R^{op}}\) sí que lo es.
Es decir, es noetheriano a derecha pero no a izquierda.

\begin{df}[Anillo noetheriano a derecha]
  Un anillo es noetheriano a derecha si el anillo opuesto es noetheriano
  a izquierda.
\end{df}
\begin{df}[Módulo a derechas]
  Definimos \(M\) módulo a derechas como \(M_R:=\subscriptbefore{R^{op}}{M}\).
\end{df}

\begin{df}[Ideal bilátero]
  Un ideal bilátero es un ideal a izquierda que es ideal a derecha también.
\end{df}

\begin{df}[Dual de un módulo]
  Sea \(\subscriptbefore{R}{M}\) un módulo. Tomamos
  \[
    \superscriptbefore{*}{M}:=\{f:M\longrightarrow R: f
    \textrm{ es homomorfismo de \(R\)-módulos}\}
  \]
  que es un grupo aditivo y un módulo a derechas, es decir,
  \(R^{op}\)-módulo, por la acción:
  \[
    (r\varphi)(m):=\varphi(m)r
  \]
  con \(r\in R\), \(m\in M\) y \(\varphi\in \superscriptbefore{*}{M}\).
  Es decir, \(\superscriptbefore{*}{M}_R\).
\end{df}

\begin{lema}
  \(\theta:{(\End_R(M))}^{op}
  \longrightarrow\End_{R^{op}}(\superscriptbefore{*}{M})\)
  tenemos que \(\theta(f)(\varphi):=\varphi\circ f\) es un homomorfismo
  de anillos.
  Nota: el producto en \({(\End_R(M))}^{op}\) es \(f*g=g\circ f\).
\end{lema}

Si \(M=\Z_n\) como \(Z\) módulos, \(\superscriptbefore{*}{M}=\{0\}\) así que
nos olvidamos de cualquier idea de reflexividad o isomorfismo.

\begin{df}[Módulos reflexivos]
  Un módulo en el que la anterior \(\theta\) es un isomorfismo.
\end{df}


    \begin{prop}
  Si \(\subscriptbefore{R}{M}\) tiene una base \(\{v_1,\ldots,v_n\}\),
  puedo definir \(\{\superscriptbefore{*}{v_1},\ldots,
  \superscriptbefore{*}{v_n}\}\), definidos mediante \(v_i(
  \superscriptbefore{*}{v_j})=\delta_{ij}\). Los \(\subscriptbefore{*}{v_i}\)
  forman una base de \(\superscriptbefore{*}{M}_R\). Además, \(\theta\) es un
  isomorfismo de anillos.
\end{prop}
\begin{proof}
  Veamos que es una base. Observemos que para cualquier \(m\in M\):
  \[
    m=\sum_{i=1}^n \superscriptbefore{*}{v_i}(m)v_i
  \]

  Sea \(\varphi\in\superscriptbefore{*}{M}\):
  \[
    \varphi(m)=\sum_{i=1}^n \superscriptbefore{*}{v_i}(m)\varphi(v_i)
    =\left(\sum_{i=1}^n \varphi(v_i)\superscriptbefore{*}{v_i}\right)(m)
  \]
  luego \(\varphi=\sum_{i=1}^n \varphi(v_i)\superscriptbefore{*}{v_i}\).
  Con lo que los \(\superscriptbefore{*}{v_i}\) generan
  \(\superscriptbefore{*}{\subscriptbefore{R^{op}}{M}}\).

  Si
  \[
    0=\sum_i r_i\superscriptbefore{*}{v_i}
  \]
  entonces:
  \[
    0=\sum_i (r_i\superscriptbefore{*}{v_i})(v_j)=r_j
  \]

  \({\End_R(M)}^{op}\overset{\theta}{\longrightarrow}\End_{R^{op}}
  (\superscriptbefore{*}{M})\) dado por \(\theta(f)(\varphi):=\varphi\circ f\).

  Veamos que \(\theta\) es inyectivo: tomo \(f\) tal que \(\theta(f)=0\).
  Dado \(m\in M\) tenemos:
  \[
    f(m)=\sum_i \superscriptbefore{*}{v_i}(f(m))v_i
    =\sum_i \theta(f)(\superscriptbefore{*}{v_i})(m)v_i=0
  \]
  luego es inyectivo. Veamos que es sobreinyectivo.

  Dado \(\psi\in\End_{R^{op}}(\superscriptbefore{*}{M})\) definimos
  \(f:M\longrightarrow M\) por:
  \[
    f(m)=\sum_i \psi(\superscriptbefore{*}{v_i})(m)v_i
  \]
  Es fácil ver que \(f\in\End_R(M)\cong{\End_R(M)}^{op}\).
  \begin{eqnarray*}
    \theta(f)(\varphi)(m)&=&(\varphi\circ f)(m)=\sum_i
    \psi(\superscriptbefore{*}{v_i})(m)\varphi(v_i)\\
    &=&\left(\sum_i \varphi(v_i)\psi(\superscriptbefore{*}{v_i})\right)(m)\\
    &=&\psi\left(\sum_i \varphi(v_i)\superscriptbefore{*}{v_i}\right)(m)\\
    &=&\psi(\varphi)(m)
  \end{eqnarray*}
  Por lo tanto, \(\theta(f)(\varphi)=\psi(\varphi)\), y se sigue
  \(\varphi=\theta(f)\), con lo que \(\theta\) es sobreyectivo.

\end{proof}

\begin{df}
  Si \(\subscriptbefore{R}{R}\) es semisimple, entonces
  \(R\cong\End_{D_1}(\Sigma_1)\times\cdots\times\End_{D_t}(\Sigma_t)\)
  donde \(D_i\) de dimensión y \(n_i=\dim_{D_i}(\Sigma_i)z\infty\) con
  \(D_i\) únicos salvo isomorfismo y reordenación y \(n_i\) únicos.
  Diremos que \(R\) es de tipo \((D_1,\ldots,D_t,n_1,\ldots,n_t)\).
\end{df}

\begin{teo}
  Si \({R}\) es semisimple de tipo
  \((D_1,\ldots,D_t,n_1,\ldots,n_t)\), entonces \(R^{op}\) es semisimple
  de tipo \((D_1^{op},\ldots,D_t^{op},n_1,\ldots,n_t)\).
\end{teo}
\begin{proof}
  Tenemos que \(R^{op}\cong{\End_{D_1}(\Sigma_1)}^{op}\times\cdots\times
{\End_{D_t}(\Sigma_t)}^{op}\cong
  {\End_{D_1^{op}}(\superscriptbefore{*}{\Sigma_1})}\times\cdots\times
  {\End_{D_t^{op}}(\superscriptbefore{*}{\Sigma_t})}\)
  y \(\dim_{D_i}(\Sigma_i)=\dim_{D_i^{op}}(\superscriptbefore{*}{\Sigma_i})\).
  \(R^{op}\) es semisimple con la estructura del enunciado.

\end{proof}


\begin{cor}
  \(R\) es semisimple si y solo si \(R^{op}\) es semisimple.
\end{cor}




    \section{Algunas aplicaciones}
\subsection{Aplicaciones a grupos finitos}

Sea \(\C\) el cuerpo de los números complejos y \(G\) un grupo con elemento
neutro \(e\). Sea \(\C G\) el \(\C\) espacio vectorial con base \(G\).

\(\mu:\C G\times \C G\longrightarrow \C G\) la aplicación bilineal dada por
\(\mu(g,h)=gh\) para \(g,h\in G\).

Si para \(r,s\in\C G\) denotamos \(rs=\mu(r,s)\) donde si
\(r=\sum_{g\in G} r_g g\) y \(s=\sum_{g\in G} s_g g\), \(r_g, s_g\in\C\),
se tiene:
\[
  rs=\sum_{g,h\in G} r_g s_h\mu(g,h)=\sum_{g,h\in G}r_g s_h gh
\]

Tenemos que \(\mu\) define un producto que es asociativo.

\(\C G\times \C G\times \C G\overset{\mu\times\id}{\longrightarrow} \C G
\times \C G\overset{\mu}{\longrightarrow} \C G\)
proporciona el mismo resultado que
\(\C G\times \C G\times \C G\overset{\id\times\mu}{\longrightarrow} \C G
\times \C G\overset{\mu}{\longrightarrow} \C G\). Pero esto es trivial
porque son dos aplicaciones trilineales que evaluadas en una base dan lo mismo
(porque el producto en \(G\) es asociativo). Este es un ejemplo de un funtor
del producto en \(G\) al producto en \(\C G\).

Al ser bilineal, es distributiva respecto de la suma.

El elemento neutro de este producto es \(1e\in \C G\).

\begin{prop}
  \(\C G\) con la estructura que hemos discutido, es un anillo.
\end{prop}

La aplicación \(\eta:\C\longrightarrow \C G\) dada por \(z\mapsto ze\) es un
homomorfismo de anillos, distinto de 0 y parte de un cuerpo, luego es
inyectivo. Luego consideraremos que \(\Im\eta\) es un subanillo de \(\C G\)
y \(\Im\eta\cong\C\). Vamos a considerar entonces que \(1e=1=e\) y que
\(\C\subseteq\C G\).

Además, \(\C\subseteq\C G\). \(gz=g\eta(z)=gze=zge=zg\), es decir, los
complejos son centrales.

\begin{df}[\(\C\)-álgebra del grupo \(G\)]
  \(\C G\) se llama \(\C\)-álgebra del grupo \(G\).
\end{df}

Definimos \(\mu(G):=\{f:G\longrightarrow\C:f\textrm{ es aplicación}\}\),
es un \(\C\)-espacio vectorial con base \(G\). Vamos a darle estructura
de módulo.

\(\mu(G)\) es un \(\C G\)-módulo definiendo para todo \(g\in G\) y
\(\varphi\in\mu(G)\) y \(x\in G\):
\[
  (g\varphi)(x) = \varphi(xg)
\]
Vemos que \(g(h\varphi)(x)=(h\varphi)(xg)=\varphi(xg)h=\varphi(x(gh))=
(gh)\varphi(x)\). Entonces \(g(h\varphi)=(gh)\varphi\).

Hemos dado una aplicación \(G\longrightarrow\Map(\mu(G),\mu(G))\)
con \(g\mapsto(\varphi\mapsto g\varphi)\). Queremos ver que
\(G\longrightarrow\End(\mu(G),\mu(G))\), es decir
\(g(\varphi+\psi)=g\varphi+g\psi\).

\[
  g(\varphi+\psi)(x)=(\varphi+\psi)(xg)=\varphi(xg)+\psi(xg)
\]
y por otro lado
\[
  g\varphi(x)+g\psi(x)=\varphi(xg)+\psi(xg)
\]

Además:
\[
  g(z\varphi)(x)=z\varphi(xg)=z(g\varphi)(x)
\]

\(\C G\longrightarrow\End_\C(\mu(G),\mu(G))\), es \(\C\)-lineal.

En resumen, tenemos la siguiente proposición:

\begin{prop}
\(\mu(G)\) es un \(\C G\)-módulo.
\end{prop}

Nuestro objetivo es demostrar que si \(G\) es finito, \(\C G\) es semisimple
y \(\mu(G)\) semisimple como \(\C G\)-módulo.

\begin{df}[Producto hermítico]
  Sea \(V\) un espacio vectorial complejo de dimensión finita.
  Un producto interno hermítico es una aplicación
  \(\langle\cdot|\cdot\rangle V\times V\longrightarrow \C\) cumpliendo:
  \begin{enumerate}
    \item \(\langle v|w\rangle=\widebar{\langle w|v\rangle}\).
    \item \(\langle v'+v|w\rangle=\langle v|w\rangle+\langle v'|w\rangle\).
    \item \(\langle av|w\rangle=a\langle v|w\rangle\).
    \item \(\langle v| v\rangle\implies v=0\).
  \end{enumerate}
  Es decir, es un espacio de Hilbert complejo de dimensión finita.
\end{df}




    Sea \(V\) un \(\C G\)-módulo. Como \(\C\subseteq\C G\) por restricción de
escalares, \(\subscriptbefore{\C}{V}\) es un espacio vectorial.
\(\rho:\C G\longrightarrow\End_\C(V)\), \(\rho\) de anillos y \(\C\)-lineal.
\[
  \rho(\sum_{g\in G} r_g g)(v)=\sum_{g\in G} r_g g v
\]
¿Qué pasa si restringimos \(\rho\) a \(G\)? Como respeta el producto en \(G\),
\(\rho|_G:G\longrightarrow GL_\C(V)\), donde \(\rho|_G\) es un homomorfismo
de grupos.
Es una represantación lineal de \(G\) con espacio de representación \(V\).

Si \(W\subseteq V\), es un \(\C G\)-submódulo si y solo si es un
\(\C\)-subespacio vectorial y \(W\) es \(G\) invariante:
para todo \(w\in W\) y todo \(g\in G\) se tiene que \(gw\in W\).

\(\mu(G)\) es el espacio de representación
\(\rho:G\longrightarrow GL_\C(\mu(G))\) dado por:
\[
  \rho(g)(\varphi)(x)=\varphi(xg)=:g\varphi(x)
\]
donde \(g,x\in G\) y \(\varphi\in\mu(G)\).

\begin{teo}
  Si \(G\) es finito entonces \(\C G\) es semisimple.
\end{teo}
\begin{proof}
  Supongamos que \(G\) es finito. Tomamos \(V\) un \(\C G\)-módulo de dimensión
  finita. Tomamos \(\langle\cdot|\cdot\rangle\) un producto interno en \(V\).

  Definimos \(\langle\cdot|\cdot\rangle_G\) producto interno sobre \(V\) así:
  \[
    \langle v|u\rangle_G=\sum_{g\in G} \langle gv| gu\rangle
  \]
  que cumple la siguiente propiedad para todo \(h\in G\):
  \[
    \langle hv|hu\rangle_G=\sum_{g\in G} \langle ghv| ghu\rangle
    =\sum_{f\in G} \langle fv| fu\rangle=\langle v|u\rangle_G
  \]
  con lo que es un operador unitario (es un operador que conserva el producto
  interno de un espacio de Hilbert) y una isometría.


  Sea \(W\) un \(\C G\)-submódulo de \(V\). Se tiene:
  \[
    V=W\dot{+} W^\perp
  \]
  donde \(\perp\) se toma respecto al producto interno nuevo:
  \[
    W^\perp :=\{v\in V: \langle v|w\rangle_G=0\quad\forall w\in W
  \]

  O sea \(W^\perp\) es \(G\)-invariante. En otras palabras, hemos de ver que
  si \(v\in W^\perp\), \(g\in G\) entonces \(gv\in\perp\), entonces para todo
  \(w\in W\) tenemos que:
  \[
    \langle gv|w\rangle_G=\langle gv|gg^{-1}w\rangle_G=
    \langle v|g^{-1} w\rangle=0
  \]
  ya que \(g^{-1}w\in W\) y \(v\in W^\perp\). Luego \(W^\perp\) es
  \(G\)-invariante.

  Como hemos demostrado que cualquier submódulo es sumando directo, tenemos
  que es semisimple.

\end{proof}

\begin{cor}
  Si \(G\) es finito, \(\mu(G)\) es un \(\C G\)-módulo semisimple.
\end{cor}

Dotamos de a \(\mu(G)\) del producto interno:
\[
  \langle \varphi|\psi\rangle:=
  \frac{1}{|G|}\sum_{g\in G}\varphi(g)\overline{\psi(g)}
\]





    
Sea \(G\) un grupo, \(V\) un \(\C G\)-módulo, de dimensión finita
como espacio vectorial complejo. Fijamos una vase de \(v_i\).
Tomamos \(x\in G\):
\[
  xv_i=\sum_j t_{ij}(x)v_j
\]

A las funciones \(t_{ij}\in\mu(G)\) se les llama funciones
matriciales de \(V\) en la base \(\{v_1,\ldots,v_n\}\).

Definimos \(C(V)\) como el subespacio vectorial de \(\mu(G)\) generado por
\(\{t_{ij}:1\le i,j\le n\}\).
Veamos que \(C(V)\) no depende de la base fijada. Recordemos que todo
cambio de bases puede interpretarse como un automorfismo.

Supongamos \(V'\) otro \(\C G\)-módulo con otra base \(\{v_1',\ldots,v_m'\}\).
Sea \(f:V\longrightarrow V'\) homomorfismo de \(\C G\)-módulos. Las funciones
matriciales de \(V'\) fijaremos una base \(v_i'\) y denotamos las funciones
matriciales \(t_{ij}'\).
\[
  f(v_i)=\sum_j a_{ij}v_j'
\]
y sea \(A\) la matriz con coeficientes \(a_{ij}\).

Tenemos que \(xf(v_i)=\sum_j a_{ij}\sum_k t_{jk}'(x) v_k'\)
y \(fx(v_i)=\sum_j t_{ij}(x)\sum_k a_{ij}v_j'\). Igualando lo anterior
\(xf(v_i)=f(xv_i)\), tenemos que \(A(t_{ij}'(x))=(t_{ij}(x))A\) o
si se quiere \(A(t_{ij}')=(t_{ij})A\).

Si \(f\) es un isomorfismo, entonces \(A\) es invertible y
se tiene que los \(t_{ij}'\) y \(t_{ij}\) son combinaciones lineales los
unos de los otros, luego si \(V'=V\) y \(f=\id\) se tiene que \(C(V)\) no
depende de la base elegida. Si \(V'\cong V\), tenemos \(C(V)=C(V')\).

\begin{lema}
  \(C(V)\) es un \(\C G\)-submódulo de \(\mu(G)\).
\end{lema}
\begin{proof}
  Sean \(x,y\in G\).
  \(
    t_{ij}(xy)
  \)
  es la matriz de la aplicación lineal dada por hacer actuar
  \(xy\) sobre cualquier vector:
  \[
    t_{ij}(xy)=\sum_k t_{ik}(y)t_{kj}(x)
  \]
  es decir, el producto de matrices (por filas, es decir, con el orden
  al revés que en la composición).

  \[
    yt_{ij}(x)=t_{ij}(xy)=\sum_k t_{ik}(y)t_{kj}(x)=
    (\sum_k t_{ik}(y)t_{kj})(x)
  \]
  con lo que:
  \[
    yt_{ij}=\sum_k t_{ik}(y)t_{kj}\in C(V)
  \]

  Luego \(C(V)\) es un submódulo.

\end{proof}

\begin{lema}
  Sea \(f:V\longrightarrow\mu(G)\) un homomorfismo de \(\C G\)-módulos.
  Entonces \(\Im f\subseteq C(V)\).
\end{lema}
\begin{proof}
  Sea \(v_i\) un elemento de la base de \(V\).
  \[
    f(v_i)(x)=f(v_i)(e x)=xf(v_i)(e)=f(xv_i)(e)=\sum_j t_{ij}(x) f(v_j)(e)
    =\left(\sum_j f(v_j)(e)t_{ij}\right)(x)
  \]
  \[
    f(v_i)=\sum_j f(v_j)(e)t_{ij}\in C(V)
  \]

\end{proof}

\begin{lema}
  Sean \(G\) finito, \(U,W\) \(\C G\)-módulos (no necesariamente de dimensión
  finita) y \(f:U\longrightarrow W\) lineal. La aplicación \(\tilde{f}:
  U\longrightarrow W\) dada por:
  \[
    \tilde{f}(u)=\sum_{x\in G}x^{-1}f(xu), \quad u\in U
  \]
  es un homomorfismo de \(\C G\)-módulos.

\end{lema}
\begin{proof}
  Hemos de ver que \(\tilde{f}(yu)=y\tilde{f}(u)\) para todo \(y\in G\)
  y todo \(u\in U\).
  \[
    \tilde{f}(yu)=\sum_{x\in G} x^{-1}f(xyu)=\sum_{z\in G} yz^{-1}
    f(zu)=y\tilde{f}(u)
  \]
  donde \(z=xy\).
\end{proof}

\begin{lema}
  Sea \(G\) finito y \(V\) un \(\C G\)-módulo de dimensión finita.
  Existe un producto interno \(\langle\cdot|\cdot\rangle\) en \(G\)
  tal que \(\langle xv|xw\rangle =\langle v|w\rangle\) con \(v,w\in V\)
  y \(x\in G\).

  Es decir, que la representación \(G\longrightarrow U(V)\), donde
  \(U(V)\) es el grupo unitario.
\end{lema}



    \begin{df}[Espacio de coeficientes]
  A \(C(V)\) se le llama espacio de coeficientes.
\end{df}


\begin{lema}[de Schur]
  Sea \(\Sigma\) un \(\C G\)-módulo simple. Entonces:
  \[
    \End_{\C G}(\Sigma)=\{\lambda \id_\Sigma :\lambda \in\C\}\cong\C
  \]
\end{lema}
\begin{proof}
  Tenemos que \(\dim_\C \Sigma<\infty\). Tomamos
  \(\phi:\Sigma\longrightarrow\Sigma\) homomorfismo de \(\C G\)-módulos.
  Sea \(\phi\) es \(\C\) lineal. Tomamos \(\lambda \in\C\) valor
  propio de \(\varphi\) y sea \(V_\lambda\) el subespacio propio asociado.

  Sea \(g\in G, v\in V_\lambda\), \(\phi(gv)=g\phi(v)=g\lambda v=\lambda g v\)
  con \(gv\in V_\lambda\) entonces \(V_\lambda\) es un \(\C G\)-submódulo de
  \(\Sigma\). Si \(\phi\neq 0\) entonces \(V_\lambda\neq\{0\}\).

  Como \(\Sigma\) es simple, \(V_\lambda = \Sigma\).
\end{proof}

\begin{df}[Matriz unitaria]
  Su inversa coincide con su conjugada transpuesta
\end{df}

\begin{teo}[de Peter-Weyl]
  Dotemos a \(\mu(G)\) con el producto interno:
  \[
    \langle\varphi|\psi\rangle=\frac{1}{|G|}\sum_{x\in G}\varphi(x)
    \overline{\psi(x)}
  \]

  Tomamos \(\Omega_{\C G}=\{\Sigma_1,\ldots, \Sigma_t\}\).

  Entonces \(\mu(G)=C(\Sigma_1)\dot{+}\cdots\dot{+}C(\Sigma_t)\),
  suma directa ortogonal de \(\C G\)-módulos.

  Además, tomando en cada \(\Sigma_i\) un producto interno tal que
  la representación asociada a \(\Sigma_i\) sea unitaria, entonces
  \(\{t_{jk}^{\Sigma_i}:i\in\{1,\ldots,s\}, j,k\in\{1,\ldots, d_i\}\}\)
  es una base ortonormal de \(\mu(G)\) siempre que
  \(d_i=\dim_\C\Sigma_i\) y \(\{t_{jk}^{\Sigma_i}\) son las funciones
  matriciales asociados a \(\Sigma_i\) con respecto de una base otronormal
  de \(\Sigma_i\).
\end{teo}
\begin{proof}
  \(\mu(G)=\Soc_{\Sigma_1}(\mu(G))\dot{+}\cdots\dot{+}\Soc_{\Sigma_t}(\mu(G))\)
  y \(\Soc_{\Sigma_i}(\mu(G))\subseteq C(\Sigma_i)\). Es suma de módulos
  isomorfos a \(\Sigma_i\) cada uno en \(C(\Sigma_i)\).
  Tenemos que:
  \[
    \mu(G)=C(\Sigma_i)+\cdots+C(\Sigma_t)
  \]

  Tomo \(V\) con base \(\{v_1,\ldots, v_n\}\) y \(W\) con base
  \(\{w_1,\ldots,w_m\}\) \(\C G\)-módulos simples. Para cada \(i,j\)
  definimos \(p_{ij}:V\longrightarrow W\) lineal dada por
  \(p_{ij}(v_k)=w_j\delta_{ki}\). Entonces tomamos
  \(\tilde{p}_{ij}\) la extensión dada por
  \[
    \tilde{p}_{ij}(v)=\sum_{x\in G}x^{-1} p_{ij}(xv)
  \]
  con \(v\in V\).
  \[
    \tilde{p}_{ij}(v_k)=\sum_{x\in G}x^{-1} p_{ij}(xv_k)=
  \sum_{x\in G}x^{-1} p_{ij}\left(\sum_l t^V_{kl}(x) v_l\right)=
\]\[
  \sum_{x\in G}x^{-1} \sum_l t^V_{kl}(x) p_{ij}(v_l)=
  \sum_{x\in G}x^{-1} t^V_{ki}(x) w_j=
  \sum_l\sum_{x\in G} t^V_{ki}(x) t_{jl}^W(x^{-1}) w_l
  \]

  En concreto si las bases \(v_i\) y \(w_j\) son ortonormales, entonces
  los coeficientes de \(t_{ki}^V\) y \(t_{jl}^W\) son de matrices
  unitarias. En ese caso la expresión anterior queda:
  \[
    \sum_l\sum_{x\in G} t^V_{ki}(x) \overline{t_{lj}^W(x)} w_l
  \]

  Si \(V\not\cong W\) entonces \(\tilde{p}_{ij}=0\) y por tanto
  \[
    \sum_l\sum_{x\in G} t^V_{ki}(x) \overline{t_{lj}^W(x)} w_l=0
  \]

  Sean \(a\neq b\) y tomamos \(V=\Sigma_a, W=\Sigma_b\), entonces
  \[
    0=\sum_{x\in G} t^V_{ki}(x) \overline{t_{lj}^W(x)}
  \]
  con lo que \(C(\Sigma_a)\perp C(\Sigma_b)\).

  Luego \(\mu(G)=C(\Sigma_1)\dot{+}\cdots\dot{+}C(\Sigma_t)\).

  Supongamos ahora \(V=W=\Sigma_a\). En ese caso, por el lema de Schur
  \(\tilde{p_{ij}}(v)=\alpha_{ij} v\). Entonces (\(v_i=w_i\)):
  \[
    \alpha_{ij} v_k = \tilde{p}_{ij}(v_k)
    \sum_l\sum_{x\in G} t^V_{ki}(x) \overline{t_{lj}^V(x)} v_l
  \]

  Si \(k\neq l\), entonces \(\alpha_{ij} v_k=0\) y por tanto:
  \[
    \sum_l\sum_{x\in G} t^V_{ki}(x) \overline{t_{lj}^V(x)} v_l=0
  \]

  Si \(k=l\) e \(i\neq j\), entonces:
  \[
    0=\sum_{x\in G}
    t_{ik}^{\Sigma_a}(x^{-1})\overline{t_{jk}^{\Sigma_a}(x^{-1})}
    =\sum_{x\in G} t_{ki}^{\Sigma_a}(x)\overline{t_{kj}^{\Sigma_a}(x)}
  \]
  luego \(\{t_{ij}^{\Sigma_a}\}\) es un sistema ortogonal generador,
  en particular es una base ortogonal. Veamos que no es ortonormal.

  \[
    \sum_{x\in G} t_{ki}^{\Sigma_a}(x)\overline{t_{ki}^{\Sigma_a}(x)}=
    |G|
  \]
  Luego son una base ortonormal.

\end{proof}

    \begin{cor}
  \[
    |G| = d_1^2+\cdots+d_t^2
  \]
\end{cor}
\begin{proof}
  \(\mu(G)=C(\Sigma_1)\dot{+}\cdots\dot{+}C(\Sigma_t)\) con lo que:
  \[|G|=\sum_{i=1}^t \dim_\C C(\Sigma_i)=\sum_{i=1}^t d_i^2\]
\end{proof}

\begin{prop}
  Sea \(G\) abeliano finito, \(\Sigma\) un \(\C G\)-módulo simple.
  Entonces \(\dim_\C \Sigma = 1\).
\end{prop}
\begin{proof}
  \(\Sigma\) tiene dimensión compleja finita, \(x\in G\),
  \(f_x:\Sigma\longrightarrow\Sigma\), \(f_x(v)=xv\)
  con \(y\in G\):
  \[
    f_x(yv)=xyv=yxv=yf_x(v)
  \]
  entonces \(f_x\in\End_{\C G}(\Sigma)=
  \{\lambda\id_\Sigma:\lambda\in\C\}\). Así que \(f_x=\lambda_x\id_\Sigma\)
  para cierto \(\lambda_x\in\C\).

  Sea \(v\in\Sigma\setminus\{0\}\), \(w\in\Sigma\),
  \(w=(\sum_{x\in G} \alpha_x x)v\) porque todo módulo simple está generado
  por cualquiera de sus elementos.
  Pero entonces: \[
    w=\sum_{x\in G} \alpha_x f_x(v)=\sum_{x\in G} \alpha_x\lambda_x v
  \]
  luego \(\dim_\C \Sigma = 1\).

\end{proof}

\begin{cor}
  Si \(G\) es abeliano, \(n=|G|\), entonces \(|\Sigma_{\C G}|=n\).
\end{cor}
\begin{proof}
  \(\Sigma_{\C G} =\{\Sigma_1,\ldots,\Sigma_t\}\), por el teorema
  de Webber-Artin,
  \[
    \C G \cong \End_{\C G}(\Sigma_1)\times\cdots\times
    \End_{\C G}(\Sigma_t)
    \cong \C\times\overset{(t)}{\cdots}\times\C
  \]
  con lo que \(n =t\).

\end{proof}

Ejemplo: \(G=\Z_n=\{0,1,\ldots, n-1\}\).
Tenemos que ver que \(\Omega_{\C G}=\{\Sigma_0,\ldots,\Sigma_{n-1}\}\), con
\(\dim_\C \Sigma_j = 1\) para todo \(j\in\{1,\ldots,n-1\}\).

Sea \(u_j\) una base de \(\Sigma_j\) (\(\Sigma_j =\C u_j\)).
Sea \(\omega =e^{2\pi i/n}\in\C\), ponemos \(k u_j :=\omega^{kj} u_j\) para
\(k\in\Z_n\).

Es claro que \((k+k')u_j=\omega^{kj+k'j}u_j=(k\circ k')u_j\) y el 0 va a la
identidad. \(\Sigma_j\) es un \(\C \Z_n\)-módulos simples (tiene
dimensión 1). Basta ver que ningún par de ellos son isomorfos entre sí.

Supongamos \(f:\Sigma_j\longrightarrow \Sigma_k\) \(\C G\)-lineal y no
nulo. Veamos que \(k=j\). \(\exists\alpha\in\C\setminus\{0\}\)
tal que \(f(u_j)=\alpha u_k\).
\[
  \omega^j \alpha u_k = \omega^j f(u_j)=f(\omega^j u_j)=f(1u_j)=1\alpha u_k
  =\alpha\omega_k u_k
\]
Luego \(\omega^j=\omega^k\) y por tanto \(j=k\).

Cada \(C(\Sigma_j)\) tiene como base \(\{t^{\Sigma_j}\}\), donde
\(t^{\Sigma_j}(k)=k u_j=\omega^{kj}\). Son una base ortonormal
de \(\mu(\Z_n)\) respecto del producto interno:
\[
  \langle\varphi|\psi\rangle=\frac{1}{n}\sum_{k\in\Z_n}\varphi(k)
  \overline{\psi(k)}
\]

Si \(\varphi\in\mu(\Z_n)\), \(\varphi=\sum_{j=0}^{n-1}\langle
\varphi|t^{\Sigma_j}\rangle t^{\Sigma_j}\). Es decir, obtenemos
la transformada de Fourier Discreta.






    Además \(t^{\Sigma_j}t^{\Sigma_k}=t^{\Sigma_j+\Sigma_k}\).

Si llamamos \(\hat{\varphi}(j)=\langle\varphi|t^{\Sigma_j}\rangle
=\frac{1}{n}\sum_{k=0}^{n-1} \varphi(k)\omega^{-kj}\) tenemos que
\[
  \varphi\psi=\sum_{j,k}\hat{\varphi}(j)\hat{\psi}(k)t^{\Sigma_j}t^{\Sigma_k}
  =\sum_{l=0}^{n-1}\left(\sum_{j=0}^{n-1}\hat{\varphi}(j)\hat{\psi}(e-j)\right)
  t^{\Sigma_e}
\]

Ejercicio: Para \(G=\Z_n\times\Z_m\), calcular \(\Omega_{\C G}\) y deducir la
correspondiente base ortonormal de \(\mu(\Z_n\times\Z_m)\).

Ejemplo: \(D_n=\langle r,s | r^n = s^2 = 1, sr=r^{-1}s\rangle\).
\(D_n=\{s^a r^j:a \in\{0,1\}, j\in\{0,\ldots,n-1\}\}\).
Veamos qué hacer con \(\mu(D_n)\). Como el grupo no es conmutativo debe haber
al menos una representación de un subgrupo simple de dimensión mayor que 1.

\(\alpha\in\C\), \(\alpha^n=1\), \(V_\alpha\) un \(\C\)-espacio vectorial
hermítico con base ortonormal \(\{v_1, v_2\}\).
Tenemos que buscar una representación: \(D_n\longrightarrow U(V_\alpha)\)
donde
\[
  r\mapsto
  \begin{pmatrix}
    \alpha&0\\
    0&\overline{\alpha}\\
  \end{pmatrix}
\quad\quad\quad
  s\mapsto
  \begin{pmatrix}
    0&1\\
    1&0\\
  \end{pmatrix}
\]
Comprobamos que (recordando que el producto por matrices se hace en el orden
inverso porque trabajamos por filas):
\[
  \begin{pmatrix}
    \alpha&0\\
    0&\overline{\alpha}\\
  \end{pmatrix}
  \begin{pmatrix}
    0&1\\
    1&0\\
  \end{pmatrix}
  =
  \begin{pmatrix}
    0&\alpha\\
    \overline{\alpha}&0\\
  \end{pmatrix}
  =
  \begin{pmatrix}
    \overline{\alpha}&0\\
    0&\alpha\\
  \end{pmatrix}
  \begin{pmatrix}
    0&1\\
    1&0\\
  \end{pmatrix}
\]

¿Cuando \(V_\alpha\) irreducible? (o sea, simple). No será simple si y solo
si existe \(v\in V_\alpha\) tal que \(\C v\) es un submódulo. Esto equivale
a que \(rv, sv\in \C v\).

Trabajando con coordenadas \(v=(x,y)\in\C^2\).
\[
  (x\quad y)
  \begin{pmatrix}
    \alpha&0\\
    0&\overline{\alpha}\\
  \end{pmatrix}
  =\beta(x\quad y)
\]
\[
  (x\quad y)
  \begin{pmatrix}
    0&1\\
    1&0\\
  \end{pmatrix}
  =\gamma(x\quad y)
\]
y resolviendo las ecuaciones obtenemos que \(\alpha^2=1\).

Luego si \(\alpha\neq\pm 1\) tenemos una representación irreducible.

Supongamos que bajo las mismas condiciones \(V_\alpha\cong V_{\alpha'}\).
Tomando trazas, tenemos que \(\alpha+\overline{\alpha}=
\alpha'+\overline{\alpha'}\). Es decir, si \(\alpha+\overline{\alpha}\neq
\alpha'+\overline{\alpha'}\) y entonces \(V_\alpha\not\cong V_{\alpha'}\).

Funciones matriciales de \(V_\alpha\):
\[
  \begin{pmatrix}
    t_{11}  & t_{12}\\
    t_{21}  & t_{22}\\
  \end{pmatrix}
\]
tenemos que
\[
  s^a r^j\mapsto
  \begin{pmatrix}
    \alpha  & 0\\
    0       & \overline{\alpha}\\
  \end{pmatrix}^j
  \begin{pmatrix}
    0 & 1\\
    1 & 0\\
  \end{pmatrix}^a
\]
si \(a=0\) tenemos:
\[
  \begin{pmatrix}
    \alpha^j  & 0\\
    0       & \overline{\alpha}^j\\
  \end{pmatrix}
\]
si \(a=1\) tenemos:
\[
  \begin{pmatrix}
    0&\alpha^j\\
    \overline{\alpha}^j& 0\\
  \end{pmatrix}
\]

Tenemos entonces que \(t_{11}(s^a r^j)=\chi_{0}(a)\alpha^j\),
\(t_{12}(s^a r^j)=\chi_{1}(a)\alpha^j\),
\(t_{21}(s^a r^j)=\chi_{1}(a)\overline{\alpha}^j\),
\(t_{22}(s^a r^j)=\chi_{1}(a)\overline{\alpha}^j\)


    Tomamos \(\omega =e^{2\pi i/n}\in\C\).
\({(\omega^2)}^j=\omega^{2j}=1\) si y solo si \(2j\frac{2\pi}{n}\)
es un múltiplo entero de \(2\pi\), es decir, que \(2j\) es múltiplo
entero de \(n\).

Luego si \(2j\) no es múltiplo entero de \(n\), \(V_{\omega^j}\) es
simple.

Caso A:\@ si \(n\) es impar, entonces \(n=2\nu+1\). Para
\(j\in\{1,\ldots,\nu\} \) se tiene que \(V_{\omega^j}\) es simple.
Si \(j'\in\{1,\ldots,\nu\} \), tenemos que:
\[
  \omega^j+\omega^{-j}
  =
  \omega^{j'}+\omega^{-j'}
\]
que significa que \(\cos\frac{2\pi j}{n}=\cos\frac{2\pi j'}{n}\), y como
entre 0 y \(\pi\) el coseno es biyectivo, \(j=j'\).

Luego \(V_{\omega^j}\) son simples y todos no isomorfos entre sí.
\[
  \Sigma_1,\ldots,\Sigma_\nu\in\Omega_{\C D_n}
\]

Como \(2n=\md{G}=d_1^2+\cdots+d_t^2\), luego \(2n-4\nu\) es el número
de elementos que nos quedan. \(2n-4\nu=4\nu+2-4\nu=2\), solo nos quedan por
considerar dos módulos de dimensión 1.

Tenemos que considerar el módulo trivial \(\Sigma_0\),
\(\C D_n\)-módulo cuya representación es la trivial, es decir,
cada \(s^a r^k\mapsto 1\in\C^\times\).

Por otro lado tenemos que \(\Sigma_{-1}\) definido por
\(s^a r^k\mapsto \sgn(1-2a)\in\C^\times\).

\[
  \Omega_{\C D_n}=
  \{\Sigma_{-1},\Sigma_0,\Sigma_1,\ldots,\Sigma_\nu\}
\]
y la siguiente base es ortonormal:
\[
  \{
    t^{\Sigma_{-1}},t^{\Sigma_0}\}\cup\{\sqrt{2}t^{\Sigma_j}_{bc}:
  j\in\{1,\ldots,\nu\}, b,c\in\{1,2\}\}
  \}
\]

Donde:
\[
  t^{\Sigma_0}(s^a r^k)=1
\]
\[
  t^{\Sigma_{-1}}(s^a r^k)=\sgn(1-2a)
\]
\[
  t^{\Sigma_j}_{11}
  = \chi_0(a)e^{2\pi i kj/n}
\]
\[
  t^{\Sigma_j}_{12}
  = \chi_1(a)e^{2\pi i kj/n}
\]
\[
  t^{\Sigma_j}_{21}
  = \chi_0(a)e^{-2\pi i kj/n}
\]
\[
  t^{\Sigma_j}_{22}
  = \chi_1(a)e^{-2\pi i kj/n}
\]


    \subsection{El caso de la circunferencia unidad}

¿Y si \(G\) no es finito? En general es intratable. Lo más fácil es estudiar
grupos compactos, habitualmente \(p\)-ádicos o grupos de Lie.

Veamos un ejemplo de un grupo de Lie compacto:
\[
  \Sphere^1=\{z\in\C:|z|=1\}
\]

Tenemos \(\C \Sphere^1\). Tomamos \(\C \Sphere^1\)-módulos de dimensión
compleja finita que provengan de representaciones continuas de \(\Sphere^1\).
Estas son los homomorfismos continuos de grupos
\(\rho:\Sphere^1\longrightarrow GL(V)\) con dimensión finita.

Dada \(\rho\) quiero definir un producto directo \(\langle\cdot|\cdot
\rangle_\Sphere\) en \(V\) tal que:
\[
  \langle\rho(z)(v)|\rho(z)(w)\rangle_\Sphere=\langle v|w\rangle_\Sphere
\]
para todo \(v,w\in V\). En notación de módulos:
\[
  \rho(z)(v)=zv
\]

En efecto, tómese un producto interno cualquiera \(\langle\cdot|\cdot\rangle\)
y definimos:
\[
  \langle v|w\rangle_{\Sphere^1}:=\int_{\Sphere^1}\langle
  zv|zw\rangle dz
\]
que es integrable por ser continua sobre un compacto. Se puede ver sin mucha
dificultad que es un producto interno.

Veamos que es unitario.
Sea \(z'\), tenemos:

\[
  \langle z'v|z'w\rangle_{\Sphere^1}=\int_{\Sphere^1}\langle
  zz'v|zz'w\rangle dz
  =
  \int_{\Sphere^1}\langle uv|uw\rangle du=
  \langle v|w\rangle_{\Sphere^1}
\]
con un cambio de variable adecuado (es una isometría), la medida
de \(\Sphere^1\) es invariante por la acción de \(\Sphere^1\).

Dicha acción es unitaria.


    Sea ahora \(U\) un subespacio invariante, es decir,
\(\C \Sphere^1\)-submódulo.
\[
  U^\perp=\{v\in V:\langle v|u\rangle =0\quad\forall u\in U\}
\]

En efecto, si \(v\in U^\perp\) y \(z\in\Sphere\), he de ver que
\(zv\in U^\perp\). Tomo \(u\in U\) con
\(\langle zv|u\rangle_\Sphere=\langle zv|zz^{-1}u\rangle=
\langle v|z^{-1}u\rangle=0\) y que \(\rho(z^{-1})(u)\in U\).

Tenemos entonces que \(V=U\dot{+}U^\perp\) y haciendo inducción sobre la
dimensión compleja de \(V\), tenemos que \(V\) es semisimple como
\(\C\Sphere^1\)-submódulo.

Busquemos las funciones matriciales: tomamos \(\{v_1,\ldots,v_n\}\) base
de \(\Sphere^1\).
\[
  z v_i=\sum_{j} t_{ij}(z)v_j
\]
con \(t_{ij}(z)\in\C\). Pero como la representación \(\rho\)
es continua, son continuas. Es decir, \(t_{ij}\in\mu(\Sphere^1)
\cap\mathcal{C}(\Sphere^1)\).

Como consecuencia \(C(V)\subset\mathcal{C}(V)\).

\begin{df}[Función representativa]
  Sea \(G\) un grupo. Llamaremos \(\varphi\in\mu(G)\) representativa si
  el submódulo cíclico generado por \(\varphi\), es decir,
  \(\C G\varphi\), es de dimensión finita como \(\C\)-espacio vectorial.
  Denotaremos por \(\mathcal{R}(G)\) al conjunto de las funciones
  representativas.
\end{df}

Ejercicio: \(\varphi\) es representativa si y solo si \(\varphi\in C(V)\)
para \(V\) cualquier \(\C G\)-módulo de dimensión finita.

\begin{prop}
  \(\mathcal{R}(G)\) es un \(\C G\)-submódulo de \(\mu(G)\).
\end{prop}

\begin{df}
  \(\mathcal{R}_c(\Sphere):=\mathcal{R}(\Sphere)\cap\mathcal{C}(\Sphere)\)
\end{df}
\begin{prop}
  \(\varphi\in\mathcal{R}_c(\Sphere)\) si y solo si \(\varphi\in C(V)\)
  para \(\rho:S\longrightarrow GL(V)\) continua \(\mathcal{R}_c(\Sphere)\)
  es un \(\C\Sphere\)-submódulo de \(\mathcal{R}(\Sphere)\).
\end{prop}

Tenemos que \(\Omega^c_{\C\Sphere^1}\) son homomorfismo continuos de grupos de
dimensión uno, entre \(\Sphere\) y \(\C^\times\).

Vamos a parametrizarlo, \(\theta\mapsto e^{i\theta}\) tenemos que
todos los homomorfismos peridicos de \(\R\) en \(\C^\times\)
son de la forma \(\chi_k(e^{i\theta})=e^{ik\theta}\) para cada \(k\in\Z\).
Es decir, \(\chi_k(z)=z^k\) para todo \(z\in\Sphere^1\).

Entonces
\[
  \Omega^c_{\C\Sphere^1}=\{\chi_k:k\in\Z\}
\]
son todas las representaciones irreducibles continuas.

Para cada una de ellas, tenemos \(C(\chi_k)=\C\chi_k\) o parametrizando
\(C(\chi)=\C e^{i\theta}\).

Tenemos que \(\C\chi_k\cong\C\chi_{k'}\) si y solo si \(k=k'\).

\[
  \dot{+}_{k\in\Z}\C\chi_k\subseteq\mathcal{R}_c(\Sphere^1)
  \subseteq\mathcal{C}(\Sphere^1)
\]
donde se usa un producto interno
\[
  \langle\varphi|\psi\rangle=\int_\Sphere\varphi\overline{\psi}
\]
con lo que los \(\chi_k\) son base ortogonal de \(\dot{+}\C \chi_k\).

El análisis funcional, obtenemos la suma directa de espacios de Hilbert,
con lo que \(\dot{+}\C \chi_k=\mathcal{R}_c(\Sphere)\cong
\mathcal{L}^2(\Z)\), mientras que \(\mathcal{C}(\Sphere)\)
se obtiene al completar \(L^2(\Sphere)\). Finalmente se obtiene un isomorfismo
entre ambos.




\end{document}
