\begin{df}[Matriz quasidiagonal]
  Sea \(A=(a_{ij})\in\mathcal{M}_{s\times t}(R)\). Diremos que
  \(A\) es quasidiagonal si \(a_{ij}=0\) para todo \(i\neq j\).
  Usaremos \(d_i=a_{ii}\) para \(i=1,\ldots,m\) con
  \(m=\min\{s,t\}\). La notación
  \[
    A=\diag_{s\times t}(d_1,\ldots, d_m)
  \]
\end{df}

Ejemplos:
\[
  \diag_{3\times 2}(1,3)=
  \begin{pmatrix}
    1&0\\
    0&3\\
    0&0
  \end{pmatrix}
\]
\[
  \diag_{2\times 3}(1,3)=
  \begin{pmatrix}
    1&0&0\\
    0&3&0\\
  \end{pmatrix}
\]

\begin{nt}
  Denotaremos \(GL_n(R)\) al grupo de unidades de \(\mathcal{M}_n(R)\):
  es decir, las matrices \(Q\) tales que existe
  otra matriz \(Q^{-1}\) que cumpla \(QQ^{-1}=Q^{-1}Q=I_n\).
\end{nt}

\begin{prop}
  Sea la presentación finita:
  \[
    E_s\overset{\psi}{\longrightarrow} F_t
    \overset{\phi}{\longrightarrow} M\longrightarrow 0
  \]
  de \(\subscriptbefore{R}{M}\). Supongamos que existen \(P\in GL_s(R)\),
  \(Q\in GL_t(R)\) y \\\(D=\diag_{s\times t}(d_1,\ldots, d_m)\)
  tales que \(PA_\psi = DQ\). Si \(\{m_1,\ldots, m_t\}\) es el conjunto
  de generadores de \(M\) y tales que \(m_i=\phi(f_i)\) con
  \[
    x_i=\sum_{j=1}^t q_{ij}m_j
  \]
  entonces \(M=\dot{+}_{i=1}^t Rx_i\) y \(\ann_R(x_i)=Rd_i\) si
  \(i\le m\) y \(\ann_R(x_i)=\{0\}\) si \(i>m\) si se da el caso.
\end{prop}

\begin{proof}
  Tomemos otra presentación:
  \[
    E_s\overset{\psi_1}{\longrightarrow} F_t
    \overset{\phi_1}{\longrightarrow} M\longrightarrow 0
  \]

  Tomemos \(\id:M\longrightarrow M\) y dos homomorfismos
  \(p:E_s\longrightarrow E_s\) y
  \(q:F_t\longrightarrow F_t\), tales que \(A_p=P\),
  \(A_q=Q\) y \(A_\psi =D\) y que conmuten todas las aplicaciones.

  Para que conmuten, definimos \(\phi_1=\phi\circ q\),
  con lo que \(\phi_1(f_i)=\phi(q(f_i))=\sum_{j=1}^t q_{ij} m_j=x_i\).

  La condición de matrices \(PA_\psi=DQ\) garantiza que
  \(\psi\circ p=q\circ A_\phi\).

  Hay que comprobar que la sucesión que nos hemos inventado
  es exacta en \(F_t\). Para demostrarlo, usamos que \(P\) y \(Q\) son
  inversibles: \(p,q\) son isomorfismos y podemos recuperar la exactitud
  de la sucesión del enunciado.

  \[
    M=Rx_1+\cdots+Rx_t
  \]
  porque \(x_i=\phi_1(f_i)\) y \(\phi_1\) es sobreyectiva. Para ver que es
  directa, tomamos el \(0=r_1 x_1+\cdots+ r_t x_t\). Hay que ver
  que cada \(r_i x_i=0\).
  \[
    0=\phi_1(r_1 f_1+\cdots+r_t f_t)\implies
    r_1 f_1+\cdots+r_t f_t\in\ker\phi_1=\Im\psi_1
  \]

  Por otro lado, \(\Im\psi_1=R\psi_1(e_1)+\cdots+ R\psi_1(e_s)\).
  Ahora bien, \(A_{\psi_1}=D\), con lo que
  \(\Im\psi_1=Rd_1 f_1+\cdots +Rd_m f_m\). Tenemos que esos módulos
  son independientes y la suma es directa:
  \(\Im\psi_1=Rd_1 f_1\dot{+}\cdots\dot{+} Rd_m f_m\).
  Entonces \(r_i\in Rd_i\) para \(i\le m\),
  y si \(t>m\), entonces \(r_i=0\) para \(i>m\).

  Así, cada \(r_i f_i = s_i d_i f_i\).
  Tomamos \(r_1 x_i\phi_1(r_i f_i) \), tenemos que
  \(r_i f_i\in\Im\psi_1=\ker\phi_1\), luego \(r_i x_i=0\).
  Luego:
  \[
    M=Rx_1\dot{+}\cdots\dot{+}Rx_t
  \]

  Se deduce también que \(\ann_R(x_i)\supseteq Rd_i\).

  \begin{eqnarray*}
    M&\cong& F_t/\Im\psi_1=
    (Rf_1\dot{+}\cdots\dot{+}Rf_t)/
    (Rd_1f_1\dot{+}\cdots\dot{+}Rd_m f_m)\\
    &\cong&
    Rf_1/Rd_1 f_1\oplus\cdots\oplus Rf_m/Rd_m f_m
    \oplus R/\{0\}\oplus\overset{(t-m)}{\cdots}\oplus R/\{0\}\\
    &\cong&
    Rf_1/Rd_1\oplus\cdots\oplus R/Rd_m
    \oplus R\oplus\overset{(t-m)}{\cdots}\oplus R
  \end{eqnarray*}

\end{proof}

Caso particular: \(R=\Z\). Aquí siempre podemos calcular P y Q.
Si \(M\) es un grupo abeliano finitamente generado como \(\Z\)-módulo,
entonces existen \(d_1,\ldots, d_m\in\N\) tales que
\[
  M\cong \Z_{d_1}\oplus\cdots\oplus\Z_{d_m}\oplus\Z^{t-m}
\]
si \(t>m\) y en otro caso:
\[
  M\cong \Z_{d_1}\oplus\cdots\oplus\Z_{d_m}
\]
Es la suma de una parte de torsión y una libre de torsión.

