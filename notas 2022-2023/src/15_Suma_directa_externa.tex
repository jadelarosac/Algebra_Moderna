\subsection{Suma directa externa}

\begin{prop}
  Sean \(R\) un anillo y \(M_1, \ldots, M_t\), con \(t \in \N\), \(R\)-módulos y se considera
  el producto cartesiano de los módulos anteriores y las operaciones:
  \begin{itemize}
  \item \((m_1, \ldots, m_t) + (m_1', \ldots, m_t') = (m_1 + m_1', \ldots, m_t + m_t')\)
    para \(m_i, m_i' \in M_i\) y para cada \(i \in \{1, \ldots, t\}\). 
  \item \(r(m_1, \ldots, m_t) = (rm_1, \ldots, rm_t\) para \(m_i \in M_i\), \(r \in R\) e
    \(i \in \{1, \ldots, t\}\).
  \end{itemize}

  Entonces \(M_1 \times \cdots \times M_t\) es un \(R\)-módulo, denotado por \(M_1
  \oplus \cdots \oplus M_t\) y llamado \textbf{suma directa externa de \(M_1,\ldots, M_t\)},
  con \(M^t\) si son todos iguales.
\end{prop}

\begin{df}[Base de un módulo libre]
  Consideramos \(R^t=R\oplus\cdots\oplus R\). Para cada \(i=1,\ldots, t\), tenemos que
  \(\{e_i: e_i=(0,\ldots, 0,1,0, \ldots, 0)\}\) forman un sistema de
  generadores de \(R^t\). Por tanto \(a=\sum_i a_i e_i\in R^t\)
  es una expresión única y a \(\{e_1, \cdots, e_t\}\) se le llama \textbf{base canónica
    de \(R^t\)}.
\end{df}

Dicha base puede no existir.

\begin{prop}
  Dado un módulo cualquiera \(\subscriptbefore{R}{M}\) y \(m_1, \ldots, m_t\in M\),
  existe un único homomorfismo de \(R\)-módulos \(f:R^t \longrightarrow M\)
  tal que \(f(e_i)=m_i\) para cada \(i = 1, \ldots, t\).
\end{prop}

\begin{proof}
  Unicidad: si existe una tal aplicación \(f\), entonces para
  cualquier \(x = \sum_{i=1}^tx_ie_i \in R^t\), con \(x_i \in R\) para cada \(i = 1, \ldots,
  t\),
  \[
    f(\sum_{i=1}^{t}x_ie_i) = \sum_{i=1}^t x_i f(e_i)=\sum_{i=1}^t x_i m_i
  \]

  Veamos la existencia,
  Definiendo \(f(x)=\sum_i x_i m_i\), obtenemos un homomorfismo de módulos
  que cumple lo exigido en el enunciado.
\end{proof}

\begin{cor}
  Si \(M\) es finitamente generado con generadores \(\{m_1, \ldots, m_t\}\),
  entonces \(M\cong R^t/L\) para \(L\) un cierto submódulo de \(M\).
\end{cor}

\begin{proof}
  Si \(M=Am_1+\cdots+Am_n\), \(f\) es sobreyectivo y tenemos que \(L=\ker f\) cumple lo
  que se pide por el teorema de isomorfía para módulos.
\end{proof}

¿Qué relación hay ebtre la suma directa externa e interna?
\begin{ejercicio}\label{ejer:rel_sumas_finito}
  Sea \(\subscriptbefore{R}{M}\), \(N_1,\ldots,
  N_t\in\mathcal{L}(\subscriptbefore{R}{M})\). Se pide demostrar que existe
  un homomorfismo \(f:N_1\oplus\cdots\oplus N_t\longrightarrow
  N_1{+}\cdots{+}N_t\)
  sobreyectivo de \(A\)-módulos tal que entre la suma directa
  externa y la suma interna, tal que \(f\) es un isomorfismo si y solo si
  la suma interna es directa. Podría ser interesante usar coordenadas.
\end{ejercicio}