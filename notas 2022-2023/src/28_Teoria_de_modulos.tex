\section{Teoría de módulos}

Sea \(R\) un anillo, \(\subscriptbefore{R}{M}\) un módulo y la
familia no vacía de submódulos
\(\Gamma\subseteq\mathcal{L}(M)\). Recordemos que entonces
\(\bigcap_{N\in\Gamma} N\in\mathcal{L}(M)\).

\begin{df}[Submódulo generado por un conjunto \(X\)]
  Si \(X\) es un subconjunto de \(M\), el menor submódulo de \(M\) que contiene
  a \(X\) se llama \textbf{submódulo generado por \(X\)}. Se denotará por \(RX\).
\end{df}

\begin{lema}
  Sea \(X \subset M\) cualquiera. Entonces
  \[
    RX=\left\{\sum_{x\in F}v_x x:F\subseteq X \textrm{ finito}, v_x\in
    R\right\}
  \]
\end{lema}
\begin{proof}
  \(X\subseteq RX\) por ser el menor submódulo que contiene a \(X\).
  \[
    C=\left\{\sum_{x\in F}v_x x:F\subseteq X \textrm{ finito}, v_x\in
    R\right\}
  \]

  Entonces \(C\subseteq RX\).
  Tenemos que, como \(C\) es un submódulo, se tiene que dar la igualdad.

\end{proof}

Si \(X=\{x_1,\ldots, x_n\}\), tenemos que \(RX=Rx_1+\cdots Rx_n\).

\begin{obs}
  El módulo \(M\) está finitamente generado si existe \(X \subset M\) finito
  tal que \(M = RX\).
\end{obs}

\begin{df}[Módulo producto]
  Sean \(M_i\) \(R\)-módulos, donde \(I\) un conjunto de índices no vacío.
  Se define el \textbf{módulo producto} como
  \[
    \prod_{i\in I} M_i=\{{(m_i)}_{i\in I}:m_i\in M_i.\}
  \]
  Si \(M_i = M\) para cada \(i \in I\), entonces al módulo producto se le
  denota por \(M^I\).
\end{df}

\begin{prop}
  El módulo producto es un módulo, con la suma término a término y el
  producto por escalares también término a término.
\end{prop}

\begin{df}[Proyecciones e inclusiones canónicas]
  Sea \(\prod_{i\in I} M_i\) un módulo producto. Definimos
  la inclusión canónica \(\iota_i\)  mediante la
  aplicación que asigna \(m_i\mapsto {(a_j)}_{j\in I}\) dado por
  \(a_j=\delta_i^j m_i\), donde \(\delta_i^j = 1\) si \(i = j\) y
  \(\delta_i^j = 0\) si \(i \neq j\) para cada \(i \in I\).
  Del mismo modo, definimos la proyección canónica \(\pi_i\) como
  la aplicación que asigna \({(a_j)}_{j\in I}\mapsto a_i\) para cada
  \(i \in I\).

  Evidentemente, \(\pi_i\circ\iota_i=\id\).
\end{df}

\begin{df}[Soporte]
  Sea \(I\) un conjunto de índices y \(M_i\) un \(R\)-módulo para cada
  \(i \in I\). El \textbf{soporte de} \((m_i)_{i \in I} \in \prod_{i \in I}M_i\)
  es el conjunto de índices
  en los que la componente de \((m_i)_{i \in I}\) es no nula, es decir,
  \[
    sop(m) = \{i \in I \ : \ m_i \neq 0\} \subset I
  \]
  para cada \(m = (m_i)_{i \in I}\).
\end{df}

\begin{lema}[Suma directa externa]
  Sea \(I\) un conjunto de índices y \(M_i\) un \(R\)-módulo para cada
  \(i \in I\). Entonces el conjunto
  \[
    \bigoplus_{i\in I} M_i:=\{{(m_i)}_{i\in I}: \ \textrm{tiene soporte
      finito}\} \subset \prod_{i \in I} M_i
  \]
  es un submódulo de \(\prod_{i \in I} M_i\), que se llama \textbf{suma directa
    externa} de \(\{M_i\}_{i \in I}\).
\end{lema}


En el caso de \(I\) finito, \(\bigoplus_{i\in I} M_i=\prod_{i\in I} M_i\).

\begin{df}[Suma de módulos]
  Sea \(M\) un \(R\)-módulo, \(I\) un conjunto de índices y\(\{N_i\}{i \in I}
  \subset \mathcal{L}(M)\).
  Definimos \(\sum_{i\in I} N_i\) como el \textbf{menor submódulo que contiene
  a cualquier} \(N_i\) o, equivalentemente:
  \[
    \sum_{i\in I} N_i=\left\{\sum_{i\in F} n_i: F \subseteq I \ \textrm{finito}\right\}
  \]
\end{df}

\begin{ejemplo}
  Cuando \(M_i = R\) para cada \(i \in I\), entonces
  \[
    R^{(I)} = \bigoplus_{i \in I} M_i \subset \prod_{i \in I} M_i = R^I.
  \]
\end{ejemplo}

\begin{lema}[Relación entre sumas]
  Sea \(\theta:\bigoplus M_i\longrightarrow \sum M_i\) tal que
  \(\theta({(m_i)}_{i\in I})=\sum_{i\in I} m_i\) es un homomorfismo
  sobreyectivo de \(R\)-módulos.

  Para \(\{N_i:i\in I\}\subseteq\mathcal{L}(M)\), son equivalentes:
  \begin{enumerate}
    \item Para todo \(j\in I\), \(N_j\cap\sum_{i\in I\setminus\{j\}} N_i
      =\{0\}\).
    \item Para todo \(F\subseteq I\) finito, y para todo \(j\in F\),
      \(N_j\cap\sum_{i\in F\setminus\{j\}} N_i
      =\{0\}\).
    \item La expresión de cada \(m \in \sum_{i \in I} N_i\) como \(m =
      \sum_{i \in I}M_i\) es única.
    \item Si \(0=\sum_{i\in I} m_i\) con \(m_i\in M_i\) para todo
      \(i\in I\), entonces \(m_i=0\) para todo \(i\in I\).
    \item \(\theta\) es inyectivo y por tanto un isomorfismo.
    \item Para cada par \(J_1,J_2\subseteq I\) con
      \(J_1\cap J_2=\emptyset\), se tiene que
      \(\left(\sum_{i\in J_1} N_i\right)\cap
      \left(\sum_{i\in J_2} N_i\right)=\{0\}\)
  \end{enumerate}
\end{lema}
\begin{proof}
  La demostración se inspira en el caso finito (Ejercicio \ref{ejer:rel_sumas_finito}
  y Proposición \ref{prop:suma-interna}).
\end{proof}

\begin{df}
  En caso de satisfacerse cualquiera de las condiciones anteriores
  equivalentes, diremos que la suma \(\sum_{i\in I} N_i\) es una suma
  directa interna, que notaremos por \(\dot{+}_{i\in I} N_i\).
\end{df}

\begin{cor}
  Si la familia \(\{N_i:i\in I\}\subseteq\mathcal{L}(M)\) verifican
  las condiciones y \(N\in\mathcal{L}(M)\) tal que
  \(N\cap\dot{+}_{i\in I} N_i=\{0\}\), entonces \(\{N_i:i\in I\}\cup\{N\}\).
\end{cor}

\begin{df}[Independencia]
  Si la familia \(\{N_i:i\in I\}\) donde cada módulo es distinto de 0 y
  satisface alguna de las condiciones anteriores equivalente, entonces
  diremos que dicha familia es independiente.
\end{df}

\begin{cor}
  Sea \(A\) es un DIP, \(\subscriptbefore{A}{M}\) módulo (no necesariamente de longitud
  finita).
  \[
    t(M)=\{m\in M:\ann_A(m)\neq\langle 0\rangle\}
  \]
  es un submódulo de \(M\), que se llama \textbf{submódulo de torsión de} \(M\).
  Tan solo sería necesario que \(A\) sea un dominio de integridad.
\end{cor}

\begin{ejemplo}
  Sea \(A\) un DIP, \(\subscriptbefore{A}{M}\) un módulo y consideramos
  su submódulo de torsión y supongamos que \(t(M)\neq\{0\}\).
  Definimos \(P\) como el conjunto de representantes de las clases de
  equivalencia, bajo la relación ser asociados, de los irreducibles de \(A\).

  Sea \(p\in P\), tomamos \(M_p=\{m\in M: p^e m=0\textrm{ para algún }
  e\ge 1\}\). Tenemos que \(M_p\subseteq t(M)\), \(M_p\) es un submódulo.
  Entonces:
  \[
    t(M)=\dot{+}_{p\in P} M_p
  \]
  Demostremos esto.

  Tomemos un \(m\in t(M)\), \(Am\) es un módulo de longitud finita.
  \[
    Am=N_1\dot{+}\cdots\dot{+}N_r
  \] donde \(N_i\) es una componente \(p_i\)-primaria.

  En particular, \(m=m_1+\cdots+ m_r\) de manera que \(m_i\in N_i\subseteq
  M_{p_i}\).

  Luego \(t(M) = \sum_{p\in P} M_p\). La unicidad es sencilla de deducir:
  cada \(m\) estaría en una componente primaria.
\end{ejemplo}

\begin{ejemplo}
  Veamos un caso particular del ejemplo anterior.
  Tomamos \(M=\Cont^\infty(\R)\), \(M\) es un
  \(\R[x]\)-módulo si \(xf=f'\). Entonces \(t(M)\) es el conjunto
  de las funciones que satisfacen una EDO con coeficientes constantes.

  \(P=\{\textrm{ Polinomios mónicos o bien lineales o bien
    cuadráticos irreducibles}\}\). Es decir, cualquier función que se puede
  definir mediante una EDO lineal con coeficientes constantes se puede
  escribir como suma de funciones que resuelven
  \({(\alpha\frac{\textrm{d}^2}{\textrm{d}x^2}+
    \beta\frac{\textrm{d}}{\textrm{d}x}+\gamma)}^e f=0\) con \(e\in\N\).
\end{ejemplo}

\begin{lema}
  Consideremos \(I\) un conjunto infinito y \(R^{(I)}\) tal y como lo hemos
  definido antes.
  Si \(M\) es un \(R\) módulo, existe una sucesión exacta de la forma
  \[
    0\longrightarrow L\longrightarrow R^{(I)}\longrightarrow M
    \longrightarrow 0.
  \]
\end{lema}
\begin{proof}
  Tomo \(\{m_i:i\in I\}\) tal que \(M=\sum_{i\in I}Rm_i\).
  Definimos \(\varphi:R^{(I)}\longrightarrow M\) dada por
  \(\varphi({(r_i)}_{i\in I})=\sum_{i\in I}r_i m_i\).

  \(L=\ker\varphi\overset{\iota}{\longrightarrow} M\).
\end{proof}

\begin{lema}[Existencia de bases]
  Para \(\{m_i:i\in I\}\subseteq M\), son equivalentes:
  \begin{enumerate}
    \item \(\sum_{i\in I} r_i m_i=0\) implica que \(r_i=0\) para todo índice.
    \item El homomorfismo \(\varphi:R^{(I)}\longrightarrow M\) con
      \(\varphi({(r_i)}_{i\in I})=\sum_i r_i m_i\) es inyectiva.
  \end{enumerate}
  Si se satisface 1, diremos que el conjunto \(\{m_i:i\in I\}\) es linealmente
  independiente. Si además estos elementos son además un conjunto de
  generadores, diremos que forman una base.
\end{lema}

La demostración es trivial.

\begin{obs}
  \(M\) tiene una base si, y solo si, \(M\cong R^I\) para algún \(I\).
\end{obs}

\begin{df}[Módulo libre]
  Un módulo se llama libre si admite una base.
\end{df}

\begin{obs}
  Hay muchos módulos que no son libres.
\end{obs}

Ejemplos de módulos no libres:
\begin{enumerate}
  \item Ningún grupo abeliano finito es libre como \(\Z\) módulo.
  \item \(t(M)\), \(\subscriptbefore{A}{M}\) con \(A\) un DIP, nunca es libre.
    En otras palabras \(A^{(I)}\) no es nunca un módulo de torsión (por
    ser un dominio de integridad).
\end{enumerate}

