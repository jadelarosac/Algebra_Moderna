\begin{teo}
  Las siguientes condiciones son equivalentes para un módulo \(M\):
  \begin{enumerate}
    \item Todo submódulo de \(M\) es un sumando directo.
    \item Todo monomorfismo \(L\longrightarrow M\) es escindido.
    \item Todo epimorfismo  \(M\longrightarrow N\) es escindido.
    \item \(\Soc(M)=M\).
    \item \(M\) es suma de una familia de submódulos simples.
    \item \(M\) es suma directa interna de una familia de submódulos simples.
  \end{enumerate}
  En cualquiera de los casos diremos que \(M\) es semisimple.
\end{teo}

\begin{proof}
  Como todas las afirmaciones son triviales ciertas si \(M=\{0\}\),
  suponemos que \(M\neq \{0\}\).

  Vamos a ver que la primera afirmación implica la tercera.
  Sea \(\phi: M\longrightarrow N\). Tomemos \(L=\ker\phi\).
  Por hipótesis \(M=L\dot{+}X\) para cierto \(X\in\mathcal{L}(M)\).
  Tenemos que, por los teoremas de isomorfía:
  \[
    N\cong M/L=(L\dot{+}X)/L\cong X/(L\cap X)\cong X/\{0\}\cong X
  \]

  Tenemos que para cada \(x\in X\) se va identificando con \(x+\{0\}\),
  y \(x+L\), que se identifica con \(\phi(x)\) a través de los isomorfismos
  anteriores.

  Es decir, la aplicación anterior es \(\phi|_X:X\longrightarrow N\).

  Definimos \(\varphi:N\longrightarrow M\) como \(\varphi:=\iota\circ
  {(\phi)}^{-1}\), que cumple que \(\phi\circ\varphi=\id_N\).

  Veamos que la tercera afirmación implica la segunda. Sea
  \(\varphi:L\longrightarrow M\) un monomorfismo. Consideramos la sucesión
  exacta corta dada por \(0\longrightarrow L\overset{\varphi}{\longrightarrow}
  M\overset{\kappa}{\longrightarrow} C\longrightarrow 0\) donde
  \(C=M/\Im\varphi\) y \(\kappa\) es la proyección canónica.

  Existe un \(g:C\longrightarrow M\) tal que \(\kappa\circ g=\id_C\).
  Defino \(h=\id_M-g\circ\kappa:M\longrightarrow M\).
  \[
    \kappa\circ h=\kappa -\kappa\circ g\circ\kappa =\kappa-\kappa =0
  \]
  con lo que \(\Im h\subseteq\ker\kappa\).

  Tenemos \(f:M\longrightarrow L\) tal que \(\varphi\circ f=h\) (es
  decir, \(h\) pero visto en \(L\)). Se dejan como ejercicio los detalles.
  \[\varphi\circ(f\circ\varphi)=h\circ\varphi=
  \varphi-g\circ\kappa\circ\varphi=\varphi\]
  donde el segundo sumando se anula por exactitud.

  Por la inyectividad de \(\varphi\), tenemos que podemos cancelar
  a izquierda y por tanto \(f\circ\varphi=\id_L\).

  Veamos que la segunda afirmación implica la primera, con lo que tendremos
  ya que las tres primeras son equivalentes.

  Tomamos \(X\in\mathcal{L}(M)\), tenemos que \(\iota:X\longrightarrow M\)
  es un monomorfismo. Por hipótesis, existe un \(p:M\longrightarrow X\) tal
  que \(p|_X=\id_X\). Entonces se tiene que:
  \[
    M=X\dot{+}\ker p
  \]
  que es un ejercicio sencillo.

  Vamos a ver que de la cuarta afirmación se deduce la quinta.
  La cuarta afirmación dice que \(M=\sum N_i\) donde \(N_i\) son los
  submódulos simples de \(M\).

  Veamos ahora que de la quinta se sigue la sexta. Por una proposición
  anterior (la 23)
  tomando \(N=0\), tenemos que es cierta.

  Trivialmente, la última afirmación implica la primera, tomando \(N\)
  cualquiera en la proposición 23.

  Basta ver ahora que la primera afirmación implica la cuarta.
  Por hipótesis \(M=\Soc(M)\dot{+}X\) para cierto \(X\). Veamos que
  \(X=\{0\}\). Si no fuera así, tomamos \(m\in X\setminus\{0\}\).
  El lema previo nos asegura que hay un epimorfismo
  \(p:Rm\longrightarrow S\) para \(S\) simple.

  De nuevo, \(Rm\) es un sumando directo de \(M\), existe un epimorfismo
  \(\pi:M\longrightarrow Rm\). Hacemos la composición \(p\circ\pi:
  M\longrightarrow S\).

  Como la hipótesis primera equivale a la tercera, existe
  \(\iota:S\longrightarrow M\) (por una vez no es inclusión) tal que
  \(p\circ\pi\circ\iota=\id_S\).
  \[
    S\cong\Im(\pi\circ\iota)\subseteq Rm\subseteq X
  \]
  donde hemos usado el primer teorema de isomorfía a una aplicación
  inyectiva.\@
  luego \(X\) contiene a una copia de un simple y no es simple. Así que
  \(X=0\) y \(M=\Soc(M)\).

\end{proof}

\begin{cor}
  Si M es finitamente generado y no nulo, existe un \(N\le M\) tal que
  \(M/N\) es simple.
\end{cor}
