\begin{df}[Espacio de coeficientes]
  A \(C(V)\) se le llama espacio de coeficientes.
\end{df}


\begin{lema}[de Schur]
  Sea \(\Sigma\) un \(\C G\)-módulo simple. Entonces:
  \[
    \End_{\C G}(\Sigma)=\{\lambda \id_\Sigma :\lambda \in\C\}\cong\C
  \]
\end{lema}
\begin{proof}
  Tenemos que \(\dim_\C \Sigma<\infty\). Tomamos
  \(\phi:\Sigma\longrightarrow\Sigma\) homomorfismo de \(\C G\)-módulos.
  Sea \(\phi\) es \(\C\) lineal. Tomamos \(\lambda \in\C\) valor
  propio de \(\varphi\) y sea \(V_\lambda\) el subespacio propio asociado.

  Sea \(g\in G, v\in V_\lambda\), \(\phi(gv)=g\phi(v)=g\lambda v=\lambda g v\)
  con \(gv\in V_\lambda\) entonces \(V_\lambda\) es un \(\C G\)-submódulo de
  \(\Sigma\). Si \(\phi\neq 0\) entonces \(V_\lambda\neq\{0\}\).

  Como \(\Sigma\) es simple, \(V_\lambda = \Sigma\).
\end{proof}

\begin{df}[Matriz unitaria]
  Su inversa coincide con su conjugada transpuesta
\end{df}

\begin{teo}[de Peter-Weyl]
  Dotemos a \(\mu(G)\) con el producto interno:
  \[
    \langle\varphi|\psi\rangle=\frac{1}{|G|}\sum_{x\in G}\varphi(x)
    \overline{\psi(x)}
  \]

  Tomamos \(\Omega_{\C G}=\{\Sigma_1,\ldots, \Sigma_t\}\).

  Entonces \(\mu(G)=C(\Sigma_1)\dot{+}\cdots\dot{+}C(\Sigma_t)\),
  suma directa ortogonal de \(\C G\)-módulos.

  Además, tomando en cada \(\Sigma_i\) un producto interno tal que
  la representación asociada a \(\Sigma_i\) sea unitaria, entonces
  \(\{t_{jk}^{\Sigma_i}:i\in\{1,\ldots,s\}, j,k\in\{1,\ldots, d_i\}\}\)
  es una base ortonormal de \(\mu(G)\) siempre que
  \(d_i=\dim_\C\Sigma_i\) y \(\{t_{jk}^{\Sigma_i}\) son las funciones
  matriciales asociados a \(\Sigma_i\) con respecto de una base otronormal
  de \(\Sigma_i\).
\end{teo}
\begin{proof}
  \(\mu(G)=\Soc_{\Sigma_1}(\mu(G))\dot{+}\cdots\dot{+}\Soc_{\Sigma_t}(\mu(G))\)
  y \(\Soc_{\Sigma_i}(\mu(G))\subseteq C(\Sigma_i)\). Es suma de módulos
  isomorfos a \(\Sigma_i\) cada uno en \(C(\Sigma_i)\).
  Tenemos que:
  \[
    \mu(G)=C(\Sigma_i)+\cdots+C(\Sigma_t)
  \]

  Tomo \(V\) con base \(\{v_1,\ldots, v_n\}\) y \(W\) con base
  \(\{w_1,\ldots,w_m\}\) \(\C G\)-módulos simples. Para cada \(i,j\)
  definimos \(p_{ij}:V\longrightarrow W\) lineal dada por
  \(p_{ij}(v_k)=w_j\delta_{ki}\). Entonces tomamos
  \(\tilde{p}_{ij}\) la extensión dada por
  \[
    \tilde{p}_{ij}(v)=\sum_{x\in G}x^{-1} p_{ij}(xv)
  \]
  con \(v\in V\).
  \[
    \tilde{p}_{ij}(v_k)=\sum_{x\in G}x^{-1} p_{ij}(xv_k)=
  \sum_{x\in G}x^{-1} p_{ij}\left(\sum_l t^V_{kl}(x) v_l\right)=
\]\[
  \sum_{x\in G}x^{-1} \sum_l t^V_{kl}(x) p_{ij}(v_l)=
  \sum_{x\in G}x^{-1} t^V_{ki}(x) w_j=
  \sum_l\sum_{x\in G} t^V_{ki}(x) t_{jl}^W(x^{-1}) w_l
  \]

  En concreto si las bases \(v_i\) y \(w_j\) son ortonormales, entonces
  los coeficientes de \(t_{ki}^V\) y \(t_{jl}^W\) son de matrices
  unitarias. En ese caso la expresión anterior queda:
  \[
    \sum_l\sum_{x\in G} t^V_{ki}(x) \overline{t_{lj}^W(x)} w_l
  \]

  Si \(V\not\cong W\) entonces \(\tilde{p}_{ij}=0\) y por tanto
  \[
    \sum_l\sum_{x\in G} t^V_{ki}(x) \overline{t_{lj}^W(x)} w_l=0
  \]

  Sean \(a\neq b\) y tomamos \(V=\Sigma_a, W=\Sigma_b\), entonces
  \[
    0=\sum_{x\in G} t^V_{ki}(x) \overline{t_{lj}^W(x)}
  \]
  con lo que \(C(\Sigma_a)\perp C(\Sigma_b)\).

  Luego \(\mu(G)=C(\Sigma_1)\dot{+}\cdots\dot{+}C(\Sigma_t)\).

  Supongamos ahora \(V=W=\Sigma_a\). En ese caso, por el lema de Schur
  \(\tilde{p_{ij}}(v)=\alpha_{ij} v\). Entonces (\(v_i=w_i\)):
  \[
    \alpha_{ij} v_k = \tilde{p}_{ij}(v_k)
    \sum_l\sum_{x\in G} t^V_{ki}(x) \overline{t_{lj}^V(x)} v_l
  \]

  Si \(k\neq l\), entonces \(\alpha_{ij} v_k=0\) y por tanto:
  \[
    \sum_l\sum_{x\in G} t^V_{ki}(x) \overline{t_{lj}^V(x)} v_l=0
  \]

  Si \(k=l\) e \(i\neq j\), entonces:
  \[
    0=\sum_{x\in G}
    t_{ik}^{\Sigma_a}(x^{-1})\overline{t_{jk}^{\Sigma_a}(x^{-1})}
    =\sum_{x\in G} t_{ki}^{\Sigma_a}(x)\overline{t_{kj}^{\Sigma_a}(x)}
  \]
  luego \(\{t_{ij}^{\Sigma_a}\}\) es un sistema ortogonal generador,
  en particular es una base ortogonal. Veamos que no es ortonormal.

  \[
    \sum_{x\in G} t_{ki}^{\Sigma_a}(x)\overline{t_{ki}^{\Sigma_a}(x)}=
    |G|
  \]
  Luego son una base ortogonal.

  \(\tilde{p}_{ii}(v) = \sum_{x\in G} x^{-1} p_{ii}(xv)\),
   con lo que \(\tilde{p}_{ii}
   =\sum_{x\in G}\rho(x^{-1})\circ p_{ii}\circ\rho(x)\)
   donde \(\rho:G\longrightarrow GL(\Sigma_a)\) donde \(\rho(x)(v):=xv\).
   Por ser homomorfismo de grupos:
  \[
    \tilde{p}_{ii} =\sum_{x\in G}{\rho(x)}^{-1}\circ p_{ii}\circ\rho(x)
  \]
  Luego la traza del endomorfismo es
  \(d_a \alpha_{ii} = |G|\). Tenemos:
  \[
    \alpha_{ii}  =
    \sum_l\sum_{x\in G} t^V_{ki}(x) \overline{t_{li}^V(x)}
    \sum_l\sum_{x\in G} |t^V_{ki}(x)|^2
  \]
  con lo que \(|t_{ij}^{\Sigma_a}|^2=
  \langle t_{ij}^{\Sigma_a}| t_{ij}^{\Sigma_a}\rangle
  =\frac{1}{|G|}\alpha_{ii}=\frac{1}{d_a}\).

  Luego la base
  \[
    \{\sqrt{d_a}t_{ij}^{\Sigma_a}:
    a\in \{1,\ldots t\}, i,j\in\{1,\ldots, d_a\}\}
  \]
  es una base ortonormal.

\end{proof}

