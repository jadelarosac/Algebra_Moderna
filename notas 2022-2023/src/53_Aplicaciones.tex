Además \(t^{\Sigma_j}t^{\Sigma_k}=t^{\Sigma_j+\Sigma_k}\).

Si llamamos \(\hat{\varphi}(j)=\langle\varphi|t^{\Sigma_j}\rangle
=\frac{1}{n}\sum_{k=0}^{n-1} \varphi(k)\omega^{-kj}\) tenemos que
\[
  \varphi\psi=\sum_{j,k}\hat{\varphi}(j)\hat{\psi}(k)t^{\Sigma_j}t^{\Sigma_k}
  =\sum_{l=0}^{n-1}\left(\sum_{j=0}^{n-1}\hat{\varphi}(j)\hat{\psi}(e-j)\right)
  t^{\Sigma_e}
\]

Ejercicio: Para \(G=\Z_n\times\Z_m\), calcular \(\Omega_{\C G}\) y deducir la
correspondiente base ortonormal de \(\mu(\Z_n\times\Z_m)\).

Ejemplo: \(D_n=\langle r,s | r^n = s^2 = 1, sr=r^{-1}s\rangle\).
\(D_n=\{s^a r^j:a \in\{0,1\}, j\in\{0,\ldots,n-1\}\}\).
Veamos qué hacer con \(\mu(D_n)\). Como el grupo no es conmutativo debe haber
al menos una representación de un subgrupo simple de dimensión mayor que 1.

\(\alpha\in\C\), \(\alpha^n=1\), \(V_\alpha\) un \(\C\)-espacio vectorial
hermítico con base ortonormal \(\{v_1, v_2\}\).
Tenemos que buscar una representación: \(D_n\longrightarrow U(V_\alpha)\)
donde
\[
  r\mapsto
  \begin{pmatrix}
    \alpha&0\\
    0&\overline{\alpha}\\
  \end{pmatrix}
\quad\quad\quad
  s\mapsto
  \begin{pmatrix}
    0&1\\
    1&0\\
  \end{pmatrix}
\]
Comprobamos que (recordando que el producto por matrices se hace en el orden
inverso porque trabajamos por filas):
\[
  \begin{pmatrix}
    \alpha&0\\
    0&\overline{\alpha}\\
  \end{pmatrix}
  \begin{pmatrix}
    0&1\\
    1&0\\
  \end{pmatrix}
  =
  \begin{pmatrix}
    0&\alpha\\
    \overline{\alpha}&0\\
  \end{pmatrix}
  =
  \begin{pmatrix}
    \overline{\alpha}&0\\
    0&\alpha\\
  \end{pmatrix}
  \begin{pmatrix}
    0&1\\
    1&0\\
  \end{pmatrix}
\]

¿Cuando \(V_\alpha\) irreducible? (o sea, simple). No será simple si y solo
si existe \(v\in V_\alpha\) tal que \(\C v\) es un submódulo. Esto equivale
a que \(rv, sv\in \C v\).

Trabajando con coordenadas \(v=(x,y)\in\C^2\).
\[
  (x\quad y)
  \begin{pmatrix}
    \alpha&0\\
    0&\overline{\alpha}\\
  \end{pmatrix}
  =\beta(x\quad y)
\]
\[
  (x\quad y)
  \begin{pmatrix}
    0&1\\
    1&0\\
  \end{pmatrix}
  =\gamma(x\quad y)
\]
y resolviendo las ecuaciones obtenemos que \(\alpha^2=1\).

Luego si \(\alpha\neq\pm 1\) tenemos una representación irreducible.

Supongamos que bajo las mismas condiciones \(V_\alpha\cong V_{\alpha'}\).
Tomando trazas, tenemos que \(\alpha+\overline{\alpha}=
\alpha'+\overline{\alpha'}\). Es decir, si \(\alpha+\overline{\alpha}\neq
\alpha'+\overline{\alpha'}\) y entonces \(V_\alpha\not\cong V_{\alpha'}\).

Funciones matriciales de \(V_\alpha\):
\[
  \begin{pmatrix}
    t_{11}  & t_{12}\\
    t_{21}  & t_{22}\\
  \end{pmatrix}
\]
tenemos que
\[
  s^a r^j\mapsto
  \begin{pmatrix}
    \alpha  & 0\\
    0       & \overline{\alpha}\\
  \end{pmatrix}^j
  \begin{pmatrix}
    0 & 1\\
    1 & 0\\
  \end{pmatrix}^a
\]
si \(a=0\) tenemos:
\[
  \begin{pmatrix}
    \alpha^j  & 0\\
    0       & \overline{\alpha}^j\\
  \end{pmatrix}
\]
si \(a=1\) tenemos:
\[
  \begin{pmatrix}
    0&\alpha^j\\
    \overline{\alpha}^j& 0\\
  \end{pmatrix}
\]

Tenemos entonces que \(t_{11}(s^a r^j)=\chi_{0}(a)\alpha^j\),
\(t_{12}(s^a r^j)=\chi_{1}(a)\alpha^j\),
\(t_{21}(s^a r^j)=\chi_{1}(a)\overline{\alpha}^j\),
\(t_{22}(s^a r^j)=\chi_{1}(a)\overline{\alpha}^j\)

