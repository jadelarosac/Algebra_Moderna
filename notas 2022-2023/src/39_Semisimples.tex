\begin{cor}
  Todo cociente de y todo submódulo de un módulo semisimple es semisimple.
\end{cor}
\begin{proof}
  Sea \(M\) semisimple y tomamos \(N\) un submódulo. Veamos que \(M/N\) es
  también semisimple. \(M\) es semisimple, luego es suma de módulos simples:
  \[
    M=\sum_{i\in I} S_i
  \]

  Consideremos \(p:M\longrightarrow M/N\) la proyección canónica (\(m\mapsto
  m+N\)). Tenemos que \(M/N=\sum_{i\in I} p(S_i)\). Para cada \(i\in S_i\)
  puede que \(p(S_i)=0\) (que sobran de la suma) o \(p(S_i)\neq 0\).

  \(p(S_i)\) es simple porque \(p:S_i\longrightarrow p(S_i)\) es un
  isomorfismo (inyectiva y definida sobre su imagen).

  Veamos que pasa con los submódulos. \(M=N\dot{+}X\) para algún \(X\).
  Esto implica que \(m=n+x\overset{\pi}{\mapsto} n\) es un epimorfismo de
  módulos entre \(M\) y \(N\).
  Entonces \(N\cong N/\ker\pi\), luego es semisimple.
\end{proof}

\begin{cor}
  \(M\) es semisimple finitamente generado si y solo si \(M=S_1\dot{+}\cdots
  \dot{+} S_n\) para \(S_i\) simple.
\end{cor}
\begin{proof}
  La implicación hacia la izquierda es una aplicación directa del teorema.

  Por otro lado, \(M=\dot{+}_{i\in I} S_i\), por ser simple.
  Sean \(m_1,\ldots, m_t\) generadores de \(M\). Entonces existe \(F\subseteq
  I\) finito tal que
  \[m_j\in \dot{+}_{i\in F} S_i\] para todo \(j\). Entonces
  \[ M\subseteq \dot{+}_{i\in F}\subseteq M\]
  con lo que \(M\) es una suma finita.
\end{proof}

\subsubsection{Anillos semisimples}
\begin{df}[Anillos semisimples]
  Un anillo \(R\) es semisimple si todo \(R\)-módulo es semisimple.
\end{df}

\begin{obs}
  Todo anillo de división es semisimple. ¿Hay más?
\end{obs}

\begin{teo}
  \(R\) es semisimple si y solo si \(\subscriptbefore{R}{R}\) es semisimple.
  Es decir, todos los módulos sobre \(R\) son semisimples si y solo si lo
  es el regular.
\end{teo}
\begin{proof}
  Sea \(\subscriptbefore{R}{M}\) un módulo.
  Está claro que \(Rm\cong R/\ann_R(m)\), que es un cociente de un semisimple,
  luego semisimple para cualquier \(m\). Tenemos que para ciertos \(m\in M\):
  \[
    M=\sum_{m\in M} Rm
  \]
  con lo que \(M\) es suma de semisimples, luego semisimple.

\end{proof}

\begin{df}[Anillo de endomorfismos]
  Sea \(M\) un \(R\)-módulo, definimos:
  \[
    \End_R(M)=\{f:M\longrightarrow M:
    f\textrm{ homomorfismo de módulos sobre }R\}
  \]
  es un subanillo de \(\End(M)\).

  Llamemos \(S=\End_R(M)\), tenemos que \(M\) es un \(S\)-módulo puesto que
  \(S\subseteq \End(M)\). \(\End_R(M)\) es el anillo de endomorfismos de
  \(M\).

  ¿Cuál es la acción en \(M\)? La inclusión: \(f\in S\), tenemos
  \(fm=f(m)\).
\end{df}
\begin{df}[Biendomorfismos]
  ¿Quién es \(\End_S(M)\)?
  Obviamente, \(\End_S(M)\subseteq\End(M)\) subanillo.

  Dado \(g\in\End(M)\), \(g\in \End_S(M)\) si y solo si \(g(fm)=fg(m)\) para
  todo \(f\in S=\End_R(M)\). Pero \(g(f(m))=g(fm)=fg(m)=f(g(m))\) con
  \(m\in M\). Pero esto es lo mismo que decir que \(g\circ f=f\circ g\).

  \[
    \End_S(M)=\{g\in\End(M):g\circ f=f\circ g \quad\forall f\in\End_R(M)\}
  \]

  Llamaremos \(T=\End_S(M)\).
\end{df}

\begin{lema}
  \(R\overset{\lambda}{\longrightarrow} \End_S(M)\) dado por
  \(\lambda(r):M\longrightarrow M\) y \(\lambda(r)(m)=rm\).
  Dicho \(\lambda\) es un homomorfismo de anillos.
\end{lema}
\begin{proof}
  Basta con ver que \(\Im\lambda\subseteq\End_S(M)\). O sea que
  \(\lambda(r)\circ f=f\circ\lambda(r)\) para todo \(r\in R\).
  En efecto:
  \[
    (\lambda(r)\circ f)(m)=rf(m)=f(rm)=(f\circ\lambda(r))(m)
  \]
  para todo \(f\in S\) y todo \(m \in M\).
\end{proof}
\begin{obs}
  El conjunto de ``triendomorfismos'' coincide con el de endomorfismos.
  Es decir, \(S=\End_T(M)\).
\end{obs}
\begin{prop}
  Los \(R\)-sumandos directos de \(M\) son los mismos que los \(T\)-sumandos
  directos de \(M\).

  Como consecuencia, si \(\subscriptbefore{R}{M}\) es semisimple, entonces
  \(\subscriptbefore{T}{M}\) también lo es.
\end{prop}
\begin{proof}
  Si \(N\) es un \(T\)-sumando directo de \(M\) tenemos que
  \(M=N\dot{+}X\) para cierto \(X\in\mathcal{L}(\subscriptbefore{T}{M})\).
  Entonces \(N\dot{+}X=M\) como \(R\)-módulos.

  Recíprocamente, \(\subscriptbefore{R}{M}=X\dot{+}Y\) con \(X,Y\in\mathcal{L}
  (\subscriptbefore{R}{M})\). Basta ver que \(X\) es un \(T\)-módulo.

  Tomo \(p:M\longrightarrow M\), tal que \(p(m)=p(x+y)=x\) y
  \(p\in S=\End_R(S)\), y \(X=\Im p\). Tomo \(g\in T\), \(x\in X\),
  \[g x=g(x)=g(p(x))=(g\circ p)(x)=(p\circ g)(x)=p(g(x))\in \Im p=X\]
  luego \(X\in\mathcal{L}(\subscriptbefore{T}{M})\).

  Veamos que si uno es semisimple lo es el otro.
  Entonces si \(N\)  es un sumando directo de \(M\)
  visto como \(R\)-módulo, entonces
  lo es como sumando directo como \(T\) módulo, luego \(M\) es semisimple.

\end{proof}

