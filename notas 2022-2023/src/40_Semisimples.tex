
\begin{cor}
  Si \(\subscriptbefore{R}{M}\) es semisimple,
  \(\ell(\subscriptbefore{R}{M})<\infty\), entonces \(\subscriptbefore{T}{M}\)
  y \(\ell(\subscriptbefore{R}{M})=\ell(\subscriptbefore{T}{M})\).
\end{cor}

Tenemos que, dado \(\subscriptbefore{R}{M}\),
\(\subscriptbefore{R}{M}^n=M\oplus\overset{(n)}{\cdots}\oplus M\).
Sea \(S'=\End_R(M^n)\).

Sea \(\iota_i\) la aplicación dada por \(m\mapsto(0,\ldots,0,m,0,\ldots0)\) y
\(\pi_j\) la que aplica \(m_i=(m_1,\ldots, m_j,\ldots, m_n)\mapsto m_j\).

Tenemos que \(\id_{M^n}=\sum_{i=1}^n\iota_i\circ\pi_i\in S'\).
Dado \(f\in\End_S(M)\), definimos \(\bar{f}=\sum_{i=1}^n\iota_i\circ
f\circ\pi_i\in\End(M^n)\), en concreto
\[
  \bar{f}(m_1,\ldots, m_n)=(f(m_1),\ldots,f(m_n))
\]

A partir de ahora prescindimos del símbolo \(\circ\) para indicar composición.

Tomando \(g\in S'\), tenemos que
\[
  g\bar{f}=\sum_{i,j=1}^n \iota_i\pi_i g \iota_j f\pi_j
  =\sum_{i,j=1}^n \iota_i f\pi_i g\iota_j\pi_j=\bar{f}g
\]
con lo que \(f\in \End_{S'}(M^n)\).

\begin{teo}[de densidad de Jacobson]
  Sea \(M\) un \(R\)-módulo semisimple. Sean \(m_1,\ldots, m_n\in M\)
  y \(S=\End_R(M)\). Para cada \(f\in\End_S(M)\) existe un \(r\in R\)
  tal que \(f(m_i)=rm_i\) para todo \(i\in\{1,\ldots,n\}\).
\end{teo}
\begin{proof}
  Sea \(m=(m_1,\ldots, m_n)\in M^n\). Sé que \(M^n\) es \(R\)-semisimple.
  \(
    Rm
  \) es un \(R\)-sumando directo de \(M^n\). Entonces \(Rm\) es un
  \(\End_{S'}(M^n)\)-submódulo de \(M^n\).

  Como \(\bar{f}\in\End_{S'}(M^n)\), entonces
  \((f(m_1),\ldots,f(m_n))=\bar{f}(m)=\bar{f} m\in Rm\), con lo que
  existe un \(r\in R\) tal que \(f(m_i)=rm_i\).

\end{proof}

\begin{lema}[de Schur]
  Sean \(\subscriptbefore{R}{M}\), \(\subscriptbefore{R}{N}\) y
  \(f:M\longrightarrow N\) es homomorfismo de \(R\)-módulos,
  entonces \(f\) o es 0 o es un isomorfismo.

  Así, \(\End_R(M)\) es una anillo de división.
\end{lema}


