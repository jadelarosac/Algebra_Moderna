\subsection{Suma directa externa}

\begin{df}
Tomando el producto cartesiano de \(t\) módulos sobre el mismo anillo
y tomando la suma usual de tuplas y definiendo el siguiente producto:
\[
  a(m_1,\ldots, m_t)=(am_1,\ldots,am_t)
\]

Es un módulo que se llama suma directa externa de \(M_1,\ldots, M_t\)
con \(M^t\) si son todos iguales.

  Se denota \(M_1\oplus\cdots\oplus M_t\).
\end{df}


Ejercicio: Sea \(\subscriptbefore{A}{M}\), \(N_1,\ldots,
N_t\in\mathcal{L}(\subscriptbefore{A}{M})\). Se pide demostrar que existe
un homomorfismo \(f:N_1\oplus\cdots\oplus N_t\longrightarrow
N_1{+}\cdots{+}N_t\)
sobreyectivo de \(A\)-módulos tal que entre la suma directa
externa y la suma interna, tal que \(f\) es un isomorfismo si y solo si
la suma interna es directa. Podría ser interesante usar coordenadas.
