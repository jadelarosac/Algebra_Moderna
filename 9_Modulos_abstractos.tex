\subsection{Módulos abstractos}

\begin{df}
  Sea \({}_AM\) un módulo cuyo homomorfismo es \(\rho\). Se dice que \({}_AM\) es
  \textbf{fiel} si \(\rho\) es inyectivo.
\end{df}
\begin{df}
  Se define el \textbf{anulador de M} como
  \[\Ann_A(M) = {a \in A \ /\ am = 0\ \text{para todo}\ m \in M}.\]
\end{df}

\begin{obs}
  Sean \(A\) un anillo, \(\subscriptbefore{A}{M}\) un \(A\)-módulo y
  un homomorfismo de anillos \(\rho:A\longrightarrow\End(M)\) y \(a \in A\).
  \(\rho(a) = 0\) siempre que \(a \in \Ann_A(M)\), luego \(\Ann_A(M) = \ker(\rho)\).
  Por eso, se puede afirmar que \(\Ann_A(M)\) es un ideal de \(A\) y se tiene que
  \(M\) es fiel si, solo si, \(\Ann_A(M) = {0}\).

  Aplicando el primer teorema de isomorfía, tenemos:
  \[
    A/\ker\rho = A/\Ann_A(M) \simeq \Im\rho \leqslant \End(M)
  \]
  \noindent y entonces \(M\) es un \(A/\ker\rho\)-módulo.
  De hecho, \((a+\ker\rho)m=\rho(m)\).
  \(M\) siempre es fiel sobre \(A/\Ann_A(M)\), aunque \(\rho\) no sea inyectiva.
\end{obs}

\begin{ejercicio}
  Si tenemos una aplicación lineal entre espacios vectoriales
  de dimensión finita, entonces el anulador está generado por un único
  polinomio, el polinomio mínimo de \(T\). Se denota como \(\Ann_{K[x]}(V) =
  < \mu(x) >\).
\end{ejercicio}

\begin{ejercicio}
  Sea \(K\) un cuerpo, \(V\) un \(K\)-espacio vectorial de dimensión finita y
  \(T:V \longrightarrow V\) una aplicación linea. Considerando que \(V\) tiene estructura
  de \(K[x]\)-módulo, probar que el anulador es no vacío.
\end{ejercicio}

\subsection{Submódulos}
\begin{df}
  Dado un módulo \({}_AM\), un \textbf{submódulo} de \({}_AM\)
  es un subgrupo aditivo \(N\subseteq M\)
  tal que \(am\in N\) para cualquier \(a\in A\) y \(m\in N\).
\end{df}

\begin{ejemplo}
  Los submódulos del módulo regular \(A\) se llaman \textbf{ideales por la izquierda de
    \(A\)}. Todo ideal es un ideal a izquierda. Si \(A\) es conmutativo, los ideales
  a izquierda coinciden con los ideales.

  Tomando \(A=\mathcal{M}_2(K)\) con \(K\) un cuerpo.
  \[
    \mathcal{M}_2(K)=\left\{
      \begin{pmatrix}
        a&b\\
        c&d
      \end{pmatrix}:
      a,b,c,d\in K
    \right\}
  \]

  Tenemos que el conjunto:
  \[
    \left\{
      \begin{pmatrix}
        0&b\\
        0&d
      \end{pmatrix}:
      b,d\in K
    \right\}
  \]
  es un ideal a izquierda de \(A\).
\end{ejemplo}

\begin{ejemplo}
  \(T:V\longrightarrow V\), \(K\)-lineal.
  ¿Qué es un \(K[x]\)-submódulo de \(V_{K[x]}\)?
  Sea \(W\) un tal submódulo.
  \(W\) es un subespacio vectorial y además \(T(w)=xw\in W\),
  es decir, un subespacio \(T\)-invariante (un ejemplo
  de subespacio \(T\) invariante es un subespacio propio).
  El recíproco es también cierto.
\end{ejemplo}

\begin{df}[Submódulo cíclico]
  Dado \(\subscriptbefore{A}{M}\), y un \(m\in M\). Es claro que
  \(Am=\{am:a\in A\}\) es un submódulo de \(\subscriptbefore{A}{M}\) que se llama
  \textbf{submódulo cíclico generado por \(m\)}.
\end{df}

\begin{ejemplo}
  \(\R[x]\sin t=\R\sin t+\R\cos t\).
\end{ejemplo}

\begin{df}[Submódulo finitamente generado]
  Dados \(m_1,\ldots, m_n\in M\), el conjunto
  \[
    Am_1+\cdots+Am_n=\{a_1 m_1+\cdots+a_n m_n: a_i\in A\}
  \]
  es un submódulo de \(\subscriptbefore{A}{M}\) llamado el \textbf{submódulo generado por
  \(m_1,\ldots, m_n\)}.
  Si \(M=Am_1+\cdots+Am_n\), diremos que \(M\) es \textbf{finitamente generado}
  con generadores
  \(m_1,\ldots, m_n\).
\end{df}

\subsubsection{Suma directa interna}

\begin{df}[Módulo suma]
  Dados \(N_1,\ldots, N_n\) submódulos de \(\subscriptbefore{A}{M}\),
  \[
    N_1+\cdots+ N_n=\{m_1+\cdots+m_n: m_i\in N_i\}
  \]
  es un submódulo de \(M\), que se llama \textbf{suma de \(N_1, \cdots, N_n\)}.
\end{df}

\begin{nt}
  Se puede expresar \(N_1+\cdots+ N_n\) como \(\sum_{i=1}^n N_i\).
\end{nt}

\begin{prop}
  Sean \(N_1,\ldots, N_t\) submódulos de \(A\). Son equivalentes:
  \begin{enumerate}
    \item Para cada \(i = 1, \cdots t\), \(N_i\cap\sum_{j\neq i}N_j =\{0\}\).
    \item Si \(0=n_1+\cdots+n_t\), \(n_i\in N_i\) entonces \(n_i=0\)
      para todo \(i = 1, \cdots t\).
    \item Cada \(n\in N_1+\cdots+N_t\) admite una representación
      única como \(n=n_1+\cdots+n_t\), con \(n_i\in N_i\).
  \end{enumerate}
\end{prop}
\begin{proof}
  Veamos que 1 implica 2. Tenemos que \(0=n_1+\cdots+n_t\),
  si despejamos, \(n_i=-\sum_{j\neq i} n_j\in N_i\cap\left(
  \sum_{j\neq i} N_j\right)=\{0\}\).

  Veamos que 2 implica 3. Si \(n=\sum n_i=\sum n_i'\),
  entonces \(0=\sum(n_i-n_i')\) lo que implica que
  \(n_i=n_i'\).

  Finalmente, tomando \(n\in N_i\cap\left(
  \sum_{j\neq i} N_j\right)\), es decir,
  \(n=\sum_{j\neq i}n_j\) con lo que
  \(0=n-\sum_{j\neq i} n_j\) y como las descomposiciones
  son únicas, \(n=0\).
\end{proof}

\begin{df}[Suma interna]
  Si \(M=N_1+\cdots+ N_t\) tales que satisfacen una de las condiciones
  equivalentes anteriores, diremos que \(M\) es la suma directa interna
  y usaremos la notación \(M=N_1\dot{+}\cdots\dot{+} N_t\).
\end{df}

\begin{df}
  Si \(\{N_1,\ldots, N_t\}\) verifican las condiciones equivalentes
  anteriores y \(N_i\neq \{0\}\), se dice que el conjunto
  \(\{N_1,\ldots, N_t\}\) es una
  familia independiente.
\end{df}

\begin{ejemplo}
  \(\Z_6\) es un \(\Z\) módulo.
\[
  \Z_6=\{0,1,2,3,4,5,6\}
\]
Tomamos
\[
  N_1=\{0,3\}
\]
y
\[
  N_2=\{0,2,4\}
\]

Tenemos que \(N_1, N_2\) es una familia independiente. Además es obvio que:
\[
  N_1\dot{+}N_2=\Z_6
\]
ya que tienen como intersección \(\{0\}\) y su suma es el total.
\end{ejemplo}