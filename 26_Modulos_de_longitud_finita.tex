\begin{obs}
  Sea \(M\) un grupo abeliano de longitud finita y \(A = \Z\).
  Los grupos abelianos son de longitud finita si, y solo si, son
  finitos.
\end{obs}
\begin{proof}
  \(\mu=p_1^{e_1}\cdots p_r^{e_r}\)
  \[
    M=\dot{+}_{i=1}^r\dot{+}_{j=1}^{n_i}\Z x_{ij}\cong
    \oplus_{i=1}^r\oplus_{j=1}^{n_i}\Z_{p_i^{e_{ij}}}
  \]
  con \(x_{ij}\). Luego es finito de cardinal:
  \[
    m=\prod_{i=1}^r\prod_{j=1}^{n_i} p_i^{e_{ij}}
    =p_1^{f_1}\cdots p_r^{f_r}
  \]
  donde \(f_i=\sum_{j=1}^{n_i} e_{ij}\).

  \(\mu| m\).

\end{proof}

\begin{ejemplo}
  Sean \(A = \Z\), \(M\) un grupo aditivo no nulo de longitud finita.
  \(\mu\Z = \Ann_{\Z}(M)\), con \(\mu = p_1^{e_1} \cdots p_r^{e_r}\); \(p_1, \ldots,
  p_r\) primos. Entonces existen \(n_1, \ldots, n_r \in \N\backslash\{0\}\),
  \(e_i = e_{i1} \ge \cdots \ge e_{in_i}\) para cada \(i = 1, \ldots, r\) tales que
  \(M \cong \oplus^r_{i=1} \oplus^{n_i}_{j = 1} \Z_{p_i}^{e_{ij}}\). Se llama
  \textbf{descomposición cíclica primaria de \(M\)} y el cardinal de \(M\) es
  \[
    m = \#M = \prod^r_{i=1}\prod^{n_i}_{j=1} \#\Z_{p_i^{e_{ij}}} =
    \prod^r_{i=1}\prod^{n_i}_{j=1} p_i^{e_{ij}}
  \]
\end{ejemplo}

\begin{ejemplo}
  Si \(m=12\), \(p_1=2\) y \(p_2=3\). Entonces \(M\cong \Z_4\oplus
  \Z_3\cong\Z_{12}\) o \(M\cong \Z_2\oplus
  \Z_2 \oplus \Z_3 \cong \Z_2 \oplus \Z_{6}\).
\end{ejemplo}

\begin{ejemplo}
  Veamos la \textbf{forma canónica de Jordan}. Sean \(A = K[x]\) y \(V\) un
  \(K[x]\)-módulo de longitud finita.
  \(V\) es dimensión finita:
  \[
    V=\dot{+}_{i=1}^r\dot{+}_{j=1}^{n_i} K[x]x_{ij}
  \]
  luego es suma directa de espacios de dimensión finita.

  \[
    V_{ij}=K[x]x_{ij}\subseteq V
  \]
  donde \(T(V_{ij})\subseteq V_{ij}\). Tenemos que
  \[
    \minpol(\left.T\right|_{V_{ij}})=p_i^{e_{ij}}
  \]
  existen \(x_{ij}\) tales que \(\{x_{ij},Tx_{ij},\ldots,
  T^{\dim V -1}x_{ij}\}\) base de \(V_{ij}\).

  Caso particular: \(\dim V=n\), \(\minpol(T)={(x-\lambda)}^n\).
  Existe un \(v\in V\) tal que
  \[
    \{v, (T-\lambda)v,\ldots,{(T-\lambda)}^{n-1} v\}
  \]

  Aplicamos \(T{(T-\lambda)}^i v=(T-\lambda+\lambda){(T-\lambda)}^i v=
  {(T-\lambda)}^{i+1}v+\lambda{(T-\lambda)}^i v\).
  La matriz asociada es:
  \[
    M_B(T)=
    \begin{pmatrix}
      \lambda&1&0&0&\cdots&0\\
      0&\lambda&1&0&\cdots&0\\
      0&0&\lambda&1&\cdots&0\\
      \vdots&\vdots&\vdots&\vdots&\cdots&\vdots\\
      0&0&0&0&\cdots&1\\
      0&0&0&0&\cdots&\lambda\\
    \end{pmatrix}
  \]
  A matrices de este tipo las llamaremos \textbf{bloque de Jordan}.

  Si le aplicamos al caso general en el que
  \(\mu={(x-\lambda_1)}^{e_1}\cdots{(x-\lambda_r)}^{e_r}\). Tomamos en
  cada \(V_{ij} =K[x]x_{ij}\) la base \(\{x_{ij},\ldots,
  {(T-\lambda)}^{e_{ij}-1} x_{ij}\}\) y obtenemos uniendo ordenadamente las
  bases una base de \(V\), llámase \(B\), tal que por bloques se expresa:
  \[
    M_B(T)=
    \begin{pmatrix}
      J_{e_{ij}}(\lambda_i)&0&0&0&\cdots&0\\
      0&J_{e_{ij}}(\lambda_i)&0&0&\cdots&0\\
      0&0&J_{e_{ij}}(\lambda_i)&0&\cdots&0\\
      \vdots&\vdots&\vdots&\vdots&\cdots&\vdots\\
      0&0&0&0&\cdots&0\\
      0&0&0&0&\cdots&J_{e_{ij}}(\lambda_i)\\
    \end{pmatrix}
  \]
\end{ejemplo}