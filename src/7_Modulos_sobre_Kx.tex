\subsection{\(K[x]\)-módulos con \(K\) cuerpo}

Tenemos \(K[x]\)-módulo \(M\). O sea, \(M\) es un grupo aditivo
y \(\rho: K[x]\longrightarrow\End(M)\) es un homomorfismo de anillos.

\(K\) se puede ver como subanillo de \(K[x]\), aplicando la
restricción de escalares aplicada a la aplicación inclusión,
\(M\) es un \(K\)-espacio vectorial.

Veamos que ocurre con la indeterminada. \(\rho(x)\in\End(M)\).

Veamos que es un endomorfismo de espacios vectoriales:
\[
  \rho(x)(km)=x\cdot (km)=x\cdot(k\cdot m)=(xk)\cdot m
  =kx\cdot m=k(xm)=k\rho(x)(m)
\]

Así que \(\rho(x)\) es \(K\)-lineal.

Si \(p=\sum_i p_i x^i\in K[x]\), tenemos que
\[
  pm=\rho(p)(m)=\sum_i p_i {\rho(x)}^i(m)
\]

\begin{prop}
  Si tengo un \(K\)-espacio vectorial \(V\) y una aplicación
  lineal \(T:V\longrightarrow V\), podemos definir para \(p\in K[x]\)
  y \(v\in V\) el operador
  \[
    pv:=p(T)(v)=\sum_i p_i T^i(v)
  \]
  resulta que \(V\) es un \(K[x]\)-módulo.
\end{prop}

Ejemplo, \(\Cont^\infty(\R)\) con \(T=\frac{d}{dt}\)
es un \(\R[x]\)-módulo.
