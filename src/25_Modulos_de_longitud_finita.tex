\begin{obs}
  Si \(A=\Z\), \(M\) grupo abeliano, \(x\in M\),
  \(\ann_\Z(x)=n\Z\), \(n\) recibe el nombre de el orden.
\end{obs}

\begin{obs}
  Si \(A=K[x]\), \(T:V\longrightarrow V\), \(n=\dim_K V<\infty\),
  \(v\in V\), \(\ann_{K[x]}(v)=\langle f(x)\rangle\).
  Tenemos que \(f\) tiene grado \(n\). \(\{v, Tv,\ldots, T^{n-1}v\}\)
  es una base de \(V\).
\end{obs}

Ejemplo: \(\mathcal{U}(\Z_8)=\{1,3,5,7\}\).
Viendo los ordenes de los elementos:
\[\mathcal{U}(\Z_8)=\langle 3\rangle\dot{+}\langle 5\rangle\]
donde \(\langle \cdot\rangle\) es la generación como subgrupo.

Ejemplo: Suponemos un espacio vectorial \(V\) de dimensión 3 y un
endomorfismo \(T\) cuyo polinomio mínimo es de la forma
\({(x-\lambda)}^2\) con \(\lambda\in K\).
Sabemos que existen dos vectores \(v_1,v_2\) tales que
\[
  V=K[x]v_1\dot{+} K[x]v_2
\]
con \(\ann_{K[x]} v=\langle {(x-\lambda)}^2\rangle
\subsetneq\langle {x-\lambda}\rangle=\ann_{K[x]} v_2\).

\begin{cor}
  Si \(\subscriptbefore{A}{M}\) es un módulo \(p\)-primario, entonces
  \[
    M\cong C_1\oplus\cdots\oplus C_n
  \]
  con \(C_i\) cíclico.

  Si \(M\cong D_1\oplus\cdots\oplus D_m\), con \(D_i\) cíclico, entonces
  \(n=m\) y tras reordenación, \(D_i\cong C_i\) para todo \(i\).
\end{cor}
\begin{proof}
  De \(M\cong C_1\oplus\cdots\oplus C_n\), se puede exigir que
  \(x_1,\ldots,x_n\in M\) tales que
  \[
    M=Ax_1\dot{+}\cdots\dot{+}Ax_n
  \]
  con \(\ann_A(x_1)\subseteq\ann_A(x_2)\subseteq\ldots\subseteq\ann_A(x_n)\)

  Con \(D_1\oplus\cdots\oplus D_m\) hago lo mismo.
  \[
    M=Ay_1\dot{+}\cdots\dot{+}Ay_n
  \]
  ordenados bajo el mismo criterio.

  El enunciado se sigue de aplicar el teorema anterior. De
  \(\ann(x_i)=\ann(y_i)\) se deduce
  \[
    C_i\cong Ax_i\cong A/\ann(x_i)=A/\ann(y_i)\cong Ay_i\cong D_i
  \]
\end{proof}

Ejercicio: Decimos que un módulo \(M\) es indescomponible si \(M\cong
L\oplus N\) implica que \(L=\{0\}\) (o \(N=\{0\}\)).
Razonar que en el corolario cada uno de los \(C_i\) es indescomponible.

Ejemplo: \(M\) grupo abeliano de longitud finita y \(p\)-primario.
Aplicando el corolario, \(M\cong C_1\oplus\cdots\oplus C_n\) con
\(C_i\) cíclico y de longitud finita \(p\)-primarios.
Tenemos que \(M\cong \Z_{p^{m_1}}\oplus\cdots\oplus\Z_{p^{m_n}}\),
\(M\) es finito de cardinal \(p^{m_1+\cdots+m_n}\).

\begin{teo}[Estructura de módulos sobre un DIP]
  \(\subscriptbefore{A}{M}\neq\{0\}\) de longitud finita. Existen
  irreducibles distintos \(p_1,\ldots,p_r\in A\) y enteros positivos
  \(n_1,\ldots, n_r\), tales que \(e_{i1}\ge\ldots\ge e_{in_i}\) con
  \(i\in\{1,\ldots,r\}\) determinados por M:\@
  \[
    M=\dot{+}_{i=1}^r\left(\dot{+}_{j=1}^{n_i} Ax_{ij}\right)
  \]
  A esa expresión
  se le llama la descomposición cíclica-primaria de \(M\) (la
  primaria sería la primera suma y luego cada factor primario se
  descompone en factores cíclicos).
  Los \(x_{ij}\in M\) son tales que verifican:
  \[
    \ann_A(x_{ij})=\langle p_i^{e_{ij}}\rangle
  \]
  con \(i\in\{1,\ldots,r\}, j\in\{1,\ldots, n_i\}\). Se le llaman
  divisores elementales de \(M\) y
  determinan \(M\) salvo isomorfismos.
\end{teo}
