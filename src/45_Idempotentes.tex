\begin{obs}
  \(R\), \(S\) anillos. Si es \(\{e_1,\ldots, e_t\}\) CCIO centrales
  indescomponibles y \(phi:R\longrightarrow S\) es un isomorfismo de anillos.
  Entonces \(\{\phi(e_1),\ldots,\phi(e_t)\}\) es el CCIO centrales
  indescomponibles de \(S\). Además se tiene
  \[
    R=Re_1\dot{+}\cdots\dot{+}Re_t
  \]
  entonces
  \[
    S=S\phi(e_1)\dot{+}\cdots\dot{+}S\phi(e_t)
  \]
  donde \(Re_i\cong S\phi(e_i)\) como anillos.
\end{obs}

Imaginemos que sabemos que \(R\) es semisimple y que disponemos de un
isomorfismo de anillos \(R\cong R_1\times\cdots\times R_s\) con \(R_i\)
indescomponibles.

Por otra parte, \(\Omega_R=\{\Sigma_1,\ldots,\Sigma_t\}\), tengo un isomorfismo
de anillos \(R\cong
\End_{D_1}(\Sigma_1)\times\cdots\times\End_{D_t}(\Sigma_t)\), donde
\(D_i=\End_R(\Sigma_i)\).

Se deduce de la observación:
\[
\End_{D_1}(\Sigma_1)\times\cdots\times\End_{D_t}(\Sigma_t)
\cong
R_1\times\cdots\times R_s
\]
sin más que componer isomorfismos. En primer lugar, \(s=t\) y tras
reordenación \(\End_{D_i}(\Sigma_i)\cong R_i\).
Conocemos los \(e_i\) CCIO centrales
del segundo factor.

\begin{teo}[Teorema de Artin-Wedderburn]
  Si \(R\cong\End_{D_1}(\Sigma_1)\times\cdots\End_{D_t}(\Sigma_t)
  \cong\End_{E_1}(T_1)\times\cdots\End_{E_s}(T_s)\)
  donde \(D_i\), \(E_i\) son todos anillos de división y \(\Sigma_i\),
  \(T_i\) de dimensión finita como espacios vectoriales.

  Entonces \(s=t\) y tras reordenación \(D_i\cong E_i\) y
  \(\dim_{D_i}(\Sigma_i)=\dim_{T_i}(T_i)\).
\end{teo}

La demostración consiste en juntar observaciones, comentarios
y teoremas anteriores.

