\begin{obs}
  Sea \(f\in\End_R(M)\). Entonces:
  \[
    f(\Soc_\Sigma(M))=f\left(\sum_{S\cong\Sigma, S\in\mathcal{L}(M)} S\right)
    =\sum_{S\cong\Sigma, S\in\mathcal{L}(M)}f(S)\subseteq\Soc_\Sigma(M)
  \]

  Tomando \(M=R\) y \(f=\rho_r\) para \(\rho_r:R\longrightarrow R\) definida
  por \(\rho_r(r')=r'r\), entonces \(\rho_r(\Soc_\Sigma(R))\subseteq
  \Soc_\Sigma(R)\) y tenemos que \(\Soc_\Sigma(R)\) es un ideal de \(R\).
\end{obs}

\begin{obs}
  \(\Omega_\Z\) es biyectivo con
  \(\{\Z_p:p\textrm{ es primo}\}\), luego es un conjunto
  infinito.
\end{obs}

\begin{teo}[Estructura de anillos semisimples]
  Sea \(R\) un anillo semisimple. Entonces \(\Omega_R\) es finito.
  Si ponemos \(\Omega_R=\{\Sigma_1,\ldots,\Sigma_t\}\) y
  \(D_i=\End_R(\Sigma_i)\), entonces
  \[
    R\cong\End_{D_i}(\Sigma_1)\times \cdots\times\End_{D_t}(\Sigma_t)
  \]
  y \(\dim_{D_i}(\Sigma_i)\) es finita.
\end{teo}
\begin{proof}
  Sé que \(\subscriptbefore{R}{R}=S_1\dot{+}\cdots\dot{+} S_n\) donde
  \(\subscriptbefore{R}{S_i}\) es simple. Así, si
  \(\subscriptbefore{R}{\Sigma}\) es simple, entonces
  \(\R\overset{p}{\longrightarrow}\Sigma\) epimorfismo, donde
  \(p|_{S_i}\) es un isomorfismo para algún \(i\) y por Schur,
  \(S_i\cong\Sigma\).
  Así que \(\Omega_R\) es finito.

  Tenemos que \(\Soc_{\Sigma_i}(R)\Soc_{\Sigma_j}(R)\subseteq
  \Soc_{\Sigma_i}(R)\cap\Soc_{\Sigma_j}(R)=\{0\}\). Eso implica
  que
  \(\Soc_{\Sigma_j}(R)\subseteq\Ann_R(\Soc_{\Sigma_j}(R))=\Ann_R(\Sigma_i)\).

  Llamo a \(I_i=\sum_{j\neq i}\Soc_{\Sigma_j}(R)\). Tenemos que
  \(I_i+I_j=R\) si \(i\neq j\). De la inclusión anterior se deduce
  \(\Ann_R(\Sigma_i)+\Ann_R(\Sigma_j)=R\) si \(i\neq j\).
  Se cumple que:
  \[
    R\longrightarrow R/\Ann_R(\Sigma_1)\times\cdots
    \times R/\Ann_R(\Sigma_t)
  \]
  tal que
  \[
    r\mapsto (r+\Ann_R(\Sigma_1),\ldots,r+\Ann_R(\Sigma_t))
  \]
  es un homomorfismo de anillos cuyo núcleo es
  \(\bigcap_{i=1}^t \Ann_R(\Sigma_i)=\bigcap_{i=1}^n \Ann_R(S_i)=\{0\}\).
  donde simplemente puede haber algún \(\Ann_R(S_i)\) repetido.

  Cada \(R/\Ann_R(\Sigma_i)\) es artiniano (de longitud finita por ser
  cociente de uno de longitud finita). \(\Sigma_i\) es un
  \(R/\Ann:R(\Sigma_i)\)-módulo simple fiel. Nuestro teorema nos garantiza
  que \(R/\Ann_R(\Sigma_i)\cong\End_{D_i}(\Sigma_i)\) para
  \(D_i=\End_{R/\Ann_R(\Sigma_i)}(\Sigma_i)\cong\End_R(\Sigma_i)\) y
  \(\dim_{D_i}(\Sigma_i)\) es finita.

\end{proof}

Ejemplo: \(R, S\) dos anillos. Sea \(T=R\times S\). Vamos a definir
\(e=(1,0)\), \(\mathcal{L}(\subscriptbefore{T}{Te})\longrightarrow
\mathcal{L}(\subscriptbefore{R}{R})\) dada por \(I\mapsto\pi(I)\) donde
\(\pi\) es la proyección en la primera componente, es una biyección que
preserva la inclusión.

Como consecuencia \(\subscriptbefore{T}{Te}\) es semisimple si y solo si
\(\subscriptbefore{R}{R}\).

Así, \(T\) es semisimple si y solo si \(R\) y \(S\) si y solo si \(T\)
son semisimples, ya que:
\[
  \subscriptbefore{T}{T}=Te\dot{+}T(1-e)
\]

Ejercicio: Sean \(D, E\) dos anillos de división, \(\subscriptbefore{D}{M}\),
\(\subscriptbefore{E}{N}\) espacios vectoriales. Se pide demostrar que
\[
  \End_D(M)\cong\End_E(N)\;\iff\;
  \left\{
  \begin{matrix}
    D\cong E\\
    \dim_D(M)=\dim_E(N)
  \end{matrix}
  \right.
\]
