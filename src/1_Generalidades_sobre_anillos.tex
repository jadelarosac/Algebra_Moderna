\section{Introducción}
Antes de comenzar con el contenido propio de la asignatura, debemos recordar
ciertos conceptos y resultados relacionados con la estructura de anillo,
impartidos en asignaturas básicas de álgebra.
\subsection{Generalidades sobre anillos}
\begin{df}[Anillo]
  Sea \(A\) un conjunto en el que existen dos operaciones
  \(+,\cdot:A\times A\longrightarrow A\) tales que:
  \begin{enumerate}
    \item \((A, +,0)\) es un grupo aditivo (conmutativo):
      \begin{itemize}
        \item \((a+b)+c=a+(b+c)\) para todos \(a,b,c\in A\).
        \item \(a+b=b+a\) para todos \(a,b\in A\).
        \item \(a+0=a\) para todo \(a\in A\).
        \item Para todo \(a\in A\) existe un \(-a\in A\)
          tal que \(-a+a=0\).
      \end{itemize}
    \item \((A, \cdot, 1)\) es un monoide:
      \begin{itemize}
        \item \((ab)c=a(bc)\) para todos \(a,b,c\in A\).
        \item \(a\cdot 1=1\cdot a=a\) para todo \(a\in A\).
      \end{itemize}
    \item Se cumplen las siguientes propiedades distributivas:
      \begin{itemize}
        \item \((a+b)c=ac+bc\) para todos \(a,b,c\in A\).
        \item \(a(b+c)=ab+ac\) para todos \(a,b,c\in A\).
      \end{itemize}
     \end{enumerate}
  \end{df}

  \begin{df}[Ideales]
    Sea \(A\) un anillo. Un subconjunto \(I\subset A\) se dice ideal de \(A\)
    si cumple las siguientes propiedades:
    \begin{itemize}
      \item \(I\) es un subgrupo aditivo de \(A\) (es decir,
        \(I\) es un conjunto no vacío que cumple
        \(a + b\in I\) para todo \(a, b\in I\)).
      \item \(ax, xa\in I\) para todo \(a\in I\) y \(x \in A\).
    \end{itemize}
  \end{df}

  \begin{obs}
      La primera condición para que un subconjunto de A sea un ideal suyo
      es equivalente a que \(b - a \in I\) para todo \(a, b \in I\). 
      Recordemos que, dado cualquier anillo \(A\), se verifica que \(0a = (0+0)a
      = 0a + 0a\), luego \(0a = 0\) para todo \(a \in A\).
      
      Esta propiedad, aunque evidentemente intuitiva, no viene explícita
      en la definición de anillo. Ahora bien, comprobemos que la propiedad
      de ideal anteriormente descrita se cumple.
      
      Sean \(a, b \in I\). Entonces \(b - a = b + (-1)a \in I\), pues \((-1) \in A\)
      y \(a \in I\). 
   \end{obs}

   \begin{df}[Homomorfismo de anillos]\label{df:homo_anillos}
     \(A,B\) anillos. Se dice que \(f:A\longrightarrow B\) se dice un
     (homo)morfismo de anillos si para todos \(a,a'\in A\) se tiene:
     \begin{enumerate}
     \item \(f(a+a')=f(a)+f(a')\)
     \item \(f(aa')=f(a)f(a')\)
     \item \(f(1)=1\)
     \end{enumerate}
   \end{df} 

  
  \begin{teo}[Teorema de Isomorfía]
    Sea \(f:A\longrightarrow B\) un homomorfismo de anillos. Entonces:
    \begin{itemize}
      \item \(\ker f\) es un ideal de \(A\),
      \item \(\Im f\) es un subanillo de \(B\),
      \item Si \(I\subset \ker f\) es un ideal de \(A\), entonces
        existe un único homomorfismo de anillos
        \(\tilde{f}:A/I\longrightarrow B\) definido por \(\tilde{f}(a+I)=f(a)\).
      \item El homomorfismo \(\tilde{f}\) es inyectivo si, y solo si,
        \(I=\ker f\).
      \item El homomorfismo \(\tilde{f}\) es sobreyectivo si, y solo si, lo es
        \(f\).
    \end{itemize}
  \end{teo}