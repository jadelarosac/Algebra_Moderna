\section{Teoría de módulos}

Sea \(R\) un anillo, \(\subscriptbefore{R}{M}\) un módulo. Sea la
familia no vacía de submódulos
\(\Gamma\subseteq\mathcal{L}(M)\) entonces
\(\bigcap_{N\in\Gamma} N\in\mathcal{L}(M)\).

\begin{df}[Submódulo generado por un conjunto \(X\)]
  Si \(X\) es un subconjunto de \(M\), el menor submódulo de \(M\) que contiene
  a \(X\) se llama submódulo generado por \(X\). Lo denotaremos por \(RX\).
\end{df}

\begin{lema}
  \[
    RX=\left\{\sum_{x\in F}v_x x:F\subseteq X \textrm{ finito}, v_x\in
    R\right\}
  \]
\end{lema}
\begin{proof}
  \(X\subseteq RX\) por ser el menor submódulo que contiene a \(X\).

  \[
    C=\left\{\sum_{x\in F}v_x x:F\subseteq X \textrm{ finito}, v_x\in
    R\right\}
  \]

  Entonces \(C\subseteq RX\).
  Tenemos que, como \(C\) es un submódulo, se tiene que dar la igualdad.

\end{proof}

Si \(X=\{x_1,\ldots, x_n\}\), tenemos que \(RX=Rx_1+\cdots Rx_n\).

\begin{df}[Módulo producto]
  Tomamos \(I\neq\emptyset\) un conjunto de índices, tal que
  \(i\in I\), tomamos un módulo \(M_i\).
  \[
    \prod_{i\in I} M_i=\{{(m_i)}_{i\in I}:m_i\in M_i\}
  \]

  Son tuplas, pero no ordenadas.
\end{df}

\begin{prop}
  El producto de módulos es un módulo, con la suma término a término y el
  producto por escalares también término a término.
\end{prop}

\begin{df}[Proyecciones e inclusiones canónicas]
  Vamos a tomar \(M_i\)y \( \prod_{i\in I} M_i\). Definimos
  la inclusión canónica \(\iota_i\)  mediante la
  aplicación que asigna \(m_i\mapsto {(a_j)}_{j\in I}\) dado por
  \(a_j=\delta_i^j m_i\).
  Del mismo modo, definimos la proyección canónica \(\pi_i\) como
  la aplicación que asigna \({(a_j)}_{j\in I}\mapsto a_i\).

  Evidentemente \(\pi_i\circ\iota_i=\id\).
\end{df}

\begin{df}[Suma directa externa]
  \[
    \bigoplus_{i\in I} M_i:=\{{(m_i)}_{i\in I}: \textrm{ tiene soporte
    finito}\}
  \]
\end{df}


En el caso de \(I\) finito \(\bigoplus_{i\in I} M_i=\prod_{i\in I} M_i\),
y en el caso general \(\bigoplus_{i\in I} M_i\subseteq\prod_{i\in I} M_i\)

\begin{df}[Suma de módulos]
  Definimos \(\sum_{i\in I} M_i\) como el menor submódulo que contiene
  a cualquier \(M_i\) o equivalentemente:
  \[
    \sum_{i\in I} M_i=\left\{\sum_{i\in F} m_i: F\subseteq I \textrm{ finito}\right\}
  \]
\end{df}

\begin{prop}[Relación entre sumas]
  Tomamos \(\theta:\bigoplus M_i\longrightarrow \sum M_i\) tal que
  \(\theta({(m_i)}_{i\in I})=\sum_{i\in I} m_i\) es un homomorfismo
  sobreyectivo de \(R\)-módulos.

  Para \(\{N_i:i\in I\}\subseteq\mathcal{L}(M)\), son equivalentes:
  \begin{enumerate}
    \item Para todo \(j\in I\), \(N_j\cap\sum_{i\in I\setminus\{j\}} N_i
      =\{0\}\).
    \item Para todo \(F\subseteq I\) finito, y para todo \(j\in F\),
      \(N_j\cap\sum_{i\in F\setminus\{j\}} N_i
      =\{0\}\).
    \item Si \(0=\sum_{i\in I} m_i\) con \(m_i\in M_i\) para todo
      \(i\in I\), entonces \(m_i=0\) para todo \(i\in I\).
    \item \(\theta\) es inyectivo y por tanto un isomorfismo.
    \item Para cada par \(J_1,J_2\subseteq I\) con
      \(J_1\cap J_2=\emptyset\), se tiene que
      \(\left(\sum_{i\in J_1} N_i\right)\cap
      \left(\sum_{i\in J_2} N_i\right)=\{0\}\)
  \end{enumerate}
\end{prop}

\begin{df}
  En caso de satisfacerse cualquiera de las condiciones anteriores
  equivalentes, diremos que la suma \(\sum_{i\in I} N_i\) es una suma
  directa interna, que notaremos por \(\dot{+}_{i\in I} N_i\).
\end{df}

\begin{cor}
  Si la familia \(\{N_i:i\in I\}\subseteq\mathcal{L}(M)\) verifican
  las condiciones y \(N\in\mathcal{L}(M)\) tal que
  \(N\cap\dot{+}_{i\in I} N_i=\{0\}\), entonces \(\{N_i:i\in I\}\cup\{N\}\).
\end{cor}

\begin{df}[Independencia]
  Si la familia \(\{N_i:i\in I\}\) donde cada módulo es distinto de 0 y
  satisface alguna de las condiciones anteriores equivalente, entonces
  diremos que dicha familia es independiente.
\end{df}

Caso particular: El módulo regular \(M_i=R\), llamamos:
\[
  R^{(I)}=\bigoplus_{i\in I} M_i=\{
    {(r_i)}_{i\in I}\in R^I:\textrm{ con soporte finito}\}
\]

