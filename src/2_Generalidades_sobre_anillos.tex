\begin{df}[Homomorfismo de anillos]
  \(A,B\) anillos. Se dice que \(f:A\longrightarrow B\) se dice un
  (homo)morfismo de anillos si para todos \(a,a'\in A\) se tiene:
  \begin{enumerate}
    \item \(f(a+a')=f(a)+f(a')\)
    \item \(f(aa')=f(a)f(a')\)
    \item \(f(1)=1\)
  \end{enumerate}
\end{df}

\begin{df}[Producto de ideales]
  Sean \(I, J\) ideales. Definimos su producto por:
  \[
    IJ=\{\sum_i x_i y_i: x_i\in I, y_i\in J\}\subseteq I\cap J
  \]
\end{df}


La suma de ideales es un ideal.

\begin{df}[Ideales coprimos]
  Dos ideales \(I, J\subset A\) se dirán primos entre sí
  o coprimos si \(I+J=A\).

  Equivalentemente, existen \(x\in I\), \(y\in J\) tales que
  \(1=x+y\).
\end{df}

La motivación de la definición anterior reside en la identidad
de Bezout, que estamos generalizando.

\begin{lema}
  Sean \(I, J, K\) ideales de \(A\),
  \(I+J=I+K=A\) si y solo si \(I+(J\cap K)=I+J\cap K=A\).

  Es decir, son coprimos entre sí si y solo si uno es coprimo
  con la intersección de los otros dos.
\end{lema}
\begin{proof}
  \[1=x+y=x'+z\]
  con \(x,x'\in I\), \(y\in J\), \(z\in K\).
  \[1=x+y=x+y1=x+y(x'+z)=x+yx'+yz\]
  \(x+yx'\in I\), y \(yz\in J\cap K\).

  Para el recíproco, \(A\supseteq I+J\supseteq I+J\cap K=A\),
  luego \(A=I+J\).
\end{proof}

\begin{lema}
  Sean \(I_1,\ldots, I_t\) ideales de \(A\).
  \(I_1\cap I_i=A\) si y solo si
  \(I_1+\bigcap_{i=2}^t I_i=A\).
\end{lema}
\begin{proof}
  Para \(t=2\) es trivial.

  Supongamos cierto \(I_1\cap I_i=A\implies
  I_1+\bigcap_{i=2}^t I_i=A\) para \(t\), veamos para \(t+1\).

  Llamo \(I=I\), \(J=\bigcap_{i=2}^t I_i\), \(K=I_{t+1}\).
  Por hipótesis de inducción \(I+J=A\) y \(I+K=A\) por ser coprimos
  (hipótesis del lema). Por el lema anterior tenemos:
  \[
    I+J\cap K=I_1+I_{t+1}\cap \bigcap_{i=2}^{t}
    I_i =I_1 + \bigcap_{i=2}^{t+1} I_i
  \]

  La otra implicación es muy sencilla.
\end{proof}

  Hipótesis de trabajo para el teorema chino del resto:
  \begin{enumerate}
    \item \(A\) un anillo.
    \item \(A_1,\ldots,A_t\) anillos.
    \item \(f_i:A\longrightarrow A_i\) un homomorfismo de anillos
      para cada \(i\in\{1,\ldots,t\}\).
    \item \(\Im f_i\subseteq A_i\) es un subanillo.
    \item A \(\Im f_1\times\cdots\times\Im f_t\) se le llama el anillo
      producto.
    \item Definimos \(f:A\longrightarrow\Im f_1\times\cdots\times\Im f_t\),
      \(f(x)=(f_1(x),\ldots,f_t(x))\) para cada \(x\in A\).
    \item Tenemos que \(f\) es un homomorfismo de anillos, cuyo núcleo es la
      intersección de todos los núcleos. Llamaremos \(I=\ker f\).
      \(x\in A\), \(x\in\ker f\) si y solo si \(f_i(x)=0\) para todo \(i\),
      es decir, \(x\in\bigcap_{i=0}^t \ker f_i\).
    \item  Además, existe \(\tilde{f}:A/I
      \longrightarrow\Im f_1\times\cdots\times\Im f_t\),
      con \(x+I\mapsto f(x)\).
    \item Cada \(\ker f_i\) es coprimo con cualquier \(\ker f_j\)
      para \(j\neq i\).
    \item Llamamos \(I_i=\ker f_i\).
  \end{enumerate}

\begin{teo}[Teorema Chino del Resto]
  \(\tilde{f}\) es isomorfismo si y solo si
  \(I_i+I_j=A\) para todo \(i\neq j\).
\end{teo}
\begin{proof}
  Veamos primero la implicación a la derecha.

  Vamos a suponer \(\tilde{f}\) sobreyectiva, es decir, que \(f\) lo es.
  Veamos que todos los \(I_i\) son coprimos entre sí.

  Dado \(i\) tomamos \(x\in A\) tal que \(f_i(x)=1\) y
  \(f_j(x)=0\) para todo \(j\neq i\).

  Observemos que \(x-1\in I_i\), \(x\in\bigcap_{j\neq i} I_j\)
  \[
    1=1-x+x\in I_i+\bigcap_{j\neq i} I_j
  \]

  Por tanto, \(I_i+\bigcap I_j =A\) y entonces por el lema anterior
  \(I_i+I_j=A\).

  Veamos el recíproco.
  Suponemos que \(I_i+I_j=A\) para cualquier \(i\neq j\).

  Tomamos \((f(b_1),\ldots,f(b_t))\in I_1\times\cdots\times I_t\).

  Para cada \(i\), tomamos
  \(
  1=a_i+p_i\) con \(a_i\in I_i\) y \(p_i\in\bigcap I_j\).

  Tomamos \(x=\sum_{i=1}^t b_i p_i\).

  \[
    f_j(x)=\sum_{k=1}^t f_j(b_k) f_j(p_k)=f_j(b_j)f_j(p_j)=f_j(b_j(1-a_j))
    =f_j(b_j)-f_j(b_j)f_j(a_j)=f_j(b_j)
  \]
  porque \(f_j(p_k)=0\) si \(k\neq j\) y \(a_j\in\ker f_j\).

\end{proof}
