\begin{prop}
  Si \(\subscriptbefore{R}{M}\) tiene una base \(\{v_1,\ldots,v_n\}\),
  puedo definir \(\{\superscriptbefore{*}{v_1},\ldots,
  \superscriptbefore{*}{v_n}\}\), definidos mediante \(v_i(
  \superscriptbefore{*}{v_j})=\delta_{ij}\). Los \(\subscriptbefore{*}{v_i}\)
  forman una base de \(\superscriptbefore{*}{M}_R\). Además, \(\theta\) es un
  isomorfismo de anillos.
\end{prop}
\begin{proof}
  Veamos que es una base. Observemos que para cualquier \(m\in M\):
  \[
    m=\sum_{i=1}^n \superscriptbefore{*}{v_i}(m)v_i
  \]

  Sea \(\varphi\in\superscriptbefore{*}{M}\):
  \[
    \varphi(m)=\sum_{i=1}^n \superscriptbefore{*}{v_i}(m)\varphi(v_i)
    =\left(\sum_{i=1}^n \varphi(v_i)\superscriptbefore{*}{v_i}\right)(m)
  \]
  luego \(\varphi=\sum_{i=1}^n \varphi(v_i)\superscriptbefore{*}{v_i}\).
  Con lo que los \(\superscriptbefore{*}{v_i}\) generan
  \(\superscriptbefore{*}{\subscriptbefore{R^{op}}{M}}\).

  Si
  \[
    0=\sum_i r_i\superscriptbefore{*}{v_i}
  \]
  entonces:
  \[
    0=\sum_i (r_i\superscriptbefore{*}{v_i})(v_j)=r_j
  \]

  \({\End_R(M)}^{op}\overset{\theta}{\longrightarrow}\End_{R^{op}}
  (\superscriptbefore{*}{M})\) dado por \(\theta(f)(\varphi):=\varphi\circ f\).

  Veamos que \(\theta\) es inyectivo: tomo \(f\) tal que \(\theta(f)=0\).
  Dado \(m\in M\) tenemos:
  \[
    f(m)=\sum_i \superscriptbefore{*}{v_i}(f(m))v_i
    =\sum_i \theta(f)(\superscriptbefore{*}{v_i})(m)v_i=0
  \]
  luego es inyectivo. Veamos que es sobreinyectivo.

  Dado \(\psi\in\End_{R^{op}}(\superscriptbefore{*}{M})\) definimos
  \(f:M\longrightarrow M\) por:
  \[
    f(m)=\sum_i \psi(\superscriptbefore{*}{v_i})(m)v_i
  \]
  Es fácil ver que \(f\in\End_R(M)\cong{\End_R(M)}^{op}\).
  \begin{eqnarray*}
    \theta(f)(\varphi)(m)&=&(\varphi\circ f)(m)=\sum_i
    \psi(\superscriptbefore{*}{v_i})(m)\varphi(v_i)\\
    &=&\left(\sum_i \varphi(v_i)\psi(\superscriptbefore{*}{v_i})\right)(m)\\
    &=&\psi\left(\sum_i \varphi(v_i)\superscriptbefore{*}{v_i}\right)(m)\\
    &=&\psi(\varphi)(m)
  \end{eqnarray*}
  Por lo tanto, \(\theta(f)(\varphi)=\psi(\varphi)\), y se sigue
  \(\varphi=\theta(f)\), con lo que \(\theta\) es sobreyectivo.

\end{proof}

\begin{df}
  Si \(\subscriptbefore{R}{R}\) es semisimple, entonces
  \(R\cong\End_{D_1}(\Sigma_1)\times\cdots\times\End_{D_t}(\Sigma_t)\)
  donde \(D_i\) de dimensión y \(n_i=\dim_{D_i}(\Sigma_i)z\infty\) con
  \(D_i\) únicos salvo isomorfismo y reordenación y \(n_i\) únicos.
  Diremos que \(R\) es de tipo \((D_1,\ldots,D_t,n_1,\ldots,n_t)\).
\end{df}

\begin{teo}
  Si \({R}\) es semisimple de tipo
  \((D_1,\ldots,D_t,n_1,\ldots,n_t)\), entonces \(R^{op}\) es semisimple
  de tipo \((D_1^{op},\ldots,D_t^{op},n_1,\ldots,n_t)\).
\end{teo}
\begin{proof}
  Tenemos que \(R^{op}\cong{\End_{D_1}(\Sigma_1)}^{op}\times\cdots\times
{\End_{D_t}(\Sigma_t)}^{op}\cong
  {\End_{D_1^{op}}(\superscriptbefore{*}{\Sigma_1})}\times\cdots\times
  {\End_{D_t^{op}}(\superscriptbefore{*}{\Sigma_t})}\)
  y \(\dim_{D_i}(\Sigma_i)=\dim_{D_i^{op}}(\superscriptbefore{*}{\Sigma_i})\).
  \(R^{op}\) es semisimple con la estructura del enunciado.

\end{proof}


\begin{cor}
  \(R\) es semisimple si y solo si \(R^{op}\) es semisimple.
\end{cor}

Ejemplo: \((\C,\R,1,2)\). Tenemos que
\(R\cong\C\times\mathcal{M}_{2\times 2}(\R)\).

Ejemplo: \((\mathbb{H},2)\). Sea \(\subscriptbefore{\mathbb{H}}{V}\) un
espacio de dimensión 2. Cuidado que \(\End_{\mathbb{H}}(V)\cong
{\mathcal{M}_{2\times 2}(\mathbb{H}^{op})}^{op}\). Se puede demostrar
que \(\mathbb{H^{op}}\cong\mathbb{H}\), pero no es automático, con lo que
añadiendo la transposición \(\End_{\mathbb{H}}(V)\cong
{\mathcal{M}_{2\times 2}(\mathbb{H})}\).

