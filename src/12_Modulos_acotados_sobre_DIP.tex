Ejemplo: \(T\) endomorfismo \(K\)-lineal. \(V=\subscriptbefore{K[x]}{V}\).

Un \(W\) es un submódulo de \(V\) es un subespacio vectorial tal que
\(T(W)\subseteq W\), es decir, \(W\) es \(T\) invariante.

Si \(\Ann_{K[x]}(V)\neq\{0\}\), tomo \(\mu(x)\in K[x]\), el polinomio
mínimo de \(T\). Es decir, \(\Ann_{K[x]}(V)=\langle\mu(x)\rangle\).

\[
  \mu=p_1^{e_1}\cdots p_t^{e_t}
\]

Entonces la descomposición primaria de \(V\) es \(V=V_1\dot{+}
\cdots\dot{+}V_t\) con
\[
  V_i=\{v\in V:p_i(x)v=0\}
\]

Caso particular: \(\dim(V)<\infty\) y que \(\mu(x)=(x-\alpha_1)\cdots
(x-\alpha_t)\) con \(\alpha_i\neq\alpha_j\).
\[
  V_i=\{v\in V:(x-\alpha_i)v=\{v\in V:T(v)=\alpha_i v\}
\]
es decir, el subespacio propio asociado al valor propio \(\alpha_i\).

Si el polinomio factoriza como producto de polinomios de grado 1 distintos,
\(T\) es diagonalizable.
Veremos en el futuro que el polinomio mínimo divide siempre al polinomio
característico.

¿Cómo se calcula el polinomio mínimo de un endomorfismo lineal?

Ejercicio: Sea \(V\) un espacio vectorial real euclídeo (con producto
escalar). Sea \(T:V\longrightarrow V\) una isometría.
Se pide demostrar que si \(W\) es un subespacio \(T\) invariante de \(V\),
entonces su ortogonal \(W^\perp\) es también \(T\) invariante.
Entonces \(V=W\dot{+}W^\perp\). Se usa inducción. Como consecuencia, usando
el teorema fundamental del álgebra, deducir que \(V\) admite una base
ortonormal con respecto de la cual la matriz de \(T\) es diagonal por
bloques, con bloques de dimensión 1 o 2. ¿Qué aspecto tienen dichos
bloques? Hay que ver que uno de los dos subespacios invariantes tienen
dimensión 1 o 2.

