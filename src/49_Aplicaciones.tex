Sea \(V\) un \(\C G\)-módulo. Como \(\C\subseteq\C G\) por restricción de
escalares, \(\subscriptbefore{\C}{V}\) es un espacio vectorial.
\(\rho:\C G\longrightarrow\End_\C(V)\), \(\rho\) de anillos y \(\C\)-lineal.
\[
  \rho(\sum_{g\in G} r_g g)(v)=\sum_{g\in G} r_g g v
\]
¿Qué pasa si restringimos \(\rho\) a \(G\)? Como respeta el producto en \(G\),
\(\rho|_G:G\longrightarrow GL_\C(V)\), donde \(\rho|_G\) es un homomorfismo
de grupos.
Es una represantación lineal de \(G\) con espacio de representación \(V\).

Si \(W\subseteq V\), es un \(\C G\)-submódulo si y solo si es un
\(\C\)-subespacio vectorial y \(W\) es \(G\) invariante:
para todo \(w\in W\) y todo \(g\in G\) se tiene que \(gw\in W\).

\(\mu(G)\) es el espacio de representación
\(\rho:G\longrightarrow GL_\C(\mu(G))\) dado por:
\[
  \rho(g)(\varphi)(x)=\varphi(xg)=:g\varphi(x)
\]
donde \(g,x\in G\) y \(\varphi\in\mu(G)\).

\begin{teo}
  Si \(G\) es finito entonces \(\C G\) es semisimple.
\end{teo}
\begin{proof}
  Supongamos que \(G\) es finito. Tomamos \(V\) un \(\C G\)-módulo de dimensión
  finita. Tomamos \(\langle\cdot|\cdot\rangle\) un producto interno en \(V\).

  Definimos \(\langle\cdot|\cdot\rangle_G\) producto interno sobre \(V\) así:
  \[
    \langle v|u\rangle_G=\sum_{g\in G} \langle gv| gu\rangle
  \]
  que cumple la siguiente propiedad para todo \(h\in G\):
  \[
    \langle hv|hu\rangle_G=\sum_{g\in G} \langle ghv| ghu\rangle
    =\sum_{f\in G} \langle fv| fu\rangle=\langle v|u\rangle_G
  \]
  con lo que es un operador unitario (es un operador que conserva el producto
  interno de un espacio de Hilbert) y una isometría.


  Sea \(W\) un \(\C G\)-submódulo de \(V\). Se tiene:
  \[
    V=W\dot{+} W^\perp
  \]
  donde \(\perp\) se toma respecto al producto interno nuevo:
  \[
    W^\perp :=\{v\in V: \langle v|w\rangle_G=0\quad\forall w\in W\}
  \]

  O sea \(W^\perp\) es \(G\)-invariante. En otras palabras, hemos de ver que
  si \(v\in W^\perp\), \(g\in G\) entonces \(gv\in\perp\), entonces para todo
  \(w\in W\) tenemos que:
  \[
    \langle gv|w\rangle_G=\langle gv|gg^{-1}w\rangle_G=
    \langle v|g^{-1} w\rangle=0
  \]
  ya que \(g^{-1}w\in W\) y \(v\in W^\perp\). Luego \(W^\perp\) es
  \(G\)-invariante.

  Como hemos demostrado que cualquier submódulo es sumando directo, tenemos
  que es semisimple.

\end{proof}

\begin{cor}
  Si \(G\) es finito, \(\mu(G)\) es un \(\C G\)-módulo semisimple.
\end{cor}

Dotamos de a \(\mu(G)\) del producto interno:
\[
  \langle \varphi|\psi\rangle:=
  \frac{1}{|G|}\sum_{g\in G}\varphi(g)\overline{\psi(g)}
\]




