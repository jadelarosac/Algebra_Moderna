\subsection{Homomorfismos y cocientes de módulos}

\begin{prop}[Módulo cociente o factor]
  Sea \(\subscriptbefore{A}{M}\) un \(A\)-módulo y \(L\in\mathcal{L}(M)\) un submódulo
  de \({}_AM\). Se considera \(M/L\) grupo aditivo cociente con la suma
  \(m + L) + (m' + L) = (m + m') + L\). Se define la acción:
  \[
    a(m+L):=am+L
  \]
  Entonces \(M/L\) es un \(A\)-módulo con la acción anterior, llamado
  \textbf{módulo cociente}.
\end{prop}

Además, la proyección canónica \(\pi: M \longrightarrow \frac{M}{L}\) dada por
\(\pi(m) = m + L\) para todo \(m \in M\) es un homomorfismo de \(A\)-módulos en el
sentido siguiente.

\begin{df}[Homomorfismo de módulos]
  Se dice que
  \(f:\subscriptbefore{A}{M}\longrightarrow \subscriptbefore{A}{N}\)
  es un homomorfismo de \(A\)-módulos si \(f(m + m') = f(m) + f(m')\) y \(f(am) = af(m)\)
  para todo \(m, m' \in M\) y \(a \in A\).
\end{df}

\begin{nt}
  La familia \(\mathcal{L}({}_AM)\) es el retículo de submódulos de \(M\).
\end{nt}

\begin{teo}[Primer teorema de isomorfía para módulos]\label{teo:iso_modulos}
  Sea \(f : M \longrightarrow N\) un homorfismo de \(A\)-módulos. Entonces
  \begin{enumerate}
  \item El núcleo \(\ker f \in \mathcal{L}(\subscriptbefore{A}{M})\).
  \item La imagen \(\Im f \in \mathcal{L}(N)\).
  \item Para cada \(L \in \mathcal{L}(\subscriptbefore{A}{M})\) tal que
    \(L \subseteq \ker f\) existe un único homomorfismo de módulos
    \(\tilde{f} : M/L \longrightarrow N\)
    tal que \(\tilde{f}\circ\pi=f\). Finalmente, \(\tilde{f}\) es
    inyectiva si y solo si \(L=\ker f\), en cuyo caso, \(\tilde{f}\)
    da un isomorfismo de \(A\)-módulos
    \(M/\ker f\cong \Im f\).

  \item El isomorfismo de grupos aditivos \(\tilde{f}: \frac{M}{\ker f} \longrightarrow
    \Im f\) es de \(A\)-módulos.
  \end{enumerate}
\end{teo}
\begin{proof}
  \begin{itemize}
  \item Se emplea el teorema de isomorfía para grupos.
  \item
  \item
  \item \(\tilde{f}(a(m + \ker f)) = \tilde{f}(am + \ker f) = f(am) = af(m) =
    a\tilde{f}(m + \ker f)\).
  \end{itemize}
\end{proof}


\begin{ejemplo}
  Sea \(m \in {}_AM\) y \(f: A \longrightarrow M\) definida como \(f(a) = am\) para cada
  \(a \in A\). Entonces \(f\) es un homomorfismo de \(A\)-módulos.

  Tenemos que \(\Im f = Am\) y
  \(\ker f=\{ a \in A \ / \ am=0 \}\) es un ideal izquierda de \(A\), que se llama
  \textbf{anulador del elemento \(m\)} y denotado por \(\ann_A(m)\). Por el primer
  teorema de isomorfía, existe un isomorfismo de módulos \(\tilde{f}: \frac{A}
  {\ann_A(m)} \longrightarrow Am\) dado por \(\tilde{f}(a + \ann_A(m)) = am\). 
\end{ejemplo}

\begin{ejemplo}
  Sea el espacio vectorial \(\Cont^\infty(\R)\), la aplicación lineal
  \(T = \frac{d}{dt}\) y \(\sin t \in \Cont^\infty(\R)\). Existe un isomorfismo
  de \(\frac{\R[x]}{\ann_{\R[x]}(\sin t)}\) en \(\R[x]\sin t\), donde
  \(\ann_{\R[x]}(\sin t) = \langle x^2 + 1 \rangle\). Sea \(f\) dicho isomorfismo.

  Entonces \(f([1]) = \sin t\) y \(f([x]) = \cos t\).

  Dada \(\varphi\),
  tenemos que
  \[
    \ann_{\R[x]}(\varphi)=\{f\in\R[x]: f(x)\varphi=0\}
    =\{f=\sum_i f_i\frac{d^i}{dt^i}: f\varphi=0\}
  \]
  \(\ann(\varphi)\neq\langle 0\rangle\) si \(\varphi\) satisface una ecuación
  diferencial lineal homogénea con coeficientes constantes. Bla bla.

  \(\R[x]/\ann_{\R[x]}(\varphi)\cong \R[x]\varphi\), donde \(\varphi\)
  satisface bla bla.

  Tenemos que \(\varphi''-\varphi'-\varphi=0\), cuya solución
  \(\varphi(t)=e^{\phi t}\), donde \(\phi\) es la razón aúrea.
\end{ejemplo}
  
\begin{ejemplo}
  Sea \(S=\Map(\N,K)\) el conjunto de las sucesiones en un cuerpo \(K\) (que forman
  un \(K\)-espacio vectorial). Tomamos \(T: S \longrightarrow S\)
  tal que \(T(s)(n) = s(n+1)\), que es lineal. A este operador se le conoce como
  retraso de ley. Entonces
  \(S\) es un \(K[x]\)-módulo, donde \(xs=T(s)\).

  Para cualquier \(f \in K[x]\), es decir \(f = \sum_i f_i x^i\), se tiene:
  \[
    (fs)(n)=\sum_i f_i s(n+i)
  \]

  Imaginémosnos que \(s\) verifica que \(\ann_{K[x]}(s)\neq\langle0\rangle\).
  Podemos tomar entonces un polinomio tal que \(fs = 0\) y que sea mónico.
  Tenemos entonces que \(s(n+m)=-\sum_{i=0}^{m-1} f_i s(n+i)\)
  para todo \(n\in\N\). Es decir, la sucesión es linealmente recursiva.

  Veamos el caso particular en el que \(s(0)=s(1)=1\). Entonces
  \[
    s(n+2)=s(n)+s(n+1)
  \]

  \[
    x^2-x-1 \in \ann_{\Q[x]}(s)
  \]

  Volviendo al caso general, tenemos que
  \[
    K[x]/\ann_{K[x]}(s)\cong K[x]s
  \]

  Tenemos que \(\dim_{K}(K[x]s)<\infty\) si y solo si
  \(\ann_{K[x]}(s)\neq\langle0\rangle\) si y solo si
  \(s\) es una sucesión linealmente recursiva.

  El generador \(p(x)\) de \(\ann_{K[x]}(s)\) se le llama el polinomio
  mínimo de \(s\). El grado de dicho polinomio, coincide con
  \(\dim_{k}(K[x]s)\) y se le llama complejidad lineal de \(s\).

  \(s,t\) dos sucesiones linealmente recursivas.
  \(K[x](s+t)\subseteq K[x]s+K[x]t\), luego la primera tiene dimensión finita.
  Luego \(s+t\) es una sucesión linealmente recursiva, de complejidad menor
  o igual a la suma de las complejidades lineales.
  Puede argumentarse lo mismo para combinaciones lineales.

  Las sucesiones linealmente recursivas forman un subespacio vectorial
  del espacio de sucesiones. De hecho forman un submódulo. Sea
  \(S^l\) el conjunto de las sucesiones linealmente recursivas, forma
  un \(S^l\) es un \(K[x]\)-submódulo de \(S\), ya que es ivariante por la
  acción de \(x\) (es \(T\)-invariante).

  Sea \(K = \R\). Queremos una sucesión \(s \in S\) tal que \(\ann_{\R[x]}(s) =
  \langle x^2 + 1 \rangle\), es decir, \(x^2 + 1)s = 0 \in S\). Entonces
  \[
    (x^2 + 1)s(n) = x^2s(n) + 1s(n) = s(n+2) + s(n),
  \]
  luego hay que resolver la recurrencia \(s(n+2) + s(n) = 0\) para cada \(n \in \N\).

  De manera más abstracta, hemos de tener un isomorfismo de \(\R[x]-módulos\) para
  \(\frac{\R[x]}{\langle x^2 + 1 \rangle} \cong \R[x]s\) y tal que \(1 \mapsto s\) y
  \(x \mapsto xs\).

  Una propuesta de lo anterior es \(s(n) = \frac{d^n}{dt}|_{t=0} \sin t\),
  que es una solución de la recurrencia.
\end{ejemplo}

\begin{ejemplo}
  ¿Qué relación hay entre los \(\R[x]\)-módulos \(\Cont^\infty(\R)\) y \(\Map(\N, \R)\)?

  Sea \(\tau : \Cont^\infty(\R) \longrightarrow \Map(\N, \R)\) dada por
  \(\tau(\phi) = \phi^{(n)}(0)\) da una sucesión. Se trata de una aplicación lineal, pero
  además es un homomorfismo de módulos. Basta con comprobar que \(\tau(x\phi) =
  x\tau(\phi)\) para cualquier \(\phi \in \Cont^\infty(\R)\). Veámoslo:
  \[
    \tau(x\phi)(n) = \tau(\phi')(n) = \phi^{n+1}(0) = \tau(\phi)(n+1) = x\tau(\phi)(n)
  \]
  para cada \(n \in \N\). Entonces \(\tau(x\phi) = x\tau(\phi)\). Por tanto,
  \(\tau\) es un homomorfismo.

  Además, los anuladores en \(\R[x]\) de \(\phi\) y \(tau(\phi)\) están relacionados.
  Por ejemplo, si \(p(x) \in \ann_{\R[x]}(\phi)\), entonces \(p(x)\tau(\phi) =
  \tau(p(x)\phi) = \tau(0) = 0\). Por tanto, \(\ann_{\R[x]}(\phi) \subset
  \ann_{\R[x]}(\tau(\phi))\). Si \(\tau\) es inyectiva y \(p(x) \in
  \ann_{\R[x]}(\tau(\phi))\), entonces, por definición,
  \(\tau(p(x)\phi) = p(x)\tau(\phi) = 0\) y \(p(x)\phi = 0\), lo que implica que
  \(p(x)\) está en el anulador de \(\phi\).

  Sin embargo, en general, que \(\phi\) sea inyectiva no es cierto, ya que hay alguna
  función no nula cuya derivada es cero.

  Al restringir \(\tau\) a las funciones que admiten un desarrollo de Taylor, es decir,
  funciones analíticas complejas (\(\Cont^{\omega}(\R)\)) en la recta \(\R\). Si
  \(\tau\) fuese inyectiva, entonces se restringe a las analíticas en la recta. Por
  tanto, si \(\phi\) es analítica, entonces \(\ann_{\R[x]}(\phi) = \ann_{\R[x]}(\tau(\phi))
  \).

  Si \(\ann_{\R[x]}(\phi) \neq \langle 0 \rangle\), entonces \(\phi\) es solución de una
  ecuación diferencial ordinaria lineal con coeficientes constantes.

  Supongamos que \(\phi\) es solución de \(\phi'' - \phi' - \phi = \). Entonces
  \(\tau(\phi)\) es solución de \(\tau(\phi)(n+2) - \tau(\phi)(n+1) - \tau(\phi)(n) = 0\).

  ¿Cuántas soluciones tiene la primera ecuación? Forma un \(\R\)-espacio vectorial de
  dimensión 2. El polinomio \(p(x) = x^2 - x - 1\) genera el anulador de todas esas
  soluciones. Las raíces de \(p\) son \(\alpha = \frac{1 + \sqrt(5)}{2}\) y
  \(\beta = \frac{1 - \sqrt(5)}{2}\). Ahora buscamos un \(\phi_1\) tal que
  \((x-\alpha)\phi_1 = 0 \ \iff \ \phi_1' - \alpha\phi_1 = 0\). Por ejemplo,
  \(\phi_1(t) = c_1\exp{\alpha t}\), para algún \(c_1 \in \R\). Entonces \(\phi_1\)
  es también solución de la primera ecuación. Análogamente se probaría para
  \(\phi_2(t) = c_2\exp{\beta t}\), para algún \(c_2 \in \R\). Son soluciones
  linealmente independientes, por tanto, una base es \(\{\phi_1, \phi_2\}\).

  Si tomamos la ecuación discreta, obtendríamos de manera directa que la bse sería
  \(\{\tau(\phi_1), \tau(\phi_2)\}\).

  Por ejemplo, la solución de \(\tau(\phi)(0) = 0\) y \(\tau(\phi)(1) = 1\) se obtiene
  ajustando coeficientes \(a, b \in \R\) tales que \(\tau(\phi) = a\tau(\phi_1) +
  b \tau(\phi_2)\), es decir, \(tau(\phi) = \tau(a\phi_1 + b\phi_2)\). Hagámoslo:
  \begin{center}
    \begin{tabular}{c c c}
      \(\tau(\phi_1)(n) = \phi_1^{(n)}(0)\) & \(\tau(\phi_1)(0) = \phi_1^{(0)}(0) = 1\) & \(\tau(\phi_2)(0) = \phi_2^{(0)}(0) = 1\) \\
      \(\tau(\phi_2)(n) = \phi_2^{(n)}(0)\) & \(\tau(\phi_1)(1) = \phi_1^{(1)}(0) = \alpha\) & \(\tau(\phi_2)(1) = \phi_2^{(1)}(0) = \beta\) \\
    \end{tabular}
  \end{center}

  Ahora buscamos \(a, b \in \R\) tales que \(a + b = 0\) y \(a\alpha + b\beta = 1\).
  Nos sirven \(a = \frac{1}{\sqrt(5)}\) y \(b = -\frac{1}{\sqrt(5)}\).
  Por tanto, \(\tau(\phi)(n) = \frac{1}{\sqrt(5)}\alpha^n - \frac{1}{\sqrt(5)}\beta^n\).
\end{ejemplo}