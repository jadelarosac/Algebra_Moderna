\begin{prop}
  Sea \(R\) tal que \(\subscriptbefore{R}{R}\) es artiniano y
  \(\subscriptbefore{R}{M}\) un módulo simple. Si \(\subscriptbefore{R}{M}\)
  es fiel (\(\Ann_R(M)=\{0\}\)),
  entonces \(\lambda:R\longrightarrow\End_D(M)\) es un isomorfismo, donde
  \(D=\End_R(M)\).
  Además, \(\dim_D M<\infty\).
\end{prop}
\begin{proof}
  Supongamos que \(\subscriptbefore{D}{M}\) no fuera de dimensión finita.
  Entonces \(M\) admite una base \(B\) infinita. Tomamos \(\{x_i:i\in N\}
  \subseteq B\) linealmente independiente.

  Dado \(i\in \N\), tomamos \(f_i:M\longrightarrow M\) la aplicación
  \(D\)-lineal que vale 0 sobre todo elemento de \(B\) y sobre
  \(f_i(x_i)=x_i\).

  Cada \(f_i\in\End_D(M)\). El teorema de densidad nos permite asegurar
  que existe \(r_i\in R\) tal que \(f_i(x_j)=r_i x_j\) para
  \(j=0,\ldots, i\).

  \(r_i\in\ann_R(x_0)\cap\ldots\cap\ann_R(x_{i-1})\), pero el
  \(r_i\in\ann_R(x_0)\cap\ldots\cap\ann_R(x_{i-1})\cap\ann_R(x_i)\).
  Tenemos que \[\ann_R(x_0)\cap\ldots\cap\ann_R(x_{i-1})
  \supsetneq\ann_R(x_0)\cap\ldots\cap\ann_R(x_{i-1})\cap\ann_R(x_i)\]
  con \(i\ge 1\) y tenemos una cadena descendente y por tanto
  \(\subscriptbefore{R}{R}\) no es artiniano.

  Tomo \(\{m_1,\ldots,m_n\}\) una base de \(M\). Dado \(f\in\End_D(M)\),
  el teorema de densidad asegura que existe \(r\in R\), entonces
  \(f(m_i)=rm_i\) para todo \(i\in\{1,\ldots,n\}\). Basta tomar
  \(\lambda(r)=f\) y tenemos que es sobreyectivo. Como además \(M\)
  es fiel, \(\lambda\) es un isomorfismo.

\end{proof}

\begin{df}[Idempotentes]
  Un elemento \(e\in R\) se dice idempotente si \(e^2=e\).

  Un conjunto \(e_1,\ldots,e_n\in R\) de idempotentes se dice un conjunto
  completo de idempotentes ortogonales (CCIO) si:
  \[
    e_i e_j=0
  \] siempre que \(i\neq j\) y además:
  \[
    e_1+\cdots+e_n = 1
  \]
\end{df}
\begin{prop}
  Si \(\{e_1,\ldots,e_n\}\) es CCIO, entonces
  \(R=Re_1\dot{+}\cdots\dot{+}Re_n\).
\end{prop}
\begin{proof}
  Sea \(r\in R\), tenemos que \(r=r1=re_1+\cdots+ren\).

  Por otro lado, si \(0=r_1 e_1+\cdots+r_n e_n\) para \(r_i\in R\),
  entonces multiplicando la identidad por \(e_i\) nos queda
  \(0=r_i e_i^2=r_i e_i\) para cada \(i\in\{1,\ldots,n\}\).

\end{proof}

\begin{df}[Anillo simple]
  \(R\) es simple si y solo si los únicos ideales de \(R\) son \(\{0\}\)
  y \(R\).
\end{df}

\begin{teo}
  Son equivalentes, para un anillo no trivial:
  \begin{enumerate}
    \item \(\subscriptbefore{R}{R}\) semisimple y todos los \(R\)-módulos
      simples son isomorfos entre sí.
    \item \(R\) es isomorfo como anillo a \(\End_D(M)\) con \(D\) de división
      y \(\subscriptbefore{D}{M}\) es de dimensión finita.
    \item \(\subscriptbefore{R}{R}\) artiniano y existe un \(R\)-módulo simple
      y fiel.
    \item \(\subscriptbefore{R}{R}\) es artiniano y simple.
  \end{enumerate}

  Además, para la segunda afirmación se da necesariamente que
  \(D\cong\End_R(\Sigma)\) para cualquier \(\subscriptbefore{R}{\Sigma}\)
  el único sumódulo simple salvo isomorfismo dado en la primera afirmación.
  Por último, necesariamente, \(\dim_D(M)=\ell(\subscriptbefore{R}{R})\).
\end{teo}
\begin{proof}
  Veamos que la primera afirmación implica la cuarta.
  Sabemos que \(\subscriptbefore{R}{R}\) tiene longitud finita.
  Sea \(I\) un ideal de \(R\) propio (\(R\neq \{0\}\)).
  \(R/I\) es semisimple como \(R\)-módulo. Como es finitamente generado,
  es suma directa finita de simples. Todos esos submódulos son isomorfos
  entre sí.
  \[
    R/I\cong \Sigma^n
  \]
  donde \(\subscriptbefore{R}{\Sigma}\) es simple.

  Tenemos que \(I=\Ann_R(R/I)\) (aquí es donde hace falta que sea ideal y no
  solo ideal por la izquierda).
  \[
    I=\Ann_R(R/I)=\Ann_R(\Sigma^n)=\Ann_R(\Sigma)
  \]

  Por otro lado \(R\cong\Sigma^m\), con  \(m=\ell(\subscriptbefore{R}{R})\).
  \[
    I=\Ann_R(\Sigma)=\Ann_R(\Sigma^m)=\Ann_R(R)=\{0\}
  \]

  Demostremos ahora que la cuarta afirmación implica la tercera.
  Tomamos \(\subscriptbefore{R}{\Sigma}\) simple (existe tomando el primero
  de la serie de descomposición, por ser artiniano).
  \(R\neq\Ann_R(\Sigma)\) por se simple, luego \(\Ann_R(\Sigma)=\{0\}\).
  Luego \(\subscriptbefore{R}{\Sigma}\) fiel.

  La segunda afirmación se deduce de la tercera por la proposición anterior.

  Veamos finalmente que la cuarta afirmación implica la primera. Tomamos
  \(S=\End_D(M)\). Si \(m, m'\in M\) con \(m\neq 0\), existe entonces un
  \(f\in S\) tal que \(f(m)=m'\).

  Así, \(Sm=M\) con lo que \(\subscriptbefore{S}{M}\) es simple.
  Sea \(\{m_1,\ldots, m_n\}\) \(D\)-base de \(M\). Para \(i\in\{1,\ldots,n\}\)
  defino \(e_i\in S\) tal que
  \[
    e_i(m_j)=
    \left\{
    \begin{matrix}
      0&\textrm{si } j\neq i\\
      m_i&\textrm{si } j = i
    \end{matrix}
    \right.
  \]

  \(\{e_1,\ldots,e_n\}\) es CCIO de \(S\), entonces \(S=Se_1\dot{+}
  \cdots\dot{+}Se_n\).
  Veamos que \(Se_i\) es simple. Basta con demostrar que si \(f\in S\)
  tal que \(fe_i\neq 0\) entonces \(Sfe_i=Se_i\).
  \[
    fe_i=f(e_i)=\sum_{j=1}^n a_j m_j
  \]
  con \(a_j\in D\). Tomamos un índice \(k\) tal que \(a_k\neq 0\) (posible
  porque \(fe_i\neq 0\)). Tenemos que \(s(m_k)=a_k^{-1} m_i\) y
  \(s(m_j)=0\) si \(j\neq k\).

  Tenemos entonces
  \[
    sfe_i(m_i)=s(\sum_j a_j m_j)=a_k^{-1}a_k m_i=m_i
  \]
  con lo que \(sfe_i=e_i\) y por tanto \(Se_i=Sfe_i\) con lo que
  \(\subscriptbefore{S}{S}\) es semisimple.

  Veamos que cualquier módulo simple es isomorfo a \(\subscriptbefore{S}{S}\).
  Para ver que cada \(Se_i\) es isomorfo a \(\subscriptbefore{S}{M}\), por
  el lema de Schur, basta encontrar un homomorfismo no nulo
  \(Se_i\longrightarrow M\). Sea \(F:Se_i\longrightarrow M\) dado por
  \(F(f):=f(m_i)=f m_i\). Es fácil ver que \(F\) es un \(S\)-módulo.
  \(F(e_i)=e_i(m_i)=m_i\neq 0\), con lo que \(F\neq 0\) y por el lema
  de Schur es un isomorfismo.

  Si \(\subscriptbefore{S}{\Sigma}\) es simple, luego existe un epimorfismo
  \(p:S\longrightarrow\Sigma\) de \(S\)-módulos (descomponer por anuladores
  de cualquiera de sus elementos). Como \(p\neq 0\), existe un \(i\) tal
  que \(p|_{Se_i}\neq 0\) y por tanto es un isomorfismo.

  Sea \(\phi:S\longrightarrow R\) un isomorfismo de anillos.
  \(\{\phi(e_1),\longrightarrow,\phi(e_n)\}\) es claramente CCIO de R.\@
  En particular, \(R=\dot{+}_{i=1}^n R\phi(e_i)\).
  Cada \(R\phi(e_i)\) es simple como \(R\)-módulo.
  \(Se_i\cong \subscriptbefore{S}{M}\), y \(\subscriptbefore{R}{M}\) por
  restricción de escalares. Comprobando que
  \(\mathcal{L}(\subscriptbefore{S}{M})=\mathcal{L}(\subscriptbefore{R}{M})\),
  deducimos que \(\subscriptbefore{R}{M}\) es simple.

  \(R\phi(e_i)\), veamos que
  \(\mathcal{L}{R\phi(e_i)}\cong\mathcal{L}{R\phi(e_i)}\)
  dados por \(I\mapsto\phi(I)\) y \(J\mapsto\phi^{-1}(M)\), luego son dos
  conjuntos ordenados por la inclusión isomorfos. Luego como uno solo tiene
  dos elementos, en el otro también.

  \(R\phi(e_i)\) es simple y por tanto \(\subscriptbefore{R}{R}\) es
  semisimple. Además:
  \[
    \dim_D(M)=n=\ell(\subscriptbefore{S}{S})=\ell(\subscriptbefore{R}{R})
  \]

  Sea \(\Sigma\) un \(R\)-módulo simple. Mediante restricción de escalares
  es un \(S\)-módulo simple, luego \(\subscriptbefore{S}{\Sigma}\cong
  \subscriptbefore{S}{M}\) con lo que \(\subscriptbefore{R}{\Sigma}\cong
  \subscriptbefore{R}{M}\).

  \[
    \lambda:D\longrightarrow\End_S(M)=\End_R(M)
  \]
  es, por densidad, un isomorfismo.

\end{proof}


