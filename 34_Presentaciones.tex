¿Es posible encontrar \(P,Q\) cuadradas inversibles tales que
\(PA_\psi Q^{-1}\) sea una matriz quasidiagonal?

\begin{df}[Matrices y operaciones elementales]
  \(E_{ij}\in\mathcal{M}_n(R)\) definida por su única entrada no nula es
  la \((i,j)\)-ésima, que vale 1.
  Se verifica:
  \[
    E_{ij}=\left\{
      \begin{matrix}
        E_{ie},&\textrm{si } j=k\\
        0,&\textrm{si } j\neq k\\
      \end{matrix}
      \right.
  \]

  Para cualquier matriz \(B\) de entradas \(b_{ij}\), se puede escribir:
  \[
    B=\sum_{i,j} b_{ij}E_{ij}=\sum_{i,j} E_{ij}b_{ij}
  \]

  Sea \(A\) una matriz rectangular de tamaño adecuado, \(r\in R,
  u\in\mathcal{U}(R)\).
  \[
    {(E_{ij}A)}_{rs}=\left\{
      \begin{matrix}
        a_{is},&\textrm{si } r=j\\
        0,&\textrm{si } r\neq j\\
      \end{matrix}
      \right.
  \]
  La matriz \(I+rE_{ij}\) es inversible para \(i\neq j\). (multiplicando
  por \(I-rE_{ij}\) sale).

  La matriz \(I+E_{ij}+E_{ji}-E_{ii}-E_{jj}\) es inversible para \(i\neq j\),
  pues al cuadrado es la identidad.

  La matrix \(I+(u-1)E_{ii}\) es inversible, se prueba multiplicando
  por \(I+(u^{-1}-1)E_{ii}\).

  A las siguientes matrices las llamamos matrices elementales:
  \begin{enumerate}
    \item \(I+rE_{ij}\) (multiplicar por un escalar una fila o columna
      y sumársela a otra).
    \item \(I+E_{ij}+E_{ji}-E_{ii}-E_{jj}\) (intercambiar sus filas o
      columnas).
    \item \(I+(u-1)E_{ii}\) (multiplicar una fila o columna por una unidad).
  \end{enumerate}
  es un grupo.
\end{df}

Ejemplo: Sea \(M\) un grupo aditivo generado por \(\{m_1,m_2,m_3\) sujeto
a las relaciones:
\begin{enumerate}
  \item \(2m_1+m_2-m_3=0\)
  \item \(4m_1+m_2-3m_3=0\)
\end{enumerate}

Tomamos \(\Z\)-módulos libres \(F_3\) con bases \(\{f_1,f_2,f_3\}\) y
\(E_2\) con bases \(\{e_1,e_2\}\).
\[
  F_3\overset{\psi}{\longrightarrow}
  F_3\overset{\phi}{\longrightarrow}
  M\longrightarrow 0
\]

Definimos \(\phi(f_i)=m_i\) y
\[
  A_\psi =
  \begin{pmatrix}
    2&1&-1\\
    4&1&-3\\
  \end{pmatrix}
\]

Solo apuntamos las operaciones por columnas porque solo nos interesa la
matriz \(Q\). Para tener una sencilla, vamos a hacer el máximo número
de matrices por filas.
\[
  \begin{pmatrix}
    2&1&-1\\
    4&1&-3\\
  \end{pmatrix}\sim
  \begin{pmatrix}
    2&1&-1\\
    0&-1&-1\\
  \end{pmatrix}\sim
\]
Ahora comenzamos a hacer operaciones por columnas, anotándolas:
\[
  \begin{pmatrix}
    2&0&-2\\
    0&-1&-1\\
  \end{pmatrix}\underset{c_3+c_2}{\sim}
  \begin{pmatrix}
    2&0&-2\\
    0&-1&0\\
  \end{pmatrix}\underset{c_3-c_1}{\sim}
  \begin{pmatrix}
    2&0&0\\
    0&-1&0\\
  \end{pmatrix}
\]
Tenemos que
\[
  D=
  \begin{pmatrix}
    2&0&0\\
    0&-1&0\\
  \end{pmatrix}
\]
y que
\[
  Q=
  \begin{pmatrix}
    1&0&0\\
    0&1&0\\
    0&0&1\\
  \end{pmatrix}\underset{c_3-c_1}{\sim}
  \begin{pmatrix}
    1&0&-1\\
    0&1&0\\
    0&0&1\\
  \end{pmatrix}\underset{c_3-c_2}{\sim}
  \begin{pmatrix}
    1&0&-1\\
    0&1&1\\
    0&0&1\\
  \end{pmatrix}
\]

\[
  x_1=\sum_{j=1}^t q_{ij}m_j=m_1-m3
\]
\[
  x_2=\sum_{j=1}^t q_{ij}m_j=m_2+m3
\]
\[
  x_3=\sum_{j=1}^t q_{ij}m_j=m3
\]

\[M=\Z x_1\dot{+}\Z x_2\dot{+}\Z x_3\dot{+}\]
\[
  \ann_\Z(x_1)=2\Z
\]
\[
  \ann_\Z(x_2)=-1\Z
\]
\[
  \ann_\Z(x_3)=\langle 0\rangle
\]

Con lo que
\[M=\Z x_1\dot{+}\Z x_3\dot{+}\cong\Z_2\oplus\Z\]
