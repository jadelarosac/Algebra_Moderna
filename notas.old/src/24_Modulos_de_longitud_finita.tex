\begin{teo}
  Sea \(\subscriptbefore{A}{M}\) \(p\)-primario de longitud finita. Existen
  \(x_1,\ldots, x_n\in M\setminus\{0\}\) tales
  que \(M=Ax_1\dot{+}\cdots\dot{+} Ax_n\) y
  \[
    \Ann_A(M)=\ann_A(x_1)\supseteq\ann_A(x_2)\supseteq\ldots
    \supseteq\ann_A(x_n)
  \]
  Además, si \(y_1,\ldots, y_n\in M\) no nulos son tales que
  \(
    M=Ay_1\dot{+}\ldots\dot{+} Ay_n
  \)
  y
  \(
    \Ann_A(M)=\ann_A(y_1)\supseteq\ann_A(y_2)\supseteq\ldots
    \supseteq\ann_A(y_m)
  \), entonces \(n=m\) y \(\ann_A(x_i)=\ann_A(y_i)\).
\end{teo}
\begin{proof}
  Tomo \(x_1\in M\) tal que \(\Ann_A(M)=\ann_A(x)\), por la proposición,
  \(M=Ax_1\dot{+} N\) para cierto submódulo \(N\) de \(M\).
  Es claro que \(\Ann_A(N)\supseteq\Ann_A(M)=\langle p^t\rangle\),
  con lo que \(\Ann_A(N)=\langle p^{t'}\rangle\) con \(t'\le t\) y
  \(\ell(N)<\ell(M)\).

  Por inducción sobre \(\ell(M)\), tenemos \(x_1,x_2,\ldots, x_n\in N\)
  y \(N=Ax_2\dot{+}\cdots\dot{+} Ax_n\).
  De esto se deduce
  \[
    M=Ax_1\dot{+}\cdots\dot{+} Ax_n
  \]
  y \(\ann_A(x_1)=\Ann_A(M)\subseteq\ann_A(x_2)\subseteq\ldots\subseteq
  \ann_A(x_n)\).

  Veamos la unicidad. Hacemos inducción sobre \(\ell(M)\).

  Si \(\ell(M)=1\), tenemso que es simple y \(M=Ax=Ay\) y \(n=1=m\).

  Si \(\ell(M)>1\), tenemos que \(M\) no es simple. Consideramos
  \(M/pM\) donde \(pM:=\{pm:m\in M\}\) que es un submódulo por ser
  \(A\) conmutativo. \(\Ann_A(pM)=\langle p\rangle\).
  \[
    \Soc(M/pM)=M/pM
  \]
  luego \(M/pM\) es semisimple.

  Tengo un homomorfismo de módulos \(M\longrightarrow
  Ax_1/Apx_1\oplus\cdots
  Ax_n/Apx_1
  \) tal que \(\sum A-ix_i\mapsto (a_1 x_1+Apx_1,\ldots,
  a_n x+Apx_n)\).

  Se puede demotrar que dicha aplicación es sobreyectivo y su núcleo es
  \(pM\).
  \[
    M/pM\cong
    Ax_1/Apx_1\oplus\cdots
    Ax_n/Apx_1
  \]
  \(n = \ell(M/pM)\). Argumentando de forma análoga para \(y\);
  obtenemos \(n= \ell(M/pM)=m\).

  Si \(pM=\{0\}\), tenemos que todos los anuladores son iguales:
  \(\ann_A(x_i)=\langle p\rangle=\ann_A(y_i)\).

  Supongamos que \(pM\neq\{0\}\).
  \[
    pM=Apx_1\dot{+}\cdots\dot{+} Apx_r
  \]
  para cierto \(r\le n\).

  Así, \(\ann_A(x_i)=\langle p\rangle\) si solo si \(i>r\).
  y también \(\ann_A(y_i)=\langle p\rangle\) si solo si \(i>r\).
  Para cualquier \(i\le r\), tenemos que \(\ann_A(px_i)=\langle
  p^{t_i-1}\rangle\) si \(\ann_A(x_i)=\langle p^{t_i}\rangle\).

  \[
   \ann_A(px_1)\supseteq\ann_A(px_2)\supseteq\ldots
    \supseteq\ann_A(px_r)
  \]
  \[
   \ann_A(py_1)\supseteq\ann_A(py_2)\supseteq\ldots
    \supseteq\ann_A(py_s)
  \]
  donde \(\ann_A(y_i)=\langle p^{s_i}\rangle\) si y solo si \(i>s\).
  Pero \(\ell(pM)<\ell(M)\), por inducción \(s=r\) y que \(s_i-1=
  r_i-1\) y como sabemos que si \(i>r=s\) tenemos que
  \(\ann_A(x_i)=\ann(y_i)=\langle p\rangle\).

\end{proof}
