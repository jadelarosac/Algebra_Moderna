
\begin{proof}
  Veamos que 1 implica 2. Tenemos que \(0=n_1+\cdots+n_t\),
  si despejamos, \(n_i=-\sum_{j\neq i} n_j\in N_i\cap\left(
  \sum_{j\neq i} N_j\right)=\{0\}\).

  Veamos que 2 implica 3. Si \(n=\sum n_i=\sum n_i'\),
  entonces \(0=\sum(n_i-n_i')\) lo que implica que
  \(n_i=n_i'\).

  Finalmente, tomando \(n\in N_i\cap\left(
  \sum_{j\neq i} N_j\right)\), es decir,
  \(n=\sum_{j\neq i}n_j\) con lo que
  \(0=n-\sum_{j\neq i} n_j\) y como las descomposiciones
  son únicas, \(n=0\).
\end{proof}

\begin{df}[Suma interna]
  Si \(M=N_1+\cdots+ N_t\) tales que satisfacen una de las condiciones
  equivalentes anteriores, diremos que \(M\) es la suma directa interna
  y usaremos la notación \(M=N_1\dot{+}\cdots\dot{+} N_t\).
\end{df}

\begin{df}
  Si \(\{N_1,\ldots, N_t\}\) verifican las condiciones equivalentes
  anteriores y \(N_i\neq \{0\}\), se dice que el conjunto
  \(\{N_1,\ldots, N_t\}\) es una
  familia independiente.
\end{df}

Ejemplo: \(\Z_6\) es un \(\Z\) módulo.
\[
  \Z_6=\{0,1,2,3,4,5,6\}
\]
Tomamos
\[
  N_1=\{0,3\}
\]
y
\[
  N_2=\{0,2,4\}
\]

Tenemos que \(N_1, N_2\) es una familia independiente. Además es obvio que:
\[
  N_1\dot{+}N_2=\Z_6
\]
ya que tienen como intersección \(\{0\}\) y su suma es el total.

