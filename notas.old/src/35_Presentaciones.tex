Ejemplo: Sea \(K\) un cuerpo, \(T:V\longrightarrow V\), con
\(\dim_K V=3\), \(\{v_1,v_2,v_3\}\) es una base de \(V\).

Sea
\[
  B=
  \begin{pmatrix}
    \phantom{-}1  & -1  & -1  \\
    -1  & \phantom{-}1  & -1  \\
    -1  & -1 &  \phantom{-}1  \\
  \end{pmatrix}
\]
la matriz de \(T\) en dicha base. Obtengamos la descomposición cíclica
primaria de \(\subscriptbefore{K[x]}{V}\).

Tenemos para
\[
  A=A_\psi=
  \begin{pmatrix}
    x{-}1  & 1  & -1  \\
    1  & x{+}1  & -1  \\
    1  & -1 &  x{-}1  \\
  \end{pmatrix}\in\mathcal{M}(K[x])
\]

Busquemos \(P, Q\) y \(D\).
\(v_i=\varphi(f_i)\). Al final obtendremos \(PAQ^{-1}=D\).

Partimos de \(A\) y hacemos operaciones por filas:
\[
  A=
  \begin{pmatrix}
    x{-}1  & 1  & -1  \\
    1  & x{+}1  & -1  \\
    1  & -1 &  x{-}1  \\
  \end{pmatrix}\sim
\]
(colocamos el polinomio de menor grado como pivote)
\[
  \begin{pmatrix}
    1  & -1 &  x{-}1  \\
    1  & x{+}1  & -1  \\
    x{-}1  & 1  & -1  \\
  \end{pmatrix}\sim
  \begin{pmatrix}
    1   & -1   &  x-1  \\
    0   & x+2  & -x  \\
    0   & x    & -x^2-2x-2  \\
  \end{pmatrix}\sim
\]
(Suponiendo que el cuerpo tiene característica
  distinta de 2)
\[
  \begin{pmatrix}
    1   & -1   &  x-1  \\
    0   & 2    & x^2-3x+2  \\
    0   & x+2  & -x  \\
  \end{pmatrix}\sim
  \begin{pmatrix}
    1   & -1   &  x-1  \\
    0   & 2    & x^2-3x+2  \\
    0   & x+2  & -x  \\
  \end{pmatrix}\sim
\]
\[
  \begin{pmatrix}
    1   & -1   &  x-1  \\
    0   & 2    & x^2-3x+2  \\
    0   & 0    & -\frac{1}{2}x^3+\frac{1}{2}x^2+x-2  \\
  \end{pmatrix}\sim
  \begin{pmatrix}
    1   & -1   &  x-1  \\
    0   & 2    & x^2-3x+2  \\
    0   & 0    & x^3-x^2-2x+4  \\
  \end{pmatrix}
\]
Empezamos con las operaciones por columnas
\[
  \begin{pmatrix}
    1   & -1   &  x-1  \\
    0   & 2    & x^2-3x+2  \\
    0   & 0    & x^3-x^2-2x+4  \\
  \end{pmatrix}
  \overset{c_2+c_1}{\sim}
  \begin{pmatrix}
    1   & 0   &  x-1  \\
    0   & 2    & x^2-3x+2  \\
    0   & 0    & x^3-x^2-2x+4  \\
  \end{pmatrix}
  \overset{c_3-(x-1)c_1}{\sim}
\]
\[
  \begin{pmatrix}
    1   & 0   &  0  \\
    0   & 2    & x^2-3x+2  \\
    0   & 0    & x^3-x^2-2x+4  \\
  \end{pmatrix}
  \overset{c_3-(\frac{1}{2})(x^2-3x+2)c_2}{\sim}
  \begin{pmatrix}
    1   & 0   &  0  \\
    0   & 2    & 0\\
    0   & 0    & x^3-x^2-2x+4  \\
  \end{pmatrix} = D
\]

Calculamos ahora \(Q\):
\[
  Q=
  \begin{pmatrix}
    1 & 1 & x\\
    0 & 1 & \frac{1}{2}(x^2-3x+2)\\
    0 & 0 & 1
  \end{pmatrix}
\]

Quiero encontrar \(x_1, x_2, x_3\) tales que
\(\subscriptbefore{K[x]}{V} = K[x]x_1\dot{+}K[x]x_2\dot{+}K[x]x_3\).

Tenemos que \(\ann_{K[x]}(x_1)=K[x]\), \(\ann_{K[x]}(x_2)=2K[x]=K[x]\) y
\(\ann_{K[x]}(x_3)=\langle x^3-x^2-2x+4 \rangle\).
Con esto, \(x_1=x_2=0\) y por tanto
\[
  \subscriptbefore{K[x]}{V}=K[x]v_3
\]
donde la última igualdad es porque \(q_{33}=1\).

Es cíclica primaria si \( x^3-x^2-2x+4 \) es una potencia del irreducible.
Vamos a estudiar según quien sea el cuerpo \(K\), al menos en un par de
casos.

Caso particular \(K=\Q\): Probando con \(\pm 1, \pm 2, \pm 4\) vemos que
no tiene raíces en \(\Q\). Por tanto, como el grado es 3,
\(\mu= x^3-x^2-2x+4\) es el polinomio mínimo y es irreducible.
\[
  \subscriptbefore{\Q[x]}{V}=\Q[x]v_3
\]
es la descomposición cíclica primaria. Además, \(\subscriptbefore{\Q[x]}{V}\)
es simple, al ser \(\mu\) maximal y \(\Q[x]v_3\cong \Q[x]/\langle\mu\rangle\).

Sobre \(\Q\) la matriz tomando la base \(\{v_3,T(v_3),T^2(v_3)\}\):
\[  \begin{pmatrix}
       0 & 1 & 0\\
       0 & 0 & 1\\
       -4 & 2 & 1\\
     \end{pmatrix}
   \]

Caso particular \(K=\R\): Por análisis, al ser grado impar, existe al menos
una raíz real. \(\mu'(x)=3x^2-2x-2\), que tiene como raíces
\(\frac{1\pm\sqrt{7}}{3}\) y tenemos una parábola con coeficiente líder
positivo. Luego hay 1 raíces o 3 si los máximos y mínimos son positivos
o negativos: \(\mu(\frac{1+\sqrt{7}}{3})>0\) luego \(\mu\) tiene una única
raíz en \(\R\).

Tenemos que, si \(\alpha\) es la raíz real y \(z\in\C\setminus\R\),
\(\mu=(x-\alpha)(x-z)(z-\bar{z})=(x-\alpha)(x^2-2\Re(z)x+|z|^2)\) en \(\R[x]\).

