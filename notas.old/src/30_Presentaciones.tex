\subsection{Presentaciones de módulos}

\begin{prop}[Módulo presentado]
  Sea \(M\) un módulo. Existe una sucesión exacta
  \[
    \cdots\overset{f_{-2}}{\longrightarrow} F_{-1}
    \overset{f_{-1}}{\longrightarrow} F_{0}
    \overset{f_{0}}{\longrightarrow} M
    \overset{}{\longrightarrow} 0
  \]
  donde \(F_{-n}\) es libre para todo \(n\in\N\). Esa sucesión se llama
  resolución libre de \(M\).
\end{prop}
\begin{proof}
  Tomo un conjunto de generadores de \(M\), y tomo un homomorfismo de módulos
  sobreyectivo \(F_0\overset{p_0}{\longrightarrow} M\).
  \[
    F_{-1}
    \overset{p_{-1}}{\longrightarrow} K_0
    \overset{\iota}{\longrightarrow} F_0
    \overset{p_{0}}{\longrightarrow} M
    \overset{}{\longrightarrow} 0
  \]
  y reiteramos el proceso.

  Exactitud vista en \(F_{-1}\) ya que otro caso sería análogo.
  \(\ker f_{-1}=:K_{-1}=\Im p_{-2}=\Im  f_{-2}\).

  La resolución puede pero no tiene por qué ser finita.

\end{proof}

\begin{df}[Módulo finitamente presentado]
  \(M\) se dice finitamente presentado si existe un presentación finita que no
  es sino una sucesión exacta de la forma
  \[
    F_{-1}\overset{f_{-1}}{\longrightarrow} F_0
    \overset{f_{0}}{\longrightarrow}M \longrightarrow 0
  \]
\end{df}

Ejercicio: dar una presentación finita del \(\Z\)-módulo \(\Z_2\oplus\Z_4\).

\begin{prop}
  Un anillo \(R\) es noetheriano a izquierda si y solo si todo módulo
  finitamente generado es finitamente presentado.
\end{prop}
\begin{proof}
  Veamos solo una implicación: que si \(\subscriptbefore{R}{R}\) es
  noetheriano entonces que submódulo finitamente generado es finitamente
  presentado.

  Como \(M\) es finitamente generado, \(K_0\) es finitamente
  generado \(F_{-1}\overset{p_{-1}}{\longrightarrow}
  K_0\overset{\iota}{\longrightarrow}
  F_0\overset{f_0}{\longrightarrow} M\longrightarrow 0\).

\end{proof}

Tenemos que \(M\cong F_0/\Im f_{-1}\). Tomemos \(E_s\), \(F_t\) módulos
libres con bases finitas de cardinales \(s\) y \(t\) respectivamente.
Diremos que \(E_s\) tiene rango \(s\) (a pesar de que no es una invariante
del módulo, problema de la base de número invariante o INB, incluso se
puede dar \(R\cong R\oplus R\)). Llamamos \(e=\{e_1,\ldots,e_s\}\) base
de \(E_s\), y \(f=\{f_1,\ldots,f_t\}\) base de \(F_t\). Sea
\(\psi: E_s\longrightarrow F_t\), definido por \(\psi(e_i)=
\sum_{j=1}^t a_{ij}f_j\). Definimos la matriz \(A_\psi={(a_{ij})}_{
  1\le i\le s, 1\le j\le t}\in\mathcal{M}_{s\times t}(R)\).

Dado \(u=\sum_{i=1}^s x_i e_i\), \(x_i\in R\). Entonces
\[
  \psi(u)=\sum_{j=1}^t y_j f_j
\]
Resulta que si \(u_e=x=(x_1,\ldots,x_s)\) y \(y=(y_1,\ldots,y_t)\),
tenemos que \(y=xA_\psi\) y por tanto
\({\psi(u)}_f=u_e A_\psi\).

Tenemos que \((\cdot)A_\psi\circ{(\cdot)}_e={(\cdot)}_f\circ\psi\).


Sean \(E_s\overset{\psi}{\longrightarrow} F_t\overset{\varphi}{\longrightarrow}
G_r\), entonces \(A_{\varphi\circ\psi}=A_\varphi A_\psi\).

Ejemplo: Sea \(T:V\longrightarrow V\) un endomorfismo de espacios vectoriales,
y \(V\) de dimensión finita. \(\subscriptbefore{K[x]}{V}\) es un módulo
finitamente presentado. Buscamos una presentación finita.



