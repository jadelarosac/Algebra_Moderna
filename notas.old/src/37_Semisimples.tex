\subsection{Módulos semisimples}

\begin{prop}
  Sea \(M\) un módulo.
  Sea \(\{M_i:i\in I\}\) una familia no vacía de submódulos simples
  (no cero y que sus únicos submódulos son el 0 y el total).

  Ponemos \(M'=\sum_{i\in I} M_i\), es decir, el menor submódulo que los
  contiene a todos. Tomamos  \(N\subsetneq M'\). Entonces existe un
  \(J\subseteq I\) tal que \(\{M_i:i\in J\}\) es independiente,
  \(N\cap\left(\dot{+}_{i\in J} M_i\right)=\{0\}\) y
  \(M'=N\dot{+}\left(\dot{+}_{j\in J} M_i\right)\).
\end{prop}
\begin{proof}
  La demostración pasa por utilizar el lema de Zorn.
  Sea \(\Gamma\) el conjunto de los subconjuntos \(J\) de \(I\) tales que
  \(\{M_i:i\in J\}\) es independiente y
  \(N\cap\left(\dot{+}_{i\in J} M_i\right)=\{0\}\).

  Veamos que \(\Gamma\neq\emptyset\). Si \(N=\{0\}\), tomamos \(i\in I\)
  y tenemos que \(\{i\}\in\Gamma\}\), que cumple trivialmente ambas
  propiedades. Si \(N\neq\{0\}\), pero \(N\cap M_i=\{0\}\), tomamos
  de nuevo \(\{i\}\in\Gamma\) que es otro caso trivial.

  Supongamos que \(N\neq\{0\}\) y \(N\cap M_i=\{0\}\) para todo \(i\in I\).
  Como cada \(M_i\) es simple, \(N\cap M_i=M_i\) para todo \(i\in I\),
  con lo que \(N=M'\), caso que hemos excluído.

  El orden que definimos en \(\Gamma\) es la inclusión. Tenemos que ver
  que cualquier cadena (subconjunto totalmente ordenado) tiene un elemento
  maximal. Sea \(\chi\subseteq\Gamma\). Definimos \(J=\bigcup_{C\in\chi} C\).
  Lo que tenemos que demostrar es que \(J\in \Gamma\).

  Veamos que \(\{M_i:i\in J\}\) es independiente. Por una proposición anterior,
  basta ver que cualquier \(\{M_i:i\in F\}\) es independiente para cualquier
  \(F\subseteq J\) finito. Por ser \(\chi\) una cadena, existe un \(C\in\chi\)
  tal que \(F\subseteq C\). Pero \(C\in\Gamma\), luego \(\{M_i:i\in C\}\)
  es idependiente, y en particular, \(\{M_i:i \in F\}\) es independiente.

  Tomamos \(m\in N\cap\left(\dot{+}_{j\in J} M_j\right)\). Entonces existe
  un \(F\subseteq J\) finito tal que \(m\in N\cap\left(\dot{+}_{j\in F}
  M_j\right)\), entonces existe un \(C\in\chi\) tal que \(F\subseteq C\)
  y por consiguiente \(m\in N\cap\left(\dot{+}_{j\in C} M_j\right)=\{0\}\).

  Por tanto \(\Gamma\) es inductivo y el lema de Zorn nos asegura que
  existe un \(J\in\Gamma\) maximal.

  Solo basta ver que \(M'=N\dot{+}\left(\dot{+}_{j\in J}M_j\right)\) y
  basta ver que es la suma (ya sabemos que es directa). Para \(i\notin J\),
  \(M_i\cap\left(N+\left(\dot{+}_{j\in J} M_j\right)\right)\neq \{0\}\).
  De lo contrario, \(J\cup\{i\}\in\Gamma\) y \(J\) no sería maximal.
  Como \(M_i\) es simple, \(M_i\subseteq\left(N+\left(\dot{+}_{j\in J}
  M_j\right)\right)\). Al final tenemos que
  \(M_i\subseteq\left(N+\left(\dot{+}_{j\in J}
  M_j\right)\right)\) para todo \(i\in I\), con lo que
  \(M'=N\dot{+}\left(\dot{+}_{j\in J}M_j\right)\).

\end{proof}

\begin{df}[Anillo de división]
  Un anillo \(D\) se dice que es un anillo de división si
  para todo \(d\in D\setminus\{0\}\) existe un \(d^{-1}\) tal que
  \[
    dd^{-1}=d^{-1}d=1
  \]
  Si \(D\) es además conmutativo, es entonces un cuerpo.
\end{df}

\begin{cor}
  Sea \(D\) un anillo de división y \(\subscriptbefore{D}{V}\) un
  \(D\)-espacio vectorial no nulo. Si \(\{v_i:i\in I\}\) es un conjunto
  de generadores no nulos de \(V\), existe \(J\subseteq I\) tal que
  \(\{v_j:j\in J\}\) es una base de \(\subscriptbefore{D}{V}\).
\end{cor}
\begin{proof}
  Tomo la familia \(\{Dv_i:i\in I\}\). Cada \(Dv_i\cong D/\ann_D (v_i)
  \cong \subscriptbefore{D}{D}\) ya que el anulador de cualquier elemento
  en un anillo de división es cero.

  \(\subscriptbefore{D}{D}\) es un módulo simple.
  \[
    V=\sum_{i\in I} Dv_i
  \]
  Tomando \(N=\{0\}\) en la proposición, existe un \(J\in I\) tal que
  \[
    V=\dot{+}_{j\in J} Dv_j
  \]
  o equivalentemente \(\{v_j: j\in J\}\) es base de \(V\).

\end{proof}

\begin{obs}
  En la proposición anterior se ve que \(V\cong D^{(J)}\).
\end{obs}

\begin{df}[Homomorfismo escindido]
  Dado homomorfismos de módulos \(N\overset{g}{\longrightarrow} M
  \overset{f}{\longrightarrow} N\) tales que \(f\circ g=\id_N\), diremos
  que \(f\) es un epimorfismo escindido (o roto o partido)
  y que \(g\) es un monomorfismo escindido (o roto o partido).
\end{df}

\begin{lema}
  Todo módulo finitamente generado no nulo contiene un submódulo propio
  maximal.
\end{lema}
\begin{proof}
  Sea \(M\) el módulo y \(\Gamma\) el conjunto de los submódulos propios
  de \(M\), o sea,
  \[
    \Gamma = \{ N:N\in\mathcal{L}(M), N\neq M\}
  \]

  Tenemos que \(\{0\}\in\Gamma\), luego es no vacío. Tomamos \(\chi\)
  cadena en \(\Gamma\) y \(N=\bigcup_{C\in\chi} C\). Veamos que
  \(N\in \Gamma\).

  Tomamos \(m_1,\ldots, m_t\) generadores de \(M\). Si \(N=M\), tendríamos
  que \(m_1,\ldots, m_t\in N\) y existiría en ese caso un \(C\in\chi\)
  tal que \(m_1,\ldots, m_t\in C\), con lo que
  \(M\subseteq C\subseteq M\) con lo que \(C=M\) y en particular
  \(C\notin \Gamma\), lo cuál es una contradicción.

  Aplicando el lema de Zorn a \(\Gamma\), tenemos que tiene elementos
  maximales.

\end{proof}


