\subsection{Módulos de longitud finita}

\begin{df}[Serie de composición]
  Sea \(M\) un módulo. Una serie de composición de \(M\)
  es una cadena de submódulos
  \[
    M=M_n\supsetneq M_{n-1}\supsetneq\ldots\supsetneq M_1
    \supsetneq M_0=\{0\}
  \]
  tal que si \(M_i\supseteq N\supseteq M_{i-1}\) para \(N\) submódulo,
  entonces \(N=M_i\) o \(N=M_{i-1}\). Es decir, cada submódulo es maximal
  en el anterior.

  A \(n\) le llamamos la longitud de la serie.
\end{df}

Ejemplo: serie de composición de \(\Z_{12}\). Tiene como subgrupos
a \(\Z_m\) con \(m\) divisor de 12.
\[
  M_3=\Z_{12}
\]
tiene como subgrupo maximal (argumentando por Lagrange):
\[
  M_2=\langle 2 \rangle
\]
que a su vez tiene como subgrupo maximal
\[
  M_1=\langle 4\rangle
\]
y ya solo tiene
\[
  M_0=\{0\}
\]

\begin{df}[Módulo simple]
  \(M\) se dice simple si \(M\supset\{0\}\) es una serie de composición.
  Es decir, si no tiene submódulos propios y no es el módulo 0.
\end{df}

\begin{prop}
  La condición de que cada submódulo sea maximal en el anterior es equivalente
  a que los factores \(M_i/M_{i-1}\) sean simples.
\end{prop}

\begin{teo}
  Toda serie de composición del mismo módulo tiene la misma longitud
  y los mismos factores salvo isomorfismo y reordenación.
\end{teo}

\(\Z_{12}\) tiene como factores \(\Z_2\), \(\Z_2\) y \(\Z_3\).

\begin{prop}
  Un módulo no nulo admite una serie de composición si y solo si es
  noetheriano y artiniano.
\end{prop}
\begin{proof}
  Sea \(M_i\) una serie de composición. Inducción sobre \(n\).
  Si \(n=1\), tenemos que \(M\) es simple y en particular noetheriano
  y artiniano.

  Si \(n>1\), entonces \(M_{n-1}\) admite una serie de composición de
  longitud \(n-1\), luego es noetheriano y artiniano. Tomamos
  la sucesión exacta corta \[0\longrightarrow M_{n-1}
  \longrightarrow M_n\longrightarrow M_n/M_{n-1}\longrightarrow 0\]
  El primer elemento es noetheriano y artiniano, el último es simple (luego
  noetheriano y artiniano), con lo que \(M_n\) es noetheriano y artiniano.

  Para el recíproco, como \(M\) es artiniano, contiene un submódulo
  simple \(M_1\). Entonces hay un \(M_2\supsetneq M_1\) donde
  \(M_2/M_1\) es simple. Reiterando el proceso, tenemos
  \(0\subsetneq M_1\subsetneq M_2\subsetneq \ldots\) y como es
  noetheriano, habrá un \(M_n\) que termine la cadena.
\end{proof}
