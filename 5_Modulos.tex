\section{Introducción al concepto de módulo}

\begin{df}
  Sean \(M\), \(N\) grupos aditivos:
  \[
    \Ad(M,N)=\{f:M\longrightarrow N| f\textrm{ homomorfismo de grupos}\}
  \]
\end{df}

El conjunto anterior es no vacío porque \(0\in\Ad(M,N)\).
\(\Ad(M,N)\) es un grupo aditivo con la suma:
\[
  (f+g)(m):=f(m)+g(m)\hspace{1cm} \forall m\in M
\]


\begin{df}[Anillo de endomorfismo de \(M\)]
  Definimos directamente \(\End(M):= \Ad(M,M)\).
\end{df}

\begin{prop}
  \((\End(M),+,0,\circ,\id) \) es un anillo.
\end{prop}
\begin{proof}
  Se comprueba que es cerrado para composición. Es obvio que la
  composición es asociativa y tiene como elemento neutro la identidad.

  Finalmente se ve que se cumplen las propiedades distributivas, que
  se siguen de que son homomorfismos.
\end{proof}


\begin{obs}
  Consideramos el grupo \(\{0\}\), es el anillo \(\{0\}\) (anillo
  cero o trivial).

  Si \(M\neq \{0\}\), entonces \(\End(M)\) no es trivial.
\end{obs}

\begin{df}[Módulo]
  Sea \(M\) un grupo aditivo y \(A\) un anillo. Una estructura de
  \(A\)-módulo sobre \(M\) es un homomorfismo de anillos
  \(\rho: A\longrightarrow\End(M)\).
\end{df}

Ejemplo: los números enteros. \(M\) grupo aditivo, \(A=\Z\).
Existe un único \(\chi:\Z\longrightarrow\End(M)\) determinado
por \(\chi(1)=\id_M\), es decir, una única estructura de
\(\Z\)-módulo sobre \(M\) (y su núcleo te da la
característica del anillo).

Ejemplo: cuerpos. Sea \(K\) un cuerpo.
Si \(V\) es un \(K\)-espacio vectorial, definimos \(\rho:K\longrightarrow
\End(V)\), tomamos \(\rho(\alpha):V\longrightarrow V\)
cumpliendo \(\rho(\alpha)(v)=\alpha v\). Trivialmente se cumple que
\(\rho\) es un homomorfismo por la estructura de espacio vectorial de \(V\).
Con lo cual tenemos una estructura de \(K\)-módulo sobre \(V\).
Se puede demostrar el recíproco trivialmente.

\begin{obs}
  Sean  \(X, Y, Z\) conjuntos. \(\Map(X,Y)\) es el conjunto de
  aplicaciones de \(X\) en \(Y\).

  Entonces:
  \[
    \psi:\Map(X\times Y, Z)\longrightarrow\Map(X,\Map(Y,Z))
  \]
  es una biyección dada por \(\psi(f)(x)(y):=f(x,y)\) y
  \(\psi^{-1}(g)(x,y):=g(x)(y)\).
\end{obs}

Ejercicio: comprobar que \(\psi^{-1}\) es realmente la inversa de
\(\psi\).

\begin{obs}
  Sean \(M, N, L\) grupos aditivos.
  \[
    \psi:\Biad(M\times N, L)\longrightarrow\Ad(M,\Ad(N,L))
  \]
  donde \(b\in\Biad(M\times N, L)\) si \(b\) es biaditiva:
  \[
    b(m+m',n)=b(m,n)+b(m',n)
  \]\[
    b(m,n+n')=b(m,n)+b(m,n')
  \]
\end{obs}

Ejercicio, demostrar que la aplicación \(\psi\) es una biyección.

\begin{teo}[Caracterización de módulos]
  Sea \(A\) anillo, \(M\) un grupo aditivo. Sea \(\Ring(A,\End(M))\),
  llamamos \(A\)-módulo a la imagen por \(\psi\) de ese conjunto.
\end{teo}

\begin{df}
  \[
    \Ring(R,S)=\{\phi:R\longrightarrow S, \phi \textrm{ es homomorfismo
    de anillos}\}
  \]
\end{df}


\begin{prop}
  Dados un grupo aditivo \(M\) y un anillo \(A\), se tiene una
  correspondencia biyectiva entre:
  \begin{enumerate}
    \item Homomorfismos de anillos \(\rho:A\longrightarrow\End(M)\)
    \item Las aplicaciones \(A\times M\longrightarrow M\)
      que satisfacen:
      \begin{itemize}
        \item \((a+a')m=am+a'm\)
        \item \(a(m+m')=am+am'\)
        \item \((aa')m=a(a'm)\)
        \item \(1\cdot m=m\)
      \end{itemize}
  \end{enumerate}
\end{prop}

\begin{proof}
  Tomamos la biyección \(\psi^{-1}:\Map(A,\Map(M,M))\longrightarrow
  \Map(A\times M,M)\).
  Tomamos \(\rho\in\Ring(A,\End(M))\), su imagen por la biyección,
  \(\psi^{-1}(\rho)\)
  son las aplicaciones que satisfacen justo las propiedades
  anteriores.

  Llamamos a \(\psi^{-1}(\rho)(a,m)=a\cdot m\). Tenemos que
  \(\psi^{-1}(\rho)(a,m)=\rho(a)(m)\). Entonces
  \(a\cdot m=\rho(a)(m)\).

  Comprobamos la tercera propiedad como ejemplo:

  Dados \(a, a'\in A\) y \(m\in M\):
  \[
    (aa')m=\rho(aa')(m)=(\rho(a)\circ\rho(a'))(m)
    =\rho(a)(\rho(a')(m))=\rho(a)(a'm)=a(a'm)
  \]

  De forma análoga se demuestran el resto de propiedades.

  Esta correspondencia responde a la fórmula \(am=\rho(a)(m)\).
\end{proof}

Un \(A\)-módulo lo veremos de cualquiera de las maneras anteriores, que
ya hemos visto que son equivalentes, según su conveniencia.

\begin{ejemplo}
  Si \(K\) es un cuerpo, un \(K\)-módulo es esencialmente
  un \(K\)-espacio vectorial.
\end{ejemplo}

\begin{ejemplo}
  Sean \(A\) un anillo y  \(\lambda:A\longrightarrow A\) definida como
  \(\lambda(a)(a'):=aa'\). Entonces \(\lambda\) es un homomorfismo de anillos que
  dota a \(A\) de una estructura de \(A\)-módulo, llamada \textbf{módulo regular}.
\end{ejemplo}
\begin{proof}
  Veamos que \(\lambda\) es un homomorfismo de
  anillos probando que se cumplen las condiciones de la definición \ref{df:homo_anillos}.
  Sean \(a_1, a_2, a' \in A\).
  \begin{itemize}
  \item \(\lambda(a_1+a_2)(a') = (a_1+a_2)a' = a_1a' + a_2a' = \lambda(a_1)(a') +
    \lambda(a_2)(a')\), luego \(\lambda(a_1 + a_2) = \lambda(a_1) + \lambda(a_2)\).
  \item \(\lambda(a_1a_2)(a') = a_1a_2a' = \lambda(a_1)(\lambda(a_2)a') =
    (\lambda(a_1) \circ \lambda(a_2))(a')\), luego \(\lambda(a_1a_2) =
    \lambda(a_1) \circ \lambda(a_2)\).
  \item \(\lambda(1)(a') = 1 \cdot a' = a'\), luego \(\lambda(1) = 1\) es la aplicación
    identidad en el anillo \(A\).
  \end{itemize}
\end{proof}.

\begin{prop}[Restricción de escalares]
  Sean \(A,\ R\) anillos, \(M\) un grupo aditivo y \(\phi:R\longrightarrow A\)
  homomorfismo de anillos. Si \(M\)es un \(A\)-módulo, vía el homomorfismo de anillos
  \(\rho : A \longrightarrow \End(M)\), tenemos que \(M\) es un \(R\)-módulo
  vía \(\rho \circ \phi\).

  Equivalentemente, si \(r\in R\) y \(m\in M\), definimos
  \[
    rm=(\rho\circ\phi)(r)(m)=\rho(\phi(r))(m)=\phi(r)m
  \]
  Este proceso se conoce como \textbf{restricción de escalares}.
\end{prop}

