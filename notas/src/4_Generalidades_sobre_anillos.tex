\subsubsection{Transformada discreta de Fourier}

Ahora vamos a reindexar. En lugar de usar \(1,\ldots, t\)
vamos a tomar los índices \(0,\ldots, n-1\).

Vamos a suponer que el cuerpo \(K\)
contiene una raíz primitiva de 1, o sea,
existe un \(\omega\in K\) tal que \(\omega^n = 1\) y
\(1, \omega, \omega^2,\ldots, \omega^{n-1}\) son distintos.

Seguro que \(\car K \not\| n\) ya que \(1, \omega, \omega^2,\ldots,
\omega^{n-1}\) son las raíces de \(x^n-1\) y son distintas.

Vamos a interpolar las raíces de la unidad.

Tomo \(\alpha_j =\omega^j\), \(j\in\{0,\ldots, n-1\}\) y
\[M=A_\omega=
\begin{pmatrix}
  1&1&\cdots1\\
  \omega^0&\omega^1&\cdots\omega^{n-1}\\
  {(\omega^0)}^2&{(\omega^1)}^2&\cdots{(\omega^{n-1})}^2\\
  \cdots&\cdots&\cdots
\end{pmatrix}= (\omega^{ij})
\]


Tenemos que \(x^n-1=(x-1)(x^{n-1}+\cdots+x+1)\)
y evaluando en \(\omega^{j}\) obtenemos
\[
  \omega^{(n-1)j}+\cdots+\omega^j+1=0
\]

Entonces \(\sum_{k=0}^{n-1}\omega^{ik}=0\) para \(0<i<n\).

\[
  \begin{pmatrix}
  	\omega^{0i}&
    \omega^{i}&
    \omega^{2i}&
    \cdots&
    \omega^{(n-1)i}
  \end{pmatrix}
  \begin{pmatrix}
  	\omega^{-0j}\\
    \omega^{-j}\\
    \omega^{-2j}\\
    \cdots\\
    \omega^{-(n-1)j}
  \end{pmatrix}
  =\sum_{k=0}^{n-1}\omega^{k(i-j)}=0
\]

Tenemos entonces que \(A_\omega A_{\omega^{-1}}^T=nI\),
es decir, \(A^{-1}_\omega=\frac{1}{n}A_{\omega^{-1}}^T\).

\(\tilde{\xi}:K[x]/\langle x^n-1\rangle\longrightarrow K^n\),
con \(\tilde{\xi}^{-1}(y)\) es el polinomio interpolador.

Tenemos unos datos \((y_0,\ldots, y_{n-1})\in K^n\). El polinomio
interpolador de esos datos en los nodos \(1,\omega,\ldots,\omega^{n-1}\)
viene dado por
\[
  \hat{y} =\sum_{j=0}^{n-1}\hat{y_j}x^j
\]
donde \(\hat{y}=y\frac{1}{n}A^T_{\omega^{-1}}\).

Explícitamente, se calcula que los coeficientes quedan:
\[
  \hat{y_j}=\frac{1}{n}\sum_{k=0}^{n-1}y_k\omega^{-jk}
\]

Tomamos \(K=\C\). \(\omega = e^{i2\pi/n}\):
\[
  \hat{y_j}=\frac{1}{n}\sum_{k=0}^{n-1}y_k\omega^{-i2\pi jk/n}
\]
que es la transformada de Fourier de \(y\).

¿Qué interpretación le damos? Supongamos una función periódica de periodo
\(2\pi\), \(f:[0,2\pi]\longrightarrow\C\) con \(f(0)=f(2\pi)\).
Dividimos el intervalo en \(n\) partes iguales, una muestra:
\(y_j=f(\frac{2\pi j}{n})\) con \(j=0,\ldots,n-1\).

Tomamos \(g:[0,2\pi]\longrightarrow\C\) con
\(g(t)=\sum_{j=0}^{n-1}\hat{y_j}e^{ijt}\).

Tenemos entonces que \(g(\frac{2\pi l}{n})=
\sum_{l=0}^{n-1}\hat{y_j}e^{i2\pi lj/n} = y_l=f(\frac{2\pi j}{n})\)

A los \(\hat{y}\) también se le llama el espectro de \(y\).
