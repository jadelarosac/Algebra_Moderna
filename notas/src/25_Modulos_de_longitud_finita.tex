\begin{obs}
  Si \(A=\Z\), \(M\) grupo abeliano, \(x\in M\),
  \(\ann_\Z(x)=n\Z\), \(n\) recibe el nombre de el orden.
\end{obs}

\begin{obs}
  Si \(A=K[x]\), \(T:V\longrightarrow V\), \(n=\dim_K V<\infty\),
  \(v\in V\), \(\ann_{K[x]}(v)=\langle f(x)\rangle\).
  Tenemos que \(f\) tiene grado \(n\). \(\{v, Tv,\ldots, T^{n-1}v\}\)
  es una base de \(V\).
\end{obs}

\begin{ejemplo}
  \(\mathcal{U}(\Z_8)=\{1,3,5,7\}\).
  Viendo los ordenes de los elementos:
  \[\mathcal{U}(\Z_8)=\langle 3\rangle\dot{+}\langle 5\rangle\]
  donde \(\langle \cdot\rangle\) es la generación como subgrupo.
\end{ejemplo}

\begin{ejemplo}
  Suponemos un espacio vectorial \(V\) de dimensión 3 y un
  endomorfismo \(T\) cuyo polinomio mínimo es de la forma
  \({(x-\lambda)}^2\) con \(\lambda\in K\).
  Sabemos que existen dos vectores \(v_1,v_2 \in V\) tales que
  \[
    V=K[x]v_1\dot{+} K[x]v_2
  \]
  con \(\ann_{K[x]} v = \langle {(x-\lambda)}^2\rangle
  \subsetneq\langle {x-\lambda}\rangle = \ann_{K[x]} v_2\).
\end{ejemplo}

\begin{cor}
  Si \(\subscriptbefore{A}{M}\) es un módulo \(p\)-primario, entonces existen
  \(C_1, \ldots, C_n\) módulos cíclicos tales que
  \[
    M\cong C_1\oplus\cdots\oplus C_n.
  \]

  Si existen otros \(D_1, \ldots, D_m\) módulos cíclicos tales que \(M\cong D_1\oplus
  \cdots\oplus D_m\), entonces
  \(n=m\) y tras reordenación, \(D_i\cong C_i\) para todo \(i \in \{1, \ldots, n\).
\end{cor}
\begin{proof}
  De \(M\cong C_1\oplus\cdots\oplus C_n\), se puede exigir que
  \(x_1,\ldots,x_n\in M\) tales que
  \[
    M=Ax_1\dot{+}\cdots\dot{+}Ax_n
  \]
  con \(\ann_A(x_1)\subseteq\ann_A(x_2)\subseteq\ldots\subseteq\ann_A(x_n)\)

  Con \(D_1\oplus\cdots\oplus D_m\) hago lo mismo.
  \[
    M=Ay_1\dot{+}\cdots\dot{+}Ay_n
  \]
  ordenados bajo el mismo criterio.

  El enunciado se sigue de aplicar el teorema anterior. De
  \(\ann(x_i)=\ann(y_i)\) se deduce
  \[
    C_i\cong Ax_i\cong A/\ann(x_i)=A/\ann(y_i)\cong Ay_i\cong D_i
  \]
\end{proof}

\begin{ejercicio}
  Decimos que un módulo \(M\) es indescomponible si \(M\cong
  L\oplus N\) implica que \(L=\{0\}\) (o \(N=\{0\}\)).
  Razonar que en el corolario cada uno de los \(C_i\) es indescomponible.
\end{ejercicio}

\begin{ejemplo}
  \(M\) grupo abeliano de longitud finita y \(p\)-primario.
  Aplicando el corolario, \(M\cong C_1\oplus\cdots\oplus C_n\) con
  \(C_i\) cíclico y de longitud finita \(p\)-primarios.
  Tenemos que \(M\cong \Z_{p^{m_1}}\oplus\cdots\oplus\Z_{p^{m_n}}\),
  \(M\) es finito de cardinal \(p^{m_1+\cdots+m_n}\).
\end{ejemplo}

\begin{teo}[Estructura de módulos sobre un DIP]
  Sea \(A\) un dominio de integridad primaria, \(M\) un grupo aditivo y el módulo
  \(\subscriptbefore{A}{M}\neq\{0\}\) de longitud finita. Entonces existen
  irreducibles distintos \(p_1,\ldots,p_r\in A\) y 
  \(n_1,\ldots, n_r \in \N\) tales que \(e_{i1}\ge\ldots\ge e_{in_i}\) con
  \(i\in\{1,\ldots,r\}\) que determinan \(M\):
  \[
    M=\dotplus_{i=1}^r\left(\dotplus_{j=1}^{n_i} Ax_{ij}\right),
  \]
  donde los \(x_{ij}\in M\) verifican:
  \[
    \ann_A(x_{ij})=\langle p_i^{e_{ij}}\rangle
  \]
  con \(i\in\{1,\ldots,r\}\) y \(j\in\{1,\ldots, n_i\}\).
  A esta expresión
  se le llama la \textbf{descomposición cíclica-primaria de \(M\)} (la
  primaria sería la primera suma y luego cada factor primario se
  descompone en factores cíclicos).
  A los \(x_{ij}\) se les llama
  \textbf{divisores elementales de \(M\)} y determinan \(M\) salvo isomorfismos.
\end{teo}

\begin{proof}
  Sea \(\mu \in A\) tal que \(\langle \mu \rangle = \Ann_A(M)\). Tomamos
  \(\mu = p_1^{e_1} \cdots p_r^{e_r}\), con \(p_1, \ldots, p_r \in A\) irreducibles.
  Entonces \(M = M_1 \dotplus \cdots \dotplus M_r\) (descomposición primaria) para
  \(M_i\) un submódulo \(p_i\)-primario. Para cada \(i \in \{1, \ldots, r\}\),
  \(M_i = \dotplus^{m_i}_{j=1} Ax_{ij}\)(descomposición cíclica de un primario),
  con \(\Ann_A(M_i) = \ann_A(x_{i1}) \subset \cdots \subset \ann_A(x_{in_i})\)
  para ciertos \(x_{ij} \in M_i\) y \(j \in \{1, \ldots, n_i\}\).

  De hecho, \(\Ann_A(M_i) = \langle p_i^{e_i} \rangle = \ann_A(x_{i1}) \subset \ldots
  \subset \ann_A(x_{in_i}) = \langle p_i^{e_{n_i}} \rangle\).
  Así, se toma la sucesión \(\{e_{ij} \ : \ j \in \{1, \ldots, n_i\}\}\) que viene dada
  por \(\ann_A(x_{ij}) = \langle p_i^{e_{ij}} \rangle \), con \(j = 1, \ldots, n_i\).
  
  Supongamos otra descomposición:
  \[
    M=N_1\dotplus \cdots \dotplus N_t
  \]
  con \(N_i\) \(s_i\)-primario para \(s_1,\ldots,s_t\in A\) irreducibles.
  Entonces
  \[
    \left\langle \mu\right\rangle =\Ann_A(M)=\bigcap_{i=1}^t \Ann_A(N_i)
    =\bigcap_{i=1}^t\left\langle s_i^{t_i}\right\rangle
    = \left\langle\mcm\{s_i^{t_i}\}\right\rangle=\left\langle\prod s_i^{t_i}\right\rangle
  \]
  y \(\mu\) es asociado con \(s_1^{t_1}\cdots s_t^{t_t}\).
  Tras reordenación, por ser \(A\) un DFU, \(t=r\) y \(s_i=p_i\).

  \(N_i\subseteq\{m\in M:p_i^{e_i} m=0\}=M_i\), entonces
  \(N_i=M_i\), argumentando sobre las longitudes.
\end{proof}
