\section{Módulos noetherianos}

\begin{df}[Sucesiones exactas]
  Una sucesión de homomorfismos de módulos
  \[
    \cdots \overset{f_{i-1}}{\longrightarrow} M_i \overset{f_i}{\longrightarrow} M_{i+1}
    \overset{f_{i+1}}{\longrightarrow} \cdots,
  \]
  donde\(f_i:M_i\longrightarrow  M_{i+1}\), se dice
  \textbf{exacta en \(M_{i+1}\)} si \(\ker f_{i+1}=\ima f_i\),\(i \in \N\).

  La sucesión anterior se dirá exacta si lo es para cualquier \(M_i\), \(i \in \N\).
\end{df}

El 0 representa a un módulo con un único elemento, el cero.

\begin{df}
  Dada una sucesión
  \[
    \{0\}\longrightarrow L 
    \overset{\alpha}{\longrightarrow} M 
    \overset{\beta}{\longrightarrow} N\longrightarrow \{0\},
  \]
  La sucesión es:
  \begin{itemize}
  \item Exacta en \(L\) si, y solo si, \(\ker \alpha= \ima 0 = \{0\}\), es decir,
    \(\alpha\) es inyectiva
  \item Exacta en \(N\) si, y solo si, \(\ima \beta = N\),
    es decir, \(\beta\) sobreyectiva
  \item Exacta en \(M\) si, y solo si, \(\ker\beta=\ima \alpha\).
  \end{itemize}
  
  A \(\alpha\) se les llama monomorfismos de módulos y a
  \(\beta\) epimorfismos de módulos.
  Se dice que la sucesión anterior es una \textbf{sucesión exacta corta} si es exacta
  en \(L, M\) y \(N\).
\end{df}

\begin{obs}
  Si la sucesión
  \[
    \{0\}\longrightarrow L 
    \overset{\alpha}{\longrightarrow} M 
    \overset{\beta}{\longrightarrow} N\longrightarrow \{0\},
  \]
  es exacta corta, entonces, por el teorema \ref{teo:iso_modulos},
  \[
    N \cong \frac{M}{\ker g} = \frac{M}{\ima f}.
  \]
\end{obs}

\begin{ejemplo}
  Si \(f:M\longrightarrow N\) es un homorfismo de módulos, obtenemos la siguiente
  sucesión exacta corta:
  \[
    0 \longrightarrow \ker f \overset{\iota}{\longrightarrow}
    M \overset{f}{\longrightarrow} \ima f \longrightarrow 0
  \]
\end{ejemplo}

\begin{prop}
  Sea \(0\longrightarrow L\overset{\psi}{\longrightarrow} M
  \overset{\varphi}{\longrightarrow}
  N\longrightarrow 0\) una sucesión exacta de \(A\)-módulos. Entonces:
  \begin{enumerate}
  \item Si \(M\) es finitamente generado, lo es también \(N\).
  \item Si \(L\) y \(N\) son finitamente generados, lo es también \(M\).
  \end{enumerate}
\end{prop}

\begin{proof}
  Veamos la primera afirmación. Sean \(\{m_1, \ldots, m_t\}\)
  generadores de \(M\).
  Es claro que \(\{\varphi(m_1), \ldots, \varphi(m_t)\}\) generan \(N\) por ser
  \(\varphi\) sobreyectiva.

  Para la segunda, sean \(\{n_1, \ldots, n_r\}\) generadores de
  \(N\), y tomamos
  \(m_i \in M\) tal que \(\varphi(m_i)= n_i\) para todo \(i \in \{1, \ldots, r\}
  \). Sean \(\{e_1, \ldots, e_s\}\) generadores de \(L\) y \(m \in M\).
  Entonces
  \[
    \varphi(m)=\sum_{i=1}^r a_i n_i = \sum_{i=1}^r a_i\varphi(m_i)
    = \varphi\left(\sum_{i=1}^r a_i m_i\right)
  \]
  con lo que \(m- \sum_{i=1}^r a_i m_i) \in \ker \varphi = \ima \psi\), donde \(a_i \in A\)
  para todo \(i \in \{1, \ldots, r\}\).
  Luego ha de existir \(l = \sum_{i=1}^s b_ie_i\), con \(b_i \in A\) para todo \(i \in
  \{1, \ldots, s\}\), tal que
  \[
    m - \sum_{i=1}^r a_i m_i= \psi(l) = \psi\left(\sum_{j=1}^s b_j e_j\right) =
    \sum_{j=1}^s b_j \psi(e_j)
  \]
  y finalmente:
  \[
    m = \sum_{i=1}^r a_i m_i + \sum_{j=1}^s b_j \varphi(e_j)
  \]
  con lo que \(\{m_1, \ldots, m_r, \psi(e_1), \ldots, \psi(e_s)\}\) es un conjunto de
  generadores de \(M\).
\end{proof}


Veamos que no se puede mejorar la proposición anterior.
\begin{ejemplo}\label{ejemplo:no_noe}
  Sea \(I\) un conjunto infinito de índices y \(R\) un anillo. Entonces el conjunto
  \[
    R^I=\{{(\alpha_i)}_{i\in I} \ : \ \alpha_i\in R\}
  \]
  con las mismas operaciones
  \begin{itemize}
  \item \((r_i)_{i \in I} + (r_i')_{i \in I} = (r_i + r_i')_{i \in I}\).
  \item \((r_i)_{i \in I} (r_i')_{i \in I} = (r_i r_i')_{i \in I}\).
  \end{itemize}
  es un \(R\)-módulo cíclico finitamente generado
  por \((\ldots,1,1,1,\ldots)\). Definimos:
  \[
    R^{(I)}=\{{(\alpha_i)}_{i\in I} \ : \ \alpha_i \in R \textrm{ y } \alpha_i=0
    \textrm{ salvo un número finito de índices}\}
  \]
  Tenemos que \(R^{(I)}\) es un ideal de \(R^I\), y por tanto ideal a
  izquierda, pero no es finitamente generado como tal.
\end{ejemplo}

Es decir, \(M\) finitamente generado no implica que un submódulo suyo
sea finitamente generado.

\begin{df}[Módulos Noetherianos]
  Un módulo finitamente generado \(M\) se dice \textbf{noetheriano} si todo
  submódulo de \(M\) es finitamente generado.
\end{df}

En el ejemplo \label{ejemplo:no_noe}, \(R^I\) no es un módulo noetheriano.

\begin{prop}\label{prop:noether}
  Equivalen:
  \begin{enumerate}
    \item \(M\) es noetheriano.
    \item Cualquier cadena ascendente \(L_1\subseteq L_2\subseteq\ldots
      \subseteq L_n\subseteq\ldots\) se estabiliza, es decir,
      a partir de un cierto \(m\) las inclusiones se vuelven igualdades.
    \item Cada subconjunto no vacío de \(\mathcal{L}(M)\) tiene un elemento
      maximal con respecto de la inclusión.
  \end{enumerate}
\end{prop}

\begin{proof}
  Veamos que la primera implica la segunda.
  Tomamos:
  \[
    L=\bigcup_{n\ge 1} L_n\in\mathcal{L}(M)
  \]
  es un submódulo porque están encajados. Por hipótesis, es finitamente
  generado. Si tomamos un conjunto finito de generadores \(F\)
  tenemos que \(F\subset L\) y como es finito, debe existir un \(m\)
  suficientemente grande tal que \(F\subseteq L_m\) y como
  genera a \(F\) se tiene que \(L\subseteq L_m\subseteq L\)
  con lo que \(L_n=L_m=L\) para todo \(n\ge m\).

  Veamos que la segunda implica la primera. Sea \(\Gamma\subseteq
  \mathcal{L}(M)\) no vacío. Si \(\Gamma\) no tiene elemento maximal
  y tomamos \(L_1\in\Gamma\), entonces existe \(L_2\in\Gamma\)
  tal que \(L_1\subsetneq L_2\).

  Reiterando el proceso, tenemos que \(L_1\subsetneq L_2\subsetneq
  \ldots\subsetneq L_n\subsetneq\ldots\) no se estabiliza.

  Veamos que la tercera afirmación implica la primera.
  Sea \(N\in\mathcal{L}(M)\).
  Tomamos el conjunto \(\Gamma\) el conjunto de todos los submódulos
  finitamente generados de \(N\). Tenemos que el módulo trivial
  es finitamente generado, luego \(\Gamma\) es no vacío.

  Sea \(L\) un elemento maximal de \(\Gamma\). Veamos que \(L=N\).

  En caso contrario, tomamos \(x\in N\) tal que \(x\notin L\). Resulta que
  \(L+Ax\) es un submódulo de \(N\) y es finitamente generado.
  \(L+Ax\in\Gamma\) y \(L\neq L+Ax\), con lo que \(L\) no sería maximal.
\end{proof}

\begin{nt}
  \(N\in\mathcal{L}(M)\), escribimos \(N\le M\).
\end{nt}

\begin{prop}[Sucesiones exactas cortas en módulos noetherianos]\label{prop:suc_noether}
  Sea \(0\longrightarrow L\overset{\varphi}{\longrightarrow}
  M\overset{\psi}{\longrightarrow} N
  \longrightarrow 0\).

  Entonces \(M\) es noetheriano si y solo si \(L\) y \(N\) son
  noetherianos.
\end{prop}

\begin{proof}
  Supongamos \(M\) noetheriano.

  \(L\cong \Im\psi\le M\) y entonces \(L\) es noetheriano trivialmente.

  Tomamos \(N_1\subseteq N_2\subseteq\ldots\subseteq N_n\subseteq\ldots\)
  una cadena ascendente en \(\mathcal{L}(N)\).

  Tenemos \(\varphi^{-1}(N_1)\subseteq \varphi^{-1}(N_2)
  \subseteq \varphi^{-1}(N_n)\subseteq\ldots\)
  cadena en \(\mathcal{L}(M)\). Existe un \(m\) a partir del cual
  se estabiliza. Entonces, para todo \(n\ge n\):
  \[
    N_n=\varphi(\varphi^{-1}(N_n))=\varphi(\varphi^{-1}(N_m))=N_m
  \]
  con lo cual \(N\) es noetheriano.


  Supongamos ahora que \(N\) y \(L\) son noeherianos.
  Tomamos una cadena ascendente \(M_n\) de submódulos de \(M\).

  Por otro lado, \(M_n\cap\Im\psi\) es una cadena de submódulos de
  \(M\), que se estabiliza por ser noetheriano \(\Im\psi\cong L\).

  Tenemos \(\varphi(M_n)\) es una cadena de submódulos de \(N\),
  que también se estabiliza.

  Tomemos el menor natural tal que ambas cadenas se hayan estabilizado.
  Sea \(n\) mayor, \(x\in M_n\), \(\varphi(x)\in\varphi(M_n)
  =\varphi(M_m)\), debe existir \(y\in M_m\). Luego \(x-y\in\ker\varphi
  =\Im\psi\), con lo que \(x-y\in M_n\cap\Im\psi=M_m\cap\Im\psi\subseteq M_m\)
  y \(x\in M_m\) ya que \(y\in M_m\).

  Por tanto \(M\) es noetheriano.
\end{proof}

\begin{cor}
  Dados dos módulos \(M_1\) y \(M_2\).
  Entonces:
  \[
    M_1\oplus M_2
  \]
  es noetheriano si y solo si \(M_1\) y \(M_2\) lo son.
\end{cor}

\begin{proof}
  Sea la sucesión exacta corta
  \[
    0\longrightarrow M_1\longrightarrow M_1\oplus M_2\longrightarrow M_2
    \longrightarrow 0
  \]
  donde la primera aplicación es \(m_1\mapsto(m_1,0)\)
  y \((m_1,m_2)\mapsto m_2\) y el núcleo de la segunda es la imagen de
  la primera. Trivialmente se sigue el corolario.
\end{proof}

\begin{teo}
  Sea \(A\) un anillo. Cada módulo sobre \(A\) finitamente generado
  es noetheriano si, y solo si, \(\subscriptbefore{A}{A}\) es noetheriano.
\end{teo}

\begin{proof}
  Una de las implicaciones es obvia.

  Veamos que si el módulo regular es noetheriano, veamos que cualquier otro
  lo es.

  Sea \(M\) finitamente generado, existe un homomorfismo sobreyectivo \(\phi\)
  tal que \(A^n\longrightarrow M\).

  Usando inductivamente el corolario, tenemos que \(A^n\) es noetheriano.
  La proposición nos dice que \(M\) es noetheriano, aplicandolo
  a la sucesión
  \[
    0\longrightarrow\ker\phi\longrightarrow A^n\longrightarrow
    M\longrightarrow 0
  \]
\end{proof}

\begin{df}[Anillo noetheriano]
  \(A\) se dice noetheriano a izquierda si el módulo regular es
  noetheriano. Si \(A\) es conmutativo diremos simplemente noetheriano.
\end{df}

\begin{teo}
  Un anillo \(A\) es noetheriano a izquierda si, y solo si, todos los módulos
  finitamente generados sobre él son noetherianos.
\end{teo}

\begin{proof}
  Supongamos que \(A\) es noetheriano y sea \({}_AM\) finitamente generado. Sea la
  sucesión exacta \(0 \longrightarrow L \longrightarrow A^m \longrightarrow {}_AM
  \longrightarrow 0\). Sabemos que \({}_AA\) es noetheriano, luego \({}_AA^m\) también
  lo es. Por tanto, \({}_AM\) es noetheriano.
\end{proof}

\begin{cor}
  Todo dominio de ideales principales es noetheriano.
\end{cor}
\begin{proof}
  Sea \(A\) un DIP. Como \({}_AA\) es noetheriano, se puede aplicar el
  teorema anterior.
\end{proof}

\begin{prop}
  Si \(A\) es un anillo noetheriano a izquierda, entonces \(A[x]\) también lo es.
\end{prop}

\begin{cor}
  Si \(A\) es un anillo noetheriano, entonces \(A[x_1, \ldots, x_n]\) es también
  noetheriano.
\end{cor}

En particular, \(\C[x_1, \ldots, x_n]\) es notheriano.