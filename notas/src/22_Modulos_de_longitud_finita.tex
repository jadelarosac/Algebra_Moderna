Sea \(\subscriptbefore{A}{M}\) un módulo, \(\mathcal{L}(M)\)
es el conjunto de todos los submódulos de \(M\).
Dado \(\Gamma\subseteq\mathcal{L}(L)\) no vacío,
tenemos \(\bigcap_{N\in\Gamma}
N\in\mathcal{L}(M)\) (no tiene por qué ocurrir que estén en \(\Gamma\),
\(\bigcap_{n\ge 1} n\Z=\{0\}\notin m\Z\) para ningún \(m\ge 1\)).


\begin{df}[Zócalo]
  \textbf{El zócalo de \(M\)} es el menor submódulo de \(M\) que contiene a todos
  los submódulos simples de \(M\).

  Si \(M\) no tiene ningún submodulo simple, definimos el zócalo como
  \(\{0\}\).

  En ambos casos usaremos la notación \(\Soc(M)\).
\end{df}

\begin{ejemplo}
  Si \(V\) es un \(K\)-espacio vectorial, \(\Soc(V)=V\).
\end{ejemplo}

\begin{ejemplo}
  \(\Soc(\Z)=\{0\}\), puesto que cada \(n\Z\) contiene
  un \(2n\Z\), luego no es simple.
\end{ejemplo}

De hecho, si \(A\) es un dominio de
integridad que no es un cuerpo, \(\Soc(A)=\{0\}\). Tienes que sus submódulos
son ideales. Para \(x\in I\), el ideal generado por \(x^2\) está dentro de
\(I\), luego \(I\) no es simple.

\begin{prop}
  Sea \(M\) un \(A\)-módulo de longitud finita. Entonces existen submódulos
  \(S_i\) simples de \(M\)
  tales que
  \[
    \Soc(M)=S_1\dot{+}\cdots\dot{+} S_n.
  \]
  Además, si existen otros submódulos \(T_i\) simples tales que
  \(
    \Soc(M)=T_1\dot{+}\cdots\dot{+} T_m
  \), entonces \(n=m\) y tras reordenación, \(S_i\cong T_i\).
\end{prop}

\begin{proof}
  Si \(\Gamma\) es el conjunto de todos los submódulos de la forma
  \(
    S_1\dot{+}\cdots\dot{+} S_n
  \)

  Si \(M\neq\{0\}\), entonces \(\Gamma\neq\emptyset\), ya que \(M\)
  contiene algún submódulo simple.

  Como \(M\) es noetheriano, existe un
  \(
    S_1\dot{+}\cdots\dot{+} S_n
  \) maximal.

  \(
    S_1\dot{+}\cdots\dot{+} S_n\subseteq\Soc(M)
  \). Sea \(S\in\mathcal{L}(M)\) simple.
  \[
    S\cap(
    S_1\dot{+}\cdots\dot{+} S_n
    )
  \]
  puesto que \(S\) es simple y la intersección es submódulo, se tiene
  que dicha intersección o es \(\{0\}\) o es \(S\).

  Consideramos
  \[
    S\cap(
    S_1\dot{+}\cdots\dot{+} S_n
    ) = \{0\}
  \]
  luego
  \[
    S\dot{+}S_1\dot{+}\cdots\dot{+} S_n\in\Gamma
  \]
  con lo que no sería maximal.

  Luego se tiene:
  \[
    S\subseteq
    S_1\dot{+}\cdots\dot{+} S_n\in\Gamma
  \]
  luego, como \(S\) era un modulo simple arbitrario, tenemos que
  \(\Soc(M)=
    S_1\dot{+}\cdots\dot{+} S_n
  \).

  Resulta que
  \[
    \{0\}\subsetneq S_1\subsetneq S_1\dot{+} S_2\subsetneq\ldots
    \subsetneq
    S_1\dot{+}\cdots\dot{+} S_n=\Soc(M)
  \]
  es una serie de composición, ya que:
  \[
    (S_1\dot{+}\cdots\dot{+} S_i)/
    (S_1\dot{+}\cdots\dot{+} S_{i-1})\cong
    S_i
  \]
  Aplicando Jordan-Hölder se obtiene el resultado.
\end{proof}

\begin{df}[Módulo semisimple]
  Sea \(M\) un \(A\)-módulo de longitud finita. Decimos que \(M\) es \textbf{semisimple}
  si es \(\Soc(M)=M\).
\end{df}

\begin{ejercicio}
  Sea \(A\) un DIP que no sea un cuerpo,
  \(I\) ideal de \(A\). Se pide demostrar
  que \(A/I\) es de longitud finita si, y solo si, \(I\neq\langle 0\rangle\).
\end{ejercicio}

¿Se puede deducir cuál es la longitud de \(A/I\) de un generador de \(I\)?

\subsection{Módulos de longitud finita sobre un DIP}

Sea de ahora en adelante \(A\) un dominio de ideales principales que no
sea un cuerpo. El objetivo es describir la estructura de los \(A\)-módulos
de longitud finita. Recordemos que \(A\) es un anillo conmutativo cualquiera.

\begin{prop}
  \(\subscriptbefore{A}{M}\) es un módulo de longitud finita si, y solo si,
  \(\subscriptbefore{A}{M}\) es finitamente generado y acotado.
\end{prop}
\begin{proof}
  Supongamos que \(M\) es distinto del 0, porque si no es trivial.

  \(M\) es de longitud finita, por tanto noetheriano, lo que implica que también
  es finitamente generado: \(M=Am_1+\cdots+Am_n\), con \(m_i\in M\) para cada
  \(i \in \{1, \ldots, n\).
  \[
    \langle\mu\rangle=\Ann_A(M)=\bigcap_{i=1}^n\ann_A(m_i)
  \]
  porque el anillo \(A\) es conmutativo, donde ademas cada anulador
  de cada elemento es un ideal (a izquierdas en un conmutativo, luego ideal).

  Sea \(\langle f_i\rangle=\ann_A(m_i)\), entonces
  \[
    \langle\mu\rangle=\bigcap_{i=1}^n\langle f_i\rangle
  \]
  donde \(\mu=\mcm\{f_i:1\le i\le n\}\).

  Veamos que \(f_i\neq 0\) para cada \(i\).
  \[
    M\subseteq Am_i\cong A/\langle f_i\rangle
  \]
  luego \(\ell(Am_i)<\infty\), como \(A\) no es un cuerpo y por
  tanto \(M\) no es artiniano, entonces
  \(\langle f_i\rangle\neq0\).

  Luego \(\langle\mu\rangle\neq 0\) y por tanto \(M\) es acotado.

  Veamos el recíproco: \(M\) acotado y finitamente generado.
  \[
    M=Am_1+\cdots+Am_n
  \]

  Vemos que cada \(Am_i\) es de longitud finita (\(\mu\neq 0\) por ser
  acotado, luego cada \(\langle f_i\rangle\neq 0\)).
  Tenmos que \(Am_i\cong A/\langle f_i\rangle\) es de longitud finita.

  Existe un epimorfismo entre \(Am_1\oplus\cdots\oplus Am_n\) (que es
  de longitud finita) y \(Am_1\oplus\cdots\oplus Am_n\), con lo que
  el segundo tiene longitud finita.
\end{proof}

\(\ell_A(M)<\infty\), entonces es acotado, o sea
\(\langle\mu\rangle=\Ann_A(M)=\langle 0\rangle\). Entonces
\[
  M=M_1\dot{+}\cdots\dot{+} M_t
\]
donde \(M_i\) es la componente \(p_i\) primaria que viene de
\(\mu={p_1}^{e_1}\cdots{p_t}^{e_t}\)
(\(M_i=\{m\in M:m\cdot{p_i}^{e_i}=0\}\)).
Además \(M_i\) es finitamente generado.
¿Se puede descomponer como suma directa de submódulos indescomponibles?


\[
  M=M_1\dot{+}\cdots\dot{+} M_t
\]
donde
\[
  M_i=\{q_i m: m\in M\}=\{m\in M: p_i^{e_i} m=0\}=\{m\in M:
  a_i q_i m = m\}
\]
con \(q_i=\frac{\mu}{p_i^{e_i}}\) y \(\sum_i a_i q_i=1\)
y \(\langle \mu\rangle=\Ann_A(M)\). Se tiene que
\(\Ann_A(M_i)=\langle p_i^{e_i}\rangle\).