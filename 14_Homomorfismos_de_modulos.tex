\[
  a+\ann_A(m)\mapsto am
\]

Ejemplo: \(S=\Map(\N,K)\), el conjunto de las sucesiones (que forman
un \(K\)-espacio vectorial). Tomamos \(T:S\longrightarrow S\)
tal que \(T(s)(n)=s(n+1)\). Es lineal. Entonces
\(\subscriptbefore{K[x]}{S}\), donde \(xs=T(s)\).

Para cualquier \(f\in K[x]\), es decir \(f=\sum_i f_i x^i\), se tiene:
\[
  (fs)(n)=\sum_i f_i s(n+i)
\]

Imaginémosnos que \(s\) verifica que \(\ann_{K[x]}(s)\neq\langle0\rangle\).
Podemos tomar entonces un polinomio tal que \(fs = 0\) y que sea mónico.
Tenemos entonces que \(s(n+m)=-\sum_{i=0}^{m-1} f_i s(n+i)\)
para todo \(n\in\N\). Es decir, la sucesión es linealmente recursiva.

Caso particular, \(s(0)=s(1)=1\), tenemos que
\[
  s(n+2)=s(n)+s(n+1)
\]

\[
  x^2-x-1\in\ann_{\Q[x]}(s)
\]

Volviendo al caso general, tenemos que
\[
  K[x]/\ann_{K[x]}(s)\cong K[x]s
\]

Tenemos que \(\dim_{K}(K[x]s)<\infty\) si y solo si
\(\ann_{K[x]}(s)\neq\langle0\rangle\) si y solo si
\(s\) es una sucesión linealmente recursiva.

El generador \(p(x)\) de \(\ann_{K[x]}(s)\) se le llama el polinomio
mínimo de \(s\). El grado de dicho polinomio, coincide con
\(\dim_{k}(K[x]s)\) y se le llama complejidad lineal de \(s\).

\(s,t\) dos sucesiones linealmente recursivas.
\(K[x](s+t)\subseteq K[x]s+K[x]t\), luego la primera tiene dimensión finita.
Luego \(s+t\) es una sucesión linealmente recursiva, de complejidad menor
o igual a la suma de las complejidades lineales.
Puede argumentarse lo mismo para combinaciones lineales.

Las sucesiones linealmente recursivas forman un subespacio vectorial
del espacio de sucesiones. De hecho forman un submódulo. Sea
\(S^l\) el conjunto de las sucesiones linealmente recursivas, forma
un \(S^l\) es un \(K[x]\)-submódulo de \(S\), ya que es ivariante por la
acción de \(x\) (es \(T\)-invariante).

Otro ejemplo: \(T\) endomorfismo de \(\Cont^\infty(\R)\) tal que
\(T(\varphi)=\varphi'\). Tenemos que
\(\subscriptbefore{R[x]}{\Cont^\infty(\R)}\). Dada \(\varphi\),
tenemos que
\[
  \ann_{\R[x]}(\varphi)=\{f\in\R[x]: f(x)\varphi=0\}
  =\{f=\sum_i f_i\frac{d^i}{dt^i}: f\varphi=0\}
\]
\(\ann(\varphi)\neq\langle 0\rangle\) si \(\varphi\) satisface una ecuación
diferencial lineal homogénea con coeficientes constantes. Bla bla.

\(\R[x]/\ann_{\R[x]}(\varphi)\cong \R[x]\varphi\), donde \(\varphi\)
satisface bla bla.

Tenemos que \(\varphi''-\varphi'-\varphi=0\), cuya solución
\(\varphi(t)=e^{\phi t}\), donde \(\phi\) es la razón aúrea.

