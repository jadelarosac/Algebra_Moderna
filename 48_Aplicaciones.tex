\section{Algunas aplicaciones}
\subsection{\(\C\)-álgebras de grupos finitos}

Sea \(\C\) el cuerpo de los números complejos y \(G\) un grupo con elemento
neutro \(e\). Sea \(\C G\) el \(\C\) espacio vectorial con base \(G\).

\(\mu:\C G\times \C G\longrightarrow \C G\) la aplicación bilineal dada por
\(\mu(g,h)=gh\) para \(g,h\in G\).

Si para \(r,s\in\C G\) denotamos \(rs=\mu(r,s)\) donde si
\(r=\sum_{g\in G} r_g g\) y \(s=\sum_{g\in G} s_g g\), \(r_g, s_g\in\C\),
se tiene:
\[
  rs=\sum_{g,h\in G} r_g s_h\mu(g,h)=\sum_{g,h\in G}r_g s_h gh
\]

Tenemos que \(\mu\) define un producto que es asociativo.

\(\C G\times \C G\times \C G\overset{\mu\times\id}{\longrightarrow} \C G
\times \C G\overset{\mu}{\longrightarrow} \C G\)
proporciona el mismo resultado que
\(\C G\times \C G\times \C G\overset{\id\times\mu}{\longrightarrow} \C G
\times \C G\overset{\mu}{\longrightarrow} \C G\). Pero esto es trivial
porque son dos aplicaciones trilineales que evaluadas en una base dan lo mismo
(porque el producto en \(G\) es asociativo). Este es un ejemplo de un funtor
del producto en \(G\) al producto en \(\C G\).

Al ser bilineal, es distributiva respecto de la suma.

El elemento neutro de este producto es \(1e\in \C G\).

\begin{prop}
  \(\C G\) con la estructura que hemos discutido, es un anillo.
\end{prop}

La aplicación \(\eta:\C\longrightarrow \C G\) dada por \(z\mapsto ze\) es un
homomorfismo de anillos, distinto de 0 y parte de un cuerpo, luego es
inyectivo. Luego consideraremos que \(\Im\eta\) es un subanillo de \(\C G\)
y \(\Im\eta\cong\C\). Vamos a considerar entonces que \(1e=1=e\) y que
\(\C\subseteq\C G\).

Además, \(\C\subseteq\C G\). \(gz=g\eta(z)=gze=zge=zg\), es decir, los
complejos son centrales.

\begin{df}[\(\C\)-álgebra del grupo \(G\)]
  \(\C G\) se llama \(\C\)-álgebra del grupo \(G\).
\end{df}

Definimos \(\mu(G):=\{f:G\longrightarrow\C:f\textrm{ es aplicación}\}\),
es un \(\C\)-espacio vectorial con base \(G\). Vamos a darle estructura
de módulo.

\(\mu(G)\) es un \(\C G\)-módulo definiendo para todo \(g\in G\) y
\(\varphi\in\mu(G)\) y \(x\in G\):
\[
  (g\varphi)(x) = \varphi(xg)
\]
Vemos que \(g(h\varphi)(x)=(h\varphi)(xg)=\varphi(xg)h=\varphi(x(gh))=
(gh)\varphi(x)\). Entonces \(g(h\varphi)=(gh)\varphi\).

Hemos dado una aplicación \(G\longrightarrow\Map(\mu(G),\mu(G))\)
con \(g\mapsto(\varphi\mapsto g\varphi)\). Queremos ver que
\(G\longrightarrow\End(\mu(G),\mu(G))\), es decir
\(g(\varphi+\psi)=g\varphi+g\psi\).

\[
  g(\varphi+\psi)(x)=(\varphi+\psi)(xg)=\varphi(xg)+\psi(xg)
\]
y por otro lado
\[
  g\varphi(x)+g\psi(x)=\varphi(xg)+\psi(xg)
\]

Además:
\[
  g(z\varphi)(x)=z\varphi(xg)=z(g\varphi)(x)
\]

\(\C G\longrightarrow\End_\C(\mu(G),\mu(G))\), es \(\C\)-lineal.

En resumen, tenemos la siguiente proposición:

\begin{prop}
\(\mu(G)\) es un \(\C G\)-módulo.
\end{prop}

Nuestro objetivo es demostrar que si \(G\) es finito, \(\C G\) es semisimple
y \(\mu(G)\) semisimple como \(\C G\)-módulo.

\begin{df}[Producto hermítico]
  Sea \(V\) un espacio vectorial complejo de dimensión finita.
  Un producto interno hermítico es una aplicación
  \(\langle\cdot|\cdot\rangle V\times V\longrightarrow \C\) cumpliendo:
  \begin{enumerate}
    \item \(\langle v|w\rangle=\overline{\langle w|v\rangle}\).
    \item \(\langle v'+v|w\rangle=\langle v|w\rangle+\langle v'|w\rangle\).
    \item \(\langle av|w\rangle=a\langle v|w\rangle\).
    \item \(\langle v| v\rangle\implies v=0\).
  \end{enumerate}
  Es decir, es un espacio de Hilbert complejo de dimensión finita.
\end{df}



