\subsection{Módulos a derecha}
\begin{df}[Anillo opuesto]
Sea \(R\) un anillo. Mantengo su estructura de grupo aditivo, pero cambiamos
el producto. El nuevo producto va a ser el producto opuesto dado por
\(r*s:=sr\).

  A \(R\) con este nuevo producto lo vamos a llamar \(R^{op}\), el anillo
  opuesto.
\end{df}

Ejemplo: \[
  R=
  \begin{pmatrix}
    \Z&\Q\\
    0&\Q
  \end{pmatrix}
  \le \mathcal{M}_2(\Q)
\]

Tenemos que \(\subscriptbefore{R}{R}\) no es noetheriano, pero
\(\subscriptbefore{R^{op}}{R^{op}}\) sí que lo es.
Es decir, es noetheriano a derecha pero no a izquierda.

\begin{df}[Anillo noetheriano a derecha]
  Un anillo es noetheriano a derecha si el anillo opuesto es noetheriano
  a izquierda.
\end{df}
\begin{df}[Módulo a derechas]
  Definimos \(M\) módulo a derechas como \(M_R:=\subscriptbefore{R^{op}}{M}\).
\end{df}

\begin{df}[Ideal bilátero]
  Un ideal bilátero es un ideal a izquierda que es ideal a derecha también.
\end{df}

\begin{df}[Dual de un módulo]
  Sea \(\subscriptbefore{R}{M}\) un módulo. Tomamos
  \[
    \superscriptbefore{*}{M}:=\{f:M\longrightarrow R: f
    \textrm{ es homomorfismo de \(R\)-módulos}\}
  \]
  que es un grupo aditivo y un módulo a derechas, es decir,
  \(R^{op}\)-módulo, por la acción:
  \[
    (r\varphi)(m):=\varphi(m)r
  \]
  con \(r\in R\), \(m\in M\) y \(\varphi\in \superscriptbefore{*}{M}\).
  Es decir, \(\superscriptbefore{*}{M}_R\).
\end{df}

\begin{lema}
  \(\theta:{(\End_R(M))}^{op}
  \longrightarrow\End_{R^{op}}(\superscriptbefore{*}{M})\)
  tenemos que \(\theta(f)(\varphi):=\varphi\circ f\) es un homomorfismo
  de anillos.
  Nota: el producto en \({(\End_R(M))}^{op}\) es \(f*g=g\circ f\).
\end{lema}

Si \(M=\Z_n\) como \(Z\) módulos, \(\superscriptbefore{*}{M}=\{0\}\) así que
nos olvidamos de cualquier idea de reflexividad o isomorfismo.

\begin{df}[Módulos reflexivos]
  Un módulo en el que la anterior \(\theta\) es un isomorfismo.
\end{df}

