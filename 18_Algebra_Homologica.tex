\begin{proof}
  Veamos que la primera implica la segunda.
  Tomamos:
  \[
    L=\bigcup_{n\ge 1} L_n\in\mathcal{L}(M)
  \]
  es un submódulo porque están encajados. Por hipótesis, es finitamente
  generado. Si tomamos un conjunto finito de generadores \(F\)
  tenemos que \(F\subset L\) y como es finito, debe existir un \(m\)
  suficientemente grande tal que \(F\subseteq L_m\) y como
  genera a \(F\) se tiene que \(L\subseteq L_m\subseteq L\)
  con lo que \(L_n=L_m=L\) para todo \(n\ge m\).

  Veamos que la segunda implica la primera. Sea \(\Gamma\subseteq
  \mathcal{L}(M)\) no vacío. Si \(\Gamma\) no tiene elemento maximal
  y tomamos \(L_1\in\Gamma\), entonces existe \(L_2\in\Gamma\)
  tal que \(L_1\subsetneq L_2\).

  Reiterando el proceso, tenemos que \(L_1\subsetneq L_2\subsetneq
  \ldots\subsetneq L_n\subsetneq\ldots\) no se estabiliza.

  Veamos que la tercera afirmación implica la primera.
  Sea \(N\in\mathcal{L}(M)\).
  Tomamos el conjunto \(\Gamma\) el conjunto de todos los submódulos
  finitamente generados de \(N\). Tenemos que el módulo trivial
  es finitamente generado, luego \(\Gamma\) es no vacío.

  Sea \(L\) un elemento maximal de \(\Gamma\). Veamos que \(L=N\).

  En caso contrario, tomamos \(x\in N\) tal que \(x\notin L\). Resulta que
  \(L+Ax\) es un submódulo de \(N\) y es finitamente generado.
  \(L+Ax\in\Gamma\) y \(L\neq L+Ax\), con lo que \(L\) no sería maximal.
\end{proof}

\begin{nt}
  \(N\in\mathcal{L}(M)\), escribimos \(N\le M\).
\end{nt}

\begin{prop}[Sucesiones exactas cortas en módulos noetherianos]
  Sea \(0\longrightarrow L\overset{\varphi}{\longrightarrow}
  M\overset{\psi}{\longrightarrow} N
  \longrightarrow 0\).

  Entonces \(M\) es noetheriano si y solo si \(L\) y \(N\) son
  noetherianos.
\end{prop}

\begin{proof}
  Supongamos \(M\) noetheriano.

  \(L\cong \Im\psi\le M\) y entonces \(L\) es noetheriano trivialmente.

  Tomamos \(N_1\subseteq N_2\subseteq\ldots\subseteq N_n\subseteq\ldots\)
  una cadena ascendente en \(\mathcal{L}(N)\).

  Tenemos \(\varphi^{-1}(N_1)\subseteq \varphi^{-1}(N_2)
  \subseteq \varphi^{-1}(N_n)\subseteq\ldots\)
  cadena en \(\mathcal{L}(M)\). Existe un \(m\) a partir del cual
  se estabiliza. Entonces, para todo \(n\ge n\):
  \[
    N_n=\varphi(\varphi^{-1}(N_n))=\varphi(\varphi^{-1}(N_m))=N_m
  \]
  con lo cual \(N\) es noetheriano.


  Supongamos ahora que \(N\) y \(L\) son noeherianos.
  Tomamos una cadena ascendente \(M_n\) de submódulos de \(M\).

  Por otro lado, \(M_n\cap\Im\psi\) es una cadena de submódulos de
  \(M\), que se estabiliza por ser noetheriano \(\Im\psi\cong L\).

  Tenemos \(\varphi(M_n)\) es una cadena de submódulos de \(N\),
  que también se estabiliza.

  Tomemos el menor natural tal que ambas cadenas se hayan estabilizado.
  Sea \(n\) mayor, \(x\in M_n\), \(\varphi(x)\in\varphi(M_n)
  =\varphi(M_m)\), debe existir \(y\in M_m\). Luego \(x-y\in\ker\varphi
  =\Im\psi\), con lo que \(x-y\in M_n\cap\Im\psi=M_m\cap\Im\psi\subseteq M_m\)
  y \(x\in M_m\) ya que \(y\in M_m\).

  Por tanto \(M\) es noetheriano.
\end{proof}

\begin{cor}
  Dados dos módulos \(M_1\) y \(M_2\).
  Entonces:
  \[
    M_1\oplus M_2
  \]
  es noetheriano si y solo si \(M_1\) y \(M_2\) lo son.
\end{cor}

\begin{proof}
  Sea la sucesión exacta corta
  \[
    0\longrightarrow M_1\longrightarrow M_1\oplus M_2\longrightarrow M_2
    \longrightarrow 0
  \]
  donde la primera aplicación es \(m_1\mapsto(m_1,0)\)
  y \((m_1,m_2)\mapsto m_2\) y el núcleo de la segunda es la imagen de
  la primera. Trivialmente se sigue el corolario.
\end{proof}

\begin{teo}
  Sea \(A\) un anillo. Cada módulo sobre \(A\) finitamente generado
  es noetheriano si y solo si \(\subscriptbefore{A}{A}\) es noetheriano.
\end{teo}

\begin{proof}
  Una de las implicaciones es obvia.

  Veamos que si el módulo regular es noetheriano, veamos que cualquier otro
  lo es.

  Sea \(M\) finitamente generado, existe un homomorfismo sobreyectivo \(\phi\)
  tal que \(A^n\longrightarrow M\).

  Usando inductivamente el corolario, tenemos que \(A^n\) es noetheriano.
  La proposición nos dice que \(M\) es noetheriano, aplicandolo
  a la sucesión
  \[
    0\longrightarrow\ker\phi\longrightarrow A^n\longrightarrow
    M\longrightarrow 0
  \]
\end{proof}

\begin{df}[Anillo noetheriano]
  \(A\) se dice noetheriano a izquierda si el módulo regular es
  noetheriano. Si \(A\) es conmutativo diremos simplemente noetheriano.
\end{df}

\begin{cor}
  Si \(A\) es noetheriano, equivalen para cualquier sucesión exacta corta:
  \begin{enumerate}
    \item \(M\) es finitamente generado.
    \item \(L\) y \(N\) son finitamente generados.
  \end{enumerate}
\end{cor}

\begin{cor}
  Todo dominio de ideales principales es noetheriano.
\end{cor}
